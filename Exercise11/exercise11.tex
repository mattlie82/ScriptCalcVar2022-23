\documentclass[12pt,a4paper]{article}


\usepackage[width=16cm,height=10cm]{geometry}
\usepackage[utf8]{inputenc}
\usepackage{microtype}
\usepackage{epsfig,amsmath,amsfonts,amssymb,latexsym,mathtools}
\usepackage{bef_alex,color,relsize,tikz}
\usepackage{hyperref}
\usepackage{fancyhdr}



\setlength\headheight{2.2cm}

\rhead{\includegraphics[height=2cm]{wiaslogo-2010.pdf}}
\chead{\begin{minipage}[b]{8cm}
    \flushleft \small
    \textsc{Calculus of Variations}\\
    Winter term 2022/23\\
    Dr. Thomas Eiter\\
    Dr. Matthias Liero\\
    \end{minipage}
}

\lhead{\includegraphics[height=2cm]{husiegel_bw_op.png}}

\cfoot{\thepage}

\setlength\parindent{0pt}

\newcounter{AUFGABE}
\def\aaa#1{\stepcounter{AUFGABE}\bigskip\par\noindent{\textbf{Exercise
      \arabic{AUFGABE}#1}}}
\def\LSG#1{\par{\footnotesize \color{blue}\textbf{Solution:} #1\color{black}}}


\newcommand{\exerciseSet}[2]{
\vspace{1.5em}
\begin{center}
\Large{\textsc{Exercise Set #1}}\\
\end{center}
\begin{center}
	\small{due on #2}\\
\end{center}
\thispagestyle{fancy}

}

\begin{document}

\exerciseSet{11}{January 31, 2023}

\setcounter{AUFGABE}{36}



\aaa{Quasi-convexity.}
The definition of quasi-convexity for a continuous function
$f:\R^{m\ti d} \to\R$ reads as follows:
\[
\forall\, A\in \R^{m\ti d} \ \forall \, \phi\in \rmC^\infty_\rmc({B_1(0)};\R^m)
: \quad \int_{B_1(0)} f(A{+}\nabla \phi(x)) \dd x \geq \int_{B_1(0)}
f(A) \dd x ,
\]
where $B_1(0)$ is the open unit ball in $\R^d$ centered at $0$. 
\begin{itemize}
\item[(a)] Show that in the definition of quasi-convexity any open  bounded
domain $\Omega \subset \R^d$ can be used without changing the
definition. 
\item[(b)] Show that the much larger set
$\rmW^{1,\infty}_0(\Omega;\R^m)$ of test functions can be used
instead of $\rmC^\infty_\rmc(\Omega;\R^m)$. 
\item[(c)] Considering $\Omega=Q={]0,1[}^d\subset \R^d$, we may look at periodic
functions $\psi\in \rmC^\infty_\text{per}(\R^d;\R^m)$, i.e.\ $\psi\in
\rmC^\infty(\R^d;\R^m)$ with $\psi(m{+}y)=\psi(y)$ 
for $m\in \Z^d$ and $y\in \R^d$. Show that quasi-convexity is
equivalent to 
\[
\forall\, A\in \R^{m\ti d} \ \forall \, \psi\in 
\rmC^\infty_\text{per}(\R^d;\R^m):\quad 
 \int_Q f(A{+}\nabla \psi(x)) \dd x \geq f(A).
\]
\end{itemize}



\aaa{Quasi-convexity implies rank-one convexity.}
Consider the periodic function $h:\R \to \R$ with period $1$, \, 
$h(t)=1{-}\theta$ for $t\in {[0,\theta[}$ and $h(t)=-\theta$ for $t\in
{[\theta,1[}$ and its primitive function $H(t)=\int_0^t h(r) \dd
r$.%  
\begin{itemize}
\item[(a)] For $a\in \R^m$ and $\eta\in \R^d$ define the sequence $u_k(x) =
\frac1k H(k \eta{\cdot} x)a$. Show that $u_k\rightharpoonup0$ in
$\rmW^{1,p}(\Omega;\R^d)$.% 
\item[(b)] For a function $f:\R^{m\ti d}\to \R$ and fixed $A \in\R^{m\ti d}$
calculate $\lim\limits_{k\to \infty}\int_\Omega f\big(A{+}\nabla u_k(x)\big)\dd
x$.% 
\item[(c)] Show for continuous $f:\R^{m\ti d} \to \R$ that quasi-convexity
implies rank-one convexity.  
\end{itemize}





\aaa{Counterexample concerning Reshetnyak's theorem.}
Take $m=d=p=2$ and $\Omega ={]-1,1[}^2$ and the sequence
\[
u^k(x_1,x_2)=\frac1{\sqrt{k}} (1{-}|x_2|)^k \big( \sin(kx_1), \cos(kx_1)\big).
\]
\begin{itemize}
\item[(a)] Show that $u^k \rightharpoonup 0$ in $\rmH^1(\Omega;\R^2)$.
\item[(b)] Prove that $\int_\Omega \det (\nabla u^k) \varphi \dd x \to 0$ 
for all $\varphi \in \rmC_\rmc(\Omega)$.
\item[(c)] Show that $\det (\nabla u^k)$ does not converge weakly to $0$ in
$\rmL^1(\Omega)$. 
\\
\textit{Hint:} Consider suitable $\varphi\in
  \rmL^\infty(\Omega)$ in (b).
\end{itemize}





\end{document}