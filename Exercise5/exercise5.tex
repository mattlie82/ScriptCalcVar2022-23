\documentclass[12pt,a4paper]{article}


\usepackage[width=16cm,height=10cm]{geometry}
\usepackage[utf8]{inputenc}
\usepackage{microtype}
\usepackage{epsfig,amsmath,amsfonts,amssymb,latexsym,mathtools}
\usepackage{bef_alex,color,relsize,tikz}
\usepackage{hyperref}
\usepackage{fancyhdr}



\setlength\headheight{2.2cm}

\rhead{\includegraphics[height=2cm]{wiaslogo-2010.pdf}}
\chead{\begin{minipage}[b]{8cm}
    \flushleft \small
    \textsc{Calculus of Variations}\\
    Winter term 2022/23\\
    Dr. Thomas Eiter\\
    Dr. Matthias Liero\\
    \end{minipage}
}

\lhead{\includegraphics[height=2cm]{husiegel_bw_op.png}}

\cfoot{\thepage}

\setlength\parindent{0pt}

\newcounter{AUFGABE}
\def\aaa#1{\stepcounter{AUFGABE}\bigskip\par\noindent{\textbf{Exercise
      \arabic{AUFGABE}#1}}}
\def\LSG#1{\par{\footnotesize \color{blue}\textbf{Solution:} #1\color{black}}}


\newcommand{\exerciseSet}[2]{
\vspace{1.5em}
\begin{center}
\Large{\textsc{Exercise Set #1}}\\
\end{center}
\begin{center}
	\small{due on #2}\\
\end{center}
\thispagestyle{fancy}

}

\begin{document}

\exerciseSet{5}{November 21, 2022}

\setcounter{AUFGABE}{14}


\aaa{Weierstraß condition in 1D.}
In the one-dimensional case, the proof of the Weierstrass criterion is much simpler 
(and even equivalent to the quasiconvexity condition):
\\
Show that if $u_*\in\rmC^1([\alpha,\beta];\R^m)$ is a strong local minimizer of $I(u)= \int_\alpha^\beta f(x,u(x),u'(x))\dd x$ with $f\in\rmC^1([\alpha,\beta]\times \R^m\times\R^m)$, 
then for all $x\in[\alpha,\beta]$ and all $B\in\R^m$
we have the inequality
\[
E(x,u_*(x),u_*'(x),B)\geq 0,
\]
where $E(x,u,A,B) = f(x,u,A+B) - f(x,u,A) - \pl_Af(x,u,A)\cdot B$.

\emph{Hint:} First show the equivalence
\[
A\mapsto \wt f (A)\text{ is quasiconvex in }A_0\quad\Longleftrightarrow\quad
\forall B\in \R^m:~\widetilde E(A_0,B) \geq 0
\]
for $\widetilde{f}\in\rmC^1(\R^m)$ and $A_0\in\R^m$,
where $\widetilde E(A,B) = \widetilde{f}(A+B) -\widetilde{f}(A) -\pl_A\widetilde{f}(A)\cdot B.$
\\
For the ``$\Rightarrow$''-direction, consider $\delta>0$ and a function $v_\delta\in\mathrm{PC}^1_0([{-1},1];\R^m)$ 
that satisfies $v'_\delta(x) = 0$ for
$x\in [{-1},0[$, $v'_\delta(x) = B$ for $x\in[0,\delta[$, and 
$v'_\delta(x) = -\delta/(1{-}\delta)B$ for $x\in[\delta,1]$.

\aaa{Weak and strong convergence and continuity.}
Consider $f\in\rmC^0(\ol\Omega\times\R^m)$ that
satisfies the growth property
\[
\exists C>0,~p\in[1,\infty[,~h\in\rmL^1(\Omega)~\forall \, (x,u)\in\Omega\times\R^m:\quad |f(x,u)|\leq C(h(x)+|u|^p).
\]
and define the functional
$I : \rmL^p(\Omega;\R^m) \to \R$; $I(u)= \int_\Omega f(x,u(x)) \dd x$.

\begin{itemize}
    \item[(a)] Show that $I$ is strongly continuous.
    \item[(b)] Show that if $I$ is weakly continuous, then $f(x,\cdot)$ is affine, i.e., there exist $a \in \rmC^0(\ol\Omega)$ and $b \in \rmC^0 (\ol\Omega; \R^m )$ 
    such that $f (x, u) = a(x) + b(x)\cdot u$.
    
    \emph{Hint:  Look at functions $u$ rapidly oscillating between two values $w_0\in\R^m$ and $w_1\in \R^m$ such that the weak limit takes the value $w_\theta := (1{-}\theta)w_0 + \theta w_1$ (comp. Exercise 13).}
\end{itemize}

\aaa{Uniform convexity.} 
Let $X$ be a Banach space, and let $I\colon X\to\R_\infty$
be uniformly convex, that is,
there exists a strictly increasing function 
$g\colon[0,\infty)\to\R$
with $g(0)=0$
such that
for every $u\in X$ there is a linear continuous operator 
$L_u\colon X\to\R$
such that
\[
\forall v\in X: \quad I(v)\geq I(u)+L_u(v-u)+g(\|u-v\|). 
\] 
Show that $\lim_{t\to0}\frac{g(t)}{t}=0$ if and only if 
$I$ is Gateaux differentiable,
that is, the first variation $\rmD I(u)[w]$ 
exists for any $u,w\in X$.
Moreover, show
that in this case $L_u=\rmD I(u)$ for all $u\in X$.

\aaa{Uniform convexity in Hilbert spaces.} 
Let $X\neq \{0\}$ be a Hilbert space.
\begin{enumerate}
    \item[(a)]
    Show that $I\colon X\to[0,\infty]$, 
    $I(u)=\|u\|^2_X$ is uniformly convex. 
    \item[(b)]
    Show that $I\colon X\to[0,\infty]$, 
    $I(u)=\|u\|_X$ is convex but not uniformly convex. 
\end{enumerate}
\end{document}
