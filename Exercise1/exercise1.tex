\documentclass[12pt,a4paper]{article}


\usepackage[width=16cm,height=10cm]{geometry}
\usepackage[utf8]{inputenc}
\usepackage{microtype}
\usepackage{epsfig,amsmath,amsfonts,amssymb,latexsym,mathtools}
\usepackage{bef_alex,color,relsize,tikz}
\usepackage{hyperref}
\usepackage{fancyhdr}



\setlength\headheight{2.2cm}

\rhead{\includegraphics[height=2cm]{wiaslogo-2010.pdf}}
\chead{\begin{minipage}[b]{8cm}
    \flushleft \small
    \textsc{Calculus of Variations}\\
    Winter term 2022/23\\
    Dr. Thomas Eiter\\
    Dr. Matthias Liero\\
    \end{minipage}
}

\lhead{\includegraphics[height=2cm]{husiegel_bw_op.png}}

\cfoot{\thepage}

\setlength\parindent{0pt}

\newcounter{AUFGABE}
\def\aaa#1{\stepcounter{AUFGABE}\bigskip\par\noindent{\textbf{Exercise
      \arabic{AUFGABE}#1}}}
\def\LSG#1{\par{\footnotesize \color{blue}\textbf{Solution:} #1\color{black}}}


\newcommand{\exerciseSet}[2]{
\vspace{1.5em}
\begin{center}
\Large{\textsc{Exercise Set #1}}\\
\end{center}
\begin{center}
	\small{due on #2}\\
\end{center}
\thispagestyle{fancy}

}

\begin{document}

\exerciseSet{1}{Oct 24, 2022}


\aaa{. Harmonic functions.} \\
A function $u:\Omega \to \R$ is called
\emph{\bfseries harmonic} on the open domain $\Omega$, if it satisfies
the following \emph{\bfseries mean-value property}:
\[
\forall\, x \in \Omega \ \forall \, r>0\text{ with }B_r(x)\Subset
\Omega : \quad u(x) = \frac1{\omega_dr^{d-1}}\int_{y\in \pl B_r(x)} 
u(y) \dd a(y).
\]
Note that the right-hand side is an average over the sphere $\pl
B_r(x)$ as $\omega_d$ denotes the $(d{-}1)$-dimensional surface
measure of the sphere $\bbS^{d-1}\subset \R^d$.  We want to show that
a function $u \in \rmC^2(\Omega)$ is harmonic if and only if $\Delta
u(x)=0$ for all $x \in \Omega$.\\

(a) For fixed $x\in \Omega$ let $R(x)=\mathrm{dist}(x , \R^d\setminus
\Omega)$ and define
\[
\Phi(r)= \frac1{\omega_dr^{d-1}}\int_{y\in \pl B_r(x)} 
u(y) \dd a(y) .
\]
Transform the integral via $y=x+rz$ with $z \in \bbS^{d-1}$, and show
that $u\in \rmC(\Omega)$ implies  
that $\Phi:{[0,R(x)[}\to \R$ is continuous with $\Phi(0)=u(x)$.\\  
%
%\LSG{We find $\Phi(r) = \frac1{\omega_d}\int_{\bbS^{d-1}} u(x{+}rz)
%    \dd a(z)$. Uniform continuity of $u$ implies continuity of $\Phi$.}

(b) Show that a harmonic function $u\in \rmC^2(\Omega)$ satisfies
$\Delta u(x)=0$ for all $x \in \Omega$.\\ (Hint: Expand $u$ in a Taylor
series at $x$ via $u(x{+}rz)=u(x)+ r\nabla u(x)\cdot z + \frac12r^2
z\cdot \nabla^2 u(x)z + o(r^2)$ for $r\to 0$.)\\   
%
%\LSG{For $r>0$ we insert the Taylor expansion into 
%\[
%0=\frac1{r^2}\Big(\omega_d u(x)-\int_{\bbS} u(x{+}rz)\dd a \Big) = 
%\frac{\omega_d}{r^2} \int_{\bbS} \!\! \big(r\nabla u(x){\otimes} z +
%\frac{r^2}2 \nabla^2u(x) {:} (z{\odot} z) +o(r^2)\big) \dd a \ \to \
%\omega_d \nabla^2 u(x) {:} \! \int_\bbS \!\! z{\odot} z \dd a, 
%\]
%where we use the uniformity opf $o(r^2)$ w.r.t.\ $z \in \bbS$ and that
%$\int_\bbS z \dd a =0$ as each $z_i$ is odd. Finally we use 
%$\int_\bbS z{\otimes} z \dd a(z) = \gamma_d I \in \R^{d\ti d}$. Hence
%$\Delta u(x)= \nabla^2 u(x) : I =0$ follows. }

(c) For $u\in \rmC^2(\Omega)$ satisfying $\Delta u(x)=0$ for
all $x \in \Omega$ show that $u$ is harmonic. \\
(Hint: Show that $\Phi'(r)=0$ by differentiating the integral over
$\bbS^{d-1}$ obtained in (a) and show $\omega_d r^{d-1} \Phi'(r) =
\int_{\pl B_r(x)} \nabla u(y) \cdot \nu(y) \dd a(y) $.)


\aaa{. Expansion in eigenfunctions.}\\ 
In the interval $\Omega= \left]0,\pi \right[$, 
we consider the wave equation $u_{tt}=\Delta
 u=u_{xx}$ with {\scshape Dirichlet} boundary conditions
 $u(t,\cdot)|_{\pl\Omega}=0$.\\

(a) Determine a complete orthonormal system $ \set{\Phi_j }{j\in \N}$
in $\rmL^2(\Omega)$, such that
$u(t,x)=\cos(\omega_j t)\Phi_j(x)$ provides a solution to the wave
equation for suitable $\omega_j$.\\

(b) Show via superposition that for $u_0 \in \rmH^2(\left] 0, \pi\right[) \cap \rmH_0^1(\left] 0,\pi \right[)$
and $v_0 \in \rmH_0^1(\left] 0,\pi\right[)$ there exists a unique solution.\\

\aaa{. Fourier transformation for the wave equation.}\\
Consider the equation $u_{tt} = \Delta u$ in $\R^d$ with initial condition $u(0,\cdot) = u_0$ and $u_t(0,\cdot) = v_0$.
Let $u_0 \in \rmH^{k+1}(\R^d), v_0\in \rmH^k(\R^d)$. We want to show via Fourier transformation, that there is a solution 
of the wave equation
whicht satisfies \[u\in \rmC(\R; \rmH^{k+1}(\R^d)) \cap \rmC^1(\R;\rmH^k(\R^d)) \cap \ldots \cap \rmC^{k+1}(\R;\rmH^0(\R^d)),
\]
where $\rmH^0(\R^d)=\rmL^2(\Omega)$. Proceed as follows:

(a) Establish an explicit formula for the Fourier transform 
$(t,\xi)\mapsto\hat{u}(t,\xi)$ of a solution $u$ to the wave equation.\\

(b) Given that $u(t,\cdot) \in \rmH^k(\R^d)$ if and only if $\xi\mapsto
(1{+}\vert \xi\vert^2)^{k/2}\hat{u}(t,\xi) \in \rmL^2(\R^d)$, 
show the above claimed regularity. (Hint: One does not have to compute the inverse Fourier transform!)

\aaa{. Solution of the wave equation from the heat equation.}\\
Consider the homogeneous wave equation 
$u_{tt}=\Delta u$ in $\R^3$ with initial conditions
$u(0,x)= u_0(x)$ and $u_t(0,x) =  0$.
We assume $u_0\in\rmC^3(\R^3)$.
 We want to establish the solution formula
 \[
u(t,x) = \frac{\pl}{\pl t}\Big(\frac{1}{4\pi t}\int_{|x{-}y|=t}u_0(y)\dd \rma_y\Big).
 \]
Proceed via the following steps:

(i) Suppose that $u:\left[0,\infty\right[\times \R^3\to \R$ is a bounded, smooth solution
to the wave equation. Letting $\tilde u$ denote the even extension
of $u$ to $\R\times \R^3$, define the function
\[
v(t,x) = \frac{1}{\sqrt{4\pi t}}\int_{-\infty}^\infty
\tilde u(s,x)\ee^{-s^2/(4t)}\dd s
\] for $x\in \R^3$ and $t>0$. Show that $v$ solves the 
homogeneous \underline{heat equation} in $\R^3$.

(ii) Argue that $v$ can be represented via the \textsc{Heat Kernel}
$H_3(t,x) = (4\pi t)^{-3/2}\exp(-|x|^2/(4t))$.

(iii) Use the following result to conclude that $u$ 
is given by the representation formula above:
The \textsc{Laplace transform} of $f:\left[0,\infty\right[\to \C$
is given as $(\calL f)(z) 
= \int_0^\infty f(t)\ee^{-zt}\dd t$ for $z\in \C$. If $f:\left[0,\infty\right[\to \C$ and
$g:\left[0,\infty\right[\to\C$ are continuous and $(\calL f)(s) =(\calL g)(s)$
for all $s\in \R$ then $f(t)=g(t)$ for all $t\in[0,\infty[$.

\end{document}
