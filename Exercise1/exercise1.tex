\documentclass[12pt,a4paper]{article}


\usepackage[width=16cm,height=10cm]{geometry}
\usepackage[utf8]{inputenc}
\usepackage{microtype}
\usepackage{epsfig,amsmath,amsfonts,amssymb,latexsym,mathtools}
\usepackage{bef_alex,color,relsize,tikz}
\usepackage{hyperref}
\usepackage{fancyhdr}



\setlength\headheight{2.2cm}

\rhead{\includegraphics[height=2cm]{wiaslogo-2010.pdf}}
\chead{\begin{minipage}[b]{8cm}
    \flushleft \small
    \textsc{Calculus of Variations}\\
    Winter term 2022/23\\
    Dr. Thomas Eiter\\
    Dr. Matthias Liero\\
    \end{minipage}
}

\lhead{\includegraphics[height=2cm]{husiegel_bw_op.png}}

\cfoot{\thepage}

\setlength\parindent{0pt}

\newcounter{AUFGABE}
\def\aaa#1{\stepcounter{AUFGABE}\bigskip\par\noindent{\textbf{Exercise
      \arabic{AUFGABE}#1}}}
\def\LSG#1{\par{\footnotesize \color{blue}\textbf{Solution:} #1\color{black}}}


\newcommand{\exerciseSet}[2]{
\vspace{1.5em}
\begin{center}
\Large{\textsc{Exercise Set #1}}\\
\end{center}
\begin{center}
	\small{due on #2}\\
\end{center}
\thispagestyle{fancy}

}

\begin{document}

\exerciseSet{1}{October 24, 2022}

\setcounter{AUFGABE}{0}

\aaa{Vector calculus.}
\begin{enumerate}
\item[(a)] Let $f:\R^n \to \R$ be a differentiable function.
Prove that $x^0$ is a critical point of $f$, i.e., $\nabla f(x^0)=0$,
if $f$ has an extreme point in $x^0 \in \R^n$.

\item[(b)] Let $x^0 \in \R^n$ be a critical point of a twice differentiable function $f: \R^n \to \R$. 
Show that $f$ has a local minimum in $x^0$ (or maximum)
if the second variation, i.e., the quadratic form 
$\langle H(x^0)w,w\rangle = \sum_{1\leq i,j\leq n}\frac{\pl^2}{\pl x_i\pl x_j}f(x_0)w_iw_j$,
is positive (or negative) definite ($H(x^0)$ is called Hessian).
\end{enumerate}


\aaa{Linear Algebra.}

Let $I(u) = \frac{1}{2} \langle u,Au \rangle - \langle b,u \rangle$
with $A \in \R^{n \times n}$, $b,u \in \R^n$, and $\langle \cdot,\cdot \rangle$
denoting the Euclidean scalar product.
\begin{enumerate}
\item[(a)] Show that $u$ is a critical point of $I$ if and only if $\frac{1}{2}(A + A^\top)u = b$ holds.

\item[(b)] Does $I$ attain its minimum if $A$ is symmetric and positive definite?

\item[(c)] Let $A$ be symmetric and positive \underline{semi}-definite. Under which conditions on
$b$ does $\inf I > - \infty$ hold? 
Is  $\inf I = \min I$ satisfied?  
\end{enumerate}


\aaa{Example of {\sc Weierstraß}.}

Consider the functional
\[
I:M\to \R, u \mapsto \int_{-1}^1 [x u'(x)]^2 \dd x \ \text{ with } \
M=\set{ u\in \rmC^1([-1,1])}{ u(-1)=-1, \: u(1)=1}.
\]
Show that \underline{every} sequence 
$(u_k)_{k\in \N}$ with $I(u_k)\to 0$ convergences to 
$u_*:x\mapsto \mathop{\rm sign}(x)$ uniformly on \underline{compact} subsets of $[-1,0)\cup (0,1]$. 

First prove that you can estimate $|1{-}u(x)|=|\int_x^1 u'(s)\dd s|$ by 
$\sqrt\alpha C(x)$, with $\alpha =I(u)$ and $C \in \rmC^0((0,1])$. 
\textit{Hint:} Set $a(x)=[x u'(x)]^2$ and use $\int_{-1}^1 a(s)\dd
s=\alpha$. 


\newpage

\section*{Notation for Chapter 2}
\begin{itemize}
    \item 
    $\Omega\subset\R^d$ bounded domain with piecewise smooth boundary
    \item
    $u\colon\Omega\to\R^m$ a (sufficiently smooth) function
    \item
    $(\nabla u(x))_{ij}=\frac{\partial u_i}{\partial x_j}(x)$,
    where $i\in\{1,\dots,m\}$, $j\in\{1,\dots,d\}$.
    \item
    $f\colon\Omega\times\R^m\times\R^{m\times d}\to\R$,
    $(x,u,A)\mapsto f(x,u,A)$ volume density
    \item
    $g\colon\Omega\times\R^m\to\R$ boundary density
    \item
    $\Gamma_\mathrm{D}\subset\partial\Omega$ Dirichlet boundary,
    $\Gamma_\mathrm{N}\coloneqq\partial\Omega\setminus\Gamma_\mathrm{D}$ Neumann boundary
\end{itemize}
We consider functionals of the form
\[
I(u)=\int_\Omega f(x,u(x),\nabla u(x))\dd x
+\int_{\partial\Omega} g(x,u(x))\dd x
\]
for $u\in M$, where for $u_0\in C^1(\overline{\Omega};\R^m)$
we define
\[
M\coloneqq \{v\in C^1(\overline{\Omega};\R^m)\mid
v\vert_{\Gamma_\mathrm{D}}=u_0\vert_{\Gamma_\mathrm{D}}\}.
\]
Then we have $M=u_0+X_0$ with
\[
X_0\coloneqq  \{v\in C^1(\overline{\Omega};\R^m)\mid
v\vert_{\Gamma_\mathrm{D}}=0\}.
\]

\bigskip

Let $x\in\Omega$, $u\in\R^m$ and $A\in\R^{m\times d}$.
We need the following partial derivatives of first and second order of $f$ ($g$ analogously):
\[
\begin{aligned}
    \partial_x f 
    \coloneqq \left(\frac{\partial f}{\partial x_j}\right)_{j=1,\dots,d} 
    &=\left(\frac{\partial f}{\partial x_1},\dots, \frac{\partial f}{\partial x_d}\right) 
    \in\R^d,
    \\
    \partial_u f 
    \coloneqq \left(\frac{\partial f}{\partial u_i}\right)_{i=1,\dots,m}
    &=\left(\frac{\partial f}{\partial u_1},\dots, \frac{\partial f}{\partial u_m}\right) 
    \in\R^m,
    \\
    \partial_A f 
    \coloneqq \left(\frac{\partial f}{\partial A_{ij}}\right)_{\substack{i=1,\dots,m\\j=1,\dots,d}}
    &=\begin{pmatrix}
        \frac{\partial f}{\partial A_{11}} & \dots &  \frac{\partial f}{\partial A_{1d}} \\
        \vdots & \ddots & \vdots\\
        \frac{\partial f}{\partial A_{m1}}
        & \dots &  \frac{\partial f}{\partial A_{md}}
    \end{pmatrix}
    \in\R^{m\times d},
\end{aligned}
\]
\[
\begin{aligned}
    \partial_u^2 f 
    &\coloneqq \left(\frac{\partial^2 f}{\partial u_i\partial u_k}\right)_{\substack{i,k=1,\dots,m}}
    \in\R^{m\times m},
    \\
    \partial_A\partial_u f 
    &\coloneqq 
    \left(\frac{\partial^2 f}{\partial A_{ij}\partial u_k}\right)_{\substack{i,k=1,\dots,m\\j=1,\dots,d}}
    \in\R^{m\times d\times m},
    \\
    \partial_A^2 f 
    &\coloneqq 
    \left(\frac{\partial^2 f}{\partial A_{ij}\partial A_{k\ell}}\right)_{\substack{i,k=1,\dots,m\\j,\ell=1,\dots,d}}
    \in\R^{m\times d\times m\times d}.
\end{aligned}
\]
Further let $v\in\R^d$, $B\in\R^{m\times d}$.
For directional derivatives we write
\[
\begin{aligned}
    \mathrm{D}_u f(x,u,A)[v]
    &=\partial_u f(x,u,A)\cdot v,
    \\
    \mathrm{D}_A f(x,u,A)[B]
    &=\partial_A f(x,u,A):B,
    \\
    \mathrm{D}_u^2 f(x,u,A)[v]
    &=\sum_{i,k=1}^m
    \frac{\partial^2 f(x,u,A)}{\partial u_i\partial u_k} v_i v_k,
    \\
    \mathrm{D}_A\mathrm{D}_u f(x,u,A)[v,B]
    &=\sum_{i,k=1}^m\sum_{j=1}^d\frac{\partial^2 f(x,u,A)}{\partial A_{ij}\partial u_k} B_{ij}v_k,
    \\
    \mathrm{D}_A^2 f(x,u,A)[B]
    &=\sum_{i,k=1}^m\sum_{j,\ell=1}^d\frac{\partial^2 f(x,u,A)}{\partial A_{ij}\partial A_{k\ell}}B_{ij}B_{k\ell}.
\end{aligned}
\]
For $\mathrm{D}_u f$ and $\mathrm{D}_A f$
we used the scalar product for vectors and matrices, respectively:
\[
u\cdot v \coloneqq \sum_{i=1}^m u_i v_i,
\qquad
A:B\coloneqq \sum_{i=1}^m\sum_{j=1}^d A_{ij}B_{ij}=\mathrm{Tr}(B^\top A).
\]
The divergence of a matrix-valued function is taken row-wise:
For $A(x)\in\R^{m\times d}$ we have $\DIV A(x) \in \R^m$ with
\[
(\DIV A(x))_i = \sum_{j=1}^d \frac{\partial}{\partial x_j}A_{ij}(x).
\]
\end{document}
