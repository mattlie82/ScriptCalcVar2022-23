\documentclass[12pt,a4paper]{article}


\usepackage[width=16cm,height=10cm]{geometry}
\usepackage[utf8]{inputenc}
\usepackage{microtype}
\usepackage{epsfig,amsmath,amsfonts,amssymb,latexsym,mathtools}
\usepackage{bef_alex,color,relsize,tikz}
\usepackage{hyperref}
\usepackage{fancyhdr}



\setlength\headheight{2.2cm}

\rhead{\includegraphics[height=2cm]{wiaslogo-2010.pdf}}
\chead{\begin{minipage}[b]{8cm}
    \flushleft \small
    \textsc{Calculus of Variations}\\
    Winter term 2022/23\\
    Dr. Thomas Eiter\\
    Dr. Matthias Liero\\
    \end{minipage}
}

\lhead{\includegraphics[height=2cm]{husiegel_bw_op.png}}

\cfoot{\thepage}

\setlength\parindent{0pt}

\newcounter{AUFGABE}
\def\aaa#1{\stepcounter{AUFGABE}\bigskip\par\noindent{\textbf{Exercise
      \arabic{AUFGABE}#1}}}
\def\LSG#1{\par{\footnotesize \color{blue}\textbf{Solution:} #1\color{black}}}


\newcommand{\exerciseSet}[2]{
\vspace{1.5em}
\begin{center}
\Large{\textsc{Exercise Set #1}}\\
\end{center}
\begin{center}
	\small{due on #2}\\
\end{center}
\thispagestyle{fancy}

}

\begin{document}

\exerciseSet{8}{January 9, 2023}

\setcounter{AUFGABE}{25}

\aaa{``$\boldsymbol{\rmW^{1,\infty} = \rmC^{0,1}}$''.}
Let $\Omega\subset\R^d$ be a bounded domain with Lipschitz boundary, and let 
\[
\rmC^\text{0,1}(\overline\Omega)\coloneqq\set{u\in\rmC^0(\overline\Omega)}{\exists L>0: |u(x)-u(y)|\leq L|x-y|
\text{ for all }x,y\in\Omega}
\]
denote the class of Lipschitz continuous functions. Show that $u\in\rmW^{1,\infty}(\Omega)$
if and only if
there is $v\in\rmC^{0,1}(\overline{\Omega})$ 
such that $u=v$ a.e.~in $\Omega$.

\aaa{Continuity of convex functions with $p$-growth.}
Let $p\in(1,\infty)$ and $M>0$. 
\begin{enumerate}
    \item [(a)]
    Let $g\colon\R\to[0,\infty)$ be convex and satisfy
    $g(x)\leq M(1+|x|^p)$ for all $x\in\R$.
    Show that there exists a constant $C>0$ such that
    \[
    |g(x)-g(y)|\leq C|x-y|(1+|x|^{p-1}+|y|^{p-1})
    \quad\text{ for all }x,y\in\R.
    \]
    \item[(b)]
    Let $f\colon\R^{m\times d}\to[0,\infty)$ be rank-one convex and satisfy
    $f(A)\leq M(1+|A|^p)$ for all $A\in\R^{m\times d}$.
    Show that there exists a constant $C>0$ such that
    \[
    |f(A)-f(B)|\leq C|A-B|(1+|A|^{p-1}+|B|^{p-1})
    \quad\text{ for all }A,B\in\R^{m\times d}.
    \]
\end{enumerate}


\aaa{Sufficient condition for weak lower semicontinuity.}
Decide how to choose $q\in[1,\infty)$ such that
the following generalization of Theorem 3.44 holds:
\\
Let $\Omega\subset\R^d$ be a bounded domain with Lipschitz boundary,
and let  $f:\Omega\times \R^m\times \R^{m\times d} \to \R$
be a Carath\'eodory function
such that $A\mapsto f(x,u,A)$ is convex for a.a.~$x\in\Omega$ and all $u\in\R^m$.
Let $\gamma\in\rmL^1(\Omega)$ and $c\in\R$ such that
\[
f(x,u,A)\geq \gamma(x)+c|u|^q
\]
for a.a.~$x\in\Omega$ and all $(u,A)\in\R^d\times\R^{m\times d}$.
Then the functional 
\[
I\colon \rmW^{1,p}(\Omega;\R^m)\to\R_\infty, \quad
I(u)=\int_\Omega f(x,u(x),\nabla u(x))\,\dd x
\]
is weakly lower semicontinuous.

\aaa{Lavrentiev gap phenomenon.} This phenomenon, discovered in 1926 by Mikhail Lavrentiev, 
refers to the situation when the infimum of a functional over different spaces
that are ``close to each other'' may 
indeed be different, i.e., for a functional $I:X\to \R_{\infty}$ 
and a dense embedding $\wt X\hookrightarrow X$ it may be the case that
\[
\inf_{u\in X} I(u) < \inf_{\wt u \in \wt X} I(\wt u).
\]
\begin{itemize}
    \item[(a)] Let $f:\Omega\times \R^m\times \R^{m\times d} \to \R$ be a Carath\'eodory function
    satisfying the growth condition
    \[
        |f(x,u,A)| \leq C(1+|u|^p +|A|^p),
    \]
    for some constant $C>0$ and $p\in[1,\infty[$. Show that the Lavrentiev phenomenon does NOT appear, i.e.\
    \[\inf_{u\in \rmW^{1,p}(\Omega;\R^m)} I(u) = \inf_{\wt u\in \rmC^\infty(\ol\Omega;\R^m)} I(\wt u)\]
    \item[(b)] (Example by Mani\`a, 1934) Consider the functional
    \[
        I(u) = \int_0^1 (u(x)^3-x)^2(u'(x))^6\dd x
        %\quad\text{subject to}\quad u(0)=0,~u(1) =1
    \]
    for $u$ either in 
    \[
    X=\{ v\in \rmW^{1,1}(0,1)\,|\,v(0)=0,~v(1) =1\}
    \]
    or in
    \[
    \wt X=\{ v\in \rmW^{1,\infty}(0,1)\,|\,v(0)=0,~v(1) =1\}.
    \]  
    Show that the infima over both spaces are different, i.e., 
    the Lavrentiev phenomena appears. Proceed as follows:
    \begin{enumerate}
        \item Show that $u_*(x) = x^{1/3}$ is a minimizer in $X$ but no element of $\wt X$.
        \item Using Exercise 26, show that for each $u\in \wt X$ there exists $x_*\in\left]0,1\right[$ such that $u(x)\leq x^{1/3}/2$ for $x\in [0,x_*]$ and such that $u(x_*)= x_*^{1/3}/2$.
        \item Conclude that $I(u)\geq \frac{7^2}{8^2}\int_0^{x_*}x^2(u'(x))^6\dd x$
        \item Use H\"older's inequality with some suitable exponent $p$ to relate the previous estimate to $\int_0^{x_*} u'(x)\dd x = x_*^{1/3}/2$ and confirm the Lavrentiev phenomenon.
    \end{enumerate}
\end{itemize}

\end{document}
