\documentclass[a4paper,12pt]{article}
\usepackage{amsmath,amssymb,amsthm}
\usepackage[width=16cm]{geometry}

\newcommand{\R}{\mathbb{R}}
\newcommand{\eps}{\varepsilon}
\newcommand{\dd}{\,\mathrm{d}}
\newcommand{\wh}{\widehat}
\newcommand{\wt}{\widetilde}

\newtheorem{theorem}{Theorem}[section]
\newtheorem{remark}[theorem]{Remark}


\begin{document}
\setcounter{section}{5}
\setcounter{subsection}{2}
\subsection{Dimension reduction via $\boldsymbol{\Gamma}$-convergence}
In the last example, we discuss the derivation of an effective lower-dimensional model for a thin structure.

Let $\omega\subset \R^{d-1}$ denote an open and bounded domain with Lipschitz boundary. For a small parameter $\eps>0$, we introduce the \emph{thin}
domain $\Omega_\eps := \omega\times \left]0,\eps\right[$ and consider a \emph{strictly} convex function $f:\R^{d} \to \R$. We assume that $f$ has quadratic growth, i.e., there exist constants $c_1,c_2,c_3>0$
such that
\[
\forall A\in\R^d:\quad c_1|A|^2 -c_2\leq f(A) \leq c_3(1+|A|^2).
\]
(The quadratic growth is not essential and can be replaced with more general $p$-growth for $1<p<\infty$.)
On the space $H^1(\Omega_\eps)$ ($m=1$), we introduce the family of functionals
\[
I_\eps(u) = \int_{\Omega_\eps}f(\nabla u)\dd x.
\]
We want to answer the question, whether the sequence $I_\eps$ defined on the $d$-dimensional domain $\Omega_\eps$ has a $\Gamma$-limit given on the lower-dimensional domain $\omega$.

To this end, we first have to make the dependence on the parameter $\eps>0$ in the domain of integration more explicit by performing a change of variables: Let $x' := (x_1,\ldots, x_{d-1})\in \omega$
denote the \emph{horizontal} coordinates. We introduce $y \in\Omega_1$ via $y = (y',y_d) = (x', x_d/\eps)$ such that $y_d\in\left]0,1\right[$. With a function $u:\Omega_\eps\to \R$, 
we associate $\wt u:\Omega_1\to \R$ via $u(x',x_d) = \wt u(x',x_d/\eps)$. We obtain the following relation for the gradients of $u$ and $\wt u$
\begin{equation}
\nabla_x u (x) = \begin{pmatrix}
\nabla_{y'} \wt u(y)\\ \frac{1}{\eps}\partial_{y_d}\wt u(y),
\end{pmatrix}
\end{equation}
where $\nabla_{y'} \wt u\in\R^{d-1}$ denotes the \emph{horizontal} gradient. Inserting the transformation into the functionals $I_\eps$
leads to
\[
I_\eps(u) = \int_{\Omega_\eps}f(\nabla u)\dd x = \eps\int_{\Omega_1}f\Big(\nabla_{y'}\wt u,\frac{1}{\eps}\partial_{y_d}\wt u\Big)\dd y=:\eps \wt I_\eps(\wt u).
\]
Note that we get the factor $\eps$ due to the substitution in the integral $\int_{\Omega_\eps} f\dd x= \eps \int_{\Omega_1}\wt f \dd y$.
 The crucial point is that the growth assumption on $f$ leads to
 the estimate
 \begin{equation}\label{eq:lowerBound}
     \wt I_\eps(\wt u) \geq c_1\Big(\|\nabla_{y'}\wt u\|^2_{L^2(\Omega_1)}+\frac{1}{\eps^2}\|\partial_{y_d}\wt u\|^2_{L^2(\Omega_1)}\Big) -c_2|\Omega_1|\qquad\forall \wt u\in H^1(\Omega_1)
 \end{equation}
 In particular, if $\wt u_\eps $ is a sequence such that $\wt u_\eps\rightharpoonup \wt u$ in $H^1(\Omega_1)$ as $\eps\to0$
 \emph{and}  $\sup_{\eps >0}\wt I_\eps(\wt u_\eps) < \infty$ we get that $\partial_{y_d}\wt u_\eps \to \partial_{y_d} \wt u = 0$ strongly in $L^2(\Omega_1)$. In  this case, we can identify $\wt u \in H^1(\Omega_1)$ with a function $w \in H^1(\omega)$ such that $\wt u(y',y_d) = w(y')$. In this sense, we have \emph{reduced the dimension} from $d$ to $d{-}1$ dimensions.

By the above discussion, the limit of $\wt I_\eps(\wt u_\eps)$ can be finite only for limits $\wt u = \lim_{\eps\to 0}\wt u_\eps$ satisfying $\partial_{y_d} \wt u = 0$. Thus, we expect that the 
$\Gamma$-limit of $\wt I_\eps$ is obtained by minimizing the effect of the derivative in this direction: We define the function
$g:\R^{d-1}\to \R$ via the minimization problem
\[
g(A') = \min\{f(A',b)\,|\, b\in \R\}.
\]
Note that $g$ satisfies the same quadratic growth. Moreover, since $f$ is strictly convex has quadratic growth, a unique minimizer
$b=b(A')\in R$ exists for $A'\in\R^{d-1}$. We check that $g$ is also strictly convex: Indeed, for $A_0',A_1'\in\R^{d-1}$ and $\theta \in [0,1]$ define $A'_\theta = (1{-}\theta)A'_0 +\theta A_1'$ and consider the unique minimizers $b_0 = b(A'_
0)$ and $b_1 = b(A'_
1)$ in the definition of $g(A_0')$ and $g(A_1')$, respectively. 
Let us set $b_\theta = (1{-}\theta)b_0 +\theta b$, then, we have that
\[
g(A_\theta') \leq f(A'_\theta, b_\theta) < (1{-}\theta)f(A_0',b_0) +\theta f(A_1',b_1) = (1{-}\theta)g(A_0') +\theta g(A_1').
\]


\setcounter{theorem}{10}
 \begin{theorem}The family of (transformed) functionals $\wt I_\eps$
 $\Gamma$-converges with respect to the weak topology on $H^1(\Omega_1)$ to the functional
 \[
 \wt I_0 (\wt u ) = \begin{cases}\displaystyle
 \int_{\Omega_1} g(\nabla_{y'} \wt u) \dd y &\text{if }\partial_{y_d}\wt u = 0\text{ a.e.\ in }\Omega_1\\
 +\infty&\text{otherwise}.
 \end{cases}
 \]
 \end{theorem}

\begin{proof} 
\emph{1. $\liminf$-estimate. }Let us consider a sequence 
$\wt u_\eps \to \wt u $ in $H^1(\Omega_1)$. We can assume that 
$\partial_{y_d}\wt u = 0$, otherwise the estimate is trivial. Indeed, if $\partial_{y_d}\wt u \neq 0$, the estimate in \eqref{eq:lowerBound} yields that  $\liminf_{\eps \to 0} \wt I_\eps(\wt u_\eps) = +\infty$. The construction of $g$ gives the estimate
\[
I_\eps(\wt u_\eps)  = \int_{\Omega_1}f\Big(\nabla_{y'}\wt u,\frac{1}{\eps}\partial_{y_d}\wt u\Big)\dd y \geq  \int_{\Omega_1} g(\nabla_{y'} \wt u_\eps)\dd y.
\]
Since $g$ is convex, the right-hand side defines a weakly lower semicontinuous functional on $H^1(\Omega_1)$, and we can pass to the limit $\eps\to 0$ to find
\[
\liminf_{\eps\to 0 }I_\eps(\wt u_\eps) \geq  \int_{\Omega_1} g(\nabla_{y'} \wt u)\dd y = \wt I_0(\wt u).
\]
\emph{2. Recovery sequence. }It is sufficient to construct a recovery sequence for limits $\wt u$ with $\partial_{y_d} \wt u = 0$.
Otherwise, the recovery sequence can be chosen as the constant sequence $\wt u _\eps \equiv  \wt u$ such that $\wt I_{\eps} (\wt u_\eps) \to \infty = \wt I_0(\wt u)$ by \eqref{eq:lowerBound}.

Hence, let $\wt u(y',y_d) = w(y')$ with $w\in H^1(\omega)$ be given and let $b=b(y')\in L^2(\omega)$ denote the unique minimizer
for $A' = \nabla_{y'} \wt u$ in the definition of $g$ (note that $f$ has quadratic growth). A recovery sequence has to satisfy
\[
\nabla_{y'} \wt u_\eps = \nabla_{y'}\wt u + o(1)\quad\text{and}
\quad \partial_{y_d}\wt u_\eps = \eps b\quad\text{as }\eps\to 0.
\]
Thus, a natural candidate would be $\wt u_\eps(y',y_d) = 
\wt u(y') + \eps y_d  b(y')$ (affine with respect to $y_d$).
Unfortunately, $b\notin H^1(\omega)$ such that $\wt u_\eps\notin H^1(\Omega_1)$. Thus, we have to use in addition an approximation argument and consider for $\delta>0$ an approximation $b_\delta\in H ^1(\omega)$ such that $\|b-b_
\delta\|_{L^2(\omega)} \leq \delta$. We then define 
$\wt u_\eps^\delta(y',y_d) = 
\wt u(y') + \eps y_d  b_\delta(y')$ with
\[
\nabla_{y'}\wt u_\eps^\delta = 
\nabla_{y'}\wt u + \eps y_d \nabla_{y'} b_\delta\quad\text{and}\quad
\partial_{y_d}\wt u_\eps = \eps b_\delta.
\]
We compute
\begin{align*}
    \wt I_\eps(\wt u_\eps^\delta) &= \int_{\Omega} f\big(\nabla_{y'}\wt u + \eps y_d \nabla_{y'} b_\delta, b_\delta\big)\dd y\\
    & = \int_{\Omega} g(\nabla_{y'}\wt u ) +\big\{f(\nabla_{y'}\wt u + \eps y_d \nabla_{y'} b_\delta, b_\delta) - f(\nabla_{y'}\wt u, b)\big\}\dd y\\
    &=\wt I_0(\wt u ) + \wt{\mathcal{R}}_\eps^\delta.
\end{align*}
It remains to show that the remainder is arbitrarily small. We use the local Lipschitz continuity of $f$ (comp. Sheet 8, Exercise 27), namely, there exists $K>0$ such that
\[
|f(A_1',b_1') - f(A'_2,b_2')|
\leq K\big(1+|A_1'| + |A_2'| + |b_1| + |b_2|\big)
(|A_1'{-}A_2'| + |b_1{-}b_2|).
\]
With this estimate and Hölder's inequality, we get
\begin{align*}
\mathcal{R}_\eps^\delta
&\leq K\int_{\Omega_1}(1+|\nabla_{y'}\wt u|+\eps|y_d\nabla_{y'}b_\delta| + |b| + |b_\delta|) \big(
\eps|y_d\nabla_{y'} b_\delta| +|b{-}b_\delta| \big)\dd y\\
&\leq \wt C(1+
\|\nabla_{y'} \wt u\|_{L^2(\Omega_1)} + 
\eps \|y_d\nabla_{y'} b_\delta\|_{} +
\|b_\delta\|_{L^2(\Omega_1)}+ \|b\|_{L^2(\Omega_1)})\\&
\qquad\times\big(\eps\|y_d\nabla_{y'} b_\delta\|_{L^2(\Omega_1)}
+\|b{-}b_\delta\|_{L^2(\Omega_1)}\big).
\end{align*}
We choose $\delta = \delta(\eps)$ such that 
$\delta(\eps) \to 0$ as $\eps\to 0$ and $\|y_d\nabla_{y'} b_{\delta(\eps)}\| \leq \frac{C} {\sqrt{\eps}}$. Thus, we conclude
that $\mathcal{R}_\eps^{\delta(\eps)} \leq C(1+\sqrt{\eps})(\sqrt{\eps} + \delta(\eps))$, which shows that $\wt u_\eps^{\delta(\eps)}$ is
indeed a recovery sequence.
\end{proof}

 \begin{remark}
 The fact that $g$ is still convex does not transfer to the poly- or quasi-convex case.
 \end{remark}
\end{document}