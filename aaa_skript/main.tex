% Allgemeines Layout
\documentclass[a4paper, 12pt]{book}
\usepackage{geometry}
\geometry{top=30mm, left=25mm, right=25mm, bottom=30mm, headheight=15mm, footskip=15mm}
\usepackage[english]{babel}
\usepackage{csquotes}
% Gestaltung der Kopf- und Fußzeilen
\usepackage{fancyhdr}
\pagestyle{fancy}
\fancyhead[L]{}
\fancyhead[R]{\nouppercase{\rightmark}}
\fancyfoot[C]{\thepage}

% Gestaltung des Inhaltsverzeichnisses
\usepackage{tocloft}
\setlength{\cftchapnumwidth}{2.5em}
\setlength{\cftsecnumwidth}{2.5em}

% Gestaltung von Überschriften
\usepackage{titlesec}
% \titleformat{\chapter}[block]{\Huge\bfseries}{\center{Chapter \thechapter}\\[1ex]}{0pt}{\normalfont\centering\huge\sffamily\bfseries}
% \titleformat{\section}{\normalfont\Large\sffamily\bfseries}{\thesection.}{1em}{}
% \titleformat{\subsection}{\normalfont\large\sffamily\bfseries}{\thesubsection.}{1em}{}
% \titleformat{\paragraph}{\center\large\sffamily\bfseries}{\theparagraph}{1em}{}

% Weitere Pakete
\usepackage[colorlinks, linkcolor = black]{hyperref} % Referenzen, Links
\usepackage{amssymb} % Mathematische Symbole
\usepackage{amsmath} % Mathematische Symbole
\usepackage{amsthm}
\usepackage{stmaryrd} % Für align-Umgebung
\usepackage{mathtools} % \xrightharpoonup
\usepackage{esint} % \fint
\usepackage{tikz} % Tikz ist kein Zeichenprogramm
\usetikzlibrary{decorations.pathreplacing} % Geschweifte Klammern (genutzt in Beispiel 1.2.3)
\usetikzlibrary{hobby,patterns.meta}
\usepackage[mathcal]{euscript} % Schönere Kalligraphen für \mathcal

\usepackage{biblatex}
\addbibresource{references.bib}
% Hilfsbefehle
\newcommand{\longhookrightarrow}{\lhook\joinrel\longrightarrow}
\newcommand{\clonghookrightarrow}{\overset{c}{\longhookrightarrow}}

\newcommand{\vol}[1]{\operatorname{vol}\mathopen{}\left(#1\right)\mathclose{}}
\newcommand{\diam}[1]{\operatorname{diam}\mathopen{}\left(#1\right)\mathclose{}}
\newcommand{\interior}[1]{\operatorname{int}\mathopen{}\left(#1\right)\mathclose{}}
\newcommand{\Lin}[2]{\operatorname{Lin}\mathopen{}\left(#1,#2\right)\mathclose{}}

\newcommand{\divergence}[1]{\operatorname{div}\mathopen{}\left(#1\right)\mathclose{}}
\newcommand{\supp}[1]{\operatorname{supp}\mathopen{}\left(#1\right)\mathclose{}}
\newcommand{\epi}[1]{\operatorname{epi}\mathopen{}\left(#1\right)\mathclose{}}
\DeclareMathOperator*{\esssup}{ess\,sup}

\newcommand{\sgn}[1]{\operatorname{sgn}\mathopen{}\left(#1\right)\mathclose{}}
\newcommand{\SO}[1]{\operatorname{SO}\mathopen{}\left(#1\right)\mathclose{}}

\newcommand{\tr}[1]{\operatorname{tr}\mathopen{}\left(#1\right)\mathclose{}}
\newcommand{\Sym}[1]{\operatorname{Sym}\mathopen{}\left(#1\right)\mathclose{}}
\newcommand{\Cof}[1]{\operatorname{Cof}\mathopen{}\left(#1\right)\mathclose{}}
\newcommand{\Rank}[1]{\operatorname{Rank}\mathopen{}\left(#1\right)\mathclose{}}


\newtheorem{theorem}{Theorem}[section]

\newtheorem{definition}[theorem]{Definition}
\newtheorem{example}[theorem]{Example}
\newtheorem{lemma}[theorem]{Lemma}
\newtheorem{remark}[theorem]{Remark}
\newtheorem{notation}[theorem]{Notation}
\newtheorem{corollary}[theorem]{corollary}


% ========== Dokument ==========
\begin{document}
    \input{mcov_titelseite}
    \tableofcontents

    \chapter{Introduction}
    \section{Minimization Problems}
The general topic in multidimensional calculus of variations is the question about minimizing problems. In the very basic setting we have a topological space $X$, some subset $M\subset X$ and a functional $I:M\longrightarrow\overline{\mathbb{R}}=\mathbb{R}\cup\{\pm\infty\}$. Typically, $X$ is a subset of an infinite-dimensional space, and $I$ is nonlinear in general. Now the task is to find a minimizer of $I$, i.e. find an element $u_*\in M$ such that $I(u_*)\leq I(u)$ for all $u\in M$.\\

Often we have that $X$ is a function space, e.g. the space of continuously differentiable functions $C^1(\Omega,\mathbb{R}^m)$ or some Sobolev space, $W^{1,p}(\Omega,\mathbb{R}^m)$ for instance, where $\Omega\subset\mathbb{R}^d$ is supposed to be a bounded domain. The functional is very often of integral type, i.e.
\[
    I(u)=\int_\Omega{f(x,u(x),\nabla u(x))\mathrm{d}x}
\]
for some function $f:\Omega\times\mathbb{R}^m\times\mathbb{R}^{m\times d}\longrightarrow\mathbb{R}$. This is the classical framework, but one can also consider higher-order derivatives of $u$ or even more general $f$. In examples from physics, $I$ often represents energy, and then $f$ is also called \textit{energy density}.
    \section{Examples}
\begin{example}
(Dido's problem / Isoperimetric problem)\\
The task is to maximize the area enclosed by a curve of prescribed length. More precisely, an interval $[a,b]\subset\mathbb{R}$ is given and we consider functions $u:[a,b]\longrightarrow\mathbb{R}$ which have length $L_0>0$. We search for some function $u_*:[a,b]\longrightarrow\mathbb{R}$ such that the area under this curve is maximal. See \hyperref[fig:example_1_2_1]{Figure I.1} for an illustration. Mathematically, this can be formulated as follows: Find $u\in C_0^1([a,b])=\{u\in C^1([a,b])\mid u(a)=u(b)=0\}$ such that the parameter
\[L(u)=\int_a^b{\sqrt{1+(u'(x))^2}\mathrm{d}x}=L_0>0\]
is fixed, and the area
\[I(u)=\int_a^b{u(x)\mathrm{d}x}\]
is maximal. This fits into our general setting we have described in the previous section by looking at $-I$ instead of $I$.\\[11pt]

\begin{figure}
	\centering
	\begin{tikzpicture}
		% Hintergrund
		\fill[cyan!20] (-0.5, -0.8) rectangle (7.8, 3.5);
		\draw[thin, cyan] (-0.5, -0.8) grid (7.8, 3.5);

		% Funktion
		\fill[red, opacity=0.2] (8, 0) -- (0, 0) plot[smooth] coordinates {(0, 0) (1.5, 3) (3, 1.5) (4.5, 2.5) (7, 0)};
		\draw[thick, red] plot[smooth] coordinates {(0, 0) (1.5, 3) (3, 1.5) (4.5, 2.5) (7, 0)};
		\node[red] at (6.5, 1.3) {$u(x)$};

		% Achsen
		\draw[very thick, ->] (-0.2, 0) -- (7.5, 0);
		\draw[very thick, ->] (0, -0.2) -- (0, 3);
		\draw[thin] (0, 0) -- (0, -0.2) node[fill=cyan!20, below] {$a$};
		\draw[thin] (7, 0) -- (7, -0.2) node[fill=cyan!20, below] {$b$};

		% Rahmen
		\draw[very thick] (-0.5, -0.8) rectangle (7.8, 3.5);
	\end{tikzpicture}
	\caption{Example of a curve for Dido's problem.}
	\label{fig:example_1_2_1}
\end{figure}
\end{example}

\begin{example}
(Brachistochrone problem / Shortest time problem)\\
This was studied mathematically first by \textsc{Johann Bernoulli} in 1696. This was the example where calculus of variations more or less was invented. As in \hyperref[fig:example_1_2_2]{Figure I.2} illustrated, a mass $m>0$ is given and it should slide from point $(a,A)$ to $(b,B)$ under the influence of gravity.\\

\begin{figure}[ht]
	\centering
	\begin{tikzpicture}
		% Hintergrund
		\fill[cyan!20] (-0.8, -0.8) rectangle (7.8, 3.5);
		\draw[thin, cyan] (-0.8, -0.8) grid (7.8, 3.5);

		% Funktion
		\node[red, fill=cyan!20] at (3.1, 1.1) {$u(x)$};
		\draw[thick, red] plot[smooth, domain=0:7] (\x, {0.7+(2-0.3*(\x)^2)*exp(-0.07*(\x)^2)});

		% Achsen
		\draw[very thick, ->] (-0.2, 0) -- (7.5, 0);
		\draw[very thick, ->] (0, -0.2) -- (0, 3);
		\draw[thin] (0, 2.7) -- (-0.2, 2.7) node[fill=cyan!20, left] {$A$};
		\draw[thin] (0, 0.2887) -- (-0.2, 0.2887) node[fill=cyan!20, left] {$B$};
		\draw[thin] (0, 0) -- (0, -0.2) node[fill=cyan!20, below] {$a$};
		\draw[thin] (7, 0) -- (7, -0.2) node[fill=cyan!20, below] {$b$};
		\fill[blue] (0, 2.7) circle (2pt);
		\node[blue] at (0.4, 2.8) {$m$};

		% Rahmen
		\draw[very thick] (-0.8, -0.8) rectangle (7.8, 3.5);
	\end{tikzpicture}
	\caption{Example of a slide.}
	\label{fig:example_1_2_2}
\end{figure}

The question is how to build the slide such that the time on the slide is short. We need some physical laws to derive our formulas. Let $u:[a,b]\longrightarrow\mathbb{R}$ be the function describing our slide. At $x\in[a,b]$ the position is given by $p(x)=\begin{pmatrix}x\\u(x)\end{pmatrix}$ and the velocity at position $p$ is
\[v(x)=\frac{\mathrm{d}}{\mathrm{d}t}p(x)=\begin{pmatrix}\dot x\\u'(x)\cdot\dot x\end{pmatrix},\]
where $\dot x=\frac{\mathrm{d}}{\mathrm{d}t}x$. The kinetic and potential energy of the mass is given by
\begin{align*}
	E_{\text{kin}}&=\frac{1}{2}\cdot m\cdot\lvert v\rvert^2=\frac{1}{2}\cdot m\cdot\dot x^2\cdot(1+(u'(x))^2),\\
	E_{\text{pot}}&=m\cdot g\cdot h=m\cdot g\cdot u(x),
\end{align*}
where $g$ denotes the gravitational acceleration. Both energies are transformed into each other, i.e. they are conserved. This means it holds
\[m\cdot g\cdot A=m\cdot g\cdot u(0)=m\cdot g\cdot u(x)+\frac{1}{2}\cdot m\cdot\dot x^2\cdot(1+(u'(x))^2).\]
After simplification and rearranging terms we obtain
\[\dot x^2=\frac{2g(A-u(x))}{1+(u'(x))^2}.\]
From this formula we can derive the sliding time. Via separation of variables we obtain
\[T(u)=\int_0^{T(u)}{1\mathrm{d}t}=\int_{x(0)}^{x(T(u))}{\frac{1}{\dot x}\mathrm{d}x}=\int_a^b{\sqrt{\frac{1+(u'(x))^2}{2g(A-u(x))}}\mathrm{d}x}.\]
The task now is to find a $C^1$-function $u$ with $u(a)=A$ and $u(b)=B$ such that $T(u)$ is minimal. This problem also falls into our class. Until now, we cannot solve this. The solution is given by a cycloid.
\begin{figure}[ht]
	\centering
	\begin{tikzpicture}
		% Hintergrund
		\fill[cyan!20] (-0.8, -0.8) rectangle (7.8, 3.5);
		\draw[thin, cyan] (-0.8, -0.8) grid (7.8, 3.5);

		% Funktion
		\node[red, fill=cyan!20] at (1.7, 1.1) {$u(x)$};
		\draw[thick, red] plot[smooth, domain=0:4.01133] ({1.46582*(\x-sin(180*\x/pi))}, {1.46582*(cos(180*\x/pi)-1)+2.7});

		% Achsen
		\draw[very thick, ->] (-0.2, 0) -- (7.5, 0);
		\draw[very thick, ->] (0, -0.2) -- (0, 3);
		\draw[thin] (0, 2.7) -- (-0.2, 2.7) node[fill=cyan!20, left] {$A$};
		\draw[thin] (0, 0.2887) -- (-0.2, 0.2887) node[fill=cyan!20, left] {$B$};
		\draw[thin] (0, 0) -- (0, -0.2) node[fill=cyan!20, below] {$a$};
		\draw[thin] (7, 0) -- (7, -0.2) node[fill=cyan!20, below] {$b$};
		\fill[blue] (0, 2.7) circle (2pt);
		\node[blue] at (0.4, 2.8) {$m$};

		% Rahmen
		\draw[very thick] (-0.8, -0.8) rectangle (7.8, 3.5);
	\end{tikzpicture}
	\caption{Solution of Brachistochrone problem.}
\end{figure}\\
\end{example}
\begin{example}
(Minimal surface / Shape of a soap film)
\begin{itemize}
	\item[(a)] Let $\Omega\subset\mathbb{R}^2$ and $g\in C^0(\partial\Omega)$ be given. Find $u\in C^0(\overline{\Omega})\cap C^1(\Omega)$ that minimizes the area
	\[I(u)=\int_\Omega{\sqrt{1+\lvert\nabla u(x)\rvert^2}\mathrm{d}x}\]
	and satisfies $u|_{\partial\Omega}=g$. This problem is illustrated in \hyperref[fig:example_1_2_3_a]{Figure I.4}.

	\begin{figure}[ht]
		\centering
		\begin{tikzpicture}
			% Ellipse
			\filldraw[fill=red, fill opacity=0.2, draw=blue] ellipse (2 and 0.8);

			% Funktion
			\filldraw[fill=blue, fill opacity=0.1, draw=violet] plot[smooth, domain=0:360, samples=100] ({2*cos(\x)}, {2+0.2*cos(5*\x)+0.8*sin(\x)});

			\fill[blue, opacity=0.1] ellipse (2 and 0.8);
			\fill[blue, opacity=0.2] (2, 0) arc (0:180:2 and 0.8) -- (-2, 1.8) plot[smooth, domain=180:360] ({2*cos(\x)}, {2+0.2*cos(5*\x)+0.8*sin(\x)}) -- (2, 0);

			% Beschriftungen
			\node[red] at (0, 0) {$\Omega$};
			\node[blue] at (2, -0.6) {$\partial\Omega$};
			\node[violet] at (2.2, 2.2) {$g$};
		\end{tikzpicture}
		\caption{Setting of minimal surface problem.}
		\label{fig:example_1_2_3_a}
	\end{figure}
	\item[(b)] There is a variant of the problem. Let $R_0,R_L>0$ be given. The goal is to find a minimal surface of the revolution.\\

	\begin{center}
		\begin{tikzpicture}
			% Rotationsvolumen Flächen einfärben
			\fill[red, opacity=0.2] (7, 0) ellipse (0.2 and 1);
			\fill[red, opacity=0.3] (0, 0) ellipse (0.15 and 0.78125);
			\fill[red, opacity=0.4] plot[smooth, domain=0:7] (\x, {(\x-3)^2/32+0.5}) arc (90:270:0.2 and 1) plot[smooth, domain=7:0] (\x, {-(\x-3)^2/32-0.5}) arc (-90:90:0.15 and 0.78125);

			% Rotationsvolumen Linien
			\draw[thick, red] plot[smooth, domain=0:7] (\x, {(\x-3)^2/32+0.5});
			\draw[thick, red] plot[smooth, domain=0:7] (\x, {-(\x-3)^2/32-0.5});
			\draw[thick, red] (7, 1) arc (90:-90:0.2 and 1);
			\draw[thick, red, dashed] (0, 0.78125) arc (90:-90:0.15 and 0.78125);

			% Achse und Radien benennen
			\draw[thick, ->] (-0.5, 0) -- (7.5, 0);
			\draw[thin] (0, 0.1) -- (0, -0.1) node[below] {0};
			\draw[thin] (7, 0.1) -- (7, -0.1) node[below] {$L$};
			\draw[decorate,decoration={brace}] (-0.6, 0) -- (-0.6, 0.78125);
			\node at (-1, 0.39) {$R_0$};
			\draw[decorate,decoration={brace}] (7.6, 1) -- (7.6, 0);
			\node at (8, 0.5) {$R_L$};

			% Rotationsvolumen elliptische Linien im Vordergrund
			\draw[thick, red] (7, 1) arc (90:270:0.2 and 1);
			\draw[thick, red] (0, 0.78125) arc (90:270:0.15 and 0.78125);
		\end{tikzpicture}
	\end{center}

	Mathematically this means: Find $u\in\{u\in C^1([0,L])\mid u(0)=R_0,u(L)=R_L\}$ that minimizes the area
	\[I(u)=\int_0^L{2\pi u(x)\sqrt{1+(u'(x))^2}\mathrm{d}x}.\]\\
\end{itemize}
\end{example}

\begin{example}
(Ground states in quantum mechanics)\\
We consider a given wave function $\Psi\in L^2(\mathbb{R}^{3N},\mathbb{C})$ that describes the state of $N$ particles. Assume that
\[\mathbb{R}^{3N}\longrightarrow[0,\infty),\qquad(x_1,\dotsc,x_N)\longmapsto\lvert\Psi(x_1,\dotsc,x_N)\rvert^2\]
is a probability density for ``particle $j$ is at place $x_j$''. Moreover, a potential $V:\mathbb{R}^{3N}\longrightarrow[0,\infty)$ describing the particle interaction should be given. With these sizes consider the Hamilton operator $H(\Psi)=-\Delta\Psi+V\cdot\Psi$ (here dimensionless) and energy functional
\[E(\Psi)=\frac{1}{2}\langle\Psi\mid H\mid\Psi\rangle=\frac{1}{2}\langle\Psi,H(\Psi)\rangle_{L^2}=\frac{1}{2}\int_{\mathbb{R}^{3N}}{\lvert\nabla\Psi(x)\rvert^2+V(x)\lvert\Psi(x)\rvert^2\mathrm{d}x}.\]
The ground states are states with least energy. I.e., $\Psi_*$ is a ground state if it minimizes the functional $E$ under the constraint $\lVert\Psi\rVert_{L^2}=1$.
\end{example}
    \section{Two Approaches}
There are two approaches to solve minimization problems.
\begin{enumerate}
	\item The \textit{classical} approach, also known as the \textit{indirect method}. The idea is to determine and characterize critical points of the functional $I$ with the help of necessary and sufficient conditions (something like ``$I'(u_*)=0$'' and ``$I''(u_*)\geq0$''). For that we need a suitable notion of derivative. Directional derivatives of $I$ are called \textit{variations}.\\

	However, there is a problem with the method. The existence of critical points is not clear. We will see that checking necessary conditions very often leads to solving a partial differential equation. Many examples are related to physics where we expect the existence of minimizers from our intuition, but this is no mathematical argument.
	\item The \textit{modern} approach, also known as the \textit{direct method}. Here the idea is to give sufficient conditions for the existence of minimizers. The general idea follows the steps:
	\begin{enumerate}
		\item Show that $I$ is bounded from below, i.e. $\inf_{u\in M}{I(u)}>-\infty$. Then choose an infimizing sequence $(u_k)_{k\in\mathbb{N}}\subset M$ with $I(u_k)\to\inf_{u\in M}{I(u)}$.
		\item Show that there exists a subsequence $(u_{k_\ell})_{\ell\in\mathbb{N}}$ of $(u_k)_{k\in\mathbb{N}}$ which converges in some sense to some $u_*\in M$.
		\item Show $\liminf_{\ell\to\infty}{I(u_{k_\ell})}\geq I(u_*)$. Then $u_*$ is a minimizer of $I$.
	\end{enumerate}
	We are talking about convergence, but what is the right topology? Strong, weak or weak* convergence? Choosing the right one is the mathematical challenge here! Verifying step (b) is easier in courser topologies, while (c) is easier in finer topologies. To see this, let $\tau_1,\tau_2$ be topologies on $X$, where $\tau_1$ is finer than $\tau_2$. Then $u_k\overset{\tau_1}{\to}u$ implies $u_k\overset{\tau_2}{\to}u$. On the other hand, if $I(u_k)\to I(u)$ for $u_k\overset{\tau_2}{\to}u$ then it also holds $I(u_k)\to I(u)$ if $u_k\overset{\tau_1}{\to}u$.\\[11pt]
\end{enumerate}

\begin{example}
(By Weierstrass)\\
Consider $X:=C^1([-1,1])$, $M:=\{u\in X\mid u(-1)=-1\text{ and }u(1)=1\}$. We want to minimize the functional
\[I:X\longrightarrow\mathbb{R},\qquad I(u):=\int_{-1}^1{(x\cdot u'(x))^2\mathrm{d}x}\]
in $M$. We claim $\inf_{u\in M}{I(u)}=0$, but that $I$ has no minimizers in $M$. For the first assertion we define the sequence
\[u_k:[-1,1]\longrightarrow\mathbb{R},\qquad u_k(x):=\frac{\arctan(kx)}{\arctan(k)}.\]
The graphs are shown in \hyperref[fig:example_1_3_1]{Figure I.5}.

\begin{figure}[h]
	\centering
	\begin{tikzpicture}
		% Hintergrund
		\fill[cyan!20] (-3, -2.5) rectangle (3, 2.5);
		\draw[thin, cyan, step=1] (-3, -2.5) grid (3, 2.5);

		% Funktionen
		\draw[thick, red] plot[smooth, domain=-2:2] (\x, {2*atan(\x/2)/atan(1)});
		\draw[thick, red, dashed] plot[smooth, domain=-2:2] (\x, {2*atan(5*\x)/atan(10)});

		% Achsen
		\draw[very thick, ->] (-2.5, 0) -- (2.5, 0);
		\draw[very thick, ->] (0, -2.2) -- (0, 2.2);
		\draw[thin] (-2, 0.1) -- (-2, -0.1) node[below, fill=cyan!20] {$-1$};
		\draw[thin] (2, 0.1) -- (2, -0.1) node[below, fill=cyan!20] {$1$};
		\draw[thin] (-0.1, -2) -- (0.1, -2) node[right, fill=cyan!20] {$-1$};
		\draw[thin] (-0.1, 2) -- (0.1, 2) node[right, fill=cyan!20] {$1$};

		% Rahmen
		\draw[very thick] (-3, -2.5) rectangle (3, 2.5);
	\end{tikzpicture}
	\caption{Graph for $u_1$ and $u_{10}$ from Weierstrass example.}
	\label{fig:example_1_3_1}
\end{figure}

The idea behind is that $u'$ should be more or less zero, and hence $u$ should be approximately piecewise constant. We compute
\[I(u_k)=\int_{-1}^1{x^2\left(\frac{\frac{1}{1+(kx)^2}\cdot k}{\arctan(k)}\right)^2\mathrm{d}x}=\frac{1}{\arctan(k)^2}\int_{-k}^k{\underbrace{y^2\frac{1}{1+y^2}}_{\leq1}\frac{\mathrm{d}y}{k}}\leq\frac{2}{\arctan(k)^2}\to0\]
for $k\to\infty$. So $(u_k)_{k\in\mathbb{N}}$ is an infimizing sequene. But we also claimed that there are no minimizers in $M$. This is true because if $u\in M$ then $I(u)>0$ since $I(u)=0$ would lead to $u'=0$ which is a contradiction to $u(-1)=-1$ and $u(1)=1$.\\

One can show that if $I(u)\ll1$ then $u|_{[0,1]}\approx1$ and $u|_{[-1,0]}\approx-1$, so a candidate for a minimizer is
\[u_*:[-1,1]\longrightarrow\mathbb{R},\qquad u_*(x):=\sgn{x}.\]
We will indeed show in the exercise lesson that every infimizing sequence $(u_k)_{k\in\mathbb{N}}$ of $I$ converges uniformly on any compact subset of $[-1,0)\cup(0,1]$ to $u_*$, i.e. $u_k\to u_*$ in $L_\text{loc}^\infty([-1,0)\cup(0,1])$. But $u_*$ does not belong to our class of functions we are considering. So the mathematical question is to find a space $\widetilde{X}\supset X$ such that $u_*\in\widetilde{X}$.
\end{example}

\begin{example}
Define the space of piecewise $C^1$-functions, i.e.
\[PC^1([a,b]):=\left\{u\in C^0([a,b])\,\middle\vert\,\exists a=x_0<x_1<\dotsc<x_n=b\text{ with }u|_{[x_{j-1},x_j]}\in C^1([x_{j-1},x_j])\right\}.\]
Now consider $X:=PC^1([0,1])$, $M:=\{u\in PC^1([0,1])\mid u(0)=0,u(1)=0\}$ with the functional
\[I:X\longrightarrow\mathbb{R},\qquad I(u):=\int_0^1{(1-(u'(x))^2)^2+u(x)^2\mathrm{d}x}.\]
We claim that $\inf_{u\in M}{I(u)}=0$, and to see this we are going to choose a sequence $(u_n)_{n\in\mathbb{N}}\subseteq M$ with the properties $\lvert u_n\rvert\leq\frac{1}{2n}$ and $u_n'\in\{-1,1\}$ for all $n\in\mathbb{N}$. A possible choice for such functions is illustrated in \hyperref[fig:example_1_3_3]{Figure I.6}.

\begin{figure}[h]
	\centering
	\begin{tikzpicture}
		% Hintergrund
		\fill[cyan!20] (-1, -2) rectangle (8, 2);
		\draw[thin, cyan, step=1.5] (-1, -2) grid (8, 2);

		% Achsen
		\draw[very thick, ->] (0, -1.8) -- (0, 1.8);
		\draw[very thick, ->] (-0.5, 0) -- (7.5, 0);
		\draw[thin] (1.5, 0.1) -- (1.5, -0.1) node[below, fill=cyan!20] {$\frac{1}{2n}$};
		\draw[thin] (3, 0.1) -- (3, -0.1) node[below, fill=cyan!20] {$\frac{1}{n}$};
		\draw[thin] (6, 0.1) -- (6, -0.1) node[below, fill=cyan!20] {$\frac{2}{n}$};
		\draw[thin] (0.1, -1.5) -- (-0.1, -1.5) node[left, fill=cyan!20] {$\frac{1}{2n}$};
		\draw[thin] (0.1, 1.5) -- (-0.1, 1.5) node[left, fill=cyan!20] {$\frac{1}{2n}$};

		% Funktion
		\draw[thick, red] (0, 0) -- (1.5, 1.5) -- (4.5, -1.5) -- (6.5, 0.5);
		\draw[thick, red, dashed] (6.5, 0.5) -- (7.5, 1.5);

		% Rahmen
		\draw[very thick] (-1, -2) rectangle (8, 2);
	\end{tikzpicture}
	\caption{Example for a function $u_n$ from Young's example.}
	\label{fig:example_1_3_3}
\end{figure}
With such a sequence we have
\[I(u_n)=\int_0^1{(1-u'_n(x)^2)^2+u_n(x)^2\mathrm{d}x}\leq\frac{1}{4n^2}\to0\]
as $n\to\infty$. Hence, $\inf_{u\in M}{I(u)}\geq0$. Moreover, it holds $u_n\to u_*$ in $C^0([0,1])$ uniformly, where $u_*\equiv0$. But $I(u_*)=1\ne0$.
\end{example}

\begin{remark}
\begin{itemize}
	\item[(a)] In \hyperlink{example_1_3_1}{Example 1.3.1} we saw that there exists a function which minimizes $I$, even if it does not lie in $M$.
	\item[(b)] In \hyperlink{example_1_3_2}{Example 1.3.2} something strange did happen. We constructed an infimizing sequence which actually converges uniformly in $C^0$ to the zero function, which even lies in $M$. So everything looks nice. But the zero function does not minimize our functional. The problem is that $I$ is not continuous with respect to uniform convergence.
\end{itemize}
\end{remark}


    \chapter{Classical Methods}
\label{ch:classical_methods}
    \section{Setting and Notation}
For the whole chapter, consider integral functionals of the form
\[I:X\longrightarrow\mathbb{R},\qquad I(u):=\int_\Omega{f(x,u(x),\nabla u(x))\mathrm{d}x}+\int_{\partial\Omega}{g(x,u(x))\mathrm{d}a},\]
where the quantities are fixed as follows:
\begin{itemize}
	\item[(a)] For $d\geq1$, $\Omega\subset\mathbb{R}^d$ is a bounded domain (domain means open and simply connected) with piecewise smooth boundary $\partial\Omega$ (this means for example $C^1$-boundary or Lipschitz boundary so that in particular the outer normal on $\partial\Omega$ does exist).
	\item[(b)] Our space of functions is $X=C^1(\overline{\Omega};\mathbb{R}^m)$ for some $m\geq1$.
	\item[(c)] $f:\Omega\times\mathbb{R}^m\times\mathbb{R}^{m\times d}\longrightarrow\mathbb{R}$, $(x,u,A)\longmapsto f(x,u,A)$, called the \textit{volume density}.
	\item[(d)] $g:\partial\Omega\times\mathbb{R}^m\longrightarrow\mathbb{R}$, $(x,u)\longmapsto g(x,u)$, called the \textit{boundary density}.
\end{itemize}
If nothing else is mentioned, there are no further assumptions on $f$ and $g$.\\

Often, $u$ shall coincide with a given function $u_0$ on a part $\Gamma_D\subset\partial\Omega$ on the boundary, i.e. $u(x)=u_0(x)$ for all $x\in\Gamma_D$. We call $\Gamma_D$ the \textit{Dirichlet boundary} and $\Gamma_N:=\partial\Omega\setminus\Gamma_D$ the \textit{Neumann boundary}. Then set
\[M:=\{u\in X\mid u|_{\Gamma_D}=u_0|_{\Gamma_D}\}.\]
Defining $X_0:=\{u\in X\mid u|_{\Gamma_D}=0\}$, we see that $M=u_0+X_0$ which is an affine linear space.\\[11pt]

We also need to fix some useful notations for derivatives. For $u\in X$, $i\in\{1,\dotsc,m\}$ and $j\in\{1,\dotsc,d\}$ we will write
\[(\nabla u(x))_{ij}=\frac{\partial u_i}{\partial x_j}(x).\]
Let $x\in\Omega$, $u\in\mathbb{R}^m$ and $A\in\mathbb{R}^{m\times d}$. For partial derivatives of first and second order of $f$ (and $g$ analogously) we write
\begin{align*}
	\partial_xf&:=\left(\frac{\partial f}{\partial x_j}\right)_{j=1,\dotsc,d}=\left(\frac{\partial f}{\partial x_1},\dotsc,\frac{\partial f}{\partial x_d}\right)\in\mathbb{R}^d,\\
	\partial_uf&:=\left(\frac{\partial f}{\partial u_i}\right)_{i=1,\dotsc,m}=\left(\frac{\partial f}{\partial u_1},\dotsc,\frac{\partial f}{\partial u_m}\right)\in\mathbb{R}^m,\\
	\partial_Af&:=\left(\frac{\partial f}{\partial A_{ij}}\right)_{\substack{i=1,\dotsc,m\\j=1,\dotsc,d}}=\begin{pmatrix}
		\frac{\partial f}{\partial A_{11}}&\cdots&\frac{\partial f}{\partial A_{1d}}\\
		\vdots&\ddots&\vdots\\
		\frac{\partial f}{\partial A_{m1}}&\cdots&\frac{\partial f}{\partial A_{md}}
	\end{pmatrix}\in\mathbb{R}^{m\times d},\\
	\partial_u^2f&:=\left(\frac{\partial^2f}{\partial u_i\partial u_k}\right)_{i,k=1,\dotsc m}\in\mathbb{R}^{m\times m},\\
	\partial_A\partial_uf&:=\left(\frac{\partial^2f}{\partial A_{ij}\partial u_k}\right)_{\substack{i,k=1,\dotsc m\\j=1,\dotsc,d}}\in\mathbb{R}^{m\times d\times m},\\
	\partial_A^2f&:=\left(\frac{\partial^2f}{\partial A_{ij}\partial A_{k\ell}}\right)_{\substack{i,k=1,\dotsc m\\j,\ell=1,\dotsc,d}}\in\mathbb{R}^{m\times d\times m\times d}.
\end{align*}
Further let $v\in\mathbb{R}^d$, $B\in\mathbb{R}^{m\times d}$. For directional derivatives we write
\begin{align*}
	D_uf(x,u,A)[v]&=\partial_uf(x,u,A)\cdot v,\\
	D_Af(x,u,A)[B]&=\partial_Af(x,u,A):B,\\
	D_u^2f(x,u,A)[v]&=\sum_{i,k=1}^m{\frac{\partial^2 f(x,u,A)}{\partial u_i\partial u_k}v_iv_k},\\
	D_AD_uf(x,u,A)[v,B]&=\sum_{i,k=1}^m{\sum_{j=1}^d{\frac{\partial^2 f(x,u,A)}{\partial A_{ij}\partial u_k}B_{ij}v_k}},\\
	D_A^2f(x,u,A)[B]&=\sum_{i,k=1}^m{\sum_{j,\ell=1}^d{\frac{\partial^2f(x,u,A)}{\partial A_{ij}\partial A_{k\ell}}B_{ij}B_{k\ell}}}.
\end{align*}
For $D_uf$ and $D_Af$ we used the scalar product for vectors and matrices, respectively:
\begin{align*}
	u\cdot v&:=\sum_{i=1}^m{u_iv_i},\\
	A:B&:=\sum_{i=1}^m{\sum_{j=1}^d{A_{ij}B_{ij}}}=\tr{B^\top A}.
\end{align*}
The divergence of a matrix-valued function is taken row-wise: For $A(x)\in\mathbb{R}^{m\times d}$ we have $\divergence{A(x)}\in\mathbb{R}^m$ with
\[(\divergence{A(x)})_i:=\sum_{j=1}^d{\frac{\partial}{\partial x_j}A_{ij}(x)}.\]
    \section{The Euler-Lagrange Equations}
The idea is to study local behaviour of $I$ in $u\in M$, i.e. to study the function $t\longmapsto I(u+tv)$ with $t\in\mathbb{R}$ for $v\in X_0$.\\

\textbf{\underline{Definition 2.2.1}}\\
Let $u\in M$, $v\in X_0$ and $n\in\mathbb{N}$. If
\[D^nI(u)[v]:=\left.\frac{\mathrm{d}^n}{\mathrm{d}t^n}I(u+tv)\right\vert_{t=0}\]
exists, it is called the \textit{$n$-th variation of $I$ in $u$ in direction $v$}.\\[11pt]

\textbf{Remark 2.2.2}\\
The $n$-th variation is the same as the $n$-th directional derivative.\\[11pt]

\textbf{\underline{Definition 2.2.3}}\\
An element $u\in M$ is called \textit{critical point of $I$} if for all $v\in X_0$ it holds $DI(u)[v]=0$.\\[11pt]

\hypertarget{lemma_2_2_4}{\textbf{Lemma 2.2.4}}\\
(Existence of first and second variation)
\begin{itemize}
	\item[(a)] If $f\in C^1(\Omega\times\mathbb{R}^m\times\mathbb{R}^{m\times d})$ and $g\in C^1(\partial\Omega\times\mathbb{R}^m)$, then the first variation of $I$ exists and for all $u\in M$, all $v\in X_0$ we have the following
	\begin{align*}
		DI(u)[v]&=\int_\Omega{\partial_uf(x,u(x),\nabla u(x))\cdot v(x)+\partial_Af(x,u(x),\nabla u(x)):\nabla v(x)\mathrm{d}x}\\
		&\qquad\qquad+\int_{\partial\Omega}{\partial_ug(x,u(x))\cdot v(x)\mathrm{d}a}.
	\end{align*}
	\item[(b)] If $f\in C^2(\Omega\times\mathbb{R}^m\times\mathbb{R}^{m\times d})$ and $g\in C^2(\partial\Omega\times\mathbb{R}^m)$, then also the second variation of $I$ exists and for all $u\in M$, $v\in X_0$ it holds
	\begin{align*}
		D^2I(u)[v]&=\int_\Omega{D_u^2f(x,u(x),\nabla u(x))[v(x)]+2D_AD_uf(x,u(x),\nabla u(x))[v(x),\nabla v(x)]\mathrm{d}x}\\
		&\qquad\qquad+\int_\Omega{D_A^2f(x,u(x),\nabla u(x))[\nabla v(x)]\mathrm{d}x}+\int_{\partial\Omega}{D_u^2g(x,u(x))[v(x)]\mathrm{d}a}.
	\end{align*}
\end{itemize}

\textit{Remark: It would be sufficient to assume $\nabla u,\nabla v\in L^\infty(\overline{\Omega};\mathbb{R}^{m\times d})$ by Lebesgue's dominated convergence theorem.}\\

\textit{Proof:}\\
We only show the claim with the first variation, i.e. (a). The second variation works analogously. So let $u\in M$, $v\in X_0$. Then, using the main theorem of calculus in the third equation and the chain rule in the fourth,
\begin{align*}
	DI(u)[v]&=\lim_{t\to0}{\frac{1}{t}[I(u+tv)-I(u)]}\\
	&=\lim_{t\to0}{\frac{1}{t}\left[\int_\Omega{f(\cdot,u+tv,\nabla u+t\nabla v)-f(\cdot,u,\nabla u)\mathrm{d}x}+\int_{\partial\Omega}{g(\cdot,u+tv)-g(\cdot,u)\mathrm{d}a}\right]}\\
	&=\lim_{t\to0}{\frac{1}{t}\left[\int_\Omega{\int_0^1{\frac{\mathrm{d}}{\mathrm{d}\tau}f(\cdot,u+\tau tv,\nabla u+\tau t\nabla v)\mathrm{d}\tau}\mathrm{d}x}+\int_{\partial\Omega}{\int_0^1{\frac{\mathrm{d}}{\mathrm{d}\tau}g(\cdot,u+\tau tv)\mathrm{d}\tau}\mathrm{d}a}\right]}\\
	&=\lim_{t\to0}{\frac{1}{t}\left[\int_\Omega{\int_0^1{D_uf(\cdot,u+\tau tv,\nabla u+\tau t\nabla v)[tv]\mathrm{d}\tau}\mathrm{d}x}\right.}\\
	&\qquad\qquad+\int_\Omega{\int_0^1{D_Af(\cdot,u+\tau tv,\nabla u+\tau t\nabla v)[t\nabla v]\mathrm{d}\tau}\mathrm{d}x}\\
	&\qquad\qquad+\left.\int_{\partial\Omega}{\int_0^1{D_ug(\cdot,u+\tau tv)[tv]\mathrm{d}\tau}\mathrm{d}a}\right].
\end{align*}
Now the term $\frac{1}{t}$ cancels with the appearing $t$'s in the integrals coming from the directions. Since $u,v,\nabla u,\nabla v$ are bounded functions and $\partial_uf,\partial_Af,\partial_ug$ are continuous, there exists $C>0$ with
\begin{align*}
	\sup_{x\in\overline{\Omega}}{\lvert D_uf(x,u(x)+\tau tv(x),\nabla u(x)+\tau t\nabla v(x))[v(x)]\rvert}&\leq C,\\
	\sup_{x\in\overline{\Omega}}{\lvert D_Af(x,u(x)+\tau tv(x),\nabla u(x)+\tau t\nabla v(x))[\nabla v(x)]\rvert}&\leq C,\\
	\sup_{x\in\partial\Omega}{\lvert D_ug(\cdot,u(x)+\tau tv(x))[v(x)]\rvert}&\leq C.
\end{align*}
Moreover, since $f,g$ are $C^1$-functions, we have pointwise convergence of
\begin{align*}
	\lim_{t\to0}{D_uf(\cdot,u+\tau tv,\nabla u+\tau t\nabla v)[v]}&=D_uf(\cdot,u,\nabla u)[v]\\
	\lim_{t\to0}{D_Af(\cdot,u+\tau tv,\nabla u+\tau t\nabla v)[\nabla v]}&=D_Af(\cdot,u,\nabla u)[\nabla v]\\
	\lim_{t\to0}{D_ug(\cdot,u+\tau tv)[v]}&=D_ug(\cdot,u)[v].
\end{align*}
(Here, the pointwise convergence is meant in $\overline{\Omega}\times[0,1]$ and $\partial\Omega\times[0,1]$.) Lebesgue's dominated convergence theorem yields the claim.\hfill$\blacksquare$\\[11pt]

\hypertarget{theorem_2_2_5}{\textbf{\underline{Theorem 2.2.5}}}\\
(Euler-Lagrange equations)\\
Let $u\in M\cap C^2(\Omega;\mathbb{R}^m)$, $f\in C^2(\Omega\times\mathbb{R}^m\times\mathbb{R}^{m\times d})$, $g\in C^1(\partial\Omega\times\mathbb{R}^m)$. Then $u$ is a critical point of $I$ if and only if it satisfies the \textit{Euler-Lagrange equations}, i.e.
\[\text{(EL)}\qquad\left\{\begin{array}{rl}
	-\divergence{\partial_Af(\cdot,u,\nabla u)}+\partial_uf(\cdot,u,\nabla u)=0&\text{in }\Omega,\\
	u=u_0&\text{on }\Gamma_D,\\
	\partial_Af(\cdot,u,\nabla u)\cdot\nu+\partial_ug(\cdot,u)=0&\text{on }\Gamma_N.
\end{array}\right.\]

\textit{Remark: With $\nu:\partial\Omega\longrightarrow B^{d-1}\subset\mathbb{R}^d$ we denote the outer unit normal vector at $\partial\Omega$ which exists by smoothness assumption. The system (EL) is a system of $m$ coupled partial differential equations.}\\

For the proof of \hyperlink{theorem_2_2_5}{Theorem 2.2.5} we use two fundamental results.\\

\hypertarget{theorem_2_2_6}{\textbf{\underline{Theorem 2.2.6}}}\\
(Divergence Theorem by Gau{\ss})\\
Let $u\in C^1(\overline{\Omega};\mathbb{R}^m)$. Then
\[\int_\Omega{\partial_{x_j}u(x)\mathrm{d}x}=\int_{\partial \Omega}{u(x)\nu_j(x)\mathrm{d}a}\in\mathbb{R}^m.\]

\textit{Proof:}\\
This is proven in course \textit{Analysis III}.\hfill$\blacksquare$\\[11pt]

\hypertarget{theorem_2_2_7}{\textbf{\underline{Theorem 2.2.7}}}\\
(Fundamental Lemma of Calculus of Variations)\\
Let $a\in C^0(\overline{\Omega})$, $b\in C^0(\partial\Omega)$ such that
\[\int_\Omega{a(x)v(x)\mathrm{d}x}+\int_{\partial\Omega}{b(x)v(x)\mathrm{d}a}=0\]
for all $v\in C_{\Gamma_D}^\infty(\overline{\Omega})=\{v\in C^\infty(\overline{\Omega})\mid v|_{\Gamma_D}=0\}$. Then $a\equiv0$ in $\Omega$ and $b\equiv0$ on $\Gamma_N$.\\

\textit{Proof:}\\
We will prove this statement, as well as some variations of it, in the exercise class.\hfill$\blacksquare$\\[11pt]

\textit{Proof of \hyperlink{theorem_2_2_5}{Theorem 2.2.5}:}\\
For all $v\in X_0$ we have with \hyperlink{lemma_2_2_4}{Lemma 2.2.4 (a)}
\begin{align*}
	DI(u)[v]&=\int_\Omega{\partial_uf(\cdot,u,\nabla u)\cdot v+\partial_Af(\cdot,u,\nabla u):\nabla v\mathrm{d}x}+\int_{\partial\Omega}{\partial_ug(\cdot,u)\cdot v\mathrm{d}a}\\
	&=\int_\Omega{\partial_uf(\cdot,u,\nabla u)\cdot v+\divergence{v^\top\partial_Af(\cdot,u,\nabla u)}-v\cdot\divergence{\partial_Af(\cdot,u,\nabla u)}\mathrm{d}x}\\
	&\qquad\qquad+\int_{\partial\Omega}{\partial_ug(\cdot,u)\cdot v\mathrm{d}a}\\
	&=\int_\Omega{\left(\partial_uf(\cdot,u,\nabla u)-\divergence{\partial_Af(\cdot,u,\nabla u)}\right)\cdot v\mathrm{d}x}\\
	&\qquad\qquad+\int_{\partial\Omega}{\left(\partial_Af(\cdot,u,\nabla u)\cdot\nu+\partial_ug(\cdot,u)\right)\cdot v\mathrm{d}a},
\end{align*}
where we have used the divergence theorem in the last line, i.e. \hyperlink{theorem_2_2_6}{Theorem 2.2.6}. Since $u$ and $f$ are $C^2$-functions, the divergence of $\partial_Af(\cdot,u,\nabla u)$ exists. Using now the fundamental lemma of calculus of variations, \hyperlink{theorem_2_2_7}{Theorem 2.2.7}, we conclude that $DI(u)[v]=0$ for all $v\in X_0$ if and only if $u$ satisfies (EL).\hfill$\blacksquare$\\[11pt]

\textbf{Remark 2.2.8}\\
A critical point $u$ is a strong/classical solution to (EL) if $u,f$ are $C^2$-functions, but in general, $u$ is only a weak solution to (EL), that is, for all $v\in X_0$ we have
\[\int_\Omega{\partial_uf(\cdot,u,\nabla u)\cdot v+\partial_Af(\cdot,u,\nabla u):\nabla v\mathrm{d}x}+\int_{\partial\Omega}{\partial_ug(\cdot,u)\cdot v\mathrm{d}a}=0.\]\\

\textbf{Example 2.2.9}\\
(Minimal surface)\\
Let $\Omega\subset\mathbb{R}^2$ and $g\in C^0(\partial\Omega;\mathbb{R})$ be given (confer \hyperlink{example_1_2_3}{Example 1.2.3}). Find $u\in C^1(\Omega;\mathbb{R})$ with $u|_{\partial\Omega}=g$ that minimizes the area
\[I(u)=\int_\Omega{\sqrt{1+\lvert\nabla u(x)\rvert^2}\mathrm{d}x}.\]
We want to compute the Euler-Lagrange equations. In this example, $d=2$, $m=1$, $\Gamma_D=\partial\Omega$ and $\Gamma_N=\emptyset$ are our quantities. Moreover, $f(x,u,A)=\sqrt{1+\lvert A\rvert^2}$ for $A\in\mathbb{R}^2$. With that,
\[\partial_Af(x,u,A)=\frac{1}{\sqrt{1+\lvert A\rvert^2}}\begin{pmatrix}A_1\\A_2\end{pmatrix}\]
and $\partial_uf=0$. In the weak form, a critical point $u\in C^1(\Omega;\mathbb{R})$ with $u|_{\partial\Omega}=g$ satisfies
\[\int_\Omega{\frac{\nabla u(x)\cdot\nabla v(x)}{\sqrt{1+\lvert\nabla u(x)\rvert^2}}\mathrm{d}x}=0\]
for all $v\in C_0^1(\overline{\Omega})$. In the strong form, a critical point $u\in C^2(\Omega;\mathbb{R})$ with $u|_{\partial\Omega}=g$ satisfies
\[\divergence{\frac{\nabla u}{\sqrt{1+\lvert\nabla u\rvert^2}}}=0\quad\text{in }\Omega.\]
From a geometric point of view this means that the mean curvature, i.e. the sum of all principal curvatures, vanishes.\\[11pt]

\hypertarget{example_2_2_10}{\textbf{Example 2.2.10}}\\
(Heat conduction)\\
We consider an open, bounded domain $\Omega\subset\mathbb{R}^d$, a given heat source $h:\Omega\longrightarrow\mathbb{R}$ and given temperature $u_0:\Gamma_D\longrightarrow\mathbb{R}$ on a part $\Gamma_D$ of the boundary and write $u:\overline{\Omega}\longrightarrow\mathbb{R}$ for the temperature.

\begin{figure}[ht]
	\centering
	\begin{tikzpicture}
		\draw[thick, red, fill=red, fill opacity=0.2] plot[smooth, domain=0:1, samples=100] ({(1.5+0.2*cos(\x*720)+0.3*sin(\x*360))*cos(\x*360)}, {(1.5+0.3*cos(\x*360))*sin(\x*360)});
		\draw[very thick, blue] plot[smooth, domain=0.85:1.05, samples=100] ({(1.5+0.2*cos(\x*720)+0.3*sin(\x*360))*cos(\x*360)}, {(1.5+0.3*cos(\x*360))*sin(\x*360)});

		\node[red] at (0, 0) {$\Omega$};
		\node[blue] at (1.8, -1) {$\Gamma_D$};
	\end{tikzpicture}
	\caption{Example of $\Omega$ and $\Gamma_D$.}
\end{figure}

Fourier's law from physics tells that the heat flux $q$ is proportional to temperature gradient $\nabla u$. Mathematically this means there exists $K:\Omega\longrightarrow\mathbb{R}^{d\times d}$ such that $K(x)$ is symmetric, positive definite and $q(x)=-K(x)\nabla u(x)$ for each $x\in\Omega$.\\

In stationary state, it holds by physical principles
\[\int_{\partial\widetilde{\Omega}}{q(x)\cdot\nu(x)\mathrm{d}a}=\int_{\widetilde{\Omega}}{h(x)\mathrm{d}x}\]
for all $\widetilde{\Omega}\subset\Omega$. Using Gau{\ss}, this is the same as
\[\int_{\widetilde{\Omega}}{\divergence{q(x)}-h(x)\mathrm{d}x}=0\]
for $\widetilde{\Omega}\subset\Omega$, so equivalently
\[\divergence{q}-h=-\divergence{K\nabla u}-h=0\qquad\text{in }\Omega.\qquad(1)\]
If $u_\text{out}:\mathbb{R}^3\setminus\overline{\Omega}\longrightarrow\mathbb{R}$ describes the temperature outside the body, then heat exchange with
\[-(K(x)\nabla u(x))\cdot\nu(x)=q(x)\cdot\nu(x)=\alpha(x)(u(x)-u_\text{out}(x))\qquad(2)\]
for all $x\in\Gamma_N$ and some $\alpha:\partial\Omega\longrightarrow\mathbb{R}$. Then (1), (2) and $u|_{\Gamma_D}=u_0$ are the Euler-Lagrange equations of the functional
\[I(u)=\int_\Omega{\frac{1}{2}(K(x)\nabla u(x))\cdot\nabla u(x)-h(x)u(x)\mathrm{d}x}+\frac{1}{2}\int_{\partial\Omega}{\alpha(x)(u(x)-u_\text{out}(x))^2\mathrm{d}a}.\]
If we use our notations with $f$ and $g$ then this corresponds to
\begin{align*}
	f(x,u,A)&=\frac{1}{2}(K(x)A)\cdot A-h(x)u,\\
	g(x,u)&=\frac{\alpha(x)}{2}(u-u_\text{out}(x))^2.
\end{align*}
It holds $\partial_uf=-h(x)$, $\partial_Af=K(x)A$ and $\partial_ug=\alpha(x)(u-u_\text{out})$.\\[11pt]

\hypertarget{example_2_2_11}{\textbf{Example 2.2.11}}\\
(Shortest Connection)\\
Given are two points $(a,A),(b,B)\in\mathbb{R}^2$ and we want to show that the straight line is the shortest connection. Other possible connections are shown in \hyperref[fig:example_2_2_11]{Figure II.2}.\\

\begin{figure}[ht]
	\centering
	\begin{tikzpicture}
		% Hintergrund
		\fill[cyan!20] (-0.5, -0.5) rectangle (6.5, 3.5);
		\draw[thin, cyan, step=1] (-0.5, -0.5) grid (6.5, 3.5);

		% Achsen
		\draw[very thick, ->] (-0.3, 0) -- (6.3, 0);
		\draw[very thick, ->] (0, -0.3) -- (0, 3.3);
		\node[fill=cyan!20] at (6, -0.3) {$x$};
		\node[fill=cyan!20] at (0.6, 3) {$u(x)$};

		% Funktionen
		\draw[thick, red] plot[smooth] coordinates {(1, 3) (2, 1.5) (4, 2.3) (5, 1)};
		\draw[thick, blue] plot[smooth, domain=1:5] (\x, {-\x/2+0.2*cos(\x*180-180)+3.3});
		\fill (1, 3) circle (1.5pt) node[right=2] {$(a,A)$};
		\fill (5, 1) circle (1.5pt) node[right] {$(b,B)$};

		% Rahmen
		\draw[very thick] (-0.5, -0.5) rectangle (6.5, 3.5);
	\end{tikzpicture}
	\caption{Connections for $(a,A)$ and $(b,B)$.}
	\label{fig:example_2_2_11}
\end{figure}

We fix the spaces $X=C^1([a,b])$, $M=\{u\in X\mid u(a)=A,u(b)=B\}$, $X_0=C_0^1([a,b])$ and $\partial\Omega=\Gamma_D=\{a,b\}$. The length of a curve $x\mapsto(x,u(x))$ is
\[I(u)=\int_a^b{\sqrt{1+(u'(x))^2}\mathrm{d}x}\]
what we want to minimize. We have $f(x,u,A)=\sqrt{1+A^2}$ with $A\in\mathbb{R}^{1\times 1}=\mathbb{R}$ in our usual notation. With that, $\partial_uf=0$ and $\partial_Af=\frac{A}{\sqrt{1+A^2}}$. Thus, the Euler-Lagrange equation is just
\[-\frac{\mathrm{d}}{\mathrm{d}x}\left(\frac{u'}{\sqrt{1+(u')^2}}\right)=0.\]
This means $\frac{u'}{\sqrt{1+(u')^2}}=c_*$ for a constant $c_*$. Rearranging this expression leads to
\[u'(x)=\frac{c_*}{\sqrt{1-c_*^2}}\]
which is constant. So, as $u_*$ has to satisfy certain boundary conditions, we obtain
\[u_*(x)=B\frac{x-a}{b-a}+A\frac{b-x}{b-a}\]
as a possible candidate for a minimizer. Until now, we aren't able to show that this is indeed a minimizer. All what we know is that there is at most one minimizer which is $u_*$.\\[11pt]

\textbf{Remark 2.2.12}\\
By computing $u_*$ we made an observation: the quantity $\frac{u'}{\sqrt{1+(u')^2}}$ is an invariant quantity, which helped solving the (EL) equations. Invariances are an important tool to solve differential equations.\\[11pt]

\hypertarget{lemma_2_2_13}{\textbf{Lemma 2.2.13}}\\
(Special case of Noether's theorem; Emmy Noether, 1919)\\
Let
\[I(u)=\int_a^b{f(x,u(x),u'(x))\mathrm{d}x}\]
with $f\in C^2([a,b]\times\mathbb{R}^m\times\mathbb{R}^m)$.
\begin{itemize}
	\item[(a)] If $f$ does not depend on $u_k$ (with $u=(u_1,\dotsc,u_m)$), i.e. $\partial_{u_k}f=0$, then every critical point $u_*$ satisfies $\partial_{A_k}f(x,u_*(x),\nabla u_*(x))=c$ (with $A=(A_1,\dotsc,A_m)$) for all $x\in[a,b]$ for a constant $c$.
	\item[(b)] If $f$ does not depend on $x$, i.e. $\partial_xf=0$, then for every critical point $u^*\in C^2([a,b];\mathbb{R}^m)$ the function $u_*'\cdot\partial_A(u_*,u_*')-f(u_*,u_*')$ is constant in $[a,b]$.\\
\end{itemize}

\textit{Remark: In its general form, Noether's theorem says: Every continuous symmetry of variational problem on $\Omega\subset\mathbb{R}^d$ corresponds to a $d$-dimensional conservation law. If $d=1$, one obtains an invariant quantity.}\\

\textit{Proof:}
\begin{itemize}
	\item[(a)] Since $d=1$, the $k$-th component of (EL) equations reads
	\[-\left(\partial_{A_k}f(\cdot,u,\nabla u)\right)'+\partial_{u_k}f(\cdot,u,\nabla u)=0.\]
	So $\left(\partial_{A_k}f(\cdot,u,\nabla u)\right)'=0$, so $\partial_{A_k}f(\cdot,u,\nabla u)$ is constant.
	\item[(b)] Let $E(u,A)=A\cdot\partial_Af(u,A)-f(u,A)$. Then
	\begin{align*}
		\frac{\mathrm{d}}{\mathrm{d}x}(E(u_*,u_*'))&=u_*''\cdot\partial_Af(u_*,u_*')+u_*'\cdot(\partial_Af(u_*,u_*'))'\\
		&\qquad\qquad-\partial_uf(u_*,u_*')\cdot u_*'-\partial_Af(u_*,u_*')\cdot u_*''\\
		&=u_*'\cdot\left((\partial_Af(u_*,u_*'))'-\partial_uf(u_*,u_*')\right)=0
	\end{align*}
	by \hyperlink{theorem_2_2_5}{Theorem 2.2.5}.\hfill$\blacksquare$\\[11pt]
\end{itemize}

\hypertarget{example_2_2_14}{\textbf{Example 2.2.14}}\\
(Brachistochrone problem)\\
Recall the mathematical setting from \hyperlink{example_1_2_2}{Example 1.2.2}: $M=\{u\in X\mid u(a)=u_a,u(b)=u_b\}$ with $X=C^1([0,1])$, and we want to minimize
\[I(u)=\int_a^b{\sqrt{\frac{1+(u'(x))^2}{2g(u_a-u(x))}}\mathrm{d}x}.\]
The volume density is given by $f(x,u,A)=\sqrt{\frac{1+A^2}{2g(u_a-u)}}$. We see that $f$ is independent of $x$, so Noether's theorem, i.e. \hyperlink{lemma_2_2_13}{Lemma 2.2.13}, says that for critical points $u\in C^2([0,1])$ the function
\[E(u,u')=u'\cdot\frac{u'}{\sqrt{2g(u_a-u)(1+(u')^2)}}-\sqrt{\frac{1+(u')^2}{2g(u_a-u)}}=\frac{-1}{\sqrt{2g(u_a-u)(1+(u')^2)}}\]
is constant in $x$. Hence, $(1+(u')^2)(u_a-u)=c$ for some $c>0$. This leads to
\[u'=\pm\sqrt{\frac{c}{u_a-u}-1}=:h(u),\]
so
\[\int_{u_a}^u{\frac{\mathrm{d}v}{h(v)}}=x+a.\]
This is an elliptic integral from which one can obtain an implicit formula for $u$. (To be precise, one needs to argue a little bit more because $h$ could be zero. However, $h$ cannot be zero on an interval $(a,x_0)$ because we then would have $I(u)=+\infty$. Hence, we obtain at least a local solution which can be extended to a global solution.)\\[11pt]

\textbf{Example 2.2.15}\\
(Minimal surface of revolution)\\
Recall from \hyperlink{example_1_2_3}{Example 1.2.3 (b)} that we want to minimize the functional
\[I(u)=\int_0^L{2\pi u(x)\sqrt{1+(u'(x))^2}\mathrm{d}x}\]
on $M=\{u\in C^1([0,L])\mid u(0)=u_0,u(L)=u_L\}$ for given $L,u_0,u_L>0$. The volume density is $f(x,u,A)=2\pi u\sqrt{1+A^2}$, and the derivatives are given by $\partial_uf(x,u,A)=2\pi\sqrt{1+A^2}$, $\partial_Af(x,u,A)=\frac{2\pi uA}{\sqrt{1+A^2}}$. Assume $u\in C^2([0,L])\cap M$ is a critical point. Then $u$ satisfies the Euler-Lagrange equations, which here is just the equation
\[-\left(\frac{2\pi uu'}{\sqrt{1+(u')^2}}\right)'+2\pi\sqrt{1+(u')^2}=0.\]
However, this is a scalar equation of second order and not so easy to solve. Observe that $f$ is independent of $x$. Thus, Noether's theorem tells us (after division by $2\pi$)
\[c=u\frac{(u')^2}{\sqrt{1+(u')^2}}-u\sqrt{1+(u')^2}=\frac{-u}{\sqrt{1+(u')^2}}\]
for some $c\in\mathbb{R}$. This transforms to $1+(u')^2=u^2c^{-2}$, i.e.
\[u'=\pm\sqrt{\frac{u^2}{c^2}-1}.\]
Note that $u$ cannot be constant in $[0,L]$ because then it would be no longer a solution for the Euler-Lagrange equation in general. So for some $x_0\in[0,L)$, the solution must be given in the form
\[u(x)=c\cdot\cosh\left(\frac{x-d}{c}\right)\]
on $(x_0,L)$ and constant in $(0,x_0)$ (via integration like in \hyperlink{example_2_2_14}{Example 2.2.14}). But since $u$ is twice continuously differentiable it must hold
\[u''(x_0)=\frac{1}{c}\cosh\left(\frac{x_0-d}{c}\right)>0,\]
and if $x_0>0$ then also $u''(x_0)=0$ which is a contradiction, so $x_0=0$. Determine $c,d\in\mathbb{R}$ by $u(0)=u_0$ and $u(b)=u_L$.
    \section{Extreme Points}\label{sec:mcov_chap2_sec3}
What we have seen so far is that satisfying the Euler-Lagrange equations is sufficient for being a critical point. That's okay, but we are more interested in extreme points, more precisely, minimizers. Maximizers can be treated analogously.\\[11pt]

\textbf{\underline{Definition 2.3.1}}\\
We consider an open, bounded set $\Omega\subset\mathbb{R}^d$ and $\emptyset\ne M\subset C^1(\Omega;\mathbb{R}^m)$. Let $I:M\longrightarrow\mathbb{R}\cup\{\pm\infty\}$. Then $u_0\in M$ is called
\begin{itemize}
	\item[(i)] \textit{weak local minimizer of $I$} if there is $\varepsilon>0$ such that for all $v\in M$ with $\lVert u_0-v\rVert_{C^1(\Omega)}\leq\varepsilon$ it holds $I(v)\geq I(u_0)$.
	\item[(ii)] \textit{strong local minimizer of $I$} if there is $\varepsilon>0$ such that for all $v\in M$ with $\lVert u_0-v\rVert_{C^0(\Omega)}\leq\varepsilon$ it holds $I(v)\geq I(u_0)$.
	\item[(iii)] \textit{global minimizer of $I$} if for all $v\in M$ we have $I(v)\geq I(u_0)$.\\[11pt]
\end{itemize}

\textbf{Remark 2.3.2}\\
We have the implications
\[\text{global minimizer}\quad\Rightarrow\quad\text{strong local minimizer}\quad\Rightarrow\quad\text{weak local minimizer}.\]
The first one is clear, and the second one follows because $\lVert\cdot\rVert_{C^0(\Omega)}\leq\lVert\cdot\rVert_{C^1(\Omega)}$, so for a weak local minimizer, $I(v)\geq I(u_0)$ has to be satisfied for a smaller class of functions $v$.\newpage

\hypertarget{example_2_3_3}{\textbf{Example 2.3.3}}\\
(Weak vs. strong minimizers)\\
Consider
\[I(u)=\int_0^1{(1-(u'(x))^2)^2\mathrm{d}x}\]
on $M=M_a:=\{u\in C^1([0,1])\mid u(0)=0,u(1)=a\}$ with parameter $a\geq0$. The volume density for $I$ is $f(A)=(1-A^2)^2$ and thus independent of $x$ and $u$.

\begin{figure}[ht]
	\centering
	\begin{tikzpicture}
		% Hintergrund
		\fill[cyan!20] (-4, -1) rectangle (4, 3);
		\draw[cyan, step=1] (-4, -1) grid (4, 3);

		% Achsen
		\draw[thick, ->] (-3.8, 0) -- (3.8, 0);
		\node at (3.8, -0.3) {$x$};
		\draw[thick, ->] (0, -0.8) -- (0, 2.8);
		\node at (0.5, 2.7) {$f(x)$};
		\draw[thin] (-2, 0.1) -- (-2, -0.1) node[below, fill=cyan!20] {$-1$};
		\draw[thin] (2, 0.1) -- (2, -0.1) node[below, fill=cyan!20] {$1$};

		% Funktion
		\draw[thick, red] plot[smooth, domain=-2.95:2.95] (\x, {2*(1-(\x/2)^2)^2});

		% Rahmen
		\draw[very thick] (-4, -1) rectangle (4, 3);
	\end{tikzpicture}
	\caption{Graph of volume density $f$.}
\end{figure}

To set up the Euler-Lagrange equations (EL), we need to compute the derivatives. It holds $\partial_Af(u')=-4u'(1-(u')^2)$ and $\partial_uf(u')=0$, so (EL) is just
\[0=[4u'(1-(u'))^2]'.\]
That's a second-order ordinary differential equation again, so not easy. Thus, we apply Noether's theorem again. Even both statements, \hyperlink{lemma_2_2_13}{Lemma 2.2.13 (a), (b)} are applicable here. We obtain:
\begin{itemize}
	\item[(a)] $\partial_Af(u')=-4u'(1-(u')^2)$ is constant in $x$.
	\item[(b)] $u'\partial_Af(u')-f(u')=-4(u')^2(1-(u')^2)-(1-(u')^2)^2=(-1-3(u')^2)(1-(u')^2)$ is constant in $x$.
\end{itemize}
With that one can find with algebraic arguments that $u'$ has to be constant. This leads to $u_a(x)=ax$ which indeed is a critical point. Now we want to analyze whether $u_a$ is a minimizer in some sense. For $w\in C_0^1([0,1])$ we have via a Taylor expansion of $f$ at $u_a'$
\begin{align*}
	I(u_a+w)&=\int_0^1{f(u_a'(x)+w'(x))\mathrm{d}x}\\
	&=I(u_a)+f'(a)\underbrace{\int_0^1{w'(x)\mathrm{d}x}}_{=w(1)-w(0)=0}+\frac{1}{2}f''(a)\int_0^1{(w'(x))^2\mathrm{d}x}\\
	&\qquad\qquad+\frac{1}{6}f'''(a)\int_0^1{(w'(x))^3\mathrm{d}x}+\frac{1}{24}f''''(a)\int_0^1{(w'(x))^4\mathrm{d}x}.
\end{align*}
The derivatives of $f$ are
\begin{align*}
	f(A)&=(1-A^2)^2\\
	f'(A)&=-4A(1-A^2)\\
	f''(A)&=12A^2-4\\
	f'''(A)&=24A\\
	f''''(A)&=24.
\end{align*}
It turns out that the existence and type of minimizers depend on $a$.
\begin{itemize}
	\item[(A)] Let $\lvert a\rvert\geq1$. For general $\xi\in\mathbb{R}$ we have
	\[\frac{1}{2}f''(a)\xi^2+\frac{1}{6}f'''(a)\xi^3+\frac{1}{24}f''''(a)\xi^4=\xi^2\underbrace{(6a^2-2+4a\xi+\xi^2)}_{=(\xi+2a)^2+2a^2-2\geq0}\geq0,\]
	so $I(u_a+w)\geq I(u_a)$ for all $w\in C_0^1([0,1])$, i.e. $I(w)\geq I(u_a)$ for all $w\in M_a$. Hence, $u_a$ is a global minimizer.
	\item[(B)] Let $\frac{1}{\sqrt{3}}<\lvert a\rvert<1$. We then have $f''(a)>0$ and thus
	\[I(u_a+w)\geq I(u_a)+\int_0^1{(w'(x))^2\underbrace{\left(\frac{1}{2}f''(a)-4a\varepsilon-\varepsilon^2\right)}_{>0\text{ for }\varepsilon>0\text{ small enough}}\mathrm{d}x}\geq I(u_a)\]
	for $w\in C_0^1([0,1])$ with $\lVert w\rVert_{C^1([0,1])}\leq\varepsilon$. This shows that $u_a$ is a weak local minimizer.\\

	But $u_a$ is not a strong local minimizer. To show this, we perturb $u_a$ by a zigzag function. So, what we actually need to show is that, for a suitable $c>0$, for all $n\in\mathbb{N}$ there exists $\widetilde{u}_n\in M_a$ with $\lVert\widetilde{u}_n-u_a\rVert_{C^0([0,1]}<\frac{c}{n}$ but $I(u_a)>I(\widetilde{u}_n)$. For that, take
	\[\hat{u}_n(x):=u_a(x)+w_n(x)\]
	with $w_n\in PC^1([0,1])$, $w_n(0)=w_n(1)=0$, such that $\hat{u}_n(x)'\in\{-1,1\}$ for each $x\in[0,1]$.\\

	\begin{figure}[ht]
		\centering
		\begin{tikzpicture}
			% Hintergrund
			\fill[cyan!20] (-1, -1) rectangle (7, 3);
			\draw[cyan, step=1] (-1, -1) grid (7, 3);

			% Achsen
			\draw[thick, ->] (-0.8, 0) -- (6.8, 0);
			\node at (6.8, -0.3) {$x$};
			\draw[thick, ->] (0, -0.8) -- (0, 2.8);
			\draw[thin] (5, 0.1) -- (5, -0.1) node[below, fill=cyan!20] {$1$};
			\draw[thin] (0.1, 2.5) -- (-0.1, 2.5) node[left, fill=cyan!20] {$a$};

			% Legende
			\draw[fill=cyan!20] (5.9, 1.7) rectangle (6.8, 2.8);
			\node[red] at (6.35, 2.5) {$u_a$};
			\node[blue] at (6.35, 2) {$\hat{u}_5$};

			% Funktionen
			\draw[thick, red] (0, 0) -- (5, 2.5);
			\draw[thick, blue] (0, 0) -- (0.75, 0.75) -- (1, 0.5) -- (1.75, 1.25) -- (2, 1) -- (2.75, 1.75) -- (3, 1.5) -- (3.75, 2.25) -- (4, 2) -- (4.75, 2.75) -- (5, 2.5);

			% Rahmen
			\draw[very thick] (-1, -1) rectangle (7, 3);
		\end{tikzpicture}
		\caption{Illustration of $\hat{u}_n$ for $n=5$.}
	\end{figure}

	A possible choice could be
	\[w_1(x)=\left\{\begin{array}{rl}
		m_1x&\text{if }0\leq x\leq x_*,\\
		m_2(x-1)&\text{if }x_*<x\leq 1,
	\end{array}\right.\]
	where $x_*:=\frac{m_2}{m_2-m_1}$ so that $w_1$ is continuous, and the quantities $m_1,m_2$ should satisfy $m_1=1-a$, $m_2=-(1+a)$ so that $\hat{u}_n'(x)\in\{-1,1\}$ is guaranteed. Then extend $w_1$ periodically to $\mathbb{R}$ such that $w_1(x+1)=w_1(x)$, and define $w_n(x)=\frac{1}{n}w_1(nx)$ for all $x\in\mathbb{R}$. Then $w_n$ has period $\frac{1}{n}$ and amplitude $\sim\frac{1}{n}$. So
	\[\lVert\hat{u}_n-u_a\rVert_{C^0([0,1])}=\lVert w_n\rVert_{C^0([0,1])}=\frac{1}{n}\lVert w_1\rVert_{C^0([0,1])}\to0\]
	as $n\to\infty$, but $\hat{u}_n'(x)=a+w_n'(x)=\pm1$ for each $x\in[0,1]$ -- except at the $n$ kinks --, i.e. $I(\hat{u}_n)=0<I(u_a)$.\\

	But the problem is that $\hat{u}_n\in PC^1([0,1])\setminus C^1([0,1])$. We obtain our desired sequence $(\widetilde{u}_n)_{n\in\mathbb{N}}$ by smoothing the $n$ kinks $x_k$ of $\hat{u}_n$, which are
	\[x_k=\frac{1+a}{2n}+\frac{k}{n}\]
	for $k=0,\dotsc,n-1$. Smoothing $\hat{u}_n$ in a small $\delta$-neighbourhood of $x_k$ quadratically gives $\widetilde{u}_n\in C^1([0,1])$ with
	\begin{align*}
		I(\widetilde{u}_n)&=\sum_{k=0}^{n-1}{\int_{x_k-\delta}^{x_k+\delta}{(1-(\widetilde{u}_n'(x))^2)^2\mathrm{d}x}}\\
		&=\sum_{k=0}^{n-1}{\int_{x_k-\delta}^{x_k+\delta}{\underbrace{\left(1-\left(\frac{1}{\delta}(x_k-x)\right)^2\right)^2}_{\leq1}\mathrm{d}x}}\\
		&\leq 2n\delta.
	\end{align*}
	Choose $\delta=\delta_n=\frac{1}{n^2}$, then $I(\widetilde{u}_n)\to0$ as $n\to\infty$. Since $I(u_a)=(1-a^2)^2>0$ for $a\in(\frac{1}{\sqrt{3}},1)$, $u_a$ is not a strong minimizer.
	\item[(C)] Let $0\leq\lvert a\rvert\leq\frac{1}{\sqrt{3}}$.\\

	At first, consider $\lvert a\rvert<\frac{1}{\sqrt{3}}$. Then we have $f''(a)<0$. If we have a look at our Taylor expansion, then a natural idea would be to construct functions with somehow small derivative, so that the terms of third and fourth order have a relatively small contribution. We choose $w_\varepsilon(x)=\varepsilon\sin(\pi x)$. Then $w_\varepsilon(0)=0=w_\varepsilon(1)$, and $\lVert w_\varepsilon\rVert_{C^1([0,1])}\to0$ as $\varepsilon\to0$. We can estimate $\lvert\cos^3\rvert,\lvert\cos^4\rvert\leq1$ so that
	\[I(u_a+w_\varepsilon)\leq I(u_a)+\underbrace{\frac{1}{2}f''(a)\varepsilon^2\int_0^1{\pi^2(\cos(\pi x))^2\mathrm{d}x}}_{<0}+4a\varepsilon^3+\varepsilon^4<I(u_a)\]
	if $\varepsilon$ is sufficiently small. Hence, $u_a$ is no weak local minimizer.\\

	Now we consider $\lvert a\rvert=\frac{1}{\sqrt{3}}$. We only concentrate on the case $a=\frac{1}{\sqrt{3}}$; the other one can be treated in a similar way. So $f'''(a)=0$, hence
	\[I(u_a+w)=I(u_a)+\int_0^1{\frac{4}{\sqrt{3}}(w'(x))^3+(w'(x))^4\mathrm{d}x}.\]
	Again, we want to conclude that $u_a$ is not a weak local minimizer. To this end, we show there exists a sequence $(w_k)_{k\in\mathbb{N}}\subset C_0^1([0,1])$ such that $\lVert w_k\rVert_{C^1([0,1])}\to0$ as $k\to\infty$ and
	\[\int_0^1{\frac{4}{\sqrt{3}}(w'_k(x))^3+(w_k'(x))^4\mathrm{d}x}<0\]
	for all sufficiently large $k$. For that, define for $\gamma\in C_0^1([0,1])$ the function $w_k(x):=\frac{1}{k}\gamma(x)$ and choose $\gamma$ in such a way that
	\[\int_0^1{\frac{4}{k^3\sqrt{3}}(\gamma'(x))^3\mathrm{d}x}+\frac{1}{k^4}\int_0^1{(\gamma'(x))^4\mathrm{d}x}<0\]
	for large $k$. It is even sufficient to construct $\gamma$ such that $\int_0^1{(\gamma'(x))^3\mathrm{d}x}<0$. First we consider
	\[\hat{\gamma}:[0,1]\longrightarrow\mathbb{R},\qquad\hat{\gamma}(x):=\left\{\begin{array}{rl}
		x/\theta&\text{for }0\leq x<\theta,\\
		\frac{1-x}{1-\theta}&\text{for }\theta\leq x\leq 1,
	\end{array}\right.\]
	which is continuous, and let $\gamma$ then be a modification that is smooth in an $\varepsilon$-neighbourhood of $\theta$. Then we obtain
	\begin{align*}
		\int_0^1{(\gamma'(x))^3\mathrm{d}x}&=C_\varepsilon+\int_0^1{(\hat{\gamma}'(x))^3\mathrm{d}x}\\
		&=C_\varepsilon+\theta\frac{1}{\theta^3}+(1-\theta)\left(\frac{-1}{1-\theta}\right)^3\\
		&=C_\varepsilon+\frac{1}{\theta^2}-\frac{1}{(1-\theta)^2},
	\end{align*}
	where $C_\varepsilon$ denotes a perturbation term. But for $\varepsilon$ small and $\theta\in(1/2,1)$, we obtain $\int_0^1{(\gamma'(x))^3\mathrm{d}x}<0$.\\[11pt]
\end{itemize}

\hypertarget{theorem_2_3_4}{\textbf{\underline{Theorem 2.3.4}}}\\
(Local weak minimizers)\\
Let $f\in C^2(\Omega\times\mathbb{R}^m\times\mathbb{R}^{m\times d})$ and $g\in C^2(\partial\Omega\times\mathbb{R}^m)$.
\begin{itemize}
	\item[(i)] Necessary condition: If $u_*\in M$ is a weak local minimizer of $I:M\longrightarrow\mathbb{R}$, then for all $v\in X_0$ it holds $D^2I(u)[v]\geq0$.
	\item[(ii)] Sufficient condition: If $u_*\in M$ is a critical point of $I:M\longrightarrow\mathbb{R}$ and there exists $\gamma_*>0$ such that for all $v\in X_0$ it holds
	\[D^2I(u_*)[v]\geq\gamma_*\left(\int_\Omega{\lvert\nabla v(x)\rvert^2+\lvert v(x)\rvert^2\mathrm{d}x}+\int_{\partial\Omega}{\lvert v(x)\rvert^2\mathrm{d}a}\right),\]
	then $u_*$ is a strict weak local minimizer of $I$.\\
\end{itemize}

\textit{Proof:}
\begin{itemize}
	\item[(i)] By hypothesis, $t=0$ is a minimizer of $t\longmapsto I(u_*+tv)$ for all $v\in X_0$. Hence, by real Analysis,
	\[0\leq\left.\frac{\mathrm{d}^2}{\mathrm{d}t^2}I(u_*+tv)\right|_{t=0}=D^2I(u_*)[v].\]
	\item[(ii)] A Taylor expansion for $f$ and $g$ yields
	\begin{align*}
		f(\cdot,u_*+v,\nabla u_*+\nabla v)&=f(\cdot,u_*,\nabla u_*)+D_uf(\cdot,u_*,\nabla u_*)[v]+D_Af(\cdot,u_*,\nabla u_*)[\nabla v]\\
		&\qquad\qquad+\frac{1}{2}D_u^2f(\cdot,u_*,\nabla u_*)[v]+D_AD_uf(\cdot,u_*,\nabla u_*)[v,\nabla v]\\
		&\qquad\qquad+\frac{1}{2}D_A^2f(\cdot,u_*,\nabla u_*)[\nabla v]+r(\cdot,v,\nabla v)\\
		g(\cdot,u_*+v)&=g(\cdot,u_*)+D_ug(\cdot,u_*)[v]+\frac{1}{2}D_u^2g(\cdot,u_*)[v]+s(\cdot,v),
	\end{align*}
	where $r(x,u,A)=o(\lvert A\rvert^2+\lvert u\rvert^2)$ and $s(x,u)=o(\lvert u\rvert^2)$ uniformly in $\overline{\Omega}$ and $\partial\Omega$, respectively. Then \hyperlink{lemma_2_2_4}{Lemma 2.2.4} yields
	\begin{align*}
		I(u_*+u)&=I(u_*)+\underbrace{DI(u_*)[v]}_{=0}+\frac{1}{2}D^2I(u_*)[v]\\
		&\qquad\qquad+\int_\Omega{r(x,v(x),\nabla v(x))\mathrm{d}x}+\int_{\partial\Omega}{s(x,v(x))\mathrm{d}a}\\
		&\geq I(u_*)+\int_\Omega{\frac{\gamma_*}{2}\left(\lvert\nabla v(x)\rvert^2+\lvert v(x)\rvert^2\right)+r(x,v(x),\nabla v(x))\mathrm{d}x}\\
		&\qquad\qquad+\int_{\partial\Omega}{\frac{\gamma_*}{2}\lvert v(x)\rvert^2+s(x,v(x))\mathrm{d}a}.
	\end{align*}
	For sufficiently small $\delta>0$ and $\lVert v\rVert_{C^1(\overline{\Omega})}<\delta$ we obtain
	\[\lvert r(\cdot,v,\nabla v)\rvert\leq\frac{\gamma_*}{4}\left(\lvert\nabla v\rvert^2+\lvert v\rvert^2\right),\qquad\lvert s(\cdot,v)\rvert\leq\frac{\gamma_*}{4}\lvert v\rvert^2.\]
	Hence we can conclude $I(u_*+v)>I(u_*)$ for all $v\in X_0\setminus\{0\}$ with $\lVert v\rVert_{C^1(\overline{\Omega})}<\delta$.\hfill$\blacksquare$\\[11pt]
\end{itemize}

In practice, it is difficult to study the complete second variation (or some kind of global condition). What we do next is treating the following pointwise (local) condition which turns out to be more useful.\\

\hypertarget{theorem_2_3_5}{\textbf{\underline{Theorem 2.3.5}}}\\
(Necessary condition by Legendre and Hadamard)\\
Let $f\in C^2(\Omega\times\mathbb{R}^m\times\mathbb{R}^{m\times d})$ and $u_*\in M$. Assume there exists $\gamma_*\geq0$ such that
\[D^2I(u_*)[v]\geq\gamma_*\int_\Omega{\lvert\nabla v(x)\rvert^2\mathrm{d}x}\]
for all $v\in C_0^1(\overline{\Omega};\mathbb{R}^m)$. Then for all $x\in\Omega$ the \textit{Legendre-Hadamard condition} holds: For any $\xi\in\mathbb{R}^m$ and all $\eta\in\mathbb{R}^d$ we have
\[\underbrace{\sum_{i,j=1}^m{\sum_{k,\ell=1}^d{\frac{\partial^2f}{\partial A_{ik}\partial A_{j\ell}}(x,u_*(x),\nabla u_*(x))\xi_i\xi_j\eta_k\eta_\ell}}}_{=:D_A^2f(x,u_*(x),\nabla u_*(x))[\xi\otimes\eta]}\geq\gamma_*\lvert\xi\rvert^2\lvert\eta\rvert^2.\]
We used the notation of tensor product (also called dyadic product)
\[\xi\otimes\eta=\xi\eta^\top=(\xi_i\eta_k)_{\substack{i=1,\dotsc,m\\k=1,\dotsc,d}}\in\mathbb{R}^{m\times d}.\]\\

\textit{Remark: In particular, if $u_*$ is a weak local minimizer, then \hyperlink{theorem_2_3_4}{Theorem 2.3.4 (a)} postulates $D^2I(u_*)[v]\geq0$ for all $v\in X_0$. So \hyperlink{theorem_2_3_5}{Theorem 2.3.5} is applicable for $\gamma_*=0$ and gives
\[D_A^2f(x,u_*(x),\nabla u_*(x))[\xi\otimes\eta]\geq0\]
for all $x\in\Omega$, $\xi\in\mathbb{R}^m$ and $\eta\in\mathbb{R}^d$. With that, candidates for minimizers (e.g. critical points) which aren't can be ruled out.}\\

\textit{Proof:}\\
The idea is to construct for given $\xi\in\mathbb{R}^m$, $\eta\in\mathbb{R}^d$ a suitable test-function $v$ that is spatially localized and related to $\xi\otimes\eta$. Abbreviate $u=u_*$. Let $x_0\in\Omega$ and $\chi\in C_c^\infty(B_1(0))$ be such that $\int_{B_1(0)}{(\chi(y))^2\mathrm{d}y}>0$. Define
\[v:\Omega\longrightarrow\mathbb{R}^d,\qquad v(x):=\chi\left(\frac{x-x_0}{\delta}\right)\cos\left(\frac{\eta\cdot(x-x_0)}{\varepsilon}\right)\xi\]
for $\delta>0$ sufficiently small such that $v$ belongs to $X_0$. Moreover, $\lVert v\rVert_\infty\leq\lVert\chi\rVert_\infty\cdot\lvert\xi\rvert$ and
\begin{align*}
	\nabla v(x)&=\frac{1}{\delta}\cos\left(\frac{\eta\cdot(x-x_0)}{\varepsilon}\right)\xi\otimes\nabla\chi\left(\frac{x-x_0}{\delta}\right)-\frac{1}{\varepsilon}\chi\left(\frac{x-x_0}{\delta}\right)\sin\left(\frac{\eta\cdot(x-x_0)}{\varepsilon}\right)\xi\otimes\eta\\
	&=:\frac{1}{\delta}V_1(\varepsilon,\delta,x)+\frac{1}{\varepsilon}V_2(\varepsilon,\delta,x).
\end{align*}
We can estimate $\lVert V_1\rVert_{C^0(\Omega)}+\lVert V_2\rVert_{C^0(\Omega)}\leq C$, independent of $x_0$, $\varepsilon$ and $\delta$. Choosing $\varepsilon=\delta^2\ll\delta$, we have $v=O(1)$ and $\nabla v=\frac{1}{\delta^2}V_2+O(\frac{1}{\delta})$ uniformly in $B_\delta(x_0)$ as $\delta\to0$. Together with \hyperlink{lemma_2_2_4}{Lemma 2.2.4 (b)} this yields
\begin{align*}
	D^2I(u_*)[v]&=\int_{B_\delta(x_0)}{D_u^2f(x,u(x),\nabla u(x))[v(x)]+2D_AD_uf(x,u(x),\nabla u(x))[v(x),\nabla v(x)]\mathrm{d}x}\\
	&\qquad\qquad+\int_{B_\delta(x_0)}{D_A^2f(x,u(x),\nabla u(x))[\nabla v(x)]\mathrm{d}x}+0\\
	&=\int_{B_\delta(x_0)}{O(1)+O\left(\frac{1}{\delta^2}\right)+O\left(\frac{1}{\delta}\right)\mathrm{d}x}\\
	&\qquad\qquad+\int_{B_\delta(x_0)}{\frac{1}{\delta^4}D_A^2f(x,u(x),\nabla u(x))[V_2(x)]+O\left(\frac{1}{\delta^3}\right)\mathrm{d}x}\\
	&=\int_{B_1(0)}{\left[O(1)+O\left(\frac{1}{\delta^2}\right)+O\left(\frac{1}{\delta}\right)\right]\delta^d\mathrm{d}x}\\
	&\qquad\qquad+\int_{B_1(0)}{\left[\frac{1}{\delta^4}D_A^2f(\cdot,u,\nabla u)[V_2](x_0+\delta y)+O\left(\frac{1}{\delta^3}\right)\right]\delta^d\mathrm{d}x}.
\end{align*}
Thus,
\begin{align*}
	\delta^{4-d}D^2I(u_*)[v]&=\int_{B_1(0)}{D_A^2f(x_0+\delta y,u(x_0+\delta y),\nabla u(x_0+\delta y))[V_2(x_0+\delta y)]\mathrm{d}y}+O(\delta)\\
	&=\int_{B_1(0)}{H(\delta y)\left[\chi(y)\sin\left(\frac{y\cdot\eta}{\delta}\right)\right]^2\mathrm{d}y}+O(\delta)\\
\end{align*}
with $H(\delta y)=D_A^2f(x_0+\delta y,u(x_0+\delta y),\nabla u(x_0+\delta y))[\xi\otimes\eta]$. Furthermore, by using the identity of $\nabla v=\frac{1}{\delta}V_1+\frac{1}{\delta^2}V_2$ and integration over $\Omega$, one can show analogously
\[\delta^{4-d}\int_\Omega{\lvert\nabla v(x)\rvert^2\mathrm{d}x}=\int_{B_1(0)}{\left[\chi(y)\sin\left(\frac{\eta\cdot y}{\delta}\right)\right]^2\lvert\xi\otimes\eta\rvert^2\mathrm{d}y}+O(\delta).\]
Lemma below implies
\begin{align*}
	\delta^{4-d}D^2I(u)[v]&\to\frac{1}{2}H(0)\int_{B_1(0)}{(\chi(y))^2\mathrm{d}y},\\
	\delta^{4-d}\int_\Omega{\lvert\nabla v(x)\rvert^2\mathrm{d}x}&\to\frac{1}{2}\lvert\xi\otimes\eta\rvert^2\int_{B_1(0)}{(\chi(y))^2\mathrm{d}y}
\end{align*}
as $\delta\to0$. Now use the assumption to see
\[\frac{1}{2}H(0)\int_{B_1(0)}{(\chi(y))^2\mathrm{d}y}\geq\gamma_*\frac{1}{2}\lvert\xi\rvert^2\lvert\eta\rvert^2\int_{B_1(0)}{(\chi(y))^2\mathrm{d}y},\]
so what we want.\hfill$\blacksquare$\\[11pt]

\textbf{Lemma 2.3.6}\\
Let $h\in C^0(B_1(0))$, $\chi\in C_c^\infty(B_1(0))$, $\eta\in\mathbb{R}^d\setminus\{0\}$. Then
\[\lim_{\delta\to0}{\int_{B_1(0)}{\chi(y)^2h(\delta y)\sin^2\left(\frac{\eta\cdot y}{\delta}\right)\mathrm{d}y}}=\frac{1}{2}h(0)\int_{B_1(0)}{\chi^2(y)\mathrm{d}x}.\]\\

\textit{Remark: The integrand on the left-hand side oscillates faster and faster as $\delta\to0$, so we have no pointwise convergence. However, one can show that $\sin^2\left(\frac{\langle\eta,\cdot\rangle}{\delta}\right)\rightharpoonup\frac{1}{2}$ in $L^2(B_1(0))$ and $\chi^2(\cdot)h(\delta\cdot)\to\chi^2(\cdot)h(0)$ in $L^2(B_1(0))$ for $\delta\to0$.}\\

\textit{Proof:}\\
We have
\begin{align*}
	&\left\lvert\int_{B_1(0)}{\chi^2(y)h(\delta y)\sin^2\left(\frac{\eta\cdot y}{\delta}\right)\mathrm{d}Y}-\frac{1}{2}h(0)\int_{B_1(0)}{\chi^2(y)\mathrm{d}y}\right\rvert\\
	\leq&\underbrace{\left\lvert\int_{B_1(0)}{\chi^2(y)[h(\delta y)-h(0)]\sin^2\left(\frac{\eta\cdot y}{\delta}\right)\mathrm{d}y}\right\rvert}_{=:A_1}+\underbrace{\left\lvert\int_{B_1(0)}{h(0)\chi^2(y)\left[\sin^2\left(\frac{\eta\cdot y}{\delta}\right)-\frac{1}{2}\right]\mathrm{d}y}\right\rvert}_{=:A_2}.
\end{align*}
Then
\[A_1\leq\sup_{y\in B_1(0)}{\lvert h(\delta y)-h(0)\rvert}\cdot\int_{B_1(0)}{\chi^2(y)\mathrm{d}y}\to0\]
as $\delta\to0$. For $A_2$, we use the identities $\cos(2\phi)=1-2\sin^2(\phi)$ and
\[\divergence{\sin\left(\frac{2\eta\cdot y}{\delta}\right)\eta}=\frac{2\lvert\eta\rvert^2}{\delta}\cos\left(\frac{2\eta\cdot y}{\delta}\right).\]
Then Gau{\ss}'s theorem yields
\begin{align*}
	A_2&=\left\lvert\frac{h(0)}{2}\int_{B_1(0)}{\chi^2(y)\cos\left(\frac{2\eta\cdot y}{\delta}\right)\mathrm{d}y}\right\rvert\\
	&=\left\lvert\frac{h(0)}{2}\int_{B_1(0)}{\chi^2(y)\frac{\delta}{2\lvert\eta\rvert^2}\divergence{\sin\left(\frac{2\eta\cdot y}{\delta}\right)\eta}\mathrm{d}y}\right\rvert\\
	&=\frac{h(0)\delta}{4\lvert\eta\rvert^2}\underbrace{\left\lvert\int_{\partial B_1(0)}{\underbrace{\chi^2(y)}_{=0}\sin\left(\frac{2\eta\cdot y}{\delta}\right)\eta\cdot\nu(y)\mathrm{d}a}-\int_{B_1(0)}{2\chi(y)\nabla\chi(y)\cdot\eta\sin\left(\frac{2\eta\cdot y}{\delta}\right)\mathrm{d}y}\right\rvert}_{\leq C\text{ as }\chi\text{ is a }C_c^\infty\text{-function}}
\end{align*}
which converges to 0 as $\delta\to0$.\hfill$\blacksquare$\\[11pt]

The Legendre-Hadamard condition is a bit complicated in its general form, but it might simplify from application to application.\\[11pt]

\textbf{Example 2.3.7}\\
(One-dimensional case, $d=1$, $m\in\mathbb{N}$)\\
We let $\Omega=(\alpha,\beta)$, $I(u)=\int_\alpha^\beta{f(t,u(t),u'(t))\mathrm{d}t}$. Then $\partial_A^2f(t,u,A)\in\mathbb{R}^{m\times m}$, $\xi\otimes\eta=\eta\xi\in\mathbb{R}^m$ for $\xi\in\mathbb{R}^m$, $\eta\in\mathbb{R}^1$. If we plug this in into Legendre-Hadamard condition we obtain
\[D_A^2f(t,u,A)[\xi\otimes\eta]=\eta^2(\partial_A^2f(t,u,A)\xi)\cdot\xi\geq\gamma_*\eta^2\lvert\xi\rvert^2,\]
i.e. $(\partial_A^2f(t,u,A)\xi)\cdot\xi\geq\gamma_*\lvert\xi\rvert^2$, which means exactly that $\partial_A^2f(t,u,A)$ is positive (semi-)definite (``semi'' if $\gamma_*=0$).\\[11pt]

Recall \hyperlink{example_2_3_3}{Example 2.3.3}. We had $f(t,u,A)=(1-A^2)^2$ and $m=d=1$. Then the condition $\partial_A^2f(t,u,A)=12A^2-4\geq0$ is equivalent to $\lvert A\rvert\geq\frac{1}{\sqrt{3}}$. Hence, if $u_*$ is a weak local minimizer of $I(u)=\int_0^1{f(u'(x))\mathrm{d}x}$, then $D^2I(u_*)[v]\geq0$. The Legendre-Hadamard condition tells us that $\partial_A^2f(u_*'(x))\geq0$, i.e. $\lvert u_*'(x)\rvert\geq\frac{1}{\sqrt{3}}$ for all $x\in(\alpha,\beta)$.\\[11pt]

\textbf{Example 2.3.8}\\
(Scalar case, $m=1$, $d\in\mathbb{N}$)\\
Consider $\Omega\subset\mathbb{R}^d$ and the functional $I(u)=\int_\Omega{f(x,u(x),\nabla u(x))\mathrm{d}x}$. Then $\partial_A^2f(x,u,A)\in\mathbb{R}^{d\times d}$, $\xi\otimes\eta=\xi\eta\in\mathbb{R}^d$ for $\xi\in\mathbb{R}^1$, $\eta\in\mathbb{R}^d$. The Legendre-Hadamard condition becomes
\[D_A^2f(x,u,A)[\xi\otimes\eta]=\xi^2(\partial_A^2f(x,u,A)\eta)\cdot\eta\geq\gamma_*\xi^2\lvert\eta\rvert^2.\]
This is equivalent to $(\partial_A^2f(x,u,A)\eta)\cdot\eta\geq\gamma_*\lvert\eta\rvert^2$, i.e. that $\partial_A^2f(x,u,A)$ is positive (semi-)definite (``semi'' if $\gamma_*=0$).\\[11pt]

Recall \hyperlink{example_2_2_10}{Example 2.2.10} (Heat conduction). We had $f(x,u,A)=\frac{1}{2}[K(x)A]\cdot A-h(x)u$ with $K:\Omega\longrightarrow\mathbb{R}^{d\times d}$. The Legendre-Hadamard condition is then equivalent to $\frac{1}{2}(K(x)+K(x)^\top)$ being positive (semi-)definite which holds true by hypothesis as $K(x)$ is supposed to by symmetric and positive definite.\\[11pt]

\textbf{Example 2.3.9}\\
(Linear elasticity)\\
We consider a body $\Omega\subset\mathbb{R}^d$ which should be deformed.

\begin{figure}[ht]
	\centering
	\begin{tikzpicture}
		\draw[red] (0, 1) arc (90:270:0.8 and 1);
		\draw[red] (4, -0.9) arc (-90:90:0.8 and 0.9);
		\draw[red] plot[smooth, domain=0:4] (\x, {0.75+(0.25-\x/40)*cos(\x*90)});
		\draw[red] plot[smooth, domain=0:4] (\x, {-0.75-(0.25-\x/40)*cos(\x*90)});
		\draw[blue, dashed] (-1, 0) to [closed, curve through = {(-1, 0) (0, 0.8) (1, 1.1) (2, 1.2) (3, 0.7) (4, 1) (5, 0) (4, -0.6) (3, -0.7) (2, -1.2) (1, -0.9) (0, -0.8)}] (-1, 0);
		\node[red] at (2, 0) {$\Omega$};
	\end{tikzpicture}
	\caption{Deformation of a body.}
\end{figure}

Let $u:\Omega\longrightarrow\mathbb{R}^d$ and $x\longmapsto x+\varepsilon u(x)$ describe an infinitesimal displacement for some $0<\varepsilon\ll1$. We use a so-called infinitesimal strain tensor as a measure of deformation. For that, define
\[e(u)=\frac{1}{2}(\nabla u+\nabla u^\top)=\Sym{\nabla u}\in\mathbb{R}_{\text{sym}}^{d\times d}.\]
Note that if $B\in\mathbb{R}^{d\times d}$ is skew-symmetric (i.e. $B=-B^\top$) then $\Sym{B}=0$. If $\nabla u$ is skew symmetric this means hat $u$ is a rigid motion. Let $f_\text{ext}:\Omega\longrightarrow\mathbb{R}^d$ describe the external forces acting on the body and $g_\text{ext}:\partial\Omega\longrightarrow\mathbb{R}^d$ the external forces acting on the surface. Then define the energy functional
\[I:C^1(\overline{\Omega};\mathbb{R}^d)\longrightarrow\mathbb{R},\qquad I(u):=\int_\Omega{f(x,\nabla u(x))-f_\text{ext}(x)\cdot u(x)\mathrm{d}x}-\int_{\partial\Omega}{g_\text{ext}(x)\cdot u(x)\mathrm{d}a}\]
with energy density
\[f(x,\nabla u(x))=\frac{\lambda(x)}{2}(\tr{e(u)(x)})^2+\mu(x)\lvert e(u)(x)\rvert^2,\]
where $\tr{B}=\sum_{i=1}^d{B_{ii}}$ and $\lvert B\rvert^2=\sum_{i,j=1}^d{B_{ij}^2}$. The functions $\lambda,\mu:\Omega\longrightarrow\mathbb{R}$ are called \textit{Lam\'e constants} of isotropic elasticity. The energy density defines a quadratic form, i.e. we have $f(x,\nabla u(x))=\frac{1}{2}\beta_x(\nabla u(x),\nabla u(x))$ with
\[\beta_x:\mathbb{R}^{d\times d}\times\mathbb{R}^{d\times d}\longrightarrow\mathbb{R},\qquad\beta_x(A,B)=\lambda(x)\tr{A}\tr{B}+\frac{\mu(x)}{2}(A+A^\top):(B+B^\top).\]
Therefore, $D_A^2f(x,\nabla u(x))[\nabla v(x)]=\beta_x(\nabla v(x),\nabla v(x))$ for $v\in C^1(\overline{\Omega};\mathbb{R}^d)$.\\[11pt]

We are interested in minimizers of $I$. The Legendre-Hadamard condition tells us that existence of minimizers requires
\[\beta_x(\xi\otimes\eta,\xi\otimes\eta)\geq0\]
for all $\xi,\eta\in\mathbb{R}^d$, $x\in\Omega$. With $\tr{\xi\otimes\eta}=\xi\cdot\eta$ and $\lvert\xi\otimes\eta+\eta\otimes\xi\rvert^2=2\lvert\xi\rvert^2\lvert\eta\rvert^2+2(\xi\cdot\eta)^2$ we obtain
\begin{align*}
	\beta_x(\xi\otimes\eta,\xi\otimes\eta)&=\lambda(x)(\xi\cdot\eta)^2+\mu(x)(\lvert\xi\rvert^2\lvert\eta\rvert^2+(\xi\cdot\eta)^2)\\
	&=(\lambda(x)+\mu(x))(\xi\cdot\eta)^2+\mu(x)\lvert\xi\rvert^2\lvert\eta\rvert^2\\
	&=(M(\eta)\xi)\cdot\xi\geq0
\end{align*}
with $M(\eta)=\mu(x)\lvert\eta\rvert^2I_d+(\lambda(x)+\mu(x))(\eta\otimes\eta)\in\mathbb{R}_{\text{sym}}^{d\times d}$. So $M(\eta)$ has to be positive semi-definite. This means all eigenvalues of $M(\eta)$ have to be non-negative, i.e.
\[\left\{\begin{array}{rl}
	(\lambda+\mu)\lvert\eta\rvert^2+\mu\lvert\eta\rvert^2\geq0,&\text{(simple eigenvalue)},\\
	\mu\lvert\eta\rvert^2\geq0,&((d-1)\text{-fold eigenvalue}),
\end{array}\right.\]
i.e. $\mu\geq0$ and $\lambda+2\mu\geq0$. Most materials satisfy $\lambda\geq0$ and $\mu\geq0$, but not all. E.g. cobalt has $\lambda<0$ but still $\lambda+2\mu\geq0$.
    \section{Quasiconvexity and the Weierstra{\ss} Condition}
In \hyperref[sec:mcov_chap2_sec3]{Section 2.3 Extreme Points} we focused on weak local minimizers, i.e. we studied the behaviour of the functional for variations $u=u_*+v$ with $\lVert v\rVert_{C^1(\Omega)}<\varepsilon$, $v\in X_0$ at critical points $u_*$. In particular, $\lVert\nabla v\rVert_{L^\infty(\Omega)}<\varepsilon$. Now we want to study strong local minimizers, i.e. the local behaviour for $u=u_*+v$ with $\lVert v\rVert_{C^0(\Omega)}<\varepsilon$. So $\lVert\nabla v\rVert_{L^\infty(\Omega)}$ is allowed to be arbitrary large.\\

The idea is to consider variations with functions of the form $u=u_*+v_\varepsilon$ with
\[v_\varepsilon(x)=\varepsilon\widetilde{w}\left(\frac{x-x_0}{\varepsilon}\right)\]
so that $\lVert v_\varepsilon\rVert_{L^\infty}=\varepsilon\lVert\widetilde{w}\rVert_{L^\infty}$ is getting smaller and smaller for $\varepsilon\to0$, but $\lVert\nabla v_\varepsilon\rVert_{L^\infty}=\lVert\nabla\widetilde{w}\rVert_{L^\infty}$ can be arbitrary large. The functions $v_\varepsilon$ are suitable for strong local minimizers, but not for weak local minimizers.\\[11pt]

\hypertarget{lemma_2_4_1}{\textbf{Lemma 2.4.1}}\\
Let $f\in C^0(\overline{\Omega}\times\mathbb{R}^m\times\mathbb{R}^{m\times d})$ and $u_*\in C^1(\overline{\Omega};\mathbb{R}^m)$. Let $\widetilde{w}\in C^1(\mathbb{R}^d;\mathbb{R}^m)$ with compact support $\supp{\widetilde{w}}\subset D\subset\mathbb{R}^d$, and set $w(x):=\widetilde{w}\left(\frac{x-x_0}{\varepsilon}\right)$ for some fixed $x_0\in\Omega$. Then
\[\lim_{\varepsilon\searrow0}{\frac{1}{\varepsilon^d}\left(I(u_*+\varepsilon w)-I(u_*)\right)}=\int_D{f(x_0,u_*(x_0),\nabla u_*(x_0)+\nabla\widetilde{w}(y))-f(x_0,u_*(x_0),\nabla u_*(x_0))\mathrm{d}y}.\]\\

\textit{Proof:}\\
Choose $\varepsilon>0$ so small that $x_0+\varepsilon D\subset\Omega$. Then the boundary terms coming from $I$ vanish and we get
\begin{align*}
	\frac{1}{\varepsilon^d}(I(u_*+\varepsilon w)-I(u_*))&=\frac{1}{\varepsilon^d}\int_{x_0+\varepsilon D}{f\left(x,u_*(x)+\varepsilon\widetilde{w}\left(\frac{x-x_0}{\varepsilon}\right),\nabla u_*(x)+\nabla\widetilde{w}\left(\frac{x-x_0}{\varepsilon}\right)\right)\mathrm{d}x}\\
	&\qquad\qquad-\frac{1}{\varepsilon^d}\int_{x_0+\varepsilon D}{f(x,u_*(x),\nabla u_*(x))\mathrm{d}x}\\
	&=\int_D{f(x_0+\varepsilon y,u_*(x_0+\varepsilon y)+\varepsilon\widetilde{w}(y),\nabla u_*(x_0+\varepsilon y)+\nabla\widetilde{w}(y))\mathrm{d}y}\\
	&\qquad\qquad-\int_D{f(x_0+\varepsilon y,u_*(x_0+\varepsilon y),\nabla u_*(x_0+\varepsilon y))\mathrm{d}y}.
\end{align*}
Now we can use Lebesgue's dominated convergence theorem by uniform continuity of $f$ and $u_*$ on compact sets.\hfill$\blacksquare$\\[11pt]

\hypertarget{definition_2_4_2}{\textbf{\underline{Definition 2.4.2}}}\\
(Quasiconvexity, Money, 1952)\\
A function $\tilde{f}\in C^0(\mathbb{R}^{m\times d};\mathbb{R})$ is called \textit{quasiconvex in a point $A\in\mathbb{R}^{m\times d}$} if
\begin{align}\label{eq:mcov_formula_qc}
	\int_D{\tilde{f}(A+\nabla\widetilde{w}(y))\mathrm{d}y}\geq\int_D{\tilde{f}(A)\mathrm{d}y}=\tilde{f}(A)\vol{D}\tag{QC}
\end{align}
for all domains $D\subset\mathbb{R}^d$, all test functions $\widetilde{w}\in C^1(\mathbb{R}^d;\mathbb{R}^m)$ with compact support $\supp{\widetilde{w}}\subset D$.\\

If $\tilde{f}$ is quasiconvex in every point, then $\tilde{f}$ is called \textit{quasiconvex}.\newpage

\textbf{Remark 2.4.3}\\
The definition of quasiconvexity is somehow complicated to verify and unnatural. However, one can show that quasiconvexity does not depend on the choice of $D$, i.e. if \eqref{eq:mcov_formula_qc} is satisfied for one domain $D$, e.g. $D=B_1(0)$, then \eqref{eq:mcov_formula_qc} holds for all domains.\\[11pt]

\hypertarget{theorem_2_4_4}{\textbf{\underline{Theorem 2.4.4}}}\\
(Quasiconvexity as necessary condition)\\
Let $f\in C^0(\Omega\times\mathbb{R}^m\times\mathbb{R}^{m\times d})$. If $u_*\in C^1(\overline{\Omega};\mathbb{R}^m)$ is a strong local minimizer of $I$, then the mapping $A\longmapsto f(x_0,u_*(x_0),A)$ is quasiconvex in $A=\nabla u_*(x_0)$ for all $x_0\in\Omega$.\\

\textit{Proof:}\\
Assume that $A\longmapsto f(x_0,u_*(x_0),A)$ is not quasiconvex in $A=\nabla u_*(x_0)$ for some $x_0\in\Omega$. Then there are $D$, $\widetilde{w}$ as in \hyperlink{definition_2_4_2}{Definition 2.4.2} such that
\[\int_D{f(x_0,u_*(x_0),\nabla u_*(x_0)+\nabla\widetilde{w}(y))\mathrm{d}y}<\int_D{f(x_0,u_*(x_0),\nabla u_*(x_0))\mathrm{d}y}.\]
Define $w(x):=\widetilde{w}\left(\frac{x-x_0}{\varepsilon}\right)$. \hyperlink{lemma_2_4_1}{Lemma 2.4.1} provides
\begin{align*}
	&\frac{1}{\varepsilon^d}\left(I(u_*+\varepsilon w)-I(u_*)\right)\\
	&\qquad\qquad=\int_D{f(x_0,u_*(x_0),\nabla u_*(x_0)+\nabla\widetilde{w}(y))-f(x_0,u_*(x_0),\nabla u_*(x_0))\mathrm{d}y}+o(1).
\end{align*}
So if $\varepsilon$ is small then $I(u_*+\varepsilon w)-I(u_*)<0$ which is a contradiction to the minimizing property of $u_*$.\hfill$\blacksquare$\\[11pt]

The idea is that if we have a list of critical points we could check which of them satisfy the quasiconvexity condition from \hyperlink{theorem_2_4_4}{Theorem 2.4.4} to shrink the list of candidates. But quasiconvexity is a global condition containing an integral, thus it is difficult to verify it in practice. We next derive a local/pointwise condition, the so-called Weierstra{\ss} condition.\\[11pt]

\hypertarget{theorem_2_4_5}{\textbf{\underline{Theorem 2.4.5}}}\\
(Weierstra{\ss}' necessary condition)\\
Let $f\in C^1(\Omega\times\mathbb{R}^m\times\mathbb{R}^{m\times d})$. If $u_*\in C^1(\overline{\Omega};\mathbb{R}^m)$ is a strong local minimizer of $I$, then it holds
\[f(x,u_*(x),\nabla u_*(x)+\xi\otimes\eta)-f(x,u_*(x),\nabla u_*(x))-D_Af(x,u_*(x),\nabla u_*(x))[\xi\otimes\eta]\geq0\]
for all $x\in\Omega$, $\xi\in\mathbb{R}^m$, $\eta\in\mathbb{R}^d$.\\[11pt]

\textbf{Remark 2.4.6}
\begin{itemize}
	\item[(a)] If $f$ is even a $C^2$-function, we can set $\xi=\alpha\tilde{\xi}$ and use a Taylor expansion with respect to $\alpha$ to obtain the Legendre-Hadamard condition $D_A^2f(x,u(x),\nabla u(x))[\tilde{\xi}\otimes\eta]\geq0$.
	\item[(b)] $E(x,u,A,\xi,\eta)=f(x,u,A+\xi\otimes\eta)-f(x,u,A)-D_Af(x,u,A)[\xi\otimes\eta]$ is called the \textit{Weierstra{\ss} excess function}.
	\item[(c)] Weierstra{\ss} only considered the case $d=1$. Then $\eta$ can be dropped in the excess function,
	\[E(x,u,A,\xi)=f(x,u,A+\xi)-\underbrace{[f(x,u,A)+\partial_Af(x,u,A)\cdot\xi]}_{\substack{\text{geometrically the tangent line}\\\text{to }f(x,u,\cdot)\text{ at }A}}.\]
	Along strong local minimizer, $f$ \glqq exceeds\grqq{} its tangent lines. Recall \hyperlink{example_2_3_3}{Example 2.3.3}.

	\begin{figure}[ht]
		\centering
		\begin{tikzpicture}
			% Hintergrund
			\fill[cyan!20] (-4, -2) rectangle (4, 3);
			\draw[cyan, step=1] (-4, -2) grid (4, 3);

			% Achsen
			\draw[thick, ->] (-3.8, 0) -- (3.8, 0);
			\node at (3.8, -0.3) {$x$};
			\draw[thick, ->] (0, -1.8) -- (0, 2.8);
			\draw[thin] (2, 0.1) -- (2, -0.1) node[below, fill=cyan!20] {$1$};

			% Funktionen
			\draw[thick, red] plot[smooth, domain=-2.95:2.95] (\x, {2*(1-(\x/2)^2)^2});
			\draw[thick, blue] plot[smooth, domain=-3.2:-1.8] (\x, {0.6328125-2.8125*(\x+2.5)});
			\fill (-2.5, 0.6328125) circle (1.5pt);
			\draw[thick, blue] plot[smooth, domain=-2.8:0.3] (\x, {0.3828125+1.3125*(\x+1.5)});
			\fill (-1.5, 0.3828125) circle (1.5pt);

			% Rahmen
			\draw[very thick] (-4, -2) rectangle (4, 3);
		\end{tikzpicture}
		\caption{Volume density $f$ with tangent lines.}
		\label{fig:remark_2_4_6}
	\end{figure}

	From \hyperref[fig:remark_2_4_6]{Figure II.6} we obtain $E\geq0$ in $(-\infty,-1]\cup[1,\infty)$, but $E<0$ in $(-1,1)$ for some values $\xi\in\mathbb{R}$.\\

	In 1D, the condition $E(x,u,A,\xi)\geq0$ for all $\xi\in\mathbb{R}^m$ is equivalent to convexity and to quasiconvexity of $A\longmapsto f(x,u,A)$.\\[11pt]
\end{itemize}

\textit{Proof of \hyperlink{theorem_2_4_5}{Theorem 2.4.5}}\\
We construct a suitable test function $\widetilde{w}$ with $\nabla\widetilde{w}=\xi\otimes\eta$. For that, we will define a certain function $\alpha\in C^1(\mathbb{R}^d;\mathbb{R})$ and put $\widetilde{w}(y)=\alpha(y)\xi\in\mathbb{R}^m$. Then $\nabla\widetilde{w}(y)=\xi\otimes\nabla\alpha(y)$.\\

\textit{Step 1:} Special case $\eta=(\eta_1,0,\dotsc,0)^\top\in\mathbb{R}^d$ with $\eta_1\geq0$.
\begin{itemize}
	\item[] We write $y=(y_1,\hat{y})\in\mathbb{R}^d$ with $y_1\in\mathbb{R}$, $\hat{y}\in\mathbb{R}^{d-1}$. Let $\mathcal{C}\subset B_1(0)$ be a cone with flat base $\{y\in\mathbb{R}^d\mid y_1=\frac{1}{2},\lvert\hat{y}\rvert\leq\frac{1}{2}\}$ and vertex $(\frac{1}{2}-\delta,0,\dotsc,0)\in\mathbb{R}^d$, where $0<\delta<\frac{1}{2}$. The mantle is given by $\{(y_1,\hat{y})\mid y_1=\frac{1}{2}-\beta(\hat{y})\}$ with $\beta(\hat{y})=\delta(1-2\lvert\hat{y}\rvert)$. Set
	\[\alpha(y)=\left\{\begin{array}{rl}
		0&\text{for }\lvert\hat{y}\rvert\geq\frac{1}{2}\text{ or }\lvert y_1\rvert\geq\frac{1}{2},\\
		\lvert\eta\rvert(y_1-\frac{1}{2})&\text{for }y_1\in\left[\frac{1}{2}-\beta(\hat{y}),\frac{1}{2}\right],\\
		\frac{-\lvert\eta\rvert\beta(\hat{y})}{1-\beta(\hat{y})}\left(y_1+\frac{1}{2}\right)&\text{for }y_1\in\left[-\frac{1}{2},\frac{1}{2}-\beta(\hat{y})\right].
	\end{array}\right.\]

	\begin{figure}[ht]
		\centering
		\begin{tikzpicture}
			% Achsen
			\draw[thick, ->] (-3, 0) -- (3, 0) node[right] {$\hat{y}$};
			\draw[thick, ->] (0, -2.5) -- (0, 2.5) node[right] {$y_1$};

			% Flächenmarkierungen im Koordinatensystem
			\fill[pattern={Lines[angle=-45, distance=1mm]}, pattern color=red] circle (2);
			\fill[white] (-1, -1) rectangle (1, 1);
			\draw[thick] (-1, 0) -- (1, 0) (0, -1) -- (0, 1);
			\fill[pattern={Dots}, pattern color=olive] (-1, 1) -- (-1, -1) -- (1, -1) -- (1, 1) -- (0, 0.2) -- (-1, 1);
			\fill[pattern={Lines[angle=45, distance=1mm]}, pattern color=yellow] (-1, 1) -- (1, 1) -- (0, 0.2) -- (-1, 1);

			% Zeichnungen im Koordinatensystem
			\draw circle (2);
			\draw[dashed] (-1, 1) -- (-1, -1) -- (1, -1) -- (1, 1);
			\draw (-1, 1) -- (1, 1) -- (0, 0.2) -- (-1, 1);
			\draw[thin] (-0.1, 0.2) -- (0.1, 0.2);

			% Beschriftungen
			\draw[thin] (-2, 0.1) -- (-2, -0.1);
			\draw[thin] (2, 0.1) -- (2, -0.1);
			\node at (-2.3, -0.3) {$-1$};
			\node at (2.1, -0.3) {$1$};
			\fill[white] (0.25, 0.04) rectangle (0.9, 0.36);
			\node at (0.55, 0.2) {\scriptsize$\frac{1}{2}-\delta$};
		\end{tikzpicture}
		\caption{Illustration of $\mathcal{C}$ in $\mathbb{R}^2$ and piecewise definition of $\alpha$.}
	\end{figure}

	By definition, $\supp{\alpha}\subset B_1(0)$, $\nabla\alpha(y)=\lvert\eta\rvert e_1=\eta$ for $y\in\mathcal{C}$ and $\nabla\alpha(y)=O(\delta)$ for $y\notin\mathcal{C}$. Moreover, $\alpha$ is Lipschitz continuous, but not necessarily differentiable everywhere. So we approximate $\alpha$ by $C^1$-functions in a neighbourhood of the kinks and use $\widetilde{w}(y)=\alpha(y)\xi$ in the quasiconvexity condition (and make use of \hyperlink{theorem_2_4_4}{Theorem 2.4.4})
	\begin{align*}
		0&\leq\int_{B_1(0)}{f(x,u,A+\nabla\widetilde{w}(y))-f(x,u,A)\mathrm{d}y}\\
		&=\int_\mathcal{C}{f(x,u,A+\xi\otimes\eta)-f(x,u,A)\mathrm{d}y}+\int_{B_1(0)\setminus\mathcal{C}}{\left[D_Af(x,u,A)[\nabla\widetilde{w}(y)]+o(\delta)\right]\mathrm{d}y}\\
		&=(f(x,u,A+\xi\otimes\eta)-f(x,u,A))\int_\mathcal{C}{1\mathrm{d}y}\\
		&\qquad\qquad+D_Af(x,u,A):\left[\int_{B_1(0)\setminus\mathcal{C}}{\xi\otimes\nabla\alpha(y)\mathrm{d}y}\right]+o(\delta).
	\end{align*}
	Gau{\ss} theorem implies
	\[\int_{B_1(0)\setminus\mathcal{C}}{\nabla\alpha(y)\mathrm{d}y}=\int_{B_1(0)}{\nabla\alpha(y)\mathrm{d}y}-\int_\mathcal{C}{\nabla\alpha(y)\mathrm{d}y}=0-\int_\mathcal{C}{\eta\mathrm{d}y}.\]
	Hence
	\[0\leq\left(f(x,u,A+\xi\otimes\eta)-f(x,u,A)-D_Af(x,u,A)[\xi\otimes\eta]\right)\underbrace{\int_\mathcal{C}{1\mathrm{d}y}}_{=c\cdot\delta}+o(\delta),\]
	i.e.
	\[0\leq f(x,u,A+\xi\otimes\eta)-f(x,u,A)-D_Af(x,u,A)[\xi\otimes\eta]+o(1)\]
	as $\delta\to0$.\\
\end{itemize}

\textit{Step 2:} General case.
\begin{itemize}
	\item[] By rotating the coordinate system, we can reduce the general case to step 1. So we assume $\eta=(\eta_1,\dotsc,\eta_d)$ is arbitrary, then there exists a rotation matrix $R\in\mathbb{R}^{d\times d}$ such that $R\eta=\overline{\eta}:=(\lvert\eta\rvert,0,\dotsc,0)$. Then let $\alpha$ be the function constructed in step 1. with respect to $\overline{\eta}$ and define $\overline{\alpha}(y)=\alpha(Ry)$. For $\overline{y}=R^{-1}y\in R^{-1}\mathcal{C}$ we then have
	\[\nabla_{\overline{y}}\overline{\alpha}(\overline{y})=\nabla_{\overline{y}}\alpha(R\overline{y})=\nabla_y\alpha(y)\cdot R=\overline{\eta}^\top\cdot R=\eta^\top.\]
	Hence, by replacing $\alpha$ with $\overline{\alpha}$ and $\mathcal{C}$ with $R^{-1}\mathcal{C}$ in the computations in step 1., we get the same result since $\vol{\mathcal{C}}=\vol{R^{-1}\mathcal{C}}$.\hfill$\blacksquare$\\[11pt]
\end{itemize}

\textbf{\underline{Theorem 2.4.7}}\\
(Weierstra{\ss}'s sufficient condition)\\
Let $\Omega=(\alpha,\beta)\subset\mathbb{R}^1$, $I(u)=\int_\alpha^\beta{f(x,u(x),u'(x))\mathrm{d}x}$, and let $u_*\in C_0^1([\alpha,\beta];\mathbb{R}^m)$ be a weak local minimizer. Suppose that
\begin{itemize}
	\item[(i)] there exists $\gamma_*>0$ such that
	\[D^2I(u_*)[v]\geq\gamma_*\int_\alpha^\beta{\lvert v'(x)\rvert^2+\lvert v(x)\rvert^2\mathrm{d}x}\]
	for all $v\in C_0^1([\alpha,\beta];\mathbb{R}^m)$, and
	\item[(ii)] there exists $\varepsilon>0$ such that for all $x\in[\alpha,\beta]$, $\tilde{u},\tilde{A}\in\mathbb{R}^m$ with $\lvert\tilde{u}-u_*(x)\rvert<\varepsilon$ and $\lvert\tilde{A}-u_*'(x)\rvert<\varepsilon$ it holds $E(x,\tilde{u},\tilde{A},B)\geq0$ for all $B\in\mathbb{R}^m$.
\end{itemize}
Then $u_*$ is a strict strong local minimizer.\\

\textit{Remark: With the first condition we are already familiar, cf. \hyperlink{theorem_2_3_4}{Theorem 2.3.4}. And in the second condition, the exceed function comes in.}\\

\textit{Proof:}\\
The proof of this theorem opens a whole new theory and is quite complicated. See \cite[Chapter ?, Section ? Theorem ?]{magnus_r_hestenes}.\hfill$\blacksquare$\\[11pt]

A common and easy assumption which guarantees existence is the standard convexity. Next, we will collect some different notions of convexity.\\[11pt]

\hypertarget{definition_2_4_8}{\textbf{\underline{Definition 2.4.8}}}\\
Let $\tilde{f}:\mathbb{R}^{m\times d}\longrightarrow\mathbb{R}$ and $A_0\in\mathbb{R}^{m\times d}$.
\begin{itemize}
	\item[(i)] $\tilde{f}$ is called \textit{convex in $A_0$} if there exists $B\in\mathbb{R}^{m\times d}$ such that
	\[\tilde{f}(A_0+C)\geq\tilde{f}(A_0)+B:C\]
	for all $C\in\mathbb{R}^{m\times d}$.
	\item[(ii)] $\tilde{f}$ is called \textit{quasiconvex in $A_0$} if for all $w\in C_0^1(B_1(0);\mathbb{R}^m)$
	\[\frac{1}{\vol{B_1(0)}}\int_{B_1(0)}{\tilde{f}(A_0+\nabla w(x))\mathrm{d}x}\geq\tilde{f}(A_0)\]
	\item[(iii)] $\tilde{f}$ is called \textit{rank-one convex in $A_0$} if there exists $B\in\mathbb{R}^{m\times d}$ such that
	\[\tilde{f}(A_0+\xi\otimes\eta)\geq\tilde{f}(A_0)+B:(\xi\otimes\eta)\]
	for all $\xi\in\mathbb{R}^m$, $\eta\in\mathbb{R}^d$.
\end{itemize}
Moreover, $\tilde{f}$ is called \textit{convex} / \textit{quasiconvex} / \textit{rank-one convex} if $\tilde{f}$ is convex / quasiconvex /rank-one convex in every point $A_0\in\mathbb{R}^{m\times d}$.\newpage

\textbf{Remark 2.4.9}
\begin{itemize}
	\item[(a)] $\tilde{f}$ is convex if and only if for all $A_1,A_2\in\mathbb{R}^{m\times d}$ it holds
	\begin{align}\label{eq:mcov_formula_c}
		\tilde{f}((1-\lambda)A_1+\lambda A_2)\leq(1-\lambda)\tilde{f}(A_1)+\lambda\tilde{f}(A_2)\tag{C}
	\end{align}
	for all $\lambda\in(0,1)$.
	\item[(b)] $\tilde{f}$ is rank-one convex if and only if (C) holds for all $A_1,A_2\in\mathbb{R}^{m\times d}$ such that $A_1-A_2$ has rank 1, i.e. $A_1-A_2=\xi\otimes\eta$ for some $\xi\in\mathbb{R}^n$, $\eta\in\mathbb{R}^d$.\\[11pt]
\end{itemize}

\hypertarget{theorem_2_4_10}{\textbf{\underline{Theorem 2.4.10}}}\\
Let $\tilde{f}:\mathbb{R}^{m\times d}\longrightarrow\mathbb{R}$ and $A_0\in\mathbb{R}^{m\times d}$. Then the following implications hold:
\[\tilde{f}\text{ convex in }A_0\quad\Rightarrow\quad\tilde{f}\text{ quasiconvex in }A_0\quad\Rightarrow\quad\tilde{f}\text{ rank-one convex in }A_0.\]
If $m=1$ or $d=1$, then $\tilde{f}$ is convex in $A_0$ if and only if it is rank-one convex in $A_0$.\\[11pt]

\textbf{Remark 2.4.11}\\
How about the other implications?
\begin{itemize}
	\item[(a)] Morrey conjectured 1952/1966, that quasiconvexity does not imply rank-one convexity.
	\item[(b)] In 1992, \v{S}ver\'ak showed for the case $m\geq3$, $d\geq2$ the existence of an analytic function $\tilde{f}:\mathbb{R}^{m\times d}\longrightarrow\mathbb{R}$ that is rank-one convex but not quasiconvex. However, the case $d=2$ or $m=2$ is still an open problem.
	\item[(c)] Quasiconvexity does not imply convexity, for example consider $\tilde{f}(A)=\det(A)$.\\
\end{itemize}

\textit{Proof of \hyperlink{theorem_2_4_10}{Theorem 2.4.10}:}
\begin{itemize}
	\item[(i)] First we show that convexity implies quasiconvexity. By definition, there exists $B\in\mathbb{R}^{m\times d}$ such that $\tilde{f}(A_0+C)\geq\tilde{f}(A_0)+B:C$ for all $C\in\mathbb{R}^{m\times d}$. Apply this pointwise to $C=\nabla w(y)$ for $w\in C_0^1(B_1(0);\mathbb{R}^m)$ with $y\in B_1(0)$. Then
	\begin{align*}
		\int_{B_1(0)}{\tilde{f}(A+\nabla w(y))\mathrm{d}y}&\geq\int_{B_1(0)}{\tilde{f}(A_0)+B:\nabla w(y)\mathrm{d}y}\\
		&=\tilde{f}(A_0)\vol{B_1(0)}+B:\underbrace{\int_{B_1(0)}{\nabla w(y)\mathrm{d}y}}_{=0}
	\end{align*}
	because
	\[\int_{B_1(0)}{(\nabla w(y))_{ij}\mathrm{d}y}=\int_{B_1(0)}{\frac{\partial}{\partial y_j}w_i(y)\mathrm{d}y}=\int_{\partial B_1(0)}{w_i(y)\nu_j(y)\mathrm{d}a}=0.\]
	\item[(ii)] For the direction from quasiconvexity to rank-one convexity one can proceed as in the proof for Weierstra{\ss}' necessary condition, \hyperlink{theorem_2_4_5}{Theorem 2.4.5}.
	\item[(iii)] Assume $\tilde{f}$ is rank-one convex and $\min\{m,d\}=1$. Then each arbitrary $C\in\mathbb{R}^{m\times d}$ has always the form $C=\xi\otimes\eta$ with $\xi\in\mathbb{R}^m$, $\eta\in\mathbb{R}^d$. Thus, convexity holds.\hfill$\blacksquare$
\end{itemize}

    \chapter{Functional Analytic Existence Results}
\label{ch:convexcase}
We already have learned the classical methods, i.e., 
searching for minimizers by studying their properties and critical points.
But the existence was somehow unclear, i.e. until now we only have some
sufficient conditions which are rather complicated to check.

To show existence, we need tools from functional analysis.
    \section{Abstract Theory}
Let $(X,\lVert\cdot\rVert_X)$ be a Banach space. Consider a functional
\[I:X\longrightarrow\mathbb{R}_\infty:=\mathbb{R}\cup\{+\infty\}.\]
The direct problem of calculus of variations is to find $u_0\in X$ with $I(u_0)=\inf_{u\in X}{I(u)}$. Such an element $u_0$ is called \textit{solution of the direct problem}. But how can we find minimizers? There is a standard recipe:
\begin{enumerate}
	\item First, consider an infimizing sequence $(u_n)_{n\in\mathbb{N}}\subset X$ such that $\lim_{n\to\infty}{I(u_n)}=\inf_{u\in X}{I(u)}$.
	\item Then ensure that $(u_n)_{n\in\mathbb{N}}$ is bounded.
	\item Then choose a subsequence $(u_{n_k})_{k\in\mathbb{N}}$ which converges in a suitable sense to some $u_*\in X$ and obtain $\lim_{k\to\infty}{I(u_{n_k})}=I(u_*)$.
\end{enumerate}
Then $u_0=u_*$ is a minimizer. In much applications, the functional $I$ satisfies a certain coercivity-property which ensures boundedness of any infimizing sequence.\\

\textbf{\underline{Definition 3.1.1}}\\
A functional $I:X\longrightarrow\mathbb{R}_\infty$ on a Banach space $X$ is called \textit{coercive} if $I(u)\to+\infty$ as $\lVert u\rVert_X\to+\infty$.\\[11pt]

\textbf{Proposition 3.1.2}\\
(Finite-dimensional case)\\
If $I:\mathbb{R}^m\longrightarrow\mathbb{R}_\infty$, $m\in\mathbb{N}$, is coercive and continuous, then the direct problem has a solution.\newpage

\textit{Proof:}\\
Let $(u_n)_{n\in\mathbb{N}}$ be an infimizing sequence. Coercivity implies that $(u_n)_{n\in\mathbb{N}}$ is bounded -- unless $I$ is infinity everywhere, but then the assertion it trivial. Then Bolzano-Weierstra{\ss} yields a convergent subsequence $u_{n_k}\to u_0$ as $k\to\infty$ for some $u_0\in\mathbb{R}^m$. Continuity of $I$ implies
\[I(u_0)=\lim_{k\to\infty}{I(u_{n_k})}=\inf_{u\in\mathbb{R}^m}{I(u)}.\]
\hfill$\blacksquare$\\[11pt]

In infinite dimensions:
\begin{itemize}
	\item[(a)] We cannot choose a convergent subsequence!
	\item[(b)] Continuity might not be suitable. For instance, consider
	\[I:\ell^2\longrightarrow\mathbb{R},\qquad x=(x_n)_{n\in\mathbb{N}}\longmapsto(1-\lVert x\rVert_{\ell^2}^2)^2+\sum_{k=1}^\infty{\frac{1}{k}x_k^2}.\]
	Then $I$ is a continuous and coercive functional, but in the exercises we have shown that $I$ admits no minimizer.\\
\end{itemize}

There are two ways out:
\begin{itemize}
	\item[(1)] Equip the functional $I$ with additional conditions that guarantee convergence in $\lVert\cdot\rVert_X$ and (norm-)continuity.
	\item[(2)] Find a suitable (weaker) notion of convergences in $X$ such that boundedness of a sequence implies convergence of a subsequence, and find conditions such that $I$ is continuous with respect to the new type of convergence.\\
\end{itemize}

First, we address variant (1).\\

\textbf{\underline{Definition 3.1.3}}\\
(Uniform convexity)\\
\begin{figure}[ht]
	\centering
	\begin{tikzpicture}
		% Hintergrund und Achsen
		\fill[cyan!20] (-7, -1) rectangle (-1, 3);
		\draw[cyan] (-7, -1) grid (-1, 3);
		\draw[thick, ->] (-6.5, 0) -- (-1.5, 0);
		\draw[thin] (-3, 0.1) -- (-3, -0.1) node[below] {$u$};
		\draw[very thick] (-7, -1) rectangle (-1, 3);

		% Funktionen
		\draw[blue] (-4.5, -0.5) -- (-1.5, 2.5);
		\draw[blue, dashed] plot[smooth, domain=-5:-1.5] (\x, {4+\x+0.2*abs(\x+3)^(3/2)});
		\draw[thick, red] plot[smooth, domain=-6:-2] (\x, {0.5+0.5*(\x+4)^2});

		% Hintergrund und Achsen
		\fill[cyan!20] (1, -1) rectangle (7, 3);
		\draw[cyan] (1, -1) grid (7, 3);
		\draw[thick, ->] (1.5, 0) -- (6.5, 0);
		\draw[thin] (5, 0.1) -- (5, -0.1) node[below] {$u$};
		\draw[very thick] (1, -1) rectangle (7, 3);

		% Funktionen
		\draw[blue] (3, -0.125) -- (6.5, 1.625);
		\draw[blue, dashed] plot[smooth, domain=2:6.5] (\x, {-1.625+\x*0.5+0.2*abs(\x-5)^(3/2)});
		\draw[thick, red] plot[smooth, domain=2:4.5] (\x, {0.5+0.5*(\x-4)^2});
		\draw[thick, red] (4.5, 0.625) -- (5.5, 1.125);
		\draw[thick, red] plot[smooth, domain=5.5:6.5] (\x, {1+0.5*(\x-5)^2});
	\end{tikzpicture}
	\caption{Illustration of uniform convexity (left) and non-uniform convexity (right).}
\end{figure}

We call $I:X\longrightarrow\mathbb{R}_\infty$ \textit{uniformly convex} if there is a function $g:[0,\infty)\longrightarrow\mathbb{R}$ with $g(0)=0$ which is strictly increasing such that for every $u\in X$ there exists a linear, continuous functional $L_u:X\longrightarrow\mathbb{R}$ satisfying
\[I(v)\geq I(u)+L_u(v-u)+g(\lVert u-v\rVert_X).\]

Very often, $g(t)=ct^p$ for some $p>1$, $c>0$ is appropriate.\\[11pt]

\textbf{\underline{Theorem 3.1.4}}\\
Let $I$ be uniformly convex and continuous with respect to $\lVert\cdot\rVert_X$. If $\inf_{u\in X}{I(u)}>-\infty$, then the direct problem has a solution. Moreover, if $I$ is not constantly $+\infty$, the solution is unique.\\

\textit{Proof:}\\
If $I$ is constantly $+\infty$ then the assertion is clear. So assume $I(u)<\infty$ for some $u\in X$ so that in particular $\alpha:=\inf_{u\in X}{I(u)}<\infty$. Let $(u_n)_{n\in\mathbb{N}}$ be an infimizing sequence, i.e. $\lim_{n\to\infty}{I(u_n)}=\alpha$. With the quantities $u=\frac{1}{2}(u_n+u_m)$ and $v=u_m$, the uniform convexity yields
\[I(u_m)\geq I\left(\frac{u_n+u_m}{2}\right)+L_{\frac{1}{2}(u_n+u_m)}\left(\frac{1}{2}(u_m-u_n)\right)+g\left(\frac{1}{2}\lVert u_n-u_m\rVert_X\right),\]
and exchanging $n$ and $m$,
\[I(u_n)\geq I\left(\frac{u_n+u_m}{2}\right)+L_{\frac{1}{2}(u_n+u_m)}\left(\frac{1}{2}(u_n-u_m)\right)+g\left(\frac{1}{2}\lVert u_m-u_n\rVert_X\right).\]
Summing up, we have
\[I(u_m)+I(u_n)\geq 2I\left(\frac{u_n+u_m}{2}\right)+2g\left(\frac{1}{2}\lVert u_n-u_m\rVert_X\right)\geq 2\alpha+2g\left(\frac{1}{2}\lVert u_n-u_m\rVert_X\right).\]
The left-hand side converges to $2\alpha$ as $m,n\to\infty$. Hence, $g\left(\frac{1}{2}\lVert u_n-u_m\rVert_X\right)\to0$ as $m,n\to\infty$. Strict monotonicity implies $\lVert u_n-u_m\rVert_X\to0$ as $m,n\to\infty$ so that $(u_n)_{n\in\mathbb{N}}$ is a Cauchy sequence. Thus, $(u_n)_{n\in\mathbb{N}}$ converges to some limit $u_*\in X$. Continuity implies $I(u_*)=\lim_{n\to\infty}{I(u_n)}=\inf_{u\in X}{I(u)}$. Hence $u_*$ is a minimizer.\\

If $v_*$ would be another minimizer, then $I(u_*)=I(v_*)$ and
\begin{align*}
	I(u_*)&\geq I\left(\frac{u_*+v_*}{2}\right)+L_{\frac{1}{2}(u_*+v_*)}\left(\frac{1}{2}(u_*-v_*)\right)+g\left(\frac{1}{2}\lVert u_*-v_*\rVert_X\right),\\
	I(v_*)&\geq I\left(\frac{u_*+v_*}{2}\right)+L_{\frac{1}{2}(u_*+v_*)}\left(\frac{1}{2}(v_*-u_*)\right)+g\left(\frac{1}{2}\lVert u_*-v_*\rVert_X\right).
\end{align*}
Therefore
\[2I(u_*)=I(u_*)+I(v_*)\geq 2I\left(\frac{u_*+v_*}{2}\right)+2g\left(\frac{1}{2}\lVert u_*-v_*\rVert_X\right)\geq 2I(u_*)+2g\left(\frac{1}{2}\lVert u_*-v_*\rVert_X\right),\]
hence $g\left(\frac{1}{2}\lVert u_*-v_*\rVert_X\right)\leq0$, i.e. $u_*=v_*$.\hfill$\blacksquare$\\[11pt]

In many applications, the assumption of uniform convexity is too restrictive. It is rather exceptional. Our next goal is to follow the second variant, i.e. to develop a weaker notion of convergence. For that, we start with a short revision from functional analysis to fix the notation.\newpage

\textbf{\underline{Definition 3.1.5}}\\
Let $X$ be a Banach space.
\begin{itemize}
	\item[(i)] The vector space $X':=\mathcal{L}(X,\mathbb{R})=\{\varphi:X\longrightarrow\mathbb{R}\text{ linear, continuous}\}$ is called the \textit{dual space of $X$}. It is equipped with the norm
	\[\lVert\varphi\rVert_{X'}:=\sup_{\lVert x\rVert_X\leq1}{\lvert\varphi(x)\rvert}\quad\left(=\sup_{\lVert x\rVert_X=1}{\lvert\varphi(x)\rvert}\quad\text{if }\dim(X)>0\right).\]
	\item[(ii)] A sequence $(u_n)_{n\in\mathbb{N}}\subset X$ is said to \textit{converge weakly to $u$ in $X$} for some $u\in X$ if for all $\varphi\in X'$ we have $\varphi(u_n)\to\varphi(u)$ as $n\to\infty$. The notation is $u_n\rightharpoonup u$. (In contrast to $u_n\to u$ for convergence with respect to norm $\lVert\cdot\rVert_X$.)
	\item[(iii)] The dual space of $X'$ is denoted by $X''=(X')'$ and is called the \textit{bidual space of $X$}.\\[11pt]
\end{itemize}

\textbf{\underline{Theorem 3.1.6}}\\
(Riesz representation theorem)\\
Let $X$ be a Hilbert space. Then for all $\varphi\in X'$ there exists a unique $v_\varphi\in X$ such that for all $u\in X$ it holds $\varphi(u)=\langle v_\varphi,u\rangle$ with $\langle\cdot,\cdot\rangle$ being the scalar product in $X$. Moreover, the map
\[X'\overset{\sim}{\longrightarrow}X,\qquad\varphi\longmapsto v_\varphi\]
is an isometric isomorphism.\\

In particular, a sequence $(u_n)_{n\in\mathbb{N}}\subset X$ converges weakly to $u\in X$ in $X$ if and only if $\langle v,u_n\rangle\to\langle v,u\rangle$ as $n\to\infty$, for all $v\in X$.\\

\textit{Proof:}\\
This is a standard result from functional analysis.\hfill$\blacksquare$\\[11pt]

\hypertarget{remark_3_1_7}{\textbf{Remark 3.1.7}}\\
The following facts are also shown in functional analysis course.
\begin{itemize}
	\item[(a)] The weak limit of a weakly convergent sequence is unique. That's a consequence from Hahn-Banach.
	\item[(b)] Strong convergence, i.e. norm convergence $u_n\to u$, implies weak convergence $u_n\rightharpoonup u$.
	\item[(c)] Weakly convergent sequences are bounded.
	\item[(d)] If $(u_n)_{n\in\mathbb{N}}\subset X$ converges weakly to some $u\in X$ in $X$ then
	\[\lVert u\rVert_X\leq\liminf_{n\to\infty}{\lVert u_n\rVert_X}.\]
	\item[(e)] A Banach space $X$ is naturally embedded into its bidual $X''$ via
	\[J:X\longhookrightarrow X'',\qquad J(u)=[X'\longrightarrow\mathbb{R},\quad\varphi\longmapsto J(u)(\varphi):=\varphi(u)].\]
	This map is isometric and hence injective in general, but not surjective.\\[11pt]
\end{itemize}

\textbf{\underline{Definition 3.1.8}}\\
A Banach space $X$ is called \textit{reflexive} if $J$ is surjective, i.e. if the map from \hyperlink{remark_3_1_7}{Remark 3.1.7 (e)} is an isometric isomorphism.\\[11pt]

\hypertarget{theorem_3_1_9}{\textbf{\underline{Theorem 3.1.9}}}\\
(Eberlin-\v{S}mulian)\\
A Banach space $X$ is reflexive if and only if every bounded sequence $(u_n)_{n\in\mathbb{N}}\subset X$ contains a weakly convergent subsequence.\\

\textit{Remark: The statement from Eberlin-\v{S}mulian tells us that weak compactness is the same as sequential weak compactness.}\\

\textit{Proof:}\\
This is also done in linear functional analysis.\hfill$\blacksquare$\\[11pt]

\textbf{\underline{Definition 3.1.10}}\\
A (nonlinear) functional $I:X\longrightarrow\mathbb{R}_\infty$ is called \textit{weakly (sequentially) continuous} if for all $(u_n)_{n\in\mathbb{N}}\subset X$, $u\in X$ with $u_n\rightharpoonup u$ (as $n\to\infty$) it follows $\lim_{n\to\infty}{I(u_n)}=I(u)$.\\[11pt]

\textbf{Remark 3.1.11}
\begin{itemize}
	\item[(a)] If $I$ is weakly sequentially continuous then it is also strongly (sequentially) continuous. The latter just means continuity with respect to $\lVert\cdot\rVert_X$.
	\item[(b)] There are ``not many'' weakly sequentially continuous functions.
	\item[(c)] Normally, continuity is defined via open sets, i.e. preimages of open sets have to be open. But we work with sequences in the definition, and that's why we should write ``sequentially'' all the time. However, we will drop this from now on because we will always work with sequences.\\[11pt]
\end{itemize}

\hypertarget{examples_3_1_12}{\textbf{Examples 3.1.12}}
\begin{itemize}
	\item[(a)] Consider $X=\ell^2$ and $I(u):=\lVert u\rVert_{\ell^2}^2=\sum_{k=1}^\infty{u_k^2}$. This is a reasonable function and we already know that this has a unique minimizer, namely $I(0)=0$. However, $I$ is not weakly continuous. Consider $u^{(n)}=(0,\dotsc,0,1,0,\dotsc)$ where the 1 appears only at the $n$-th position. Then $u^{(n)}\rightharpoonup0$ as $n\to\infty$ in $\ell^2$, but $I(u^{(n)})=1\not\to0$ for $n\to\infty$.
	\item[(b)] Consider $f\in C^0(\overline{\Omega}\times\mathbb{R}^m)$ that satisfies the growth property: There exists $C>0$, $p\in[1,\infty)$ and $h\in L^1(\Omega)$ such that
	\[\lvert f(x,u)\rvert\leq C(h(x)+\lvert u\rvert^p)\]
	for all $(x,u)\in\Omega\times\mathbb{R}^m$. Define the functional
	\[I:L^p(\Omega;\mathbb{R}^m)\longrightarrow\mathbb{R},\qquad I(u):=\int_\Omega{f(x,u(x))\mathrm{d}x}.\]
	In this setting, one can show that $I$ is always strongly continuous (i.e. this will be shown in the exercises). But $I$ is weakly continuous if and only if $f(x,\cdot)$ is affine, i.e., there exist $a\in C^0(\overline{\Omega})$ and $b\in C^0(\overline{\Omega};\mathbb{R}^m)$ such that $f(x,u)=a(x)+b(x)\cdot u$.\newpage
\end{itemize}

So we see that the notion of weak continuity is not a good choice. We need to relax it.\\[11pt]

\textbf{\underline{Definition 3.1.13}}\\
A functional $I:X\longrightarrow\mathbb{R}_\infty$ is called \textit{weakly (sequentially) lower semicontinuous} if for all $(u_n)_{n\in\mathbb{N}}\subset X$, $u\in X$ with $u_n\rightharpoonup u$ in $X$ as $n\to\infty$, it follows
\[I(u)\leq\liminf_{n\to\infty}{I(u_n)}.\]

\begin{figure}[ht]
	\centering
	\begin{tikzpicture}
		% Hintergrund und Achsen
		\fill[cyan!20] (-7, -1) rectangle (-1, 3);
		\draw[cyan] (-7, -1) grid (-1, 3);
		\draw[thick, ->] (-6.5, 0) -- (-1.5, 0) node[below] {$x$};
		\draw[very thick] (-7, -1) rectangle (-1, 3);

		% Funktionen
		\draw[thick, red] plot[smooth, domain=-5.9:-4] (\x, {0.8+(\x+4)^2/2});
		\draw[thick, red] plot[smooth, domain=-4:-1.8] (\x, {0.2+(\x+4)^2/2});
		\draw[thick, red, fill=cyan!20] (-4, 0.8) circle (1.5pt);
		\fill[red] (-4, 0.2) circle (1.5pt);

		% Hintergrund und Achsen
		\fill[cyan!20] (1, -1) rectangle (7, 3);
		\draw[cyan] (1, -1) grid (7, 3);
		\draw[thick, ->] (1.5, 0) -- (6.5, 0) node[below] {$x$};
		\draw[very thick] (1, -1) rectangle (7, 3);

		% Funktionen
		\draw[thick, red] plot[smooth, domain=1.8:6.2] (\x, {0.2+(\x-4)^2/2});
		\draw[thick, red, fill=cyan!20] (4, 0.2) circle (1.5pt);
		\fill[red] (4, 0.8) circle (1.5pt);
	\end{tikzpicture}
	\caption{Example of a function which is lower semicontinuous (left) and not (right).}
\end{figure}

In general, the norm in a Banach space is not weakly continuous by \hyperlink{examples_3_1_12}{Examples 3.1.12 (a)}. But it is weakly lower semicontinuous by \hyperlink{remark_3_1_7}{Remark 3.1.7 (d)}.\\

\hypertarget{theorem_3_1_14}{\textbf{\underline{Theorem 3.1.14}}}\\
(Abstract existence theorem)\\
Let $X$ be a reflexive Banach space. Let $I:X\longrightarrow\mathbb{R}_\infty$ be coercive and weakly lower semicontinuous. Then the direct problem has a solution.\\

\textit{Proof:}\\
Let $\alpha:=\inf_{u\in X}{I(u)}$. If $\alpha=\infty$ then $I(u)=+\infty$ for all $u\in X$, so every $u\in X$ is a minimizer. Now let us assume $\alpha<\infty$. By definition, there exists an infimizing sequence $(u_n)_{n\in\mathbb{N}}\subset X$, i.e. $\lim_{n\to\infty}{I(u_n)}=\alpha$. Since $\alpha<\infty$, we have $I(u_n)\leq C<\infty$ for some $C\in\mathbb{R}$ and large $n$. Coercivity implies that $(u_n)_{n\in\mathbb{N}}$ is bounded in $X$. Theorem from Eberlein-\v{S}mulian, i.e. \hyperlink{theorem_3_1_9}{Theorem 3.1.9}, yields a weakly convergent subsequence $(u_{n_k})_{k\in\mathbb{N}}\subset(u_n)_{n\in\mathbb{N}}$. Let $u_*\in X$ be the weak limit. Since $I$ is weakly lower semicontinuous, we obtain
\[\inf_{u\in X}{I(u)}\leq I(u_*)\leq\liminf_{k\to\infty}{I(u_{n_k})}=\alpha=\inf_{u\in X}{I(u)}.\]
\hfill$\blacksquare$\\[11pt]

We see that lower weak semicontinuity fits perfectly into our setting. This motivates to study this continuity-type a bit more and to find equivalent notions.\\[11pt]

\hypertarget{theorem_3_1_15}{\textbf{\underline{Theorem 3.1.15}}}\\
Let $X$ be a Banach space, $I:X\longrightarrow\mathbb{R}_\infty$. The following statements are equivalent.
\begin{itemize}
	\item[(i)] $I$ is $\left\{\begin{array}{l}
		\text{(strongly) lower semicontinuous}\\
		\text{weakly lower semicontinuous}
	\end{array}\right\}$.
	Strong lower semicontinuity means that for all $(u_n)_{n\in\mathbb{N}}\subset X$, $u\in X$ with $u_n\to u$ in $X$ we have $I(u)\leq\liminf_{n\to\infty}{I(u_n)}$.
	\item[(ii)] The epigraph of $I$ is $\left\{\begin{array}{l}
		\text{strongly sequentially closed}\\
		\text{weakly sequentially closed}
	\end{array}\right\}$.

	The epigraph is defined as $\epi{I}:=\{(u,\alpha)\in X\times\mathbb{R}\mid I(u)\leq\alpha\}$.
	\item[(iii)] The sublevel sets of $I$ are $\left\{\begin{array}{l}
		\text{strongly sequentially closed}\\
		\text{weakly sequentially closed}
	\end{array}\right\}$.

	The sublevel sets are all the sets of the form $S_\alpha(I):=\{u\in X\mid I(u)\leq\alpha\}$ with $\alpha\in\mathbb{R}$.\\
\end{itemize}

\textit{Proof:}\\
In the following, ``$\rightsquigarrow$'' denotes either strong or weak convergence because the arguments are always the same.
\begin{itemize}
	\item[(i)] $\Rightarrow$ (ii): Take a sequence $(u_n,\alpha_n)_{n\in\mathbb{N}}\subset\epi{I}$ such that $u_n\rightsquigarrow u$ in $X$ and $\alpha_n\to\alpha$ in $\mathbb{R}$ for some $u\in X$, $\alpha\in\mathbb{R}$. Then
	\[\alpha=\lim_{n\to\infty}{\alpha_n}\geq\liminf_{n\to\infty}{I(u_n)}\geq I(u),\]
	hence $(u,\alpha)\in\epi{I}$.
	\item[(ii)] $\Rightarrow$ (iii): Let $\alpha\in\mathbb{R}$ and $(u_n)_{n\in\mathbb{N}}\subset S_\alpha(I)$ such that $u_n\rightsquigarrow u$ in $X$ for some $u\in X$. Then $(u_n,\alpha)\rightsquigarrow(u,\alpha)$ in $X\times\mathbb{R}$. Since $(u_n,\alpha)\in\epi{I}$ for all $n\in\mathbb{N}$ and $\epi{I}$ is closed with respect to ``$\rightsquigarrow$'', we have $(u,\alpha)\in\epi{I}$, so $I(u)\leq\alpha$, so $u\in S_\alpha(I)$.
	\item[(iii)] $\Rightarrow$ (i): Let $(u_n)_{n\in\mathbb{N}}\subset X$, $u\in X$ such that $u_n\rightsquigarrow u$ in $X$ for $n\to\infty$, and write $c:=\liminf_{n\to\infty}{I(u_n)}$. If $c=\infty$ then $I(u)\leq\infty$ trivially and we have nothing to do. If $c<\infty$, then consider a subsequence $(u_{n_k})_{k\in\mathbb{N}}\subset(u_n)_{n\in\mathbb{N}}$ with $c=\lim_{k\to\infty}{I(u_{n_k})}$.\\

	For all $\alpha>c$, there exists $K_\alpha$ such that $I(u_{n_k})\leq\alpha$ for all $k\geq K_\alpha$. This means $u_{n_k}\in S_\alpha(I)$ for all $k\geq K_\alpha$, and since $S_\alpha(I)$ is closed with respect to ``$\rightsquigarrow$'', we obtain $u\in S_\alpha(I)$, i.e. $I(u)\leq\alpha$ for all $\alpha>c$. This means nothing else but
	\[I(u)\leq c=\lim_{k\to\infty}{I(u_{n_k})}=\liminf_{n\to\infty}{I(u_n)}.\]
	\hfill$\blacksquare$\\[11pt]
\end{itemize}

We want to learn more about weakly lower semicontinuous functions. The question about this property for a functional $I$ can be turned over by \hyperlink{theorem_3_1_15}{Theorem 3.1.15} to the question, when the sublevel sets are weakly sequentially closed. It turns out that convex functionals play an important role.
            \section{Convex Functionals}
We already had some different types of convexity, namely \hyperlink{definition_2_4_2}{Definition 2.4.2} and \hyperlink{definition_2_4_8}{Definition 2.4.8}. We also want to fix the most common notion in a more general setting.\\[11pt]

\textbf{\underline{Definition 3.2.1}}\\
(Convexity)\\
Let $X$ be a linear space.
\begin{itemize}
	\item[(i)] A set $M\subset X$ is called \textit{convex} if for all $x,y\in M$ and all $\lambda\in(0,1)$ we have $(1-\lambda)x+\lambda y\in M$.
	\item[(ii)] A mapping $I:X\longrightarrow\mathbb{R}_\infty$ is called \textit{convex} if for all $x,y\in X$ and all $\lambda\in(0,1)$ it holds $I((1-\lambda)x+\lambda y)\leq(1-\lambda)I(x)+\lambda I(y)$.\\[11pt]
\end{itemize}

\hypertarget{remark_3_2_2}{\textbf{Remark 3.2.2}}\\
A functional $I:X\longrightarrow\mathbb{R}_\infty$ is convex if and only if the epigraph $\epi{I}$ is convex. Furthermore, $I$ is convex if and only if all the sublevel sets $S_\alpha(I)$ for $\alpha\in\mathbb{R}$ are convex.\\[11pt]

\textbf{Examples 3.2.3}
\begin{itemize}
	\item[(a)] For a normed space $X$ and any $p\geq1$, the functional $I:X\longrightarrow\mathbb{R}$, $I(x)=\lVert x\rVert_X^p$ is convex. (This is because $I$ is the composition of the norm $X\longrightarrow[0,\infty)$, $u\longmapsto\lVert u\rVert_X$ (which is convex by triangular inequality) and the mapping $[0,\infty)\longrightarrow[0,\infty)$, $c\longmapsto c^p$ (which is convex and monotonically increasing for all $p\geq1$).)

	\begin{figure}[ht]
		\centering
		\begin{tikzpicture}
			% Hintergrund und Achsen
			\fill[cyan!20] (-3, -1) rectangle (3, 3);
			\draw[cyan] (-3, -1) grid (3, 3);
			\draw[thick, ->] (-2.8, 0) -- (2.8, 0) node[below] {$x$};
			\draw[thick, ->] (0, -0.8) -- (0, 2.6) node[right] {$I(x)$};
			\draw[very thick] (-3, -1) rectangle (3, 3);

			% Funktion
			\draw[thick, red] plot[smooth, domain=-1.6:1.6] (\x, {(\x)^2});
		\end{tikzpicture}
		\caption{Illustration for $(X,\lVert\cdot\rVert_X)=(\mathbb{R},\lvert\cdot\rvert)$ and $p=2$.}
	\end{figure}
	\item[(b)] For a set $M\subset X$ we define the so-called \textit{indicator function}

	\begin{figure}[ht]
		\centering
		\begin{tikzpicture}
			% Hintergrund und Achsen
			\fill[cyan!20] (-3, -1) rectangle (3, 3);
			\draw[cyan] (-3, -1) grid (3, 3);
			\draw[thick, ->] (-2.8, 0) -- (2.8, 0) node[below] {$x$};
			\draw[thick, ->] (0, -0.8) -- (0, 2.6) node[right] {$I(x)$};

			% Funktion
			\draw[thick, red] (-1, 0) -- (1, 0);
			\draw[thick, red, dashed] (-1, 0) -- (-1, 3);
			\draw[thick, red, dashed] (1, 0) -- (1, 3);

			% Menge M
			\draw[thick, blue] (-1, 0.2) -- (-1, -0.2);
			\draw[thick, blue] (1, 0.2) -- (1, -0.2);
			\node[blue] at (0.5, -0.3) {$M$};

			\draw[very thick] (-3, -1) rectangle (3, 3);
		\end{tikzpicture}
		\caption{Illustration for $(X,\lVert\cdot\rVert_X)=(\mathbb{R},\lvert\cdot\rvert)$ and $M=[-1,1]$.}
	\end{figure}

	\[I(x)=\chi_M(x)=\left\{\begin{array}{rl}
		0&\text{for }x\in M,\\
		\infty&\text{for }x\notin M.
	\end{array}\right.\]

	One can easily show that $I$ is convex if and only if $M$ is convex.
	\item[(c)] For a convex set $M$, $I(x)=\lVert x\rVert_X^p+\chi_M(x)$ is convex.

	\begin{figure}[ht]
		\centering
		\begin{tikzpicture}
			% Hintergrund und Achsen
			\fill[cyan!20] (-3, -1) rectangle (3, 3);
			\draw[cyan] (-3, -1) grid (3, 3);
			\draw[thick, ->] (-2.8, 0) -- (2.8, 0) node[below] {$x$};
			\draw[thick, ->] (0, -0.8) -- (0, 2.6) node[right] {$I(x)$};

			% Menge M
			\draw[thick, blue] (-1, 0.2) -- (-1, -0.2);
			\draw[thick, blue] (1, 0.2) -- (1, -0.2);
			\draw[thick, blue] (-1, 0) -- (1, 0);
			\node[blue] at (0.5, -0.3) {$M$};

			% Funktion
			\draw[thick, red] plot[smooth, domain=-1:1] (\x, {(\x)^2});
			\draw[thick, red, dashed] (-1, 1) -- (-1, 3);
			\draw[thick, red, dashed] (1, 1) -- (1, 3);

			\draw[very thick] (-3, -1) rectangle (3, 3);
		\end{tikzpicture}
		\caption{Illustration for $(X,\lVert\cdot\rVert_X)=(\mathbb{R},\lvert\cdot\rvert)$, $p=2$ and $M=[-1,1]$.}
	\end{figure}

	More general, the sum of two convex functions again is convex.\\[11pt]
\end{itemize}

\hypertarget{theorem_3_2_4}{\textbf{\underline{Theorem 3.2.4}}}\\
(Hahn-Banach separation theorem)\\
Let $X$ be a Banach space, and let $A,B\subset X$ be non-empty, convex sets with $A\cap B=\emptyset$.
\begin{itemize}
	\item[(i)] If $A$ is open, then there exists $\varphi\in X'$ such that $\varphi(a)<\varphi(b)$ for all $a\in A$, $b\in B$.
	\item[(ii)] If $A$ is closed and $B$ compact, then there exist $\varphi\in X'$ and $\alpha,\beta\in\mathbb{R}$ such that for all $a\in A$, $b\in B$ it holds $\varphi(a)\leq\alpha<\beta\leq\varphi(b)$.\\
\end{itemize}

\textit{Proof:}\\
This is done in course \textit{Functional Analysis}.\hfill$\blacksquare$\\[11pt]

\hypertarget{lemma_3_2_5}{\textbf{Lemma 3.2.5}}\\
(Mazur's lemma)\\
Let $X$ be a Banach space and $M\subset X$ convex. Then $M$ is strongly sequentially closed if and only if $M$ is weakly sequentially closed.\\

\textit{Proof:}
\begin{itemize}
	\item[($\Rightarrow$)] Consider $(u_n)_{n\in\mathbb{N}}\subset M$, $u\in X$ with $u_n\rightharpoonup u$ in $X$ as $n\to\infty$. We need to see $u\in M$. Assume that $u\notin M$. Choose $A=M$ and $B=\{u\}$. Then $A$, $B$ are both clearly non-empty, convex, and by assumption $u\notin M=A$, so $A\cap B=\emptyset$. Moreover, $A$ is strongly sequentially closed, i.e. closed. Thus, \hyperlink{theorem_3_2_4}{Theorem 3.2.4 (ii)} yields $\varphi\in X'$, $\alpha,\beta\in\mathbb{R}$ with $\varphi(a)\leq\alpha<\beta\leq\varphi(u)$ for all $a\in A$. But $u_n\rightharpoonup u$ for $n\to\infty$, so in particular
	\[\varphi(u)=\lim_{n\to\infty}{\varphi(u_n)}\leq\alpha<\beta\leq\varphi(u)\]
	which is a contradiction. Hence, $u\in M$.
	\item[($\Leftarrow$)] This is clear because any strongly convergent sequence also converges weakly to the same limit.\hfill$\blacksquare$\\[11pt]
\end{itemize}

\hypertarget{theorem_3_2_6}{\textbf{\underline{Theorem 3.2.6}}}\\
Let $X$ be a Banach space and let $I:X\longrightarrow\mathbb{R}_\infty$ be convex and lower semicontinuous. Then $I$ is weakly lower semicontinuous.\\

\textit{Proof:}\\
Since $I$ is convex, all sublevel sets $S_\alpha(I)$ for $\alpha\in\mathbb{R}$ are convex by \hyperlink{remark_3_2_2}{Remark 3.2.2}. And since $I$ is lower semicontinuous, all $S_\alpha(I)$ are (strongly) closed by \hyperlink{theorem_3_1_15}{Theorem 3.1.15}. By \hyperlink{lemma_3_2_5}{Lemma 3.2.5}, i.e. Mazur's lemma, all $S_\alpha(I)$ are therefore weakly closed. Hence, \hyperlink{theorem_3_1_15}{Theorem 3.1.15} tells us that $I$ is weakly lower semicontinuous.\hfill$\blacksquare$\\[11pt]

\hypertarget{corollary_3_2_7}{\textbf{Corollary 3.2.7}}\\
(Abstract existence theorem for convex functionals)\\
Let $X$ be a reflexive Banach space. Let $I:X\longrightarrow\mathbb{R}_\infty$ be coercive, convex and lower semicontinuous. Then the direct problem has a solution, i.e. there exists $u_*\in X$ such that $I(u_*)=\inf\{I(v)\mid v\in X\}$.\\

\textit{Proof:}\\
Combine the abstract existence theorem, \hyperlink{theorem_3_1_14}{Theorem 3.1.14}, with \hyperlink{theorem_3_2_6}{Theorem 3.2.6}.\hfill$\blacksquare$\\[11pt]

Previously, we investigated minimizers via critical points for which we needed some differentiability condition. In our abstract existence results such restrictive regularity assumptions on $I$ are no longer needed. So we have an existence result for a huge class of special convex functionals. To further investigate minimizers, we need a good notion of derivative for convex functionals.\\

\textbf{\underline{Definition 3.2.8}}\\
(Subdifferential)\\
Let $X$ be a Banach space, $I:X\longrightarrow\mathbb{R}_\infty$ a functional. Let $u\in X$.
\begin{itemize}
	\item[(i)] A functional $\varphi\in X'$ is called a \textit{subgradient of $I$ in $u$} if for all $v\in X$
	\[I(v)\geq I(u)+\varphi(v-u).\]
	\item[(ii)] The set $\partial I(u)=\{\varphi\in X'\mid\varphi\text{ is a subgradient of }I\text{ in }u\}$ is called the \textit{subdifferential of $I$ in $u$}.\\
\end{itemize}

To get a feeling for subdifferentials, one can think of the right-hand side ``$I(u)+\varphi(v-u)$'' as an affine function in $v$, like a tangential plane or a Taylor expansion of first order which always has to be below $I$. We will discuss some examples.\\[11pt]

\hypertarget{remark_3_2_9}{\textbf{Remark 3.2.9}}
\begin{itemize}
	\item[(a)] It is possible that $\partial I(u)=\emptyset$ for some $u\in X$, so this is allowed.
	\item[(b)] If $I\not\equiv+\infty$ and $\partial I(u)\ne\emptyset$ for some $u\in X$, then $I(u)<\infty$. Indeed, if $\varphi\in\partial I(u)$, then, if $v\in X$ satisfies $I(v)<\infty$, we obtain $\infty>I(v)-\varphi(v-u)\geq I(u)$.\\[11pt]
\end{itemize}

\textbf{Example 3.2.10}\\
(Modulus functional)\\
Consider $I:\mathbb{R}\longrightarrow\mathbb{R}$, $I(x):=\lvert x\rvert$. To compute the subdifferential, we have to ask what kind of affine function lies below $I$.

\begin{figure}[ht]
	\centering
	\begin{tikzpicture}
		% Hintergrund und Achsen
		\fill[cyan!20] (-3, -2) rectangle (3, 2);
		\draw[cyan] (-3, -2) grid (3, 2);
		\draw[thick, ->] (-2.8, 0) -- (2.8, 0) node[below] {$\mathbb{R}$};
		\draw[thick, ->] (0, -1.8) -- (0, 1.6) node[right] {$\mathbb{R}$};
		\draw[very thick] (-3, -2) rectangle (3, 2);

		% Funktionen
		\draw[thick, red] (-1.6, 1.6) -- (0, 0) -- (1.6, 1.6);
		\draw[thick, blue, dashed] (-1.8, -1.8) -- (0.9, 0.9);
		\draw[thick, blue, dashed] (-2.6, 1.3) -- (2.6, -1.3);
		\draw[thick, blue, dashed] (-2.7, -0.9) -- (2.7, 0.9);
	\end{tikzpicture}
	\caption{Graph of modulus functional (red) and some subdifferentials in 0 (blue).}
\end{figure}

We obtain
\[\partial I(x)=\left\{\begin{array}{rl}
	1&\text{if }x>0,\\
	{[-1,1]}&\text{if }x=0,\\
	-1&\text{if }x<0.
\end{array}\right.\]\\

\textbf{Example 3.2.11}\\
(Indicator function)\\
Let $\emptyset\ne M\subset X$ and consider
\[\chi_M:X\longrightarrow\mathbb{R}_\infty,\qquad\chi_M(x):=\left\{\begin{array}{rl}
	0&\text{if }u\in M,\\
	+\infty&\text{if }u\notin M.
\end{array}\right.\]
If $u\notin M$, we clearly have $\partial\chi_M(u)=\emptyset$ by \hyperlink{remark_3_2_9}{Remark 3.2.9 (b)}. If $u\in M$, then $\varphi\in\partial\chi_M(u)$ if and only if $I(v)\geq I(u)+\varphi(v-u)$ for all $v\in X$. This is equivalent to $0\geq\varphi(v-u)$ for all $v\in M$ (since $I(v)=I(u)=0$ for $v\in M$).\\

For $u\in\interior{M}$, we have $B_{\varepsilon_0}(u)\subset M$ for some suitable $\varepsilon_0>0$. Choose $v=u+\varepsilon h$ for $0<\varepsilon<\varepsilon_0$ and $h\in X$ with $\lVert h\rVert_X=1$. Then
\[0\geq\varphi(v-u)=\varphi(\varepsilon h)=\varepsilon\varphi(h),\]
i.e. $0\geq\varphi(h)$ for all $h\in X$ with $\lVert h\rVert_X=1$. Repeat the argument with $\tilde{h}=-h$ to get $0\leq\varphi(h)$. Thus, $0=\varphi(h)$ for all $h\in X$ with $\lVert h\rVert_X=1$. As a consequence $\lVert\varphi\rVert_{X'}=0$, i.e. $\varphi=0$. Altogether we see
\[\partial\chi_M(u)=\left\{\begin{array}{rl}
	\emptyset&\text{if }u\notin M,\\
	\{0\}&\text{if }u\in\interior{M},\\
	\{\varphi\in X'\mid\varphi(u)\geq\varphi(v)\text{ for all }v\in M\}&\text{if }u\in M.
\end{array}\right.\]\\

As an example, take $X=\mathbb{R}$ and $M=[0,\infty)$. Then
\[\partial\chi_M(u)=\left\{\begin{array}{rl}
	\emptyset&\text{if }u<0,\\
	\{0\}&\text{if }u>0,\\
	(-\infty,0]&\text{if }u=0
\end{array}\right.\]
since for $u=0$ we have $\varphi(v)\leq\varphi(u)=0$ for all $v\in[0,\infty)$, so, if $\varphi(x)=ax$ for some $a\in\mathbb{R}$, then $a\leq0$.\\

\begin{figure}[ht]
	\centering
	\begin{tikzpicture}
		% Hintergrund und Achsen
		\fill[cyan!20] (-3, -2) rectangle (3, 2);
		\draw[cyan] (-3, -2) grid (3, 2);
		\draw[thick, ->] (-2.8, 0) -- (2.8, 0) node[below] {$\mathbb{R}$};

		% Funktionen
		\draw[thick, red] (0, 0) -- (2.5, 0);
		\draw[thick, red, dashed] (0, 0) -- (0, 2);
		\draw[thick, blue, dashed] (-1.2, 1.2) -- (1.2, -1.2);
		\draw[thick, blue, dashed] (-0.6, 1.2) -- (0.6, -1.2);
		\fill circle (1.5pt);
		\node at (-0.15, -0.2) {0};

		\draw[very thick] (-3, -2) rectangle (3, 2);
	\end{tikzpicture}
	\caption{Graph of indicator function (red) and some subdifferentials in 0 (blue).}
\end{figure}
\leavevmode\\[11pt]

\textbf{Proposition 3.2.12}\\
(Global minimizers and subdifferentials)\\
Let $X$ be a Banach space, not necessarily reflexive, $I:X\longrightarrow\mathbb{R}_\infty$ a functional and $u_*\in X$. Then $I(u_*)=\inf\{I(u)\mid u\in X\}$ if and only if $0\in\partial I(u_*)$.\\

\textit{Proof:}
\begin{align*}
	I(u_*)=\inf\{I(u)\mid u\in X\}\quad&\Longleftrightarrow\quad I(v)\geq I(u)\text{ for all }v\in X\\
	&\Longleftrightarrow\quad I(v)\geq I(u_*)+0(v-u)\text{ for all }v\in X\\
	&\Longleftrightarrow\quad0\in\partial I(u_*).
\end{align*}
\hfill$\blacksquare$\\[11pt]

How are subdifferentials related to classical derivative? For that, we introduce a new type of differentiability.\\

\textbf{\underline{Definition 3.2.13}}\\
(G\^ateaux derivative)\\
A functional $I:X\longrightarrow\mathbb{R}_\infty$ is called \textit{G\^ateaux differentiable in $u\in X$} with $I(u)<\infty$ if there exists $\varphi_u\in X'$ with
\[\lim_{t\to0}{\frac{1}{t}(I(u+th)-I(u))}=\varphi_u(h)\]
for all $h\in X$. In this case, we write $DI(u):=\varphi_u$ and this is called the \textit{G\^ateaux derivative of $I$ in $u$}.\\[11pt]

The G\^ateaux derivative is strongly related to the notion of the first variation. Indeed, $I$ is G\^ateaux differentiable if and only if the first variation exists and defines a linear continuous functional.\\[11pt]

\textbf{Proposition 3.2.14}\\
Let $X$ be a Banach space, $I:X\longrightarrow\mathbb{R}_\infty$ convex. Let $u\in X$ with $I(u)<\infty$. If $I$ is G\^ateaux differentiable in $u$ then $\partial I(u)=\{DI(u)\}$.\\

\textit{Remark: The subdifferential is a global concept, while G\^ateaux differentiability is local. Therefore we need $I$ to be convex.}\\

\textit{Proof:}\\
First we are going to show $DI(u)\in\partial I(u)$. For this, let $v\in X$ and define $\gamma(t)=I(u+t(v-u))$. Then $\gamma$ is convex since $I$ is, so
\[I(v)=\gamma(1)\geq\gamma(0)+\gamma'(0)(1-0)=I(u)+DI(u)[v-u],\]
therefore $DI(u)\in\partial I(u)$. Now show that $\partial I(u)\subset\{DI(u)\}$. Let $\varphi\in\partial I(u)$. For $h\in X$, $t\in\mathbb{R}$, we have $I(u+th)-I(u)\geq\varphi(th)$ (remind that $I(u)<\infty$). Therefore
\[DI(u)[h]=\lim_{t\nearrow0}{\frac{I(u+th)-I(u)}{t}}\leq\varphi(h)\leq\lim_{t\searrow0}{\frac{I(u+th)-I(u)}{t}}=DI(u)[h],\]
i.e. $DI(u)[h]=\varphi(h)$, i.e. $\varphi=DI(u)$, i.e. $\varphi\in\{DI(u)\}$.\hfill$\blacksquare$\\[11pt]

When is $I$ subdifferentiable? This means, when is $\partial I(u)\ne\emptyset$? We are interested in an answer for more general functionals.\\[11pt]

\hypertarget{theorem_3_2_15}{\textbf{\underline{Theorem 3.2.15}}}\\
Let $X$ be a Banach space, $I:X\longrightarrow\mathbb{R}_\infty$ convex. If $I$ is finite and continuous in $u\in X$, then $\partial I(u)\ne\emptyset$.\\

For the proof we need a lemma first.\\

\hypertarget{lemma_3_2_16}{\textbf{Lemma 3.2.16}}\\
Let $I:X\longrightarrow\mathbb{R}_\infty$ be convex, $I$ finite and continuous in $u\in X$. Then $(u,I(u)+1)\in\interior{\epi{I}}$.\\

\textit{Proof:}\\
If $I$ is continuous in $u$, then there exists $\delta_0>0$ such that for all $v\in X$ with $\lVert u-v\rVert_X<\delta_0$ we have $\lvert I(u)-I(v)\rvert\leq\frac{1}{2}$. In particular, for all $\alpha\in B_{1/2}(I(u)+1)$ and $v\in B_{\delta_0}(u)$ it holds $I(v)\leq\alpha$. Hence,
\[B_{\delta_0}(u)\times B_{1/2}(I(u)+1)\subset\epi{I}\subset X\times\mathbb{R},\]
so $(u,I(u)+1)\in\interior{\epi{I}}$.\hfill$\blacksquare$\\[11pt]

\textit{Proof of \hyperlink{theorem_3_2_15}{Theorem 3.2.15}:}\\
\hyperlink{lemma_3_2_16}{Lemma 3.2.16} implies $B:=\interior{\epi{I}}\ne\emptyset$. Since $I$ is convex, also $\epi{I}$ is convex, so $B$ is convex. Moreover, let $A:=\{(u,I(u))\}$ and apply Hahn-Banach separation theorem, i.e. \hyperlink{theorem_3_2_4}{Theorem 3.2.4 (i)}. Then there exists $\phi\in(X\times\mathbb{R})'\simeq X'\times\mathbb{R}$ such that $\phi(u,I(u))<\phi(v,\alpha)$ for all $(v,\alpha)\in B$.\\

We can write $\phi(v,\alpha)=\varphi(v)+\beta\alpha$ for some $\varphi\in X'$ and $\beta\in\mathbb{R}$. Choose $(u,I(u)+1)\in\interior{\epi{I}}$. Then
\[\varphi(u)+\beta I(u)<\varphi(u)+\beta(I(u)+1),\]
so $\beta>0$. Let $v\in X$ with $I(v)<\infty$. Then $(v,I(v))\in\epi{I}$. Since $\epi{I}$ is convex and $\interior{\epi{I}}\ne\emptyset$, we have that $\epi{I}\subset\overline{\interior{\epi{I}}}$ and hence we obtain
\[\varphi(u)+\beta I(u)\leq\varphi(v)+\beta I(v),\]
i.e. $I(u)\leq I(v)-\frac{1}{\beta}\varphi(u-v)$. But this also holds true if $I(v)=\infty$, whence $-\frac{1}{\beta}\varphi\in\partial I(u)$.\hfill$\blacksquare$\\[11pt]

\textbf{Proposition 3.2.17}\\
(Subdifferential calculus)\\
Let $X$ be a Banach space.
\begin{itemize}
	\item[(a)] If $I:X\longrightarrow\mathbb{R}_\infty$ is a functional and $\alpha>0$, then $\partial(\alpha I)(u)=\alpha\partial I(u)$ for all $u\in X$.
	\item[(b)] For two functionals $I_1,I_2:X\longrightarrow\mathbb{R}_\infty$ and $u\in X$ it holds
	\[\partial(I_1+I_2)(u)\supset\partial I_1(u)+\partial I_2(u).\]
	Equality holds if in addition
	\begin{itemize}
		\item[(i)] $I_1$, $I_2$ are both convex,
		\item[(ii)] $I_1(u_0),I_2(u_0)<\infty$ and $I_i$ is continuous in $u_0$ for some $u_0\in X$ and some $i\in\{1,2\}$.\\
	\end{itemize}
\end{itemize}

\textit{Proof:}
\begin{itemize}
	\item[(a)] We have
	\begin{align*}
		\varphi\in\partial(\alpha I)(u)\quad&\Longleftrightarrow\quad\alpha I(v)\geq\alpha I(u)+\varphi(v-u)\text{ for all }v\in X,\\
		&\Longleftrightarrow\quad I(v)\geq I(u)+\frac{1}{\alpha}\varphi(v-u)\\
		&\Longleftrightarrow\quad\frac{1}{\alpha}\varphi\in\partial I(u).
	\end{align*}
	\item[(b)] Let $\varphi_1\in\partial I_1(u)$ and $\varphi_2\in\partial I_2(u)$. Then for all $v\in X$
	\begin{align*}
		(I_1+I_2)(v)&=I_1(v)+I_2(v)\\
		&\geq I_1(u)+\varphi_1(v-u)+I_2(u)+\varphi_2(v-u),\\
		&=(I_1+I_2)(u)+(\varphi_1+\varphi_2)(v-u),
	\end{align*}
	whence $\varphi_1+\varphi_2\in\partial(I_1+I_2)(u)$.\\

	Now we assume the additional assumptions (i) and (ii) and let $\varphi\in\partial(I_1+I_2)(u)$. We need a functional in one of the subdifferentials $\partial I_1(u)$ or $\partial I_2(u)$ to work with. To this end, we will apply Hahn-Banach separation theorem in an appropriate way. First note that $I_1(u)<\infty$ for $i=1,2$ because $I_1+I_2\not\equiv\infty$. Without loss of generality we may assume that $I_1$ is continuous in $u_0$. Define
	\[G_u:X\longrightarrow\mathbb{R}_\infty,\qquad G_u(v):=I_1(v)-I_1(u)-\varphi(v-u).\]
	Then $G_u(v)\geq I_2(u)-I_2(v)$ for all $v\in X$. Moreover, we define
	\begin{align*}
		A&:=\epi{G_u}=\{(v,\alpha)\in X\times\mathbb{R}\mid G_u(v)\leq\alpha\},\\
		B&:=\{(w,\beta)\in X\times\mathbb{R}\mid\beta\leq I_2(u)-I_2(w)\}.
	\end{align*}
	Then $A$ and $B$ are convex because $I_1,I_2$ are convex. By hypothesis (ii) and \hyperlink{lemma_3_2_16}{Lemma 3.2.16} it holds $\interior{A}\ne\emptyset$. Furthermore, $\interior{A}\cap B=\emptyset$. Hence, Hahn-Banach separation theorem yields $(\tilde{\varphi},\gamma)\in X'\times\mathbb{R}$ such that $\tilde{\varphi}(w)+\gamma\beta<\tilde{\varphi}(v)+\gamma\alpha$ for all $(w,\beta)\in B$, $(v,\alpha)\in\interior{A}$, i.e. $\tilde{\varphi}(w)+\gamma\beta\leq\tilde{\varphi}(v)+\gamma\alpha$ for all $(w,\beta)\in B$, $(v,\alpha)\in A$.\\

	\textit{Claim:} $\gamma>0$.
	\begin{itemize}
		\item[] \hyperlink{lemma_3_2_16}{Lemma 3.2.16} and (ii) imply $(u_0,G_u(u_0)+1)\in\interior{\epi{G_u}}=\interior{A}$. Moreover, $(u_0,I_2(u)-I_2(u_0))\in B$. Then
		\[\tilde{\varphi}(u_0)+\gamma(I_2(u)-I_2(u_0))<\tilde{\varphi}(u_0)+\gamma(G_u(u_0)+1),\]
		so rearranging terms gives
		\[\gamma(\underbrace{I_2(u)-I_2(u_0)-G_u(u_0)}_{\leq0}-1)<0,\]
		so $\gamma>0$ as claimed.
	\end{itemize}

	Moreover, $(u,0)\in A\cap B$. Thus,
	\[\tilde{\varphi}(w)+\gamma\beta\leq\tilde{\varphi}(u)+\gamma\cdot0\leq\tilde{\varphi}(v)+\gamma\alpha\]
	for all $(w,\beta)\in B$, $(v,\alpha)\in A$. Using $(v,G_u(v))\in A$ and $(w,I_2(u)-I_2(w))\in B$ for $w,v\in X$ with $I_1(v),I_2(w)<\infty$, we obtain
	\[\tilde{\varphi}(w)+\gamma(I_2(u)-I_2(w))\leq\tilde{\varphi}(u)\leq\tilde{\varphi}(v)+\gamma G_u(v),\]
	hence $I_2(w)\geq I_2(u)+\frac{1}{\gamma}\tilde{\varphi}(w-u)$ and $I_1(v)\geq I_1(u)+\varphi(v-u)-\frac{1}{\gamma}\tilde{\varphi}(v-u)$. But these estimates hold also true for $I_1(v)=\infty$ and $I_2(w)=\infty$. We infer $\frac{1}{\gamma}\tilde{\varphi}\in\partial I_2(u)$ and $\varphi-\frac{1}{\gamma}\tilde{\varphi}\in\partial I_1(u)$. So
	\[\varphi=\left(\varphi-\frac{1}{\gamma}\tilde{\varphi}\right)+\frac{1}{\gamma}\tilde{\varphi}\in\partial I_1(u)+\partial I_2(u).\]
\end{itemize}
\hfill$\blacksquare$
    \section{Lebesgue and Sobolev Spaces}
For the abstract results on existence of solutions to the direct problem, \hyperlink{theorem_3_1_14}{Theorem 3.1.14} and \hyperlink{corollary_3_2_7}{Corollary 3.2.7}, we needed \textit{reflexive} Banach spaces. But the spaces of continuous, continuously differentiable or piecewise continuously differentiable functions aren't reflexive. For example, consider the bounded sequence $(u_n)_{n\in\mathbb{N}}\subset C^0([0,1])$ with
\[u_n:[0,1]\longrightarrow\mathbb{R},\qquad u_n(x):=\left\{\begin{array}{rl}
	1-nx&\text{if }x\in[0,1/n],\\
	0&\text{if }x\in(1/n,1].
\end{array}\right.\]

\begin{figure}[ht]
	\centering
	\begin{tikzpicture}
		% Hintergrund und Achsen
		\fill[cyan!20] (-2, -1) rectangle (4, 3);
		\draw[cyan, step=2] (-2, -1) grid (4, 3);
		\draw[thick, ->] (-1.8, 0) -- (3.8, 0) node[below] {$x$};
		\draw[thin] (0.1, 2) -- (-0.1, 2) node[left, fill=cyan!20] {1};
		\draw[thick, ->] (0, -0.8) -- (0, 2.7) node[right] {$u_n(x)$};
		\draw[thin] (1, 0.1) -- (1, -0.1) node[below] {$\frac{1}{n}$};
		\draw[very thick] (-2, -1) rectangle (4, 3);

		% Funktionen
		\draw[thick, red] (0, 2) -- (1, 0) -- (3.5, 0);
	\end{tikzpicture}
	\caption{Graph of $u_n$.}
\end{figure}

Then $\lVert u_n\rVert_{C^0([0,1])}=1$. For $x_0\in[0,1]$ consider the dirac measure
\[\delta_{x_0}:C^0([0,1])\longrightarrow\mathbb{R},\qquad\delta_{x_0}(u):=u(x_0).\]
This is linear and continuous, whence $\delta_{x_0}\in C^0([0,1])'$, and
\[\delta_{x_0}(u_n)=u_n(x_0)\to\left\{\begin{array}{rl}
	0&\text{if }x_0\in(0,1],\\
	1&\text{if }x_0=0.
\end{array}\right.\]
Hence, if a subsequence of $(u_n)_{n\in\mathbb{N}}$ would be weakly convergent, then the limit $u_*$ is
\[u_*(x)=\left\{\begin{array}{rl}
	1&\text{if }x=0,\\
	0&\text{if }x\in(0,1].
\end{array}\right.\]
However, $u_*\notin C^0([0,1])$, so $(u_n)_{n\in\mathbb{N}}$ has no weakly convergent subsequence. Hence, $C^0([0,1])$ cannot be reflexive by Eberlin-\v{S}mulian, \hyperlink{theorem_3_1_9}{Theorem 3.1.9}. In a same way, one can construct counterexamples for $C^1([0,1])$ and $PC^1([0,1])$.\\[33pt]

\paragraph{Lebesgue Spaces}
Let $\Omega\subset\mathbb{R}^d$ be open. A function $u:\Omega\longrightarrow\mathbb{R}$ is called \textit{measurable} if for all $a,b\in\mathbb{R}$ the preimage $u^{-1}([a,b))\subset\Omega$ is Lebesgue measurable. There are many other equivalent definitions for measurability. For $p\in[1,\infty)$, define
\[\lVert u\rVert_{L^p(\Omega)}:=\left(\int_\Omega{\lvert u(x)\rvert^p\mathrm{d}x}\right)^{1/p},\]
and for $p=\infty$ define
\[\lVert u\rVert_{L^\infty(\Omega)}:=\esssup_{x\in\Omega}{\lvert u(x)\rvert}=\inf\left\{\sup_{x\in\Omega\setminus N}{\lvert u(x)\rvert}\,\middle\vert\,N\subset\Omega\text{ null set }\right\}.\]
Let $\mathcal{L}^p(\Omega):=\{u:\Omega\longrightarrow\mathbb{R}\mid u\text{ is measurable with }\lVert u\rVert_{L^p(\Omega)}<\infty\}$. Then $\lVert\cdot\rVert_{L^p(\Omega)}$ is a seminorm on $\mathcal{L}^p(\Omega)$, and it becomes a norm on $L^p(\Omega):=\mathcal{L}^p(\Omega)/\sim$, where $u_1\sim u_2$ if $u_1(x)=u_2(x)$ for almost all $x\in\Omega$.\\

These objects, as well as the following list of properties for Lebesgue spaces, should already be known from \textit{Analysis} or \textit{Functional Analysis} course.
\begin{itemize}
	\item[(a)] For all $p\in[1,\infty]$ the pair $(L^p(\Omega),\lVert\cdot\rVert_{L^p(\Omega)})$ forms a Banach space.
	\item[(b)] We have some dense subsets:
	\begin{itemize}
		\item[(1)] The simple functions
		\[S(\Omega):=\operatorname{span}\{\chi_E:\Omega\longrightarrow\{0,1\}\mid E\subset\Omega\text{ measurable with finite measure}\}\]
		are dense in $L^p(\Omega)$ for all $p\in[1,\infty]$.
		\item[(2)] The smooth functions with compact support, i.e. $C_c^\infty(\Omega)$, are dense in $L^p(\Omega)$ for all $p\in[1,\infty)$.
	\end{itemize}
	\item[(c)] H\"older inequality: If $p,q\in[1,\infty]$ are such that $\frac{1}{p}+\frac{1}{q}=1$ and $u\in L^p(\Omega)$, $v\in L^q(\Omega)$, then
	\[\lVert u\cdot v\Vert_{L^1(\Omega)}\leq\lVert u\rVert_{L^p(\Omega)}\cdot\lVert v\Vert_{L^q(\Omega)}.\]
	In particular, if $\Omega$ is bounded, then $L^p(\Omega)\subset L^{\tilde{p}}(\Omega)$ for $p>\tilde{p}$.
	\item[(d)] For $p=2$, $L^2(\Omega)$ is a Hilbert space with scalar product
	\[\langle u,v\rangle=\int_\Omega{u(x)v(x)\mathrm{d}x}.\]
	\item[(e)] Riesz-Fischer representation theorem: Let $p\in[1,\infty)$ and $q\in(1,\infty]$ with $\frac{1}{p}+\frac{1}{q}=1$. Then $(L^p(\Omega))'$ is isometrically isomorphic to $L^q(\Omega)$: For every $\varphi\in(L^p(\Omega))'$ there exists a unique $v_\varphi\in L^q(\Omega)$ such that
	\[\varphi(u)=\int_\Omega{u(x)v_\varphi(x)\mathrm{d}x}\]
	for all $u\in L^p(\Omega)$, and $\lVert\varphi\rVert_{(L^p(\Omega))'}=\lVert v_\varphi\rVert_{L^q(\Omega)}$. In particular, $(L^p(\Omega))''=L^p(\Omega)$ for $p\in(1,\infty)$ and hence $L^p(\Omega)$ is reflexive if $p\in(1,\infty)$.
	\item[(f)] One the one hand we have $(L^1(\Omega))'\simeq L^\infty(\Omega)$, but $L^1(\Omega)$ is a proper subspace of $(L^\infty(\Omega))'$ in general, so that $L^p(\Omega)$ is not reflexive for $p\in\{1,\infty\}$.
	\item[(g)] Characterization of weak convergence in $L^p(\Omega)$ for $p\in(1,\infty)$: A sequence $(u_n)_{n\in\mathbb{N}}\subset L^p(\Omega)$ converges weakly to $u\in L^p(\Omega)$ in $L^p(\Omega)$ if and only if $\lVert u_n\rVert_{L^p(\Omega)}\leq C$ and for all $\varphi\in C_c^\infty(\Omega)$ it holds
	\[\int_\Omega{u_n(x)\varphi(x)\mathrm{d}x}\to\int_\Omega{u(x)\varphi(x)\mathrm{d}x}.\]
	We can also replace $C_c^\infty(\Omega)$ by $S(\Omega)$, and thanks to linearity of the integral we can even reduce the test-functions to characteristic functions which then is just integration over subsets.\\[11pt]
\end{itemize}

\textbf{\underline{Definition 3.3.1}}\\
(Weak derivative)\\
Let $u\in L_\text{loc}^1(\Omega):=\{u:\Omega\longrightarrow\mathbb{R}\mid u|_\omega\in L^1(\omega)\text{ for all }\omega\subset\mathrel\subset\Omega\}$, where $\omega\subset\mathrel\subset\Omega$ means that $\omega$ is a compact subset of $\Omega$. Let $\alpha=(\alpha_1,\dotsc,\alpha_d)\in\mathbb{N}_0^d$ be a multi-index. A function $v_\alpha\in L_\text{loc}^1(\Omega)$ is called an \textit{$\alpha$-th weak derivative of $u$} if
\[\int_\Omega{u(x)D^\alpha\varphi(x)\mathrm{d}x}=(-1)^{\lvert\alpha\rvert}\int_\Omega{v_\alpha(x)\varphi(x)\mathrm{d}x}\]
for all $\varphi\in C_c^\infty(\Omega)$, where $\lvert\alpha\rvert:=\sum_{i=1}^d{\alpha_i}$ and $D^\alpha\varphi=\frac{\partial^{\lvert\alpha\rvert}\varphi}{\partial x_1^{\alpha_1}\cdots\partial x_d^{\alpha_d}}$. We set $D^\alpha u:=v_\alpha$. This is justified by \hyperlink{remark_3_3_2}{Remark 3.3.2 (b)}.\\[11pt]

\hypertarget{remark_3_3_2}{\textbf{Remark 3.3.2}}
\begin{itemize}
	\item[(a)] If $u$ is differentiable in the classical sense, then the weak derivative coincides with the classical derivative.
	\item[(b)] The weak derivative is unique. So, if $v_\varphi$ and $w_\varphi$ are $\alpha$-th weak derivatives of $u$, then
	\[\int_\Omega{v_\alpha(x)\varphi(x)\mathrm{d}x}=(-1)^{\lvert\alpha\rvert}\int_\Omega{u(x)D^\alpha\varphi(x)\mathrm{d}x}=\int_\Omega{w_\alpha(x)\varphi(x)\mathrm{d}x}\]
	for all $\varphi\in C_c^\infty(\Omega)$. Therefore, $\int_\Omega{(v_\alpha(x)-w_\alpha(x))\varphi(x)\mathrm{d}x}=0$ for all $\varphi\in C_c^\infty(\Omega)$, so $v_\alpha-w_\alpha=0$ almost everywhere by the fundamental lemma of calculus of variations (in a version for locally integrable functions).\\[11pt]
\end{itemize}

\textbf{Example 3.3.3}\\
Let $u:\mathbb{R}\longrightarrow\mathbb{R}$, $u(x):=\lvert x\rvert$. Then $u$ is continuous, but not (classically) differentiable in 0. But $u$ is weakly differentiable, and we want to compute its weak derivative. We would expect that the weak derivative coincides (almost everywhere) on the set of points where $u$ is classically differentiable. So, we would expect that the weak derivative is $v(x)=\sgn{x}$ for $x\ne0$, and it doesn't matter how $v$ is defined at zero because pointwise it is only defined almost everywhere. For $\varphi\in C_c^\infty(\mathbb{R})$, integration by parts yields
\begin{align*}
	\int_\mathbb{R}{u(x)\varphi'(x)\mathrm{d}x}&=\int_{-\infty}^0{-x\varphi'(x)\mathrm{d}x}+\int_0^\infty{x\varphi'(x)\mathrm{d}x}\\
	&=-0\cdot\varphi(0)+\lim_{x\to-\infty}{(-x\varphi(x))}+\int_{-\infty}^0{1\cdot\varphi(x)\mathrm{d}x}\\
	&\qquad\qquad+\lim_{x\to+\infty}{x\varphi(x)}-0\cdot\varphi(0)-\int_0^\infty{1\cdot\varphi(x)\mathrm{d}x}\\
	&=-\int_\mathbb{R}{\sgn{x}\varphi(x)\mathrm{d}x}=-\int_\mathbb{R}{v(x)\varphi(x)\mathrm{d}x}.
\end{align*}
Hence, $v$ is the weak derivative of $u$, i.e. $u'=v$ in the weak sense.\\

We can ask further if $v$ is weakly differentiable. It is not continuous at 0, but on $\mathbb{R}\setminus\{0\}$ this function is infinitely many often differentiable. If $v$ would be weakly differentiable, then its weak derivative $w=v'$ would satisfy
\begin{align*}
	-\int_\mathbb{R}{w(x)\varphi(x)\mathrm{d}x}&=\int_\mathbb{R}{v(x)\varphi'(x)\mathrm{d}x}\\
	&=\int_{-\infty}^0{-\varphi'(x)\mathrm{d}x}+\int_0^\infty{\varphi'(x)\mathrm{d}x}\\
	&=-\varphi(0)+\lim_{x\to-\infty}{\varphi(x)}+\lim_{x\to+\infty}{\varphi(x)}-\varphi(0)=-2\varphi(0)
\end{align*}
for all $\varphi\in C_c^\infty(\mathbb{R})$. This would mean that $w$ represents a Dirac measure which is impossible for an $L_\text{loc}^1(\mathbb{R})$-function. Therefore, $v$ is \textit{not} weakly differentiable.\\[11pt]

\textbf{\underline{Definition 3.3.4}}\\
(Sobolev space)\\
Let $\Omega\subset\mathbb{R}^d$ be open, $p\in[1,\infty]$, $k\in\mathbb{N}$. Define
\[W^{k,p}(\Omega):=\left\{u\in L^p(\Omega)\,\middle\vert\,\begin{array}{c}
	\text{For all }\alpha\in\mathbb{N}_0^d\text{ with }\lvert\alpha\rvert\leq k,\text{ the weak}\\
	\text{derivative }D^\alpha u\text{ exists and }D^\alpha u\in L^p(\Omega).
\end{array}\right\}\]
with norm
\[\lVert u\rVert_{W^{k,p}(\Omega)}:=\left(\sum_{\alpha\in\mathbb{N}_0^d,\lvert\alpha\rvert\leq k}{\lVert D^\alpha u\rVert_{L^p(\Omega)}^p}\right)^{1/p}\]
if $p<\infty$, and
\[\lVert u\rVert_{W^{k,\infty}(\Omega)}:=\max\left\{\lVert D^\alpha u\rVert_{L^\infty(\Omega)}\,\middle\vert\,\alpha\in\mathbb{N}_0^d,\lvert\alpha\rvert\leq k\right\}.\]
The definition makes sense because $L^p(\Omega)\subset L_\text{loc}^1(\Omega)$.\\

Moreover, we set $W_0^{k,p}(\Omega):=\overline{C_c^\infty(\Omega)}^{W^{k,p}(\Omega)}$, i.e. $W_0^{k,p}(\Omega)$ is the closure of $C_c^\infty(\Omega)$ in $W^{k,p}(\Omega)$ with respect to norm $\lVert\cdot\rVert_{W^{k,p}(\Omega)}$. We also abbreviate $H^k(\Omega):=W^{k,2}(\Omega)$.\\[11pt]

\textbf{Remark 3.3.5}\\
(most important properties of Sobolev spaces)\\
The following can be looked up in the literature, for instance in \cite[Chapter 3 ??]{adams_fournier} or \cite[Part II, 5. Sobolev Spaces]{lawrence_evans}.
\begin{itemize}
	\item[(a)] $(W^{k,p}(\Omega),\lVert\cdot\rVert_{W^{k,p}(\Omega)})$ is a Banach space for all $p\in[1,\infty]$, $k\in\mathbb{N}_0$.
	\item[(b)] $(H^k(\Omega),\langle\cdot,\cdot\rangle_k)$ are Hilbert spaces with scalar product
	\[\langle u,v\rangle_k:=\sum_{\alpha\in\mathbb{N}_0^d,\lvert\alpha\rvert\leq k}{\langle D^\alpha u,D^\alpha v\rangle_{L^2(\Omega)}}.\]
	\item[(c)] $W_0^{k,p}(\Omega)$ is a Banach space.
	\item[(d)] For $p\in(1,\infty)$, the dual space of $W^{k,p}(\Omega)$ is isomorphic to $W^{k,q}(\Omega)$, where $\frac{1}{p}+\frac{1}{q}=1$. More precisely, for all $\varphi\in(W^{k,p}(\Omega))'$ there is a unique $w\in W^{k,q}(\Omega)$ such that for all $u\in W^{k,p}(\Omega)$ we have
	\[\varphi(u)=\sum_{\alpha\in\mathbb{N}_0^d,\lvert\alpha\vert\leq k}{\int_\Omega{D^\alpha u(x)D^\alpha w(x)\mathrm{d}x}}.\]
	In particular, $W^{k,p}(\Omega)$ is reflexive for $p\in(1,\infty)$. This is in the same spirit as for the Lebesgue spaces.
	\item[(e)] For $p\in[1,\infty)$, a sequence $(u_n)_{n\in\mathbb{N}}\subset W^{k,p}(\Omega)$ converges weakly to $u\in W^{k,p}(\Omega)$ if and only if $D^\alpha u_n\rightharpoonup D^\alpha u$ in $L^p(\Omega)$ as $n\to\infty$ for all $\alpha\in\mathbb{N}_0^d$ with $\lvert\alpha\rvert\leq k$.
	\item[(f)] For $p\in[1,\infty)$,
	\begin{itemize}
		\item[(1)] $W^{k,p}(\Omega)\cap C^\infty(\Omega)$ is dense in $W^{k,p}(\Omega)$;
		\item[(2)] $C^\infty(\overline{\Omega})=\{v|_\Omega\mid v\in C^\infty(\mathbb{R}^d)\}$ is dense in $W^{k,p}(\Omega)$ if $\Omega\subset\mathbb{R}^d$ is a bounded domain with Lipschitz boundary (i.e. $\partial\Omega$ is locally the graph of a Lipschitz continuous function).
	\end{itemize}
	\item[(g)] If $\Omega\subset\mathbb{R}^d$ is bounded, then $W^{1,\infty}(\Omega)$ is isomorphic to the space of Lipschitz continuous functions.
	\item[(h)] In general, $W_0^{k,p}(\Omega)\subsetneq W^{k,p}(\Omega)$. However, $W_0^{k,p}(\mathbb{R}^d)=W^{k,p}(\mathbb{R}^d)$ if $p<\infty$.\\[11pt]
\end{itemize}

We further need embedding theorems for Sobolev spaces. Here, the Sobolev number $k^*:=k-\frac{d}{p}$ plays a crucial role, where as before $k$ denotes the order of differentiability, $p$ the order of integrability and $d$ the dimension.\\

To understand where this number comes from, we look as a motivation at $\Omega=(0,1)$, $u_a(x)=x^a$ for some $a\in\mathbb{R}$. It depends on $a$ (especially when $a<0$) if $u_a\in L^p(\Omega)$ for some $p$. When do we have $u_a\in W^{k,p}(0,1)$? One can show that the crucial point somehow is the highest order derivative, so $u_a\in W^{k,p}(0,1)$ if and only if
\[\int_0^1{\lvert u_a^{(k)}(x)\rvert^p\mathrm{d}x}=\int_0^1{x^{(a-k)p}\mathrm{d}x}<\infty.\]
This is finite if $(a-k)p>-1$, i.e. if $a>k-\frac{1}{p}=k^*$ for ($d=1$). Similarly on can show $(x\mapsto\lvert x\rvert^a)\in W^{k,p}((0,1)^d)$ if and only if $a>k-\frac{d}{p}=k^*$.\\[11pt]

\textbf{\underline{Definition 3.3.6}}\\
(H\"older space)\\
Let $\Omega\subset\mathbb{R}^d$ be open and bounded. A function $u:\Omega\longrightarrow\mathbb{R}$ is called \textit{H\"older continuous} for exponent $\gamma\in(0,1]$ if
\[[u]_\gamma:=\sup\left\{\frac{\lvert u(x)-u(y)\rvert}{\lvert x-y\rvert^\gamma}\,\middle\vert\,x,y\in\Omega,x\ne y\right\}<\infty.\]
For $k\in\mathbb{N}_0$, we define the space
\[C^{k,\gamma}(\overline{\Omega}):=\left\{u\in C^k(\overline{\Omega})\,\middle\vert\,[D^\alpha u]_\gamma<\infty\text{ for all }\alpha\in\mathbb{N}_0^d,\lvert\alpha\rvert\leq k\right\},\]
endowed with the norm
\[\lVert u\rVert_{C^{k,\gamma}(\overline{\Omega})}:=\sum_{\alpha\in\mathbb{N}_0^d,\lvert\alpha\rvert\leq k}{\lVert D^\alpha u\rVert_{C^0(\overline{\Omega})}}+\sum_{\alpha\in\mathbb{N}_0^d,\lvert\alpha\rvert=k}{[D^\alpha u]_\gamma}.\]\\
	
\textbf{Remark 3.3.7}\\
One can show that $(C^{k,\gamma}(\overline{\Omega}),\lVert\cdot\rVert_{C^{k,\gamma}(\overline{\Omega})})$ is a Banach space, but it is not reflexive. Also note that the norm is formed with the $C^0$-norms over all derivatives, while the sum over $[D^\alpha u]_\gamma$ is only taken over the derivatives of order $k$.\\[11pt]

\textbf{\underline{Definition 3.3.8}}\\
Let $(X,\lVert\cdot\rVert_X)$, $(Y,\lVert\cdot\rVert_Y)$ be normed spaces and $X\subseteq Y$. We say that \textit{$X$ embeds continuously (compactly) into $Y$} if the identical embedding $\iota:X\longrightarrow Y$, $x\longmapsto x$, is a continuous (compact) operator.\\

For continuous embeddings we write $X\longhookrightarrow Y$, and for compact embeddings $X\clonghookrightarrow Y$.\\[11pt]

\textbf{Remark 3.3.9}
\begin{itemize}
	\item[(a)] $X$ embeds continuously into $Y$ if there exists $C>0$ such that $\lVert x\rVert_Y\leq C\lVert x\rVert_X$ for all $x\in X$.
	\item[(b)] $X$ embeds compactly into $Y$ if every weakly convergent sequence in $X$ converges strongly in $Y$, i.e. if $(u_n)_{n\in\mathbb{N}}\subset X$, $u\in X$, then $u_n\rightharpoonup u$ in $X$ implies $u_n\to u$ in $Y$.\\[11pt]
\end{itemize}

\hypertarget{theorem_3_3_10}{\textbf{\underline{Theorem 3.3.10}}}\\
(Sobolev embedding theorem)\\
Let $\Omega\subset\mathbb{R}^d$ be a (not necessarily bounded) domain with Lipschitz boundary $\partial\Omega$. Let $\tilde{k},k\in\mathbb{N}_0$ with $k\geq\tilde{k}$.
\begin{itemize}
	\item[(i)] If $p,\tilde{p}\in[1,\infty)$ with $\tilde{p}\geq p$ and $k-\frac{d}{p}\geq\tilde{k}-\frac{d}{\tilde{p}}$, then $W^{k,p}(\Omega)\longhookrightarrow W^{\tilde{k},\tilde{p}}(\Omega)$.
	\item[(ii)] If $p\in[1,\infty)$ and $\gamma\in(0,1)$ are such that $k-\frac{d}{p}\geq\tilde{k}+\gamma$, then $W^{k,p}(\Omega)\longhookrightarrow C^{\tilde{k},\gamma}(\overline{\Omega})$.\\
\end{itemize}

\textit{Proof:}\\
This is treated in the course \textit{Partial Differential Equations}, but the proof can also be looked up in appropriate literature, e.g. \cite[Kapitel 8, 8.9 Einbettungssatz in Sobolev-R\"aumen, 8.13 Einbettungssatz von Sobolev-R\"aumen in H\"older-R\"aume]{hans_wilhelm_alt} for bounded domains.\hfill$\blacksquare$\\[11pt]

\textbf{\underline{Theorem 3.3.11}}\\
(Rellich compactness theorem)\\
If $\Omega\subset\mathbb{R}^d$ is a \textit{bounded} domain with Lipschitz boundary, and if we have strict inequalities in \hyperlink{theorem_3_3_10}{Theorem 3.3.10}, i.e. $k-\frac{d}{p}>\tilde{k}-\frac{d}{p}$ or $k-\frac{d}{p}>\tilde{k}+\gamma$, then the above embeddings are compact.\\

\textit{Proof:}\\
As before, we refer to the literature.\hfill$\blacksquare$\\[11pt]

\textbf{Examples 3.3.12}
\begin{itemize}
	\item[(a)] Consider $k=1$, $p=2$.
	\begin{itemize}
		\item[(1)] If $d=1$, then the Sobolev number is $1-\frac{1}{2}=\frac{1}{2}$. So in case (i), for $\tilde{k}=0$ we can choose any $\tilde{p}\in[2,\infty)$ to obtain an embedding $H^1(\Omega)\longhookrightarrow L^{\tilde{p}}(\Omega)$, and in (ii) we have $H^1(\Omega)\longhookrightarrow C^{0,\gamma}(\overline{\Omega})$ for $\gamma\leq\frac{1}{2}$.
		\item[(2)] If $d=2$, then the Sobolev number is $1-\frac{2}{2}=0$. So case (ii) is no longer applicable, but we still have $H^1(\Omega)\longhookrightarrow L^{\tilde{p}}(\Omega)$ for $\tilde{p}\in[2,\infty)$.
		\item[(3)] If $d=3$, the then Sobolev number is $1-\frac{3}{2}=-\frac{1}{2}$. This gives $H^1(\Omega)\longhookrightarrow L^{\tilde{p}}(\Omega)$ for $\tilde{p}\in[2,6]$.
	\end{itemize}
	If $\Omega$ is bounded, the embeddings $H^1(\Omega)\longhookrightarrow L^{\tilde{p}}(\Omega)$ also hold for $\tilde{p}\in[1,2)$ by H\"older's inequality.
	\item[(b)] For $k=1$ and $p\in[1,\infty)$, we have the continuous embedding $W^{1,p}(\Omega)\longhookrightarrow L^{\tilde{p}}(\Omega)$ for
	\begin{itemize}
		\item[(1)] $p<d$ and $p\leq\tilde{p}\leq\frac{dp}{d-p}$,
		\item[(2)] $p=d$ and $p\leq\tilde{p}<\infty$,
		\item[(3)] $p>d$ and $p\leq\tilde{p}\leq\infty$. (Infinity via the embedding $C^{0,\gamma}(\overline{\Omega})\longhookrightarrow C^0(\overline{\Omega})\longhookrightarrow L^\infty(\Omega)$.)
	\end{itemize}
	If we have $p>d$, we further have $W^{1,p}(\Omega)\longhookrightarrow C^{0,\gamma}(\overline{\Omega})$ with $\gamma\leq1-\frac{d}{p}$.\\

	These embeddings are compact if $\Omega$ is bounded and if $\tilde{p}<\frac{dp}{d-p}$ in case $p<d$, or if $\gamma<1-\frac{d}{p}$ in case $p>d$. In particular, if $u_n\rightharpoonup u_*$ in $W^{1,p}(\Omega)$ for some $(u_n)_{n\in\mathbb{N}}\subset W^{1,p}(\Omega)$ and $u_*\in W^{1,p}(\Omega)$, then
	\begin{itemize}
		\item[(i)] $u_n\to u_*$ in $L^{\tilde{p}}(\Omega)$ if $\tilde{p}<d$ and $\frac{1}{\tilde{p}}>\frac{1}{p}-\frac{1}{d}$,
		\item[(ii)] $u_n\to u$ in $C^0(\overline{\Omega})$ if $p>d$, i.e. uniform convergence.\\[11pt]
	\end{itemize}
\end{itemize}

In \hyperref[chap:mcov_chap2]{Chapter II. Classical Methods}, functionals were defined on classes of functions $u$ with prescribed boundary values $u|_{\Gamma_D}$ with $\Gamma_D\subset\partial\Omega$. Since $\Gamma_D$ has Lebesgue measure zero, the expression $u|_{\Gamma_D}$ does not make sense for $u\in L^p(\Omega)$. However, for Sobolev functions $u$, there is a notion of so-called ``trace'' $u|_{\partial\Omega}$.\\

\textbf{\underline{Theorem 3.3.13}}\\
(Trace theorem)\\
Let $\Omega\subset\mathbb{R}^d$ be a bounded domain with Lipschitz boundary, and let $p,q\in[1,\infty]$ such that
\[\left\{\begin{array}{rl}
	q<\frac{(d-1)p}{d-p}&\text{if }p<d,\\
	q<\infty&\text{if }d=p,\\
	q\leq\infty&\text{if }p>d.
\end{array}\right.\]
(In particular, $q=p$ is allowed in all three cases.) Then the operator $C^1(\overline{\Omega})\longrightarrow C^0(\partial\Omega)$, $u\longmapsto u|_{\partial\Omega}$ has a unique extension to a bounded linear operator $\gamma:W^{1,p}(\Omega)\longrightarrow L^q(\partial\Omega)$, called \textit{trace operator}, where $\partial\Omega$ is equipped with surface measure. This means, $\gamma$ is a bounded linear operator with $\gamma(u)=u|_{\partial\Omega}$ for all $u\in C^1(\overline{\Omega})$.\\

\textit{Proof:}\\
As before, this is treated in \textit{Partial Differential Equations} in detail, but can be looked up in the literature as well.\hfill$\blacksquare$\\[11pt]

\textbf{Remark 3.3.14}\\
If $\Omega\subset\mathbb{R}^d$ is a bounded Lipschitz domain, then $\{u\in W^{1,p}(\Omega)\mid\gamma(u)=0\}=W_0^{1,p}(\Omega)$.\\[11pt]

\textbf{\underline{Theorem 3.3.15}}\\
(Poincar\'e's inequality)\\
Let $\Omega\subset\mathbb{R}^d$ be open and bounded, $p\in[1,\infty)$. Then for all $u\in W_0^{1,p}(\Omega)$ it holds
\[\lVert u\rVert_{L^p(\Omega)}\leq\diam{\Omega}\lVert\nabla u\rVert_{L^p(\Omega;\mathbb{R}^d)},\]
where $\diam{\Omega}:=\sup\{\lvert x-y\rvert:x,y\in\Omega\}$.\\

\textit{Proof:}\\
Since $C_c^\infty(\Omega)$ is dense in $W_0^{1,p}(\Omega)$ by definition, it suffices to consider $u\in C_c^\infty(\Omega)$ instead of $u\in W_0^{1,p}(\Omega)$. Let $\delta:=\diam{\Omega}$, and extend $u$ by zero to $\mathbb{R}^d$. For $x\in\Omega$ we obtain
\[u(x)=u(x)-u(x-\delta e_1)=\int_0^1{\frac{\mathrm{d}}{\mathrm{d}s}[u(x-(1-s)\delta e_1)]\mathrm{d}s}=\int_0^1{\delta\partial_1u(x-(1-s)\delta e_1)\mathrm{d}s},\]
hence, by H\"older's inequality,
\begin{align*}
	\lvert u(x)\rvert^p&\leq\delta^p\left(\left(\int_0^1{1^{p'}\mathrm{d}x}\right)^{1/p'}\left(\int_0^1{\lvert\partial_1u(x-(1-s)\delta e_1)\rvert^{p}\mathrm{d}s}\right)^{1/p}\right)^p\\
	&\leq\delta^p\int_0^1{\lvert\nabla u(x-(1-s)\delta e_1)\rvert^p\mathrm{d}s}.
\end{align*}
Integrating over $\Omega$ yields, together with Fubini,
\begin{align*}
	\int_\Omega{\lvert u(x)\rvert^p\mathrm{d}x}&\leq\delta^p\int_0^1{\int_\Omega{\lvert\nabla u(x-(1-s)\delta e_1)\rvert^p\mathrm{d}x}\mathrm{d}s}\\
	&\leq\delta^p\int_0^1{\int_\Omega{\lvert\nabla u(y)\rvert^p\mathrm{d}y}\mathrm{d}s}\\
	&=\delta^p\int_\Omega{\lvert\nabla u(y)\rvert^p\mathrm{d}y}.
\end{align*}
\hfill$\blacksquare$\\[11pt]

\textbf{Remark 3.3.16}
\begin{itemize}
	\item[(a)] The proof can be changed by choosing another $i$-th unit vector than $e_1$. That's why one could expect that Poincar\'e's inequality even hold true for more general domains, e.g., if $\Omega$ lies between two hyperplanes. So in particular, $\Omega$ could be unbounded. Indeed, one can adapt this proof for such domains, but then the constant $\diam{\Omega}$ has to be changed to a certain distance between the hyperplanes.
	\item[(b)] Poincar\'e's inequality shows
	\[\lVert\nabla u\rVert_{L^p(\Omega;\mathbb{R}^d)}\leq\lVert u\rVert_{W^{1,p}(\Omega)}\leq\diam{\Omega}\lVert\nabla u\rVert_{L^p(\Omega;\mathbb{R}^d)}\]
	for all $u\in W_0^{1,p}(\Omega)$. This means that $\lVert\cdot\rVert_{W^{1,p}(\Omega)}$ and $\lVert\nabla(\cdot)\rVert_{L^p(\Omega;\mathbb{R}^d)}$ are equivalent norms on $W_0^{1,p}(\Omega)$.
\end{itemize}
\leavevmode
    \section{Integral Functionals in the Convex Case}
Consider the minimization problem for
\[I(u)=\int_\Omega{f(x,u(x),\nabla u(x))\mathrm{d}x}\]
on a suitable subset of the Sobolev space $W^{1,p}(\Omega;\mathbb{R}^m)$, for $1<p<\infty$. For example, on $W_0^{1,p}(\Omega;\mathbb{R}^m)$ or on $u_0+W_0^{1,p}(\Omega;\mathbb{R}^m)$, where $u_0\in W^{1,p}(\Omega;\mathbb{R}^m)$. In order to apply our abstract theory, we need to ask under which assumptions on $f:\Omega\times\mathbb{R}^m\times\mathbb{R}^{m\times d}\longrightarrow\mathbb{R}_\infty$ we have
\begin{itemize}
	\item[(a)] well-definedness for $I$ on $W^{1,p}(\Omega;\mathbb{R}^m)$,
	\item[(b)] coercivity of $I$ on $W^{1,p}(\Omega;\mathbb{R}^m)$ and
	\item[(c)] weak lower semicontinuity of $I$.\\
\end{itemize}

What could fail for well-definedness? First we need to ensure that $x\longmapsto f(x,u(x),\nabla u(x))$ is measurable. The second thing is that the integrals of the positive and negative part of $f(x,u(x),\nabla u(x))$ both may be infinite. But we even want more, namely, that $I(u)\ne-\infty$. This breaks down to having a finite negative part. To ensure well-definedness, we make the following assumption.\\

\textbf{\underline{Definition 3.4.1}}\\
(Constantin Carath\'eodory, * 1873, $\dagger$ 1950)\\
If $\Omega\subset\mathbb{R}^d$ is an open measurable set, then $f:\Omega\times\mathbb{R}^m\times\mathbb{R}^{m\times d}\longrightarrow\mathbb{R}$ is called a \textit{Carath\'eodory function} if
\begin{itemize}
	\item[(i)] $f(x,\cdot,\cdot)$ is continuous for almost all $x\in\Omega$,
	\item[(ii)] $f(\cdot,u,A)$ is measurable for all $(u,A)\in\mathbb{R}^m\times\mathbb{R}^{m\times d}$.\\
\end{itemize}

There is also a weaker notion which uses a so-called normal integrand and where continuity is replaced by lower semicontinuity.\\[11pt]

\hypertarget{lemma_3_4_2}{\textbf{Lemma 3.4.2}}\\
Let $f:\Omega\times\mathbb{R}^m\times\mathbb{R}^{m\times d}\longrightarrow\mathbb{R}$ be a Carath\'eodory function.
\begin{itemize}
	\item[(a)] If $u:\Omega\longrightarrow\mathbb{R}^m$, $A:\Omega\longrightarrow\mathbb{R}^{m\times d}$ are measurable, then
	\[\Omega\longrightarrow\mathbb{R}_\infty,\qquad x\longmapsto f(x,u(x),A(x))\]
	is measurable.
	\item[(b)] Let $p\in[1,\infty)$ and $a\in L^1(\Omega)$, $b\in L^\infty(\Omega)$ be such that
	\[f(x,u,A)\geq a(x)+b(x)(\lvert u\rvert^p+\lvert A\rvert^p)\]
	for all $(u,A)\in\mathbb{R}^m\times\mathbb{R}^{m\times d}$ and almost all $x\in\Omega$. Then
	\[I(u)=\int_\Omega{f(x,u(x),\nabla u(x))\mathrm{d}x}\in(-\infty,\infty]\]
	is well-defined for all $u\in W^{1,p}(\Omega;\mathbb{R}^m)$.\\
\end{itemize}

\textit{Remark: This lemma gives us well-definedness for the integral functional from the beginning.}\\

\textit{Proof:}
\begin{itemize}
	\item[(a)] At first, let $(u,A)$ be a simple function, i.e.
	\[u(x)=\sum_{i=1}^M{\alpha_i\chi_{B_i}(x)}\quad\text{and}\quad A(x)=\sum_{i=1}^M{\beta_i\chi_{B_i}(x)}\]
	for measurable $B_1,\dotsc,B_M\subset\Omega$, $\Omega=\dot\bigcup_{i=1}^M{B_i}$, $B_i\cap B_j=\emptyset$ for $i\ne j$, and $\alpha_i\in\mathbb{R}^m$, $\beta_i\in\mathbb{R}^{m\times d}$. Let $h(x):=f(x,u(x),A(x))$, then we want to see that $h$ is measurable. For $\gamma\in\mathbb{R}$ we have
	\[\{x\in\Omega\mid h(x)\leq\gamma\}=\bigcup_{i=1}^M{\{x\in B_i\mid f(x,\alpha_i,\beta_i)\leq\gamma\}},\]
	where all sets in the union are measurable because $f$ is a Carath\'eodory function. So $h^{-1}((-\infty,\gamma])\in\mathcal{B}(\Omega)$, i.e. $h$ is measurable.\\

	Now let $(u,A)$ be measurable. Then $(u,A)$ is the pointwise limit of simple functions $(u_n,A_n)_{n\in\mathbb{N}}$. Since $f$ is continuous in the second and third argument,
	\[h(x):=f(x,u(x),A(x))=\lim_{n\to\infty}{f(x,u_n(x),A_n(x))}\]
	for almost all $x\in\Omega$, so $h$ is measurable as the limit of measurable functions.
	\item[(b)] For $u\in W^{1,p}(\Omega;\mathbb{R}^m)$ let $F(x):=a(x)+b(x)(\lvert u(x)\rvert^p+\lvert\nabla u(x)\rvert^p)$. Then
	\begin{align*}
		\int_\Omega{\lvert F(x)\rvert\mathrm{d}x}&\leq\int_\Omega{\lvert a(x)\rvert\mathrm{d}x}+\int_\Omega{\lvert b(x)\rvert(\lvert u(x)\rvert^p+\lvert\nabla u(x)\rvert^p)\mathrm{d}x}\\
		&\leq\lVert a\rVert_{L^1(\Omega)}+\lVert b\rVert_{L^\infty(\Omega)}\int_\Omega{(\lvert u(x)\rvert^p+\lvert\nabla u(x)\rvert^p)\mathrm{d}x}<\infty.
	\end{align*}
	Hence $F\in L^1(\Omega)$. Therefore, the negative part of $f(\cdot,u(\cdot),\nabla u(\cdot))$ is integrable.\hfill$\blacksquare$\\[11pt]
\end{itemize}

For coercivity, we need $I(u)\to+\infty$ as $\lVert u\rVert_{W^{1,p}(\Omega)}\to+\infty$. A sufficient condition for that would be $f(x,u,A)\geq c(\lvert A\rvert^p+\lvert u\rvert^p)-\gamma(x)$ for some $c>0$ and $\gamma\in L^1(\Omega)$, since then
\[I(u)\geq c\int_\Omega{\lvert\nabla u(x)\rvert^p+\lvert u(x)\rvert^p\mathrm{d}x}-\lVert\gamma\rVert_{L^1(\Omega)}\to+\infty\]
as $\lVert u\rVert_{W^{1,p}(\Omega)}\to+\infty$. However there is a problem. Very often, $f$ is independent of $u$. As a simple example we may consider
\[I(u)=\int_\Omega{\frac{1}{2}\lvert\nabla u(x)\rvert^p\mathrm{d}x}\]
which corresponds to the Laplace equation $-\Delta u=0$, but the above condition is not satisfied. There is a way out by including Poincar\'e's inequality.\\[11pt]

\hypertarget{theorem_3_4_3}{\textbf{\underline{Theorem 3.4.3}}}\\
(Sufficient condition for coercivity)\\
Let $\Omega\subset\mathbb{R}^d$ be open and bounded, and let $f:\Omega\times\mathbb{R}^m\times\mathbb{R}^{m\times d}\longrightarrow\mathbb{R}$ be a Carath\'eodory function that satisfies
\[f(x,u,A)\geq c\lvert A\rvert^p-\delta(x)\lvert u\rvert^p-h(x)\]
for all $(x,u,A)\in\Omega\times\mathbb{R}^m\times\mathbb{R}^{m\times d}$, where $p\in[1,\infty)$, $c>0$, $r\in[1,p)$, $\delta\in L^{\frac{p}{p-r}}(\Omega)$, $h\in L^1(\Omega)$. Then there exist constants $\tilde{c}>0$, $\beta\in\mathbb{R}$ such that
\[I(u):=\int_\Omega{f(x,u(x),\nabla u(x))\mathrm{d}x}\geq\tilde{c}\lVert u\rVert_{W^{1,p}(\Omega)}^p+\beta\]
for all $u\in W_0^{1,p}(\Omega;\mathbb{R}^m)$. In particular, $I$ is coercive on $W_0^{1,p}(\Omega;\mathbb{R}^m)$.\\

\textit{Remark: $W_0^{1,p}(\Omega;\mathbb{R}^m)$ can be replaced with any other subspace of $W^{1,p}(\Omega)$ where the Poincar\'e inequality holds.}\\

\textit{Proof:}\\
For $u\in W_0^{1,p}(\Omega;\mathbb{R}^m)$ we have with H\"older's inequality (applied to $\frac{p}{p-r}$ and $\frac{p}{r}$) in the second, Poincar\'e in the third line
\begin{align*}
	I(u)&\geq c\int_\Omega{\lvert\nabla u(x)\vert^p\mathrm{d}x}-\int_\Omega{\delta(x)\lvert u(x)\rvert^r\mathrm{d}x}-\int_\Omega{h(x)\mathrm{d}x}\\
	&\geq c\lVert\nabla u\rVert_{L^p(\Omega)}^p-\lVert\delta\rVert_{L^{\frac{p}{p-r}}(\Omega)}\lVert u\rVert_{L^p(\Omega)}^r-\lVert h\rVert_{L^1(\Omega)}\\
	&\geq c\lVert\nabla u\rVert_{L^p(\Omega)}^p-\lVert\delta\rVert_{L^{\frac{p}{p-r}}(\Omega)}\diam{\Omega}^r\lVert\nabla u\rVert_{L^p(\Omega)}^r-\lVert h\rVert_{L^1(\Omega)}.
\end{align*}
By Young's inequality ($ab\leq\frac{1}{p}a^p+\frac{1}{q}b^q$ for $\frac{1}{p}+\frac{1}{q}=1$, $a,b\geq0$) we have
\[\diam{\Omega}^r\lVert\delta\rVert_{L^{\frac{p}{p-r}}(\Omega)}\lVert\nabla u\rVert_{L^p(\Omega)}^r\leq\frac{r}{p}\lVert\nabla u\rVert_{L^p(\Omega)}^p\varepsilon^p+\underbrace{\frac{p-r}{p}\varepsilon^{-\frac{p}{p-r}}\diam{\Omega}^{\frac{pr}{p-r}}\lVert\delta\rVert_{L^{\frac{p}{p-r}}(\Omega)}^{\frac{p}{p-r}}}_{=:\tilde{\beta}}.\]
Choosing $\varepsilon$ small, e.g. such that $\varepsilon^p\leq\frac{c}{2}$, then we have
\[I(u)\geq\frac{c}{2}\lVert\nabla u\rVert_{L^p(\Omega)}^p-\tilde{\beta}-\lVert h\rVert_{L^1(\Omega)}\geq\tilde{c}\lVert u\rVert_{W^{1,p}(\Omega)}^p+\beta,\]
with $\tilde{c}=\frac{c}{4}\min\{1,\diam{\Omega}^{-p}\}$ and $\beta=-\tilde{\beta}-\lVert h\rVert_{L^1(\Omega)}$.\hfill$\blacksquare$\\[11pt]

\textbf{Example 3.4.4}\\
Let $\Omega\subset\mathbb{R}^d$ be a bounded Lipschitz domain and consider
\[I(u)=\int_\Omega{\frac{1}{2}\lvert\nabla u(x)\rvert^2\mathrm{d}x}.\]
Then $I$ is coercive in $W_0^{1,p}(\Omega;\mathbb{R}^m)$ for $p\in[1,2]$ by \hyperlink{theorem_3_4_3}{Theorem 3.4.3} since it holds $f(x,u,A)=\frac{1}{2}\lvert A\rvert^2\geq c_p\lvert A\rvert^p-\tilde{c}_p$, where $c_p,\tilde{c}_p$ will be determined now. This is again a setting for Young's inequality with exponents $\frac{2}{p}$, $\frac{2}{2-p}$, i.e. we have
\[\lvert A\rvert^p=\lvert A\rvert^p\cdot 1\leq\frac{p}{2}\lvert A\rvert^2+\frac{2-p}{2}\cdot 1^{\frac{2}{2-p}}.\]
So $c_p=\frac{1}{p}$ and $\tilde{c}_p=\frac{2-p}{2p}$.\\

But $I$ is not coercive for $p>2$. To see this, we take $u\in W_0^{1,2}(\Omega;\mathbb{R}^m)\setminus W_0^{1,p}(\Omega;\mathbb{R}^m)$ and $(u_n)_{n\in\mathbb{N}}\subset C_c^\infty(\Omega;\mathbb{R}^m)$ such that $u_n\to u$ in $W^{1,2}(\Omega;\mathbb{R}^m)$. Then
\[I(u_n)=\int_\Omega{\frac{1}{2}\lvert\nabla u_n(x)\rvert^2\mathrm{d}x}\to\int_\Omega{\frac{1}{2}\lvert\nabla u(x)\rvert^2\mathrm{d}x}=I(u)\]
for $n\to\infty$. If $I$ is coercive, we would have $\lVert u_n\rVert_{W^{1,p}(\Omega;\mathbb{R}^m)}\leq C$, and reflexivity implies the weak convergence of a subsequence $u_{n_k}\rightharpoonup\tilde{u}$ as $k\to\infty$ for some $\tilde{u}\in W_0^{1,p}(\Omega;\mathbb{R}^m)$. But then $u_{n_k}\to\tilde{u}$ in $W^{1,2}(\Omega;\mathbb{R})$ by Rellich's compact embedding. Thus, $u=\tilde{u}\in W_0^{1,p}(\Omega;\mathbb{R}^m)$ which is a contradiction to the choice of $u$.\\[11pt]

Finally, we address the question regarding weak lower semicontinuity.\\

\textbf{Remark 3.4.5}\\
The functional $I$ can be lower semicontinuous without being continuous, for example
\[I:W^{1,2}(\Omega)\longrightarrow\mathbb{R}_\infty,\qquad I(u):=\int_\Omega{\lvert\nabla u(x)\rvert^4\mathrm{d}x}.\]
Then $f(A)=\lvert A\rvert^4$ is continuous (hence $I$ is lower semicontinuous) and convex (hence $I$ is convex), so \hyperlink{theorem_3_2_6}{Theorem 3.2.6} tells us that $I$ is weakly lower semicontinuous. However, $I$ is not continuous in all elements $u\in W^{1,4}(\Omega)$. Indeed, let $b\in W^{1,2}(\Omega)\setminus W^{1,4}(\Omega)$ and define $u_\varepsilon=u+\varepsilon b\in W^{1,2}(\Omega)$ for $\varepsilon>0$. Then $u_\varepsilon\not\in W^{1,4}(\Omega)$ and $u_\varepsilon\to u$ in $W^{1,2}(\Omega)$ strongly, but $I(u_\varepsilon)=+\infty$ for all $\varepsilon>0$, while $I(u)<\infty$.\\[11pt]

Asking for continuity of $I$ might be too restrictive in some applications. To ensure weak lower semicontinuity, we can assume convexity.\\

\hypertarget{theorem_3_4_6}{\textbf{\underline{Theorem 3.4.6}}}\\
Let $\Omega\subset\mathbb{R}^d$ be open, bounded. Let $f:\Omega\times\mathbb{R}^m\times\mathbb{R}^{m\times d}\longrightarrow\mathbb{R}$ be a Carath\'eodory function such that
\begin{itemize}
	\item[(i)] the assignment $(u,A)\longmapsto f(x,u,A)$ is convex for almost all $x\in\Omega$,
	\item[(ii)] $f(x,u,A)\geq\gamma(x)$ for some $\gamma\in L^1(\Omega)$ and almost all $x\in\Omega$.
\end{itemize}
Then
\[I:W^{1,p}(\Omega;\mathbb{R}^m)\longrightarrow\mathbb{R}_\infty,\qquad I(u):=\int_\Omega{f(x,u(x),\nabla u(x))\mathrm{d}x}\]
defines a weakly lower semicontinuous functional for any $p\in[1,\infty]$.\\

\textit{Proof:}\\
Assumption (i) shows that $I$ is convex. Hence, by \hyperlink{theorem_3_2_6}{Theorem 3.2.6} it suffices to show that $I$ is lower semicontinuous. Let $(u_n)_{n\in\mathbb{N}}\subset W^{1,p}(\Omega;\mathbb{R}^m)$, $u\in W^{1,p}(\Omega;\mathbb{R}^m)$ such that $u_n\to u$ in $W^{1,p}(\Omega;\mathbb{R}^m)$ for $n\to\infty$.\\

We can find a subsequence $(u_{n_k})_{k\in\mathbb{N}}\subseteq(u_n)_{n\in\mathbb{N}}$ with $u_{n_k}(x)\to u(x)$ and $\nabla u_{n_k}(x)\to\nabla u(x)$ for almost all $x\in\Omega$ as $k\to\infty$. Continuity of $f(x,\cdot,\cdot)$ implies
\[f(x,u_{n_k}(x),\nabla u_{n_k}(x))\to f(x,u(x),\nabla u(x))\]
for $k\to\infty$ and almost all $x\in\Omega$. By Fatou's lemma, we conclude
\begin{align*}
	I(u)&=\int_\Omega{(f(x,u(x),\nabla u(x))-\gamma(x))\mathrm{d}x}+\int_\Omega{\gamma(x)\mathrm{d}x}\\
	&\leq\liminf_{k\to\infty}{\int_\Omega{(f(x,u_{n_k}(x),\nabla u_{n_k}(x))-\gamma(x))}\mathrm{d}x}+\int_\Omega{\gamma(x)\mathrm{d}x}\\
	&=\liminf_{k\to\infty}{I(u_{n_k})}.
\end{align*}
The whole argument applies to every subsequence of $(u_n)_{n\in\mathbb{N}}$, i.e., for every subsequence we can find a further subsequence for which the liminf-inequality holds. With that we conclude that the liminf-inequality holds for the whole sequence.\hfill$\blacksquare$\\[11pt]

As mentioned before, continuity of $I$ is far too much. But asking for convexity in $u$ and $A$ in $f$ is still very restrictive. So we will work on other types of conditions which ensures weak lower semicontinuity. For example, it could be enough to ask for convexity only in $A$, because weak convergence $u_n\rightharpoonup u$ in $W^{1,p}(\Omega)$ can be improved to strong convergence in $L^q(\Omega)$ for suitable $q$ by Rellich's compact embedding. Therefore, we may drop in some sense the convexity assumption on $u$.\\[11pt]

\hypertarget{theorem_3_4_7}{\textbf{\underline{Theorem 3.4.7}}}\\
(Sufficient condition for weak lower semicontinuity)\\
Let $\Omega\subset\mathbb{R}^d$ be an open and bounded Lipschitz domain. Let $f:\Omega\times\mathbb{R}^m\times\mathbb{R}^{m\times d}\longrightarrow\mathbb{R}$ be a Carath\'eodory function such that $A\longmapsto f(x,u,A)$ is convex for all $u\in\mathbb{R}^m$ and almost all $x\in\Omega$, and $f(x,u,A)\geq\gamma(x)$ for some $\gamma\in L^1(\Omega)$. Then the functional
\[I:W^{1,p}(\Omega)\longrightarrow\mathbb{R}_\infty,\qquad I(u):=\int_\Omega{f(x,u(x),\nabla u(x))\mathrm{d}x}\]
is weakly lower semicontinuous.\\

See \cite[Theorem 3.23]{bernard_dacorogna_direct} for a general proof. Here, we only prove two more restrictive versions.\\[11pt]

\hypertarget{theorem_3_4_8}{\textbf{\underline{Theorem 3.4.8}}}\\
Let $\Omega,f$ and $I$ be as in \hyperlink{theorem_3_4_7}{Theorem 3.4.7}. Moreover, let $p>d$ and assume that for each $R>0$ there exist $h_R\in L^1(\Omega)$, $C_R>0$ and a modulus of continuity, that is, a continuous, nondecreasing function $\omega_R:[0,\infty)\longrightarrow[0,\infty)$ with $\omega_R(0)=0$ such that
\[\lvert f(x,u,A)-f(x,v,A)\rvert\leq\omega_R(\lvert u-v\rvert)\cdot\left(h_R(x)+C_R\lvert A\rvert^p\right)\]
for almost all $x\in\Omega$, all $u,v\in B_R(0)\subset\mathbb{R}^m$ and all $A\in\mathbb{R}^{m\times d}$.\\

Then $I$ is weakly lower semicontinuous.\\

\textit{Proof:}\\
First note $I$ is well-defined by \hyperlink{lemma_3_4_2}{Lemma 3.4.2} since $f(\cdot,u,A)\geq\gamma$ in $\Omega$. To show that $I$ is weakly lower semicontinuous, let $(u_n)_{n\in\mathbb{N}}\subset W^{1,p}(\Omega;\mathbb{R}^m)$ and $u_*\in W^{1,p}(\Omega;\mathbb{R}^m)$ be such that $u_n\rightharpoonup u_*$ in $W^{1,p}(\Omega;\mathbb{R}^m)$ for $n\to\infty$. Then $u_n\to u_*$ in $C^0(\overline{\Omega};\mathbb{R}^m)$ by Rellich's compactness theorem. Therefore, there exists $R>0$ such that $\lvert u_n(x)\rvert\leq R$ and $\lvert u_*(x)\rvert\leq R$ for all $x\in\Omega$. We write
\[I(u_n)=\underbrace{\int_\Omega{(f(x,u_n(x),\nabla u_n(x))-f(x,u_*(x),\nabla u_n(x)))\mathrm{d}x}}_{=:\mathcal{T}_n}+\int_\Omega{\underbrace{f(x,u_*(x),\nabla u_n(x))}_{=:F(x,\nabla u_n(x))}\mathrm{d}x}.\]

\textit{Step 1:} $\mathcal{T}_n\to0$ as $n\to\infty$.
\begin{itemize}
	\item[] We have
	\begin{align*}
		\lvert\mathcal{T}_n\rvert&\leq\int_\Omega{\lvert f(x,u_n(x),\nabla u_n(x))-f(x,u_*(x),\nabla u_n(x))\rvert\mathrm{d}x}\\
		&\leq\int_\Omega{\omega_R(\lvert u_n(x)-u_*(x)\rvert)\left(h_R(x)+C_R\lvert\nabla u_n(x)\rvert^p\right)\mathrm{d}x}\\
		&\leq\omega_R\left(\lVert u_n-u_*\rVert_{C^0(\overline{\Omega})}\right)\left(\lVert h_R\rVert_{L^1(\Omega)}+C_R\lVert\nabla u_n\rVert_{L^p(\Omega)}^p\right),
	\end{align*}
	since $\omega_R$ is monotone. Since $(u_n)_{n\in\mathbb{N}}$ is weakly convergent, it is bounded, i.e. $\lVert\nabla u_n\rVert_{L^p(\Omega)}^p$ is bounded. Moreover, $\omega_R(\lVert u_n-u_*\rVert_{C^0(\overline{\Omega})})\to0$ as $n\to\infty$ because $\omega_R$ is continuous. Hence, $\mathcal{T}_n\to0$ as $n\to\infty$.\\
\end{itemize}

\textit{Step 2:} $\tilde{I}(u):=\int_\Omega{F(x,\nabla u(x))\mathrm{d}x}$ is weakly lower semicontinuous.
\begin{itemize}
	\item[] The function $\tilde{f}(x,u,A):=F(x,A)$ is a Carath\'eodory function, and $(u,A)\longmapsto\tilde{f}(x,u,A)$ is convex, and $\tilde{f}(x,u,A)\geq\gamma(x)$ for almost all $x\in\Omega$ (with $\gamma$ from the statement). Hence, \hyperlink{theorem_3_4_6}{Theorem 3.4.6} gives weak lower semicontinuity for $\tilde{I}$.\\
\end{itemize}

\textit{Step 3:} It holds $I(u_*)\leq\liminf_{n\to\infty}{I(u_n)}$.
\begin{itemize}
	\item[] Step 1. and 2. imply
	\[\liminf_{n\to\infty}{I(u_n)}\geq\liminf_{n\to\infty}{\mathcal{T}_n}+\liminf_{n\to\infty}{\tilde{I}(u_n)}\geq0+\tilde{I}(u_*)=\tilde{I}(u_*)=I(u_*).\]
\end{itemize}
\hfill$\blacksquare$\\[11pt]

If $p\leq d$, the space $W^{1,p}(\Omega;\mathbb{R}^m)$ does no longer embed (compactly) into $C^0(\overline{\Omega})$, but into $L^q(\Omega)$ if $1-\frac{d}{p}>-\frac{d}{q}$. By adjusting the assumptions on growth and continuity of $f$, we obtain a version for which we can show weak lower semicontinuity.\\

\hypertarget{theorem_3_4_9}{\textbf{\underline{Theorem 3.4.9}}}\\
Let $\Omega$, $f$ and $I$ be as in \hyperlink{theorem_3_4_7}{Theorem 3.4.7}. Let $1\leq p\leq d$ and $1\leq q<p^*:=\frac{dp}{d-p}$. Assume that there are $\widetilde{C}>0$, $\theta\in(0,1)$ and $\delta\in L^{q'}(\Omega)$, $\gamma\in L^{(\frac{q}{\theta})'}(\Omega)$ such that
\begin{align*}
	\lvert f(x,u,A)-f(x,v,A)\rvert&\leq\lvert u-v\rvert\left[\delta(x)+\widetilde{C}\left(\lvert u\rvert^{q-1}+\lvert v\rvert^{q-1}+\lvert A\rvert^{\frac{p}{q'}}\right)\right]\\
	&\qquad\qquad+\lvert u-v\rvert^\theta\left[\gamma(x)+\widetilde{C}\left(\lvert u\rvert^{q-\theta}+\lvert v\rvert^{q-\theta}+\lvert A\rvert^{p\frac{q-\theta}{q}}\right)\right]
\end{align*}
for almost all $x\in\Omega$, all $u,v\in\mathbb{R}^m$, $A\in\mathbb{R}^{m\times d}$. Here, $\frac{1}{q}+\frac{1}{q'}=1$ and $\frac{1}{(q/\theta)}+\frac{1}{(q/\theta)'}=1$. Then $I$ is weakly lower semicontinuous on $W^{1,p}(\Omega;\mathbb{R}^m)$.\\

\textit{Remark: In comparison to the previous case where we had the modulus of continuous, we here have to add the $\lvert u-v\rvert^\theta$-term.}\\

\textit{Proof:}\\
Let $(u_n)_{n\in\mathbb{N}}\subset W^{1,p}(\Omega;\mathbb{R}^m)$ such that $u_n\rightharpoonup u_*$ in $W^{1,p}(\Omega;\mathbb{R}^m)$. Then $u_n\to u_*$ in $L^q(\Omega;\mathbb{R}^m)$ by Rellich's compactness theorem. We proceed as in the proof for \hyperlink{theorem_3_4_8}{Theorem 3.4.8}, i.e., we split $I(u_n)$ into the two terms. The claim follows in the same way, and if we can show $\mathcal{T}_n\to0$ for $n\to\infty$ then we are done already (because the proofs for step 2. and 3. can literally be copied). We have
\begin{align*}
	\lvert\mathcal{T}_n\rvert&\leq\int_\Omega{\lvert f(x,u_n(x),\nabla u_n(x))-f(x,u_*(x),\nabla u_n(x))\rvert\mathrm{d}x}\\
	&\leq\int_\Omega{\lvert u_n(x)-u_*(x)\rvert\left[\delta(x)+\widetilde{C}\left(\lvert u_n(x)\rvert^{q-1}+\lvert u_*(x)\rvert^{q-1}+\lvert\nabla u_n(x)\rvert^{\frac{p}{q'}}\right)\right]\mathrm{d}x}\\
	&\qquad\qquad+\int_\Omega{\lvert u_n(x)-u_*(x)\rvert^\theta\left[\gamma(x)+\widetilde{C}\left(\lvert u_n(x)\rvert^{q-\theta}+\lvert u_*(x)\rvert^{q-\theta}+\lvert\nabla u_n(x)\rvert^{p\frac{q-\theta}{q}}\right)\right]\mathrm{d}x},
\end{align*}
and applying H\"older's inequality yields
\begin{align*}
	\lvert\mathcal{T}_n\rvert&\leq C\lVert u_n-u_*\rVert_{L^q(\Omega)}\left[\lVert\delta\rVert_{L^{q'}(\Omega)}+\big\lVert\lvert u_n\rvert^{q-1}\big\rVert_{L^{q'}(\Omega)}+\big\lVert\lvert u_*\rvert^{q-1}\big\rVert_{L^{q'}(\Omega)}+\big\lVert\lvert\nabla u_n\rvert^{p\frac{q-1}{q}}\big\rVert_{L^{q'}(\Omega)}\right]\\
	&\qquad\qquad+C\big\lVert\lvert u_n-u_*\rvert^\theta\big\rVert_{L^{\frac{q}{\theta}}(\Omega)}\left[\lVert\delta\rVert_{L^{(\frac{q}{\theta})'}(\Omega)}+\big\lVert\lvert u_n\rvert^{q-\theta}\big\rVert_{L^{(\frac{q}{\theta})'}(\Omega)}\right]\\
	&\qquad\qquad+C\big\lVert\lvert u_n-u_*\rvert^\theta\big\rVert_{L^{\frac{q}{\theta}}(\Omega)}\left[\big\lVert\lvert u_*\rvert^{q-\theta}\big\rVert_{L^{(\frac{q}{\theta})'}(\Omega)}+\big\lVert\lvert\nabla u_n\rvert^{p\frac{q-\theta}{q}}\big\rVert_{L^{(\frac{q}{\theta})'}(\Omega)}\right].
\end{align*}
We use that $\big\lVert\lvert v\rvert^{q-1}\big\rVert_{L^{q'}(\Omega)}=\lVert v\rVert_{L^q(\Omega)}^{q-1}$, and $\big\lVert\lvert\nabla v\rvert^{p\frac{q-1}{q}}\big\rVert_{L^{q'}(\Omega)}=\lVert\nabla v\rVert_{L^p(\Omega)}^{p/q'}$, and $\big\lVert\lvert v\rvert^\theta\big\rVert_{L^{\frac{q}{\theta}}(\Omega)}=\lVert v\rVert_{L^q(\Omega)}^\theta$, and $\big\lVert\lvert v\rvert^{q-\theta}\big\rVert_{L^{(\frac{q}{\theta})'}(\Omega)}=\lVert v\rVert_{L^q(\Omega)}^{q-\theta}$, and $\big\lVert\lvert \nabla v\rvert^{p\frac{q-\theta}{q}}\big\rVert_{L^{(\frac{q}{\theta})'}(\Omega)}=\lVert\nabla v\rVert_{L^p(\Omega)}^{p(q-\theta)/q}$.\\

The weak convergence $u_n\rightharpoonup u$ in $W^{1,p}(\Omega;\mathbb{R}^m)$ implies that all the terms inside the brackets $[\dotsc]$ above can be controlled, i.e., are uniformly bounded in $n$, so that
\[\lvert\mathcal{T}_n\rvert\leq M\left(\lVert u_n-u_*\rVert_{L^q(\Omega)}+\lVert u_n-u_*\rVert_{L^q(\Omega)}^\theta\right)\to0\]
as $n\to\infty$.\hfill$\blacksquare$\\[11pt]

\textbf{Remark 3.4.10}\\
The assumptions of \hyperlink{theorem_3_4_8}{Theorem 3.4.8} and \hyperlink{theorem_3_4_9}{Theorem 3.4.9} demand continuity of $f$ with respect to $u$ which is not the case in \hyperlink{theorem_3_4_7}{Theorem 3.4.7}.\\[11pt]

\hypertarget{theorem_3_4_11}{\textbf{\underline{Theorem 3.4.11}}}\\
(Existence result in convex case; Ioffe's theorem)\\
Let $\Omega\subset\mathbb{R}^d$ be an open, bounded Lipschitz domain. Let $f:\Omega\times\mathbb{R}^m\times\mathbb{R}^{m\times d}\longrightarrow\mathbb{R}$ be such that
\begin{itemize}
	\item[(i)] $f$ is a Carath\'eodory function,
	\item[(ii)] $A\longmapsto f(x,u,A)$ is convex for almost all $x\in\Omega$ and all $u\in\mathbb{R}^m$, and there is $\gamma\in L^1(\Omega)$ such that $f(x,u,A)\geq\gamma(x)$ for almost all $x\in\Omega$,
	\item[(iii)] there are quantities $p\in(1,\infty)$, $c_A>0$, $r\in[1,p)$, $\delta\in L^{\frac{p}{p-r}}(\Omega)$, $h\in L^1(\Omega)$ such that for all $(x,u,A)\in\Omega\times\mathbb{R}^m\times\mathbb{R}^{m\times d}$ it holds
	\[f(x,u,A)\geq c_A\lvert A\rvert^p-\delta(x)\lvert u\rvert^r-h(x).\]
\end{itemize}
Let $u_0\in W^{1,p}(\Omega;\mathbb{R}^m)$ and $\varphi\in(W^{1,p}(\Omega;\mathbb{R}^m))'$ be given. Then there exists a minimizer of the functional
\[I(u):=\int_\Omega{f(x,u(x),\nabla u(x))\mathrm{d}x}-\varphi(u)\]
on $u_0+W_0^{1,p}(\Omega;\mathbb{R}^m)$, i.e., there exists $u_*\in u_0+W_0^{1,p}(\Omega;\mathbb{R}^m)$ such that
\[I(u_*)=\inf\left\{I(u)\,\middle\vert\,u\in u_0+W_0^{1,p}(\Omega;\mathbb{R}^m)\right\}.\]

\textit{Proof:}\\
We obtain well-definedness of $I$ by \hyperlink{lemma_3_4_2}{Lemma 3.4.2} and assumptions (i), (ii). If $I(u)=+\infty$ for all $u\in u_0+W_0^{1,p}(\Omega;\mathbb{R}^m)$ then there is nothing to do. Assume that $I(\tilde{u})<\infty$ for some $\tilde{u}\in u_0+W_0^{1,p}(\Omega;\mathbb{R}^m)$, and consider the functional
\[\tilde{I}:W_0^{1,p}(\Omega;\mathbb{R}^m)\longrightarrow\mathbb{R}_\infty,\qquad\tilde{I}(v):=I(u_0+v).\]
Hence, if we put $\tilde{f}(x,v,A):=f(x,u_0(x)+v,\nabla u_0(x)+A)$, we can write
\[\tilde{I}(v)=\int_\Omega{\tilde{f}(x,v(x),\nabla v(x))\mathrm{d}x}-\varphi(u_0+v).\]
Make the following observations:
\begin{itemize}
	\item[(a)] Since $p\in(1,\infty)$, the space $W_0^{1,p}(\Omega;\mathbb{R}^m)$ is reflexive as a closed subspace of the reflexive space $W^{1,p}(\Omega;\mathbb{R}^m)$.
	\item[(b)] With assumption (iii) we can estimate
	\begin{align*}
		\tilde{f}(x,v,A)&=f(x,u_0(x)+v,\nabla u_0(x)+A)\\
		&\geq c_A\lvert\nabla u_0(x)+A\rvert^p-\delta(x)\lvert u_0(x)+v\rvert^r-h(x)\\
		&\geq c_Ac_p\left(\lvert A\rvert^p-\lvert\nabla u_0(x)\rvert^p\right)-c_r\lvert\delta(x)\rvert\left(\lvert u_0(x)\rvert^r+\lvert v\rvert^r\right)-\lvert h(x)\rvert\\
		&\geq c_Ac_p\lvert A\vert^p-c_r\lvert\delta(x)\rvert\lvert v\rvert^r-\left[c_Ac_p\lvert\nabla u_0(x)\rvert^p+c_r\lvert\delta(x)\rvert\lvert u_0(x)\rvert^r+\lvert h(x)\vert\right].
	\end{align*}
	The term in brackets is integrable, so that \hyperlink{theorem_3_4_3}{Theorem 3.4.3} implies for $\tilde{c}>0$, $\beta\in\mathbb{R}$
	\[\tilde{I}(v)\geq\tilde{c}\lVert v\rVert_{W^{1,p}(\Omega)}^p+\beta-\lVert\varphi\rVert_{(W^{1,p}(\Omega))'}\left(\lVert u_0\rVert_{W^{1,p}(\Omega)}+\lVert v\rVert_{W^{1,p}(\Omega)}\right)\to+\infty\]
	as $\lVert v\rVert_{W^{1,p}(\Omega)}\to+\infty$, because $p>1$. Hence, $\tilde{I}$ is coercive.
	\item[(c)] Since $A\longmapsto\tilde{f}(x,u,A)$ is convex and $\tilde{f}(x,u,A)\geq\gamma(x)$, \hyperlink{theorem_3_4_7}{Theorem 3.4.7} implies that $\tilde{I}$ is weakly lower semicontinuous.
\end{itemize}
Hence, existence of a minimizer $\bar{u}\in W_0^{1,p}(\Omega;\mathbb{R}^m)$ for $\tilde{I}$ follows with \hyperlink{theorem_3_1_14}{Theorem 3.1.14}. Then $u_*:=\bar{u}+u_0$ is a minimizer for $I$ as desired.\hfill$\blacksquare$\\[11pt]

\textbf{Remark 3.4.12}
\begin{itemize}
	\item[(a)] The constraint $u\in u_0+W_0^{1,p}(\Omega;\mathbb{R}^m)$ corresponds to Dirichlet boundary conditions $u\vert_{\partial\Omega}=u_0\vert_{\partial\Omega}$ in the trace sense.
	\item[(b)] Uniqueness does not hold without some form of strict convexity.
	\item[(c)] The case $r=p$ can be considered if $\delta$ is small, i.e. $\lvert\delta(x)\rvert<\frac{c_A}{C_P^p}$ with $C_P$ being the optimal constant in Poincar\'e inequality, so $\lVert u\rVert_{L^p(\Omega)}\leq C_P\lVert\nabla u\rVert_{L^p(\Omega)}$.
	\item[(d)] If $\min\{m,d\}=1$ then convexity of $A\longmapsto f(x,u,A)$ is also necessary. See exercises and discussion of quasiconvexity later for general case.\\[11pt]
\end{itemize}

\textbf{Example 3.4.13}\\
Let $\Omega\subset\mathbb{R}^d$ be an open, bounded Lipschitz domain. We want to apply the existence result to investigate existence for a semilinear elliptic partial differential equation. We search for $u:\Omega\longrightarrow\mathbb{R}^1$ satisfying
\[\left\{\begin{array}{rl}
	-\divergence{\alpha\nabla u}+g(\cdot,u)=0&\text{in }\Omega,\\
	u=0&\text{on }\partial\Omega
\end{array}\right.\]
for some constant $\alpha>0$ and $g:\Omega\times\mathbb{R}\longrightarrow\mathbb{R}$ continuous. This is the Euler-Lagrange equation for the functional
\[I:W_0^{1,p}(\Omega)\longrightarrow\mathbb{R}_\infty,\qquad u\longmapsto\int_\Omega{\frac{\alpha}{2}\lvert\nabla u(x)\rvert^2+G(x,u(x))\mathrm{d}x}\]
with $\partial_uG(x,u)=g(x,u)$, i.e. $G(x,u)=\int_0^u{g(x,v)\mathrm{d}v}$. Notice that the choice of the integration domain for $G$ does not change the minimum of $I$.\\

As the squared norm of the gradient appears in the integral, the natural choice is $p=2$ to ensure coercivity. We choose $g(x,u)=\beta(x)u+c_qu\lvert u\rvert^{q-2}+h(x)$, where $\beta,h\in L^\infty(\Omega)$, $c_q>0$ and $q>2$. Thus,
\[G(x,u)=\frac{\beta(x)}{2}u^2+\frac{c_q}{q}\lvert u\rvert^q+h(x)u.\]
We estimate
\[G(x,u)\geq-\frac{\lVert\beta\rVert_{L^\infty(\Omega)}}{2}u^2-\lVert h\rVert_{L^\infty(\Omega)}\lvert u\rvert+\frac{c_q}{q}\lvert u\rvert^q\geq\widetilde{C}\in\mathbb{R},\]
because $q>2$, and therefore
\[I(u)\geq\frac{\alpha}{2}\lVert\nabla u\rVert_{L^2(\Omega)}^2+\widetilde{C}\vol{\Omega}\]
which gives coercivity via Poincar\'e's inequality. Hence, \hyperlink{theorem_3_4_11}{Theorem 3.4.11} is applicable, i.e., we have existence of minimizers.\\

However, also \hyperlink{theorem_3_4_8}{Theorem 3.4.8} and \hyperlink{theorem_3_4_9}{Theorem 3.4.9} can be applied since
\begin{align*}
	\lvert G(x,u)-G(x,v)\rvert&=\left\lvert\frac{\beta(x)}{2}(u-v)(u+v)+\frac{c_q}{q}(\lvert u\rvert^q-\lvert v\rvert^q)+h(x)(u-v)\right\rvert\\
	&\leq\frac{\lVert\beta\rVert_{L^\infty(\Omega)}}{2}(\lvert u\rvert+\lvert v\rvert)\lvert u-v\rvert+\lVert h\rVert_{L^\infty(\Omega)}\lvert u-v\rvert+\widehat{C}\left(\lvert u\rvert^{q-1}+\lvert v\rvert^{q-1}\right)\lvert u-v\rvert,
\end{align*}
where we have used convexity $\frac{1}{q}\lvert v\rvert^q\geq\frac{1}{q}\lvert u\rvert^q+u\lvert u\rvert^{q-2}(v-u)$.\\

If $d=1$ we have $p=2>d$, and then \hyperlink{theorem_3_4_8}{Theorem 3.4.8} can be directly applied for any $q\in(2,\infty)$. If $d\geq2$ then \hyperlink{theorem_3_4_9}{Theorem 3.4.9} holds for $q\in(2,\frac{2d}{d-2})$.\\[11pt]

\hypertarget{example_3_4_14}{\textbf{Example 3.4.14}}\\
(Karl Weierstra{\ss}, * 1815, $\dagger$ 1897)\\
Consider the functional
\[I:W^{1,p}(-1,1)\longrightarrow\mathbb{R},\qquad I(u):=\frac{1}{2}\int_{-1}^1{x^2(u'(x))^2\mathrm{d}x},\]
so with volume density $f(x,A)=\frac{1}{2}\lvert x\rvert^2A^2$. This is convex in $A$, but $I$ is not coercive on $W^{1,p}(-1,1)$ for any $p\in(1,\infty)$. In the exercises we saw that a minimizer does not exist.\\[11pt]

\hypertarget{example_3_4_15}{\textbf{Example 3.4.15}}\\
(Oskar Bolza, * 1857, $\dagger$ 1942)\\
We consider
\[I:W_0^{1,4}(0,1)\longrightarrow\mathbb{R},\qquad I(u):=\int_0^1{(u(x))^4+(1-(u'(x))^2)^2\mathrm{d}x}.\]
It turns out that this function is coercive on $W^{1,4}(0,1)$, but $A\longmapsto(1-A^2)^2$ is not convex. One can show via a sequence of zig-zag functions $(u_k)_{k\in\mathbb{N}}$ (similar as in \hyperlink{example_2_3_3}{Example 2.3.3}) that converges weakly to 0 in $W^{1,4}(0,1)$ that $I(u_k)\to0\ne I(0)=1$. In particular, a minimizer does not exist.\\[11pt]

\hyperlink{example_3_4_14}{Example 3.4.14} and \hyperlink{example_3_4_15}{Example 3.4.15} lead to the question whether our developed conditions in Ioffe's theorem are even necessary. But the answer is no.\\[11pt]

\textbf{Example 3.4.16}\\
Let's consider
\[I(u)=\int_0^1{f(u'(x))\mathrm{d}x}\quad\text{subject to}\quad u(0)=a,u(1)=b,\]
with a convex function $f:\mathbb{R}\longrightarrow[0,\infty)$. Jensen's inequality gives
\[I(u)=\int_0^1{f(u'(x))\mathrm{d}x}\geq f\left(\int_0^1{u'(x)\mathrm{d}x}\right)=f(b-a),\]
and $u_*(x)=(b-a)x+a$ is therefore a minimizer. But coercivity is missing here.
    \section{Weak Euler-Lagrange Equations}
In the classical theory, we have seen that for densities $f\in C^2(\Omega\times\mathbb{R}^m\times\mathbb{R}^{m\times d})$ and minimizers $u_*\in C^2(\overline{\Omega};\mathbb{R}^m)$, the Euler-Lagrange equations are satisfied
\[\left\{\begin{array}{rl}
	-\divergence{\partial_Af(\cdot,u_*,\nabla u_*)}+\partial_uf(\cdot,u_*,\nabla u_*)=0&\text{in }\Omega,\\
	u_*=u_0&\text{on }\Gamma_D,\\
	\partial_Af(\cdot,u_*,\nabla u_*)\cdot\nu+\partial_ug(\cdot,u_*)=0&\text{on }\Gamma_N.
\end{array}\right.\]
The question is in what sense are Euler-Lagrange equations satisfied in the case of Sobolev functions?\\[11pt]

\textbf{Example 3.5.1}\\
Consider $\Omega=(-1,1)$, $d=m=1$ and for $1<p<\infty$ the functional
\[I:M\longrightarrow\mathbb{R},\qquad I(u)=\int_{-1}^1{(x^4-(u(x))^6)^2\lvert u'(x)\rvert^p\mathrm{d}x}\]
with $M=\{u\in H^1(\Omega)\mid u(1)=1,u(-1)=-1\}=u_0+H_0^1(\Omega)$ and $u_0(x)=x$ for $x\in(-1,1)$. Thus, the density $f(x,u,A)=(x^4-u^6)^2\lvert A\rvert^p$ is convex in $A$, non-negative, hence also $I\geq0$, but coercivity is missing. However, this can be cured by adding the term $\frac{\varepsilon}{2}\lvert A\rvert^2$, but we neglect this for simplicity. A minimizer of $I$ is given by
\[u_*(x)=\left\{\begin{array}{rl}
	x^{2/3}&\text{if }x>0,\\
	-(-x)^{2/3}&\text{if }x\leq0.
\end{array}\right.\]

\begin{figure}[ht]
	\centering
	\begin{tikzpicture}
		% Hintergrund und Achsen
		\fill[cyan!20] (-3.5, -2.5) rectangle (3.5, 2.5);
		\draw[cyan] (-3.5, -2.5) grid (3.5, 2.5);
		\draw[thick, ->] (-3.3, 0) -- (3.3, 0) node[below] {$x$};
		\draw[thick, ->] (0, -2.3) -- (0, 2.3);
		\draw[thin] (-2, 0.1) -- (-2, -0.1) node[below] {$-1$};
		\draw[thin] (2, 0.1) -- (2, -0.1) node[below] {$1$};
		\draw[thin] (0.1, -2) -- (-0.1, -2) node[left] {$-1$};
		\draw[thin] (0.1, 2) -- (-0.1, 2) node[left] {$1$};
		\draw[very thick] (-3.5, -2.5) rectangle (3.5, 2.5);

		% Funktion
		\draw[thick, red] plot[smooth, domain=-2:0, samples=200] (\x, {-2*(-\x/2)^(2/3)});
		\draw[thick, red] plot[smooth, domain=0:2, samples=200] (\x, {2*(\x/2)^(2/3)});
	\end{tikzpicture}
	\caption{Illustration of minimizer $u_*$.}
\end{figure}

Clearly, $u_*\in H^1(\Omega)$ with $u_*(x)=\frac{2}{3}\lvert x\rvert^{-1/3}$, and $I(u_*)=0$. We ask: Does the first variation of $I$ in $u_*$ exist? Does $u_*$ satisfy $DI(u_*)[v]=0$ for suitable directions $v$ (e.g. $v\in C_c^\infty(\Omega)$)? We compute the derivative of $t\longmapsto I(u_*+tv)$ in $t=0$.\\

We claim that, if $v(0)\ne0$, then $I(u_*+tv)=+\infty$ for all $t\ne0$. In particular the variation in $u_*$ would not exist. Our functional $I$ splits into two integrals (one over $(-1,0)$ and one over $(0,1)$), but we only do the following calculation for the second one:
\begin{align*}
	&\int_0^1{(x^4-(x^{2/3}+tv(x))^6)^2\left\lvert\frac{2}{3}x^{-1/3}+tv'(x)\right\rvert^p\mathrm{d}x}\\
	&\qquad\qquad=\int_0^1{(x^4-x^4-6x^\frac{10}{3}tv-15x^\frac{8}{3}(tv)^2-20x^\frac{6}{3}(tv)^3-15x^\frac{4}{3}(tv)^4-6x^\frac{2}{3}(tv)^5-(tv)^6)^2}\\
	&\qquad\qquad\qquad\qquad\cdot\lvert\frac{2}{3}x^{-\frac{1}{3}}+tv'\rvert^p\mathrm{d}x\\
	&\qquad\qquad=\int_0^1{\tilde{c}t^{12}(v(x))^{12}x^{-p/3}\mathrm{d}x}+R(t),
\end{align*}
where $R(t)$ collects all lower order terms. Since $v(0)\ne0$ we have $v(x)\approx ax$ for $x\sim0$ for some constant $a\ne0$ which yields
\[\int_0^1{x^{12}x^{-p/3}\mathrm{d}x}=+\infty\]
if $p$ is sufficiently large, i.e. $12-\frac{p}{3}\leq-1$.\\[11pt]

To ensure that first variation of $I$ exists, we have to impose additional conditions on $f,\partial_Af$ and $\partial_uf$.\\

\hypertarget{theorem_3_5_2}{\textbf{\underline{Theorem 3.5.2}}}\\
(Sufficient conditions for G\^ateaux differentiability; Ren\'e G\^ateaux, * 1889, $\dagger$ 1914)\\
Let $\Omega\subset\mathbb{R}^d$ be bounded and open, and $f\in C^1(\overline{\Omega}\times\mathbb{R}^m\times\mathbb{R}^{m\times d})$ (in fact, $f\in C^0(\overline{\Omega}\times\mathbb{R}^m\times\mathbb{R}^{m\times d})$ and $f(x,\cdot,\cdot)\in C^1(\mathbb{R}^m\times\mathbb{R}^{m\times d})$ for all $x\in\overline{\Omega}$ would be enough). Let $p,q,r\in[1,\infty)$ such that $\frac{1}{r}\geq\frac{d-p}{dp}=:\frac{1}{p^*}$, $q\leq 1+\frac{r(p-1)}{p}$ and
\begin{align}
	\lvert f(x,u,A)\rvert&\leq C_1(1+\lvert u\rvert^r+\lvert A\rvert^p)\label{eq:mcov_formula_3_1},\\
	\lvert\partial_uf(x,u,A)\rvert&\leq C_2(1+\lvert u\rvert^{r-1}+\lvert A\rvert^{p-1})\label{eq:mcov_formula_3_2},\\
	\lvert\partial_Af(x,u,A)\rvert&\leq C_3(1+\lvert u\rvert^{q-1}+\lvert A\rvert^{p-1})\label{eq:mcov_formula_3_3}.
\end{align}
Then
\begin{itemize}
	\item[(i)] $I(u)\in\mathbb{R}$ for every $u\in W^{1,p}(\Omega;\mathbb{R}^m)$;
	\item[(ii)] $I$ is G\^ateaux differentiable, and for all $v\in W^{1,p}(\Omega;\mathbb{R}^m)$
	\[DI(u)[v]=\int_\Omega{\partial_Af(x,u(x),\nabla u(x)):\nabla v(x)+\partial_uf(x,u(x),\nabla u(x))\cdot v(x)\mathrm{d}x}\]
	is well-defined.
\end{itemize}

\textit{Remark: If $p>d$, we can replace $\lvert u\rvert^r,\lvert u\rvert^{r-1},\lvert u\rvert^{q-1}$ by some $b(u)$, where $b:\mathbb{R}^m\longrightarrow\mathbb{R}$ is locally bounded.}\\

\textit{Proof:}\\
The Sobolev embedding implies $W^{1,p}(\Omega;\mathbb{R}^m)\longhookrightarrow L^r(\Omega;\mathbb{R}^m)$ so that in particular it holds $\lVert v\rVert_{L^r(\Omega)}\leq C_{r,p}\lVert v\rVert_{W^{1,p}(\Omega)}$ for some $C_{r,p}>0$ and all $v\in W^{1,p}(\Omega;\mathbb{R}^m)$. Therefore, \eqref{eq:mcov_formula_3_1} gives
\begin{align*}
	\lvert I(u)\rvert&\leq\int_\Omega{\lvert f(x,u(x),\nabla u(x))\rvert\mathrm{d}x}\\
	&\leq C_1\left(\vol{\Omega}+\lVert u\rVert_{L^r(\Omega)}^r+\lVert\nabla u\rVert_{L^p(\Omega)}^p\right)<\infty
\end{align*}
for $u\in W^{1,p}(\Omega;\mathbb{R}^m)$. This shows assertion (i). For (ii), we first note that, via \eqref{eq:mcov_formula_3_2} and H\"older's inequality (and with $p'$, $r'$ being the dual exponents of $p$ and $r$, respectively),
\begin{align*}
	&\int_\Omega{\lvert\partial_uf(x,u(x),\nabla u(x))\rvert\lvert v(x)\rvert\mathrm{d}x}\\
	&\qquad\qquad\leq\int_\Omega{C_2(1+\lvert u(x)\rvert^{r-1}+\lvert\nabla u(x)\rvert^{p-1})\lvert v(x)\rvert\mathrm{d}x}\\
	&\qquad\qquad\leq C_2\left(\left(\int_\Omega{1\mathrm{d}x}\right)^{\frac{1}{p'}}\lVert v\rVert_{L^p(\Omega)}+\lVert\lvert u\rvert^{r-1}\rVert_{L^{r'}(\Omega)}\lVert v\rVert_{L^r(\Omega)}+\lVert\lvert\nabla u\rvert^{p-1}\rVert_{L^{p'}(\Omega)}\lVert v\rVert_{L^p(\Omega)}\right)\\
	&\qquad\qquad\leq C_2\left(\vol{\Omega}^{\frac{1}{p'}}+C_{r,p}\lVert u\rVert_{L^r(\Omega)}^{r-1}+\lVert\nabla u\rVert_{L^p(\Omega)}^{p-1}\right)\lVert v\rVert_{W^{1,p}(\Omega)}<\infty,
\end{align*}
since $u,v\in W^{1,p}(\Omega;\mathbb{R}^m)$. Analogously,
\begin{align*}
	&\int_\Omega{\lvert\partial_Af(x,u(x),\nabla u(x))\rvert\lvert\nabla v(x)\rvert\mathrm{d}x}\\
	&\qquad\qquad\leq\int_\Omega{C_3(1+\lvert u(x)\rvert^{q-1}+\lvert\nabla u(x)\rvert^{p-1})\lvert\nabla v(x)\rvert\mathrm{d}x}\\
	&\qquad\qquad\leq C_3\left(\vol{\Omega}^{\frac{1}{p'}}\lVert\nabla v\rVert_{L^p(\Omega)}+\lVert\lvert u\rvert^{q-1}\rVert_{L^{p'}(\Omega)}\lVert\nabla v\rVert_{L^p(\Omega)}+\lVert\lvert\nabla u\rvert^{p-1}\rVert_{L^{p'}(\Omega)}\lVert\nabla v\rVert_{L^p(\Omega)}\right)\\
	&\qquad\qquad\leq C_3\left(\vol{\Omega}^{\frac{1}{p'}}+\left(\int_\Omega{\lvert u(x)\rvert^{\frac{(q-1)p}{p-1}}\mathrm{d}x}\right)^{\frac{1}{p'}}+\lVert\nabla u\rVert_{L^p(\Omega)}^{p-1}\right)\lVert v\rVert_{W^{1,p}(\Omega)}.
\end{align*}
The condition for $q$ implies $q-1\leq\frac{r}{p'}$, i.e. $\frac{(q-1)p}{p-1}\leq r$. Thus,
\[\partial_Af(\cdot,u,\nabla u):\nabla v+\partial_uf(\cdot,u,\nabla u)\cdot v\in L^1(\Omega)\]
for all $u,v\in W^{1,p}(\Omega;\mathbb{R}^m)$. It remains to check G\^ateaux differentiability, and to this end we use the fundamental theorem of calculus: if $g\in C^1([0,1])$ then
\[g(1)-g(0)=\int_0^1{\frac{\mathrm{d}}{\mathrm{d}s}g(s)\mathrm{d}s}.\]
With that, we write the difference quotient for G\^ateaux differential. For any $t\ne0$ it holds via chain rule
\begin{align*}
	\frac{1}{t}(I(u+tv)-I(u))&=\int_\Omega{\int_0^1{\frac{1}{t}\frac{\mathrm{d}}{\mathrm{d}s}f(x,u(x)+stv(x),\nabla u(x)+st\nabla v(x))\mathrm{d}s}\mathrm{d}x}\\
	&=\int_\Omega{\int_0^1{\frac{1}{t}\partial_uf(x,u(x)+stv(x),\nabla u(x)+st\nabla v(x))\cdot tv(x)\mathrm{d}s}\mathrm{d}x}\\
	&\qquad\qquad+\int_\Omega{\int_0^1{\frac{1}{t}\partial_Af(x,u(x)+stv(x),\nabla u(x)+st\nabla v(x)):t\nabla v(x)\mathrm{d}s}\mathrm{d}x}\\
	&=\int_\Omega{\int_0^1{h(t,s,x)\mathrm{d}s}\mathrm{d}x}
\end{align*}
with $h(t,s,\cdot):=\partial_uf(\cdot,u+stv,\nabla u+st\nabla v)\cdot v+\partial_Af(\cdot,u+stv,\nabla u+st\nabla v):\nabla v$. We are interested into the limit $t\to0$, and in order to conclude we need to interchange the limit and integration. For that, we will make use of Lebesgue's dominated convergence theorem. We prove existence of a dominating function for $h$. Using the growth assumptions \eqref{eq:mcov_formula_3_2} and \eqref{eq:mcov_formula_3_3} we estimate
\begin{align*}
	\lvert h(t,s,\cdot)\rvert&\leq C_2(1+\lvert u+stv\rvert^{r-1}+\lvert\nabla u+st\nabla v\rvert^{p-1})\lvert v\rvert\\
	&\qquad\qquad+C_3(1+\lvert u+stv\rvert^{q-1}+\lvert\nabla u+st\nabla v\rvert^{p-1})\lvert\nabla v\rvert\\
	&\leq C_2\left\{1+\left(\lvert u\rvert+\lvert st\rvert\lvert v\rvert\right)^{r-1}+\left(\lvert\nabla u\rvert+\lvert st\rvert\lvert\nabla v\rvert\right)^{p-1}\right\}\lvert v\rvert\\
	&\qquad\qquad+C_3\left\{1+\left(\lvert u\rvert+\lvert st\rvert\lvert v\rvert\right)^{q-1}+\left(\lvert\nabla u\rvert+\lvert st\rvert\lvert\nabla v\rvert\right)^{p-1}\right\}\lvert\nabla v\rvert\\
	&\leq C_2\left\{1+\left(\lvert u\rvert+\lvert v\rvert\right)^{r-1}+\left(\lvert\nabla u\rvert+\lvert\nabla v\rvert\right)^{p-1}\right\}\lvert v\rvert\\
	&\qquad\qquad+C_3\left\{1+\left(\lvert u\rvert+\lvert v\rvert\right)^{q-1}+\left(\lvert\nabla u\rvert+\lvert\nabla v\rvert\right)^{p-1}\right\}\lvert\nabla v\rvert\\
	&=:\tilde{h}.
\end{align*}
So $\tilde{h}$ is an integrable dominating function for $h$ since it does not depend on $s$, $t$ (but on $x$ which is ok). Thus, Lebesgue's dominated convergence is applicable and we have
\begin{align*}
	DI(u)[v]&=\lim_{t\to0}{\frac{1}{t}(I(u+tv)-I(u))}=\lim_{t\to0}{\int_\Omega{\int_0^1{h(t,s,x)\mathrm{d}s}\mathrm{d}x}}\\
	&=\int_\Omega{\int_0^1{\lim_{t\to0}{h(t,s,x)}\mathrm{d}s}\mathrm{d}x}=\int_\Omega{\int_0^1{h(0,0,x)\mathrm{d}s}\mathrm{d}x}=\int_\Omega{h(0,0,x)\mathrm{d}x}\\
	&=\int_\Omega{\partial_Af(x,u(x),\nabla u(x)):\nabla v(x)+\partial_uf(x,u(x),\nabla u(x))\cdot v(x)\mathrm{d}x},
\end{align*}
where we have used that $h(0,s,x)=h(0,0,x)$ for all $s\in[0,1]$. We conclude that $I$ is G\^ateaux differentiable with derivative as asserted.\hfill$\blacksquare$\\[11pt]

\textbf{Corollary 3.5.3}\\
(Necessary condition)\\
If $I:u_0+W_0^{1,p}(\Omega;\mathbb{R}^m)\longrightarrow\mathbb{R}$ satisfies the assumptions of \hyperlink{theorem_3_5_2}{Theorem 3.5.2} and \hyperlink{theorem_3_4_11}{Theorem 3.4.11}, then all minimizers $u_*$ of $I$ satisfy the weak Euler-Lagrange equations, i.e., for all $v\in W_0^{1,p}(\Omega;\mathbb{R}^m)$ it holds $DI(u_*)[v]=0$. More explicitly, we have
\[\int_\Omega{\partial_Af(x,u_*(x),\nabla u_*(x)):\nabla v(x)+\partial_uf(x,u_*(x),\nabla u_*(x))\cdot v(x)\mathrm{d}x}=0\]
for all $v\in W_0^{1,p}(\Omega;\mathbb{R}^m)$, which is the weak formulation of the classical Euler-Lagrange equations
\[-\divergence{\partial_Af(\cdot,u_*,\nabla u_*)}+\partial_uf(\cdot,u_*,\nabla u_*)=0.\]\\

\textit{Remark: One can show that, if $f$ is not only convex in $A$ but also in $u$, then satisfying the weak Euler-Lagrange equations is also sufficient for being a minimizer.}\\

\textit{Proof:}\\
This follows with simple Analysis I*.\hfill$\blacksquare$\\[11pt]

\textbf{Example 3.5.4}\\
We consider
\[f(x,u,A):=\frac{\alpha(x)}{p}\lvert A\rvert^p+\frac{\beta(x)}{r}\lvert u\rvert^r,\]
where $\alpha,\beta\in L^\infty(\Omega)$, $\alpha(x)\geq\alpha_0>0$, $\beta(x)\geq\beta_0>0$ for almost all $x\in\Omega$, and $1<p<\infty$, $1<r<\infty$. The previous theorem, \hyperlink{theorem_3_5_2}{Theorem 3.5.2}, gives G\^ateaux differentiability on $W^{1,p}(\Omega)$ of
\[I(u)=\int_\Omega{\frac{\alpha(x)}{p}\lvert\nabla u(x)\rvert^p+\frac{\beta(x)}{r}\lvert u(x)\rvert^r\mathrm{d}x}\]
if $\frac{1}{r}\geq\frac{d-p}{dp}$, with directional derivative
\[DI(u)[v]=\int_\Omega{\alpha(x)\lvert\nabla u(x)\rvert^{p-2}\nabla u(x):\nabla v(x)+\beta(x)\lvert u(x)\rvert^{r-2}u(x)v(x)\mathrm{d}x}.\]
The strong form of the Euler-Lagrange equation is
\[-\divergence{\alpha\lvert\nabla u\rvert^{p-2}\nabla u}+\beta\lvert u\rvert^{r-2}u=0\quad\text{in }\Omega.\]
If $\alpha\equiv1$ then this partial differential equation is the so-called \textit{$p$-Laplacian}.
    \section{Minimization Problems with Constraints}
In applications, minimizers of our problem often have to satisfy additional conditions besides boundary conditions. These conditions are called constriants. The study of such minimization problems is the main subject in optimization theory.\\[11pt]

\begin{example}
\begin{itemize}
	\item[(a)] (Isovolumetric problem) Enclose given area with shortest possible circumference
	\[I(u):=\int_a^b{\sqrt{1+(u'(x))^2}\mathrm{d}x}.\]
	This we want to minimize subject to $\mathcal{J}(u):=\int_a^b{u(x)\mathrm{d}x}=A_0$, where $A_0\in\mathbb{R}$ is the given area. The minimization problem can be compactly written as
	\[\min\{I(u)\mid u(a)=u(b)=0\text{ and }\mathcal{J}(u)=A_0\}.\]
	Observe that in this example the constraint $\mathcal{J}$ is linear in $u$.
	\item[(b)] (Eigenvalue problem) Let $M\in\mathbb{R}^{n\times n}$ be symmetric and positive definite. We want to minimize
	\[I(u):=\frac{1}{2}(Mu)\cdot u\qquad\text{subject to}\qquad u\in\mathbb{R}^n\text{ and }\lvert u\rvert_2=2.\]
	Here, the constraint ``$\lvert u\rvert_2=2$'' is nonlinear. The Problem can be written compactly as
	\[\min\{I(u)\mid\mathcal{J}(u)=1\},\]
	where $\mathcal{J}(u):=\frac{1}{2}\lvert u\rvert_2^2$. Later we will see that minimizers $u_*\in\mathbb{R}^n$ necessarily have to satisfy $DI(u_*)=\lambda_* D\mathcal{J}(u_*)$ for some $\lambda_*\in\mathbb{R}$. But this means $Mu_*=\lambda_*u_*$, i.e., $u_*$ is an eigenvector for the eigenvalue $\lambda_*$. In particular we have $I(u_*)=\frac{1}{2}(Mu_*)\cdot u_*=\frac{\lambda_*}{2}\lvert u_*\rvert_2^2=\lambda_*$.\\

	In infinite-dimensional setting, we can equivalently ask to minimize $\min\{I(u)\mid\mathcal{J}(u)=1\}$ where
	\[I(u)=\int_\Omega{\lvert\nabla u(x)\rvert^2\mathrm{d}x}\quad\text{and}\quad\mathcal{J}(u)=\int_\Omega{(u(x))^2\mathrm{d}x}\]
	for $u\in H_0^1(\Omega)$. A minimizer $u_*$ then has to satisfy
	\[\int_\Omega{\nabla u_*(x)\cdot\nabla v(x)\mathrm{d}x}=\lambda_*\int_\Omega{u_*(x)v(x)\mathrm{d}x}\]
	for all $v\in H_0^1(\Omega)$. So $\lambda_*$ is an eigenvalue of the Laplacian $-\Delta$ on $H_0^1(\Omega)$, so $-\Delta u_*=\lambda_*u_*$, $u_*=0$ on $\partial\Omega$, holds in the weak sense.
	\item[(c)] (Obstacle problem) Let $u:\Omega\longrightarrow\mathbb{R}$ describe a membrane stretched over an obstacle which is denoted by $\psi:\Omega\longrightarrow\mathbb{R}$.\\

	\begin{figure}[ht]
		\centering
		\begin{tikzpicture}
			\fill[cyan!10] (-0.5, -0.7) rectangle (5.7, 2.5);
			\draw[cyan] (-0.5, -0.7) grid (5.7, 2.5);
			\draw[thick, ->] (-0.3, 0) -- (5.5, 0);
			\draw[thick, ->] (0, -0.5) -- (0, 2.3);

			\draw[thick, red] plot[smooth, domain=-2:2, samples=200] ({\x+3}, {(3-\x)*((\x)^6-15/2*(\x)^4+12*(\x)^2+8)/24});
			\draw[blue] (1, 0) -- (1.522, 1.648);
			\draw[blue] plot[smooth, domain=-1.48:-1] ({\x+3}, {(3-\x)*((\x)^6-15/2*(\x)^4+12*(\x)^2+8)/24});
			\draw[blue] (2, 2.25) -- (4, 1.125);
			\draw[blue] plot[smooth, domain=1:1.23] ({\x+3}, {(3-\x)*((\x)^6-15/2*(\x)^4+12*(\x)^2+8)/24});
			\draw[blue] (4.228, 0.920) -- (5, 0);

			\draw[thin] (1, 0.1) -- (1, -0.1) node[below, fill=cyan!10] {$a$};
			\draw[thin] (5, 0.1) -- (5, -0.1) node[below, fill=cyan!10] {$b$};
			\node at (5.5, -0.3) {$x$};
			\node[red] at (4.3, 0.3) {$\psi$};
			\node[blue] at (4.7, 0.6) {$u$};

			\draw[very thick] (-0.5, -0.7) rectangle (5.7, 2.5);
		\end{tikzpicture}
		\caption{Illustration of Obstacle problem.}
	\end{figure}

	Then minimize the functional $I(u)=\int_\Omega{\sqrt{1+\lvert\nabla u(x)\rvert^2}\mathrm{d}x}$ subject to $u(x)\geq\psi(x)$ for almost every $x\in\Omega$, and $u\vert_{\partial\Omega}=0$.\\

	Caveat: $f(A)=\sqrt{1+\lvert A\rvert^2}$ grows linear. Hence, coercivity holds only on $W_0^{1,1}(\Omega)$ which is a bad space as it is not reflexive. But there is a way out by simplification the model via Taylor expansion
	\[f(A)=f(0)+\partial_Af(0)\cdot A+\frac{1}{2}\partial_A^2f(0)A\cdot A+o(\lvert A\rvert^2).\]
	So for $\lvert A\rvert$ small it holds $f(A)=1+\frac{1}{2}\lvert A\rvert^2$. So we consider
	\[\widetilde{I}(u)=\int_\Omega{\left(1+\frac{1}{2}\lvert\nabla u(x)\rvert^2\right)\mathrm{d}x},\]
	which is coercive on $H_0^1(\Omega)$, subject to $u(x)\geq\psi(x)$ for almost every $x\in\Omega$.\\[11pt]
\end{itemize}
\end{example}

\begin{theorem}
Let $X$ be a reflexive Banach space, $M\subset X$ non-empty, weakly sequentially closed, $I:M\longrightarrow\mathbb{R}_\infty$ coercive on $M$ and weakly sequentially lower semicontinuous on $X$. Then there exists $u_*\in M$ such that
\[I(u_*)=\inf_{u\in M}{I(u)}.\]
Note that the constraint is encoded in the subset $M$.\\
\end{theorem}

\begin{proof}
The proof follows the same strategy as \hyperlink{theorem_3_1_14}{Theorem 3.1.14}.\hfill$\blacksquare$\\[11pt]
\end{proof}


In \hyperlink{examples_3_6_1}{Examples 3.6.1}, we had the sets
\begin{itemize}
	\item[(a)] $M=\{u\in H^1(\Omega)\mid u(a)=u(b)=0,\int_a^b{u(x)\mathrm{d}x}=A_0\}$,
	\item[(b)] $M=\{u\in H_0^1(\Omega)\mid\lVert u\rVert_{L^2(\Omega)}=1\}$,
	\item[(c)] $M=\{u\in H_0^1(\Omega)\mid u(x)\geq\psi(x)\text{ for almost every }x\in\Omega\}$.
\end{itemize}
Sometimes, like in (c) if e.g. $\psi\vert_{\partial\Omega}>0$, it is not clear that $M$ is non-empty. But mostly the crucial question is: When is $M\subset X$ weakly sequentially closed?\\[11pt]

\begin{example}
\begin{itemize}
	\item[(a)] If $M$ is convex, then Mazur's lemma tells us that $M$ is weakly sequentially closed if it is strongly closed.
	\item[(b)] But $M$ does not need to be convex for weak sequential closedness. As an example, consider a set-valued map $\Phi:\Omega\longrightarrow\mathcal{P}(\mathbb{R})$ with $\mathcal{P}(\mathbb{R})$ being the power set of $\mathbb{R}$, satisfying $\Phi(x)\subset\mathbb{R}$ closed, non-empty for all $x\in\Omega$. Then, for $p\geq1$ define
	\[M=\{u\in W^{1,p}(\Omega)\mid u(x)\in\Phi(x)\text{ for almost every }x\in\Omega\}.\]
	First we should mention that it can happen that $M$ is non-empty.

	\begin{figure}[ht]
		\centering
		\begin{tikzpicture}
			\fill[cyan!10] (-0.5, -0.5) rectangle (6.5, 3.5);
			\draw[cyan] (-0.5, -0.5) grid (6.5, 3.5);
			\draw[thick, ->] (-0.3, 0) -- (6.3, 0);
			\node at (6.3, -0.3) {$x$};
			\draw[thick, ->] (0, -0.3) -- (0, 3.3);

			\fill[red, opacity=0.2] plot[smooth, domain=0:3] (\x, {sqrt(\x+0.5)}) -- (3, 2.871) plot[smooth, domain=3:0] (\x, {1+sqrt(\x+0.5)}) -- (0, 0.707);
			\fill[red, opacity=0.2] plot[smooth, domain=3:6, samples=200] (\x, {2-sqrt(1-\x/6)}) -- (6, 1) plot[smooth, domain=6:3, samples=200] (\x, {1-sqrt(1-\x/6)}) -- (3, 1.293);
			\draw[red] plot[smooth, domain=0:3] (\x, {sqrt(\x+0.5)});
			\draw[red] plot[smooth, domain=0:3] (\x, {1+sqrt(\x+0.5)});
			\draw[red] plot[smooth, domain=3:6, samples=200] (\x, {1-sqrt(1-\x/6)});
			\draw[red] plot[smooth, domain=3:6, samples=200] (\x, {2-sqrt(1-\x/6)});

			\draw[thick, blue] plot[smooth, domain=0:3] (\x, {0.6+ln(\x+2)});
			\draw[thick, blue] plot[smooth, domain=3:6] (\x, {ln(\x-1)+0.3*cos((\x-3)*90)});

			\node[red] at (1.5, 2.7) {$\Phi$};
			\node[blue] at (3.2, 2.2) {$u$};
			\draw[very thick] (-0.5, -0.5) rectangle (6.5, 3.5);
		\end{tikzpicture}
		\caption{$\Phi$ with a sort of jump.}
		\label{fig:example_3_6_3_empty}
	\end{figure}

	\begin{figure}[ht]
		\centering
		\begin{tikzpicture}
			\fill[cyan!10] (-0.5, -0.7) rectangle (6.5, 3.5);
			\draw[cyan] (-0.5, -0.7) grid (6.5, 3.5);
			\draw[thick, ->] (-0.3, 0) -- (6.3, 0);
			\node at (6.3, -0.3) {$x$};
			\draw[thick, ->] (0, -0.5) -- (0, 3.3);
			\draw[thin] (1, 0.1) -- (1, -0.1) node[below, fill=cyan!10] {$a$};
			\draw[thin] (5, 0.1) -- (5, -0.1) node[below, fill=cyan!10] {$b$};

			\fill[red, opacity=0.2] plot[smooth, domain=1:5] (\x, {2+(\x-2)^2/10}) -- (5, 1.5) -- (4.9, 1.5) arc (0:180:0.7 and 0.3) -- (1, 1.5) -- (1, 2.1);
			\fill[red, opacity=0.2] plot[smooth, domain=1:5] (\x, {0.8-(\x-2.5)^2/20}) -- (5, 1.5) -- (4.9, 1.5) arc (360:180:0.7 and 0.3) -- (1, 1.5) -- (1, 0.9125);
			\draw[red] plot[smooth, domain=1:5] (\x, {2+(\x-2)^2/10});
			\draw[red] plot[smooth, domain=1:5] (\x, {0.8-(\x-2.5)^2/20});
			\draw[red] (4.2, 1.5) ellipse (0.7 and 0.3);

			\draw[thick, blue] plot[smooth, domain=1:5] (\x, {1.7+(\x-2.5)^2/8});
			\draw[thick, blue] plot[smooth, domain=1:5] (\x, {1+0.2*sin(\x*60)});
			\draw[thick, olive] plot[smooth, domain=1:5] (\x, {1.35+(\x-2.5)^2/16+sin(\x*60)/10});

			\node[red] at (3.5, 2.5) {$\Phi$};
			\node[blue, fill=cyan!10] at (5.8, 2.5) {$u_1\in M$};
			\node[olive, fill=cyan!10] at (5.7, 1.7) {$u\notin M$};
			\node[blue, fill=cyan!10] at (5.8, 0.9) {$u_2\in M$};
			\draw[very thick] (-0.5, -0.7) rectangle (6.5, 3.5);
		\end{tikzpicture}
		\caption{$\Phi(x)$ not connected for all $x\in[a,b]$.}
		\label{fig:example_3_6_3_not_connected}
	\end{figure}

	For example, for a $\Phi$ such as in \hyperref[fig:example_3_6_3_empty]{Figure III.11}, $M$ is empty because any $u$ with $u(x)\in\Phi(x)$ has a jump. Moreover, if e.g. $\Phi(x)$ is not connected for all $x$ (illustrated in \hyperref[fig:example_3_6_3_not_connected]{Figure III.12}), then $M$ is not convex.\\

	But $M$ is weakly sequentially closed as a subset of $W^{1,p}(\Omega)$. Indeed, let $(u_n)_{n\in\mathbb{N}}\subset M$, $u\in W^{1,p}(\Omega)$ such that $u_n\rightharpoonup u$ in $W^{1,p}(\Omega)$ for $n\to\infty$. We need $u\in M$. By Rellich's compact theorem we have $u_n\to u$ in $L^p(\Omega)$. So there exists a subsequence $(u_{n_k})_{k\in\mathbb{N}}\subset(u_n)_{n\in\mathbb{N}}$ such that $u_{n_k}(x)\to u(x)$ for $k\to\infty$ and almost all $x\in\Omega$. Since $\Phi(x)\subset\mathbb{R}$ is closed, we therefore have $u(x)\in\Phi(x)$ for almost all $x\in\Omega$. Thus, $u\in M$.\\[11pt]
\end{itemize}
\end{example}

\begin{corollary}
(\hyperlink{examples_3_6_1}{Examples 3.6.1 (c)}, Obstacle problem)\\
Let $\Omega\subset\mathbb{R}^d$ be bounded, open with Lipschitz boundary. For $\psi\in H_0^1(\Omega)$ define
\[M_\psi:=\{u\in H_0^1(\Omega)\mid u\geq\psi\text{ almost everywhere in }\Omega\}.\]
Set $\Phi(x)=[\psi(x),+\infty)$ which is closed in $\mathbb{R}$. As $\psi\in M_\psi$, we have $M_\psi\ne\emptyset$.\\

Hence $M_\psi$ is weakly sequentially closed. \hyperlink{theorem_3_6_2}{Theorem 3.6.2} therefore yields a minimizer $u_*\in M_\psi$ of
\[\widetilde{I}(u)=\int_\Omega{1+\frac{1}{2}\lvert\nabla u(x)\rvert^2\mathrm{d}x}\]\\[11pt]
\end{corollary}

\begin{example}
Consider $X=W^{1,p}(\Omega)$, $p\in(1,\infty)$, $\mathcal{J}(u)=\lVert u\rVert_{L^r(\Omega)}^r$ and $M=\{u\in X\mid\mathcal{J}(u)=1\}$. We ask when $M$ is weakly sequentially closed. The idea is to use Rellich's compact embedding, so we need $1\geq\frac{1}{r}\geq\frac{d-p}{dp}$ in order to have the compact embedding $W^{1,p}(\Omega)\clonghookrightarrow L^r(\Omega)$. Hence, $u_n\rightharpoonup u$ in $W^{1,p}(\Omega)$ yields $u_n\to u$ in $L^r(\Omega)$, and then $\mathcal{J}(u)=\lim_{n\to\infty}{\mathcal{J}(u_n)}=1$.
\end{example}

Next we want to treat the question whether we can characterize minimizers $u_*$ via suitable Euler-Lagrange equations. To answer this question, we restrict our discussion to equality constraints, i.e. of the form $\mathcal{J}(u)=a$ for some $a\in\mathbb{R}$. For applications this is really restrictive and usually inequality constraints are of more interest. The obstacle problem for example is no equality constraint. Inequality constraints are related to so-called (KKT) conditions (``Karush-Kuhn-Tucker''). This is program in optimization theory.\\

In the following we will write for Banach spaces $X,Y$
\[C^1(X,Y):=\{\mathcal{J}:X\longrightarrow Y\mid\mathcal{J}\text{ is Fr\'echet differentiable}\},\]
that means $\mathcal{J}$ is G\^ateaux differentiable and $D\mathcal{J}:X\longrightarrow\Lin{X}{Y}$ is continuous, where $\Lin{X}{Y}$ denotes the space of all continuous, linear maps from $X$ to $Y$. The notion goes back to \textsc{Maurice Fr\'echet} (* 1878; $\dagger$ 1973). He was a student of Hadamard.\\

\begin{theorem}[Lagrange multipliers]
Let $X$ be a reflexive Banach space, $I,\mathcal{J}\in C^1(X;\mathbb{R})$. Put $M_\alpha:=\{u\in X\mid\mathcal{J}(u)=\alpha\}$ for $\alpha\in\mathbb{R}$ and suppose $M_\alpha\ne\emptyset$. Further assume that
\begin{itemize}
	\item[(a)] $I$ is weakly sequentially lower semicontinuous on $X$;
	\item[(b)] $\mathcal{J}$ is weakly sequentially continuous on $X$;
	\item[(c)] $I:M_\alpha\longrightarrow\mathbb{R}$ is coercive.
\end{itemize}
Then
\begin{itemize}
	\item[(i)] $I:M_\alpha\longrightarrow\mathbb{R}$ has a minimizer $u_*\in M_\alpha$.
	\item[(ii)] If in addition $D\mathcal{J}(u_*)\ne0$ then there exists $\lambda_*\in\mathbb{R}$, a so-called \textit{Lagrange multiplier}, such that $DI(u_*)=\lambda_*D\mathcal{J}(u_*)\in X'$.\\
\end{itemize}
\end{theorem}

\begin{remark}
\begin{itemize}
	\item[(a)] Let's consider a more specific setting. Let $X=W_0^{1,p}(\Omega;\mathbb{R}^m)$ for $p\in(1,\infty)$, and let $f:\Omega\times\mathbb{R}^m\times\mathbb{R}^{m\times d}\longrightarrow\mathbb{R}$, $h:\Omega\times\mathbb{R}^m\longrightarrow\mathbb{R}$ be such that
	\[I(u):=\int_\Omega{f(x,u(x),\nabla u(x))\mathrm{d}x}\quad\text{and}\quad\mathcal{J}(u):=\int_\Omega{h(x,u(x))\mathrm{d}x}\]
	satisfy the assumptions of \hyperlink{theorem_3_6_6}{Theorem 3.6.6}. Then the weak Euler-Lagrange equations with constraint given by $\mathcal{J}$ is
	\[\int_\Omega{\partial_Af(\cdot,u_*,\nabla u_*):\nabla v+\partial_uf(\cdot,u_*,\nabla u_*)\cdot v\mathrm{d}x}=\lambda_*\int_\Omega{\partial_uh(\cdot,u_*)\cdot v\mathrm{d}x}\]
	for all $v\in W_0^{1,p}(\Omega;\mathbb{R}^m)$.
	\item[(b)] If $\mathcal{J}:X\longrightarrow Y$, i.e. $\mathcal{J}$ maps into a general (possibly infinite-dimensional) Banach space, and $M_{y_0}=\{u\in X\mid\mathcal{J}(u)=y_0\}$ for some $y_0\in Y$, then there is a Theorem by Ljusternik that gives $\lambda_*\in Y'$ so that the Euler-Lagrange equations are
	\[DI(u_*)[v]=\lambda_*(\underbrace{D\mathcal{J}(u_*)[v]}_{\in Y})\]
	for all $v\in X$. See \cite[Chapter 4, 4.2 Ljusternik's Theorems]{blanchard_bruening}.\\
\end{itemize}
\end{remark}

For the proof of \hyperlink{theorem_3_6_6}{Theorem 3.6.6} we need the following preliminary result.\\

\begin{theorem}[Implicit Function Theorem]
Let $X,Y,Z$ be Banach spaces, $F:X\times Y\longrightarrow Z$ a $C^1$-function. Let $(x_0,y_0)\in X\times Y$ such that $F(x_0,y_0)=0\in Z$ and $(D_yF(x_0,y_0))^{-1}\in\operatorname{Lin}(Z,Y)$ exists. Then, there exist $\varepsilon>0$, $\Phi_0\in C^1(B_\varepsilon(x_0),Y)$ (with $B_\varepsilon(x_0)=\{x\in X\mid \lVert x-x_0\rVert_X<\varepsilon\}$) such that
\begin{itemize}
	\item[(i)] $\Phi_0(x_0)=y_0$,
	\item[(ii)] $F(x,\Phi_0(x))=0$ for all $x\in B_\varepsilon(x_0)$,
	\item[(iii)] From $F(x,y)=0$ and $\lVert x-x_0\rVert+\lVert y-y_0\rVert<\varepsilon$ it follows $\Phi_0(x)=y$.\\
\end{itemize}
\end{theorem}

\begin{proof}
The proof can be found in \cite[Kapitel III, III.5 Differentiation nichtlinearer Abbildungen, Satz III.5.4 (e)]{dirk_werner}.\\[11pt]

\textit{Proof of \hyperlink{theorem_3_6_6}{Theorem 3.6.6}:}\\
First note that the existence of a minimizer $u_*\in M_\alpha$ follows from weak closedness of the set $M_\alpha\subset X$ and \hyperlink{theorem_3_6_2}{Theorem 3.6.2}. The proof of assertion (ii) is divided into three steps.\\

\textit{Step 1:}
\begin{itemize}
	\item[] We describe $M_\alpha$ locally around $u_*\in M_\alpha$ over tangent space. The tangent space of $M_\alpha$ in $u_*$ is
	\[V_{u_*}=\{v\in X\mid D\mathcal{J}(u_*)[v]=0\},\]
	and $V_{u_*}$ is a subspace of $X$ because $D\mathcal{J}(u_*)$ is linear. By assumption $D\mathcal{J}(u_*)\ne0$, there exists $v_1\in X$ such that $D\mathcal{J}(u_*)[v_1]=1$. Define the function
	\[F:V_{u_*}\times\mathbb{R}\longrightarrow\mathbb{R},\qquad F(v,\delta):=\mathcal{J}(u_*+v+\delta v_1)-\alpha.\]
	We know
	\begin{itemize}
		\item[(1)] $F(0,0)=0$ since $u_*\in M_\alpha$,
		\item[(2)] $F\in C^1(V_{u_*}\times\mathbb{R})$ since $\mathcal{J}$ is a $C^1$-function, and
		\item[(3)] $\partial_\delta F(0,0)=D\mathcal{J}(u_*)[v_1]=1$.
	\end{itemize}

	\begin{figure}[ht]
		\centering
		\begin{tikzpicture}
			\draw[thick, red] ellipse (3 and 1.5);
			\draw[blue] (-1, -2.4597) -- (5, 0.2236);

			\draw[thick, violet] (0.7, -1.6994) -- (3.3, -0.5366);
			\draw[violet] (3.3, -0.5366) arc (24.1:54.1:0.3);
			\draw[violet] (3.3, -0.5366) arc (24.1:-5.9:0.3);
			\draw[violet] (0.7, -1.6994) arc (204.1:234.1:0.3);
			\draw[violet] (0.7, -1.6994) arc (204.1:174.1:0.3);

			\draw[thick, ->] (2, -1.118) -- (3, -0.6708);
			\draw[thick, ->] (3, -0.6708) -- (2.8844, -0.4124);

			\node[red] at (-2, -0.6) {$M_\alpha$};
			\node[blue] at (5.4, 0.2236) {$V_{u_*}$};
			\node at (3, -1) {\scriptsize$v\in V_{u_*}$};
			\node[fill=white] at (3, 0) {\scriptsize$\hat{\delta}(v)v_1$};
			\node[violet] at (1.9, -1.8) {$V_{u_*}\cap B_\varepsilon(0)$};
			\fill (2, -1.118) circle (1.5pt) node[above] {$u_*$};
		\end{tikzpicture}
		\caption{Illustration of tangent space $V_{u_*}$ and function $\hat{\delta}$.}
	\end{figure}

	Hence, the implicit function theorem, i.e. \hyperlink{theorem_3_6_8}{Theorem 3.6.8}, gives $\varepsilon>0$ and a mapping $\hat{\delta}\in C^1(V_{u_*}\cap B_\varepsilon(0);\mathbb{R})$ with $\hat{\delta}(0)=0$ and $F(v,\hat{\delta}(v))=0$ for all $v\in V_{u_*}\cap B_\varepsilon(0)$ (here, $B_\varepsilon(0)=\{x\in X\mid\lVert x\rVert_X<\varepsilon\}$). The last property means $\mathcal{J}(u_*+v+\hat{\delta}(v)v_1)=\alpha$ for all $v\in V_{u_*}\cap B_\varepsilon(0)$, so that
	\begin{align}\label{eq:mcov_formula_3_4}
		\{u_*+v+\hat{\delta}(v)v_1\mid v\in V_{u_*}\cap B_\varepsilon(0)\}\subseteq M_\alpha.
	\end{align}\\
\end{itemize}

\textit{Step 2:} We show that $DI(u_*)[w]=0$ for all $w\in V_{u_*}$.
\begin{itemize}
	\item[] Define $\widetilde{I}(v):=I(u_*+v+\hat{\delta}(v)v_1)$ for $v\in V_{u_*}\cap B_\varepsilon(0)$. As $u_*$ minimizes $I$, we get $\widetilde{I}(0)\leq\widetilde{I}(v)$ for all $v\in V_{u_*}\cap B_\varepsilon(0)$ by \eqref{eq:mcov_formula_3_4}. Thus,
	\[\lim_{h\searrow0}{\frac{\widetilde{I}(hw)-\widetilde{I}(0)}{h}}\geq0\]
	for all $w\in V_{u_*}$, and hence $D\widetilde{I}(0)[w]\geq0$. By replacing $w$ by $-w$ (remind $V_{u_*}$ is a linear space) we even get $D\widetilde{I}(0)[w]=0$ for all $w\in V_{u_*}$. Via chain rule this means
	\begin{align}\label{eq:mcov_formula_3_5}
		D\widetilde{I}(0)=DI(u_*)[w+D\hat{\delta}(0)[w]v_1]=0
	\end{align}
	for all $w\in V_{u_*}$. We claim $D\hat{\delta}(0)[w]=0$ for all $w\in V_{u_*}$. We know $G(v):=F(v,\hat{\delta}(v))=0$ for all $v\in V_{u_*}\cap B_\varepsilon(0)$. Fix $w\in V_{u_*}$. Then
	\[0=D_vG(v)[w]=D_vF(v,\hat{\delta}(v))[w]+D_\delta F(v,\hat{\delta}(v))[D\hat{\delta}(v)[w]].\]
	Using the definition of $F$ in terms of $\mathcal{J}$ we get
	\[0=D\mathcal{J}(u_*+v+\hat{\delta}(v)v_1)[w]+D\mathcal{J}(u_*+v+\hat{\delta}(v)v_1)[v_1]D\hat{\delta}(v)[w].\]
	Thus, for $v=0$ we get
	\[0=\underbrace{D\mathcal{J}(u_*)[w]}_{=0\text{ as }w\in V_{u_*}}+\underbrace{D\mathcal{J}(u_*)[v_1]}_{=1\text{ choice of }v_1}D\hat{\delta}(0)[w],\]
	i.e., $D\hat{\delta}(0)[w]=0$ as claimed. Therefore, $DI(u_*)[w]=0$ for all $w\in V_{u_*}$ by \eqref{eq:mcov_formula_3_5}.\newpage
\end{itemize}

\textit{Step 3:}
\begin{itemize}
	\item[] Write $X$ as the direct sum of $V_{u_*}$ and $\operatorname{span}\{v_1\}$, so $X=V_{u_*}\oplus\operatorname{span}\{v_1\}$. Indeed, if $w\in X$ is given, write $D\mathcal{J}(u_*)[w]=\beta_w$ and $v=w-\beta_wv_1$. We check
	\[D\mathcal{J}(u_*)[v]=D\mathcal{J}(u_*)[w]-\beta_wD\mathcal{J}(u_*)[v_1]=\beta_w-\beta_w\cdot 1=0.\]
	Therefore, $v\in V_{u_*}$. Moreover, for $w=v+\beta_wv_1$ we have
	\[DI(u_*)[w]=\underbrace{DI(u_*)[v]}_{=0\text{ by step 2.}}+\beta_w\underbrace{DI(u_*)[v_1]}_{=:\lambda_*}=\beta_w\lambda_*=\lambda_*D\mathcal{J}(u_*)[w].\]
\end{itemize}
\end{proof}

\begin{example}[Eigenvalue of Laplacian]
Let $\Omega\subset\mathbb{R}^d$ be an open, bounded domain with Lipschitz boundary. Consider $X=H_0^1(\Omega)$ and
\[I(u)=\int_\Omega{\lvert\nabla u(x)\rvert^2\mathrm{d}x},\qquad\mathcal{J}(u)=\int_\Omega{\lvert u(x)\rvert^2\mathrm{d}x}.\]
\textit{Definition:} We call $\lambda\in\mathbb{C}$ an \textit{eigenvalue of Laplacian $(-\Delta)$ on $H_0^1(\Omega)$} if $u_\lambda\in H_0^1(\Omega)\setminus\{0\}$ exists such that for all $v\in H_0^1(\Omega)$
\[\int_\Omega{\nabla u_\lambda(x)\cdot\nabla v(x)\mathrm{d}x}=\lambda\int_\Omega{u_\lambda(x)v(x)\mathrm{d}x}.\]
This is the weak formulation of the partial differential equation
\[\left\{\begin{array}{rl}
	-\Delta u_\lambda=\lambda u_\lambda&\text{in }\Omega,\\
	u=0&\text{on }\partial\Omega.
\end{array}\right.\]
In this case, $u_\lambda$ is called an \textit{eigenfunction}.\\

We claim:
\begin{itemize}
	\item[(a)] All eigenvalues are real and positive.
	\item[(b)] The smallest eigenvalue $\lambda_{\text{min}}>0$ exists and is given by
	\[\lambda_\text{min}=\min_{\substack{u\in H_0^1(\Omega)\\u\ne0}}{\frac{I(u)}{\mathcal{J}(u)}}=\min_{\substack{u\in H_0^1(\Omega)\\\mathcal{J}(u)=1}}{I(u)},\]
	and all minimizers are eigenfunctions.\\
\end{itemize}

We want to prove this.
\begin{itemize}
	\item[(a)] Set $v=u_\lambda$ in the weak formulation to obtain $\lambda\in\mathbb{R}$ and $\lambda>0$.
	\item[(b)] Consider
	\[\min\{I(u)\mid u\in H_0^1(\Omega)\text{ and }\mathcal{J}(u)=1\}.\]
	Part (i) of \hyperlink{theorem_3_6_6}{Theorem 3.6.6} is applicable on $M_1=\{u\in H_0^1(\Omega)\mid\mathcal{J}(u)=1\}$ because this is weakly sequentially closed as $H_0^1(\Omega)\clonghookrightarrow L^2(\Omega)$. So there exists $u_*\in M_1$ so that
	\begin{align*}
		I(u_*)&=\min_{u\in M_1}{\int_\Omega{\lvert\nabla u(x)\rvert^2\mathrm{d}x}}=\min_{\substack{\tilde{u}\in H_0^1(\Omega)\\\tilde{u}\ne0}}{\int_\Omega{\left\lvert\nabla\frac{\tilde{u}(x)}{\lVert\tilde{u}\rVert_{L^2(\Omega)}}\right\rvert^2\mathrm{d}x}}\\
		&=\min_{\substack{\tilde{u}\in H_0^1(\Omega)\\\tilde{u}\ne0}}{\frac{\int_\Omega{\lvert\nabla\tilde{u}(x)\rvert^2\mathrm{d}x}}{\int_\Omega{\lvert\tilde{u}(x)\rvert^2\mathrm{d}x}}}=\min_{\substack{\tilde{u}\in H_0^1(\Omega)\\\tilde{u}\ne0}}{\frac{I(\tilde{u})}{\mathcal{J}(\tilde{u})}}.
	\end{align*}
	We have $D\mathcal{J}(u_*)[w]=2\int_\Omega{u_*(x)w(x)\mathrm{d}x}$ for all $w\in H_0^1(\Omega)$. For $w=u_*$ it follows $D\mathcal{J}(u_*)[u_*]=2\lVert u_*\rVert_{L^2(\Omega)}^2=2\ne0$. Thus, assertion (ii) of \hyperlink{theorem_3_6_6}{Theorem 3.6.6} gives existence of $\lambda_*\in\mathbb{R}$ such that $DI(u_*)=\lambda_*D\mathcal{J}(u_*)$. This means explicitly that
	\[\int_\Omega{\nabla u_*(x)\cdot\nabla v(x)\mathrm{d}x}=\lambda_*\int_\Omega{u_*(x)v(x)\mathrm{d}x}\]
	for all $v\in H_0^1(\Omega)$ (after division by 2). So $u_*$ is indeed an eigenfunction for the eigenvalue $\lambda_*$, and we see, if we set $v=u_*$ again, that $\lambda_*$ is the minimum, i.e.
	\[\min_{\substack{u\in H_0^1(\Omega)\\u\ne0}}{\frac{I(u)}{\mathcal{J}(u)}}=I(u_*)=\int_\Omega{\lvert\nabla u_*(x)\rvert^2\mathrm{d}x}=\lambda_*\int_\Omega{\lvert u_*(x)\rvert^2\mathrm{d}x}=\lambda_*.\]
	It remains to show that $\lambda_*$ is the smallest eigenvalue. Suppose $\tilde{\lambda}>0$ is an eigenvalue with $\lambda_*>\tilde{\lambda}$ and some eigenfunction $\tilde{u}\in H_0^1(\Omega)\setminus\{0\}$. Then in particular
	\[\int_\Omega{\lvert\nabla\tilde{u}(x)\rvert^2\mathrm{d}x}=\tilde{\lambda}\int_\Omega{\lvert\tilde{u}(x)\rvert^2\mathrm{d}x}.\]
	After rescaling we can assume $1=\lVert\tilde{u}\rVert_{L^2(\Omega)}^2=\mathcal{J}(\tilde{u})$ and thus $\tilde{u}\in M_1$. Hence,
	\[\tilde{\lambda}=\int_\Omega{\lvert\nabla\tilde{u}(x)\rvert^2\mathrm{d}x}=I(\tilde{u})\geq I(u_*)=\lambda_*\]
	which is a contradiction.\\
\end{itemize}

Other eigenvalues of the Dirichlet Laplacian can be obtained by iterating this process and further restriction of $M_1$, e.g. by looking at functions which lie in the complement of eigenspaces.
\end{example}

    \chapter{Nonconvex Problems}
In the last chapter, we treated functionals where we have assumed that the volume density is convex in the third argument, i.e. $A\longmapsto f(x,u,A)$ convex. This, as well as some additional assumptions, ensured that the functional $I(u)=\int_\Omega{f(x,u(x),\nabla u(x))\mathrm{d}x}$ become weakly lower semicontinuous on $W^{1,p}(\Omega;\mathbb{R}^m)$. In general, this convexity-assumption is sufficient but not necessary. In fact, if $\min\{m,d\}=1$, it is also necessary as we have seen in the exercise lessons. Moreover, in some applications convexity in $A$ is a too strong assumption. We will have a look at an application from continuum mechanics (nonlinear elasticity) as a motivating example. In fact, this example is one of the main reasons why the theory we are going to treat in this chapter was developed.\\

Let's consider $\Omega\subset\mathbb{R}^d$ describing a reference configuration of some elastic body and a function $u:\Omega\longrightarrow\mathbb{R}^d$ which describes the deformation of the body in presence of some external forces $F$.

\begin{figure}[ht]
	\centering
	\begin{tikzpicture}
		\draw[red] rectangle (7, 1.5);
		\draw[red] (1, 1) -- (1, 2.5) -- (8, 2.5) -- (8, 1);
		\draw[red, dashed] (1, 1) -- (8, 1);
		\draw[red] (0, 0) -- (1, 1) (7, 0) -- (8, 1) (0, 1.5) -- (1, 2.5) (7, 1.5) -- (8, 2.5);

		\fill[red] (7.4, 0.8) circle (1.5pt);
		\draw[red, dotted] (7.6, 1) -- (7.4, 0.8) -- (7.8, 0.8);

		\draw[thick, blue, fill=blue, fill opacity=0.2] (0, 0) -- (1, 1) -- (1, 2.5) -- (0, 1.5) -- (0, 0);
		\draw[thick, blue] (0, 0) arc (90:0:4 and 2);
		\draw[thick, blue, dashed] (1, 1) arc (90:0:5.5 and 3);
		\draw[thick, blue] (0, 1.5) arc (90:0:4 and 2);
		\draw[thick, blue] (1, 2.5) arc (90:0:5.5 and 3);
		\draw[thick, blue] (4, -2) rectangle (6.5, -0.5);

		\fill[blue] (5.95, -1.6) circle (1.5pt);
		\draw[blue, dotted] (6, -2) -- (5.95, -1.6) -- (6.4, -1.5);

		\draw[thick, ->] (5, 0.6) -- node[left, fill=white] {$F$} (5, -0.4);

		\node[red] at (8.4, 1.75) {$\Omega$};
		\node[red] at (7.25, 0.65) {$x$};
		\node[blue] at (5.4, -1.6) {$u(x)$};
		\node[blue] at (0.5, 1.1) {$\Gamma_D$};
	\end{tikzpicture}
	\caption{Illustration of an elastic deformation with fixed side.}
\end{figure}

The elastic energy of the body is
\[\int_\Omega{f(x,\nabla u(x))\mathrm{d}x},\]
and it is independent of $u$ since constant shifting or rotation should not change the energy. The function $f$ is called ``stored elastic energy density''. The deformation $u$ is the minimizer of
\[I(u)=\int_\Omega{f(x,\nabla u(x))\mathrm{d}x}-\ell(u),\]
where $\ell\in(W^{1,p}(\Omega;\mathbb{R}^d))'$, e.g.
\[\ell(u)=\int_\Omega{F(x)\cdot u(x)\mathrm{d}x}+\int_{\Gamma_N}{G(x)\cdot u(x)\mathrm{d}a}.\]
The first integral could describe gravity and the second some Neumann boundary condition. Physically relevant properties of $f$ are:
\begin{itemize}
	\item[(a)] If we think of a sponge, the energy should increase if we stretch or shrink it. The displacement $u(x)=sx$ describes such an action, and for such $u$ it holds $\nabla u(x)=s\mathrm{Id}$. Thus, we would like to have $f(x,s\cdot\mathrm{Id})>f(x,\mathrm{Id})$ for all $s\ne1$.
	\item[(b)] We also would like to have an invariance property of the form: For all rotations $R\in\SO{d}$ and all $A\in\mathbb{R}^{d\times d}$ it holds $f(RA)=f(A)$.\\

	For example, consider a displacement $u:\Omega\longrightarrow\mathbb{R}^d$ and $\widetilde{u}(x)=Rx+r$, where $Rx$ describes a rotation and $r$ a translation. The gradient of the composition $\widehat{u}=\widetilde{u}\circ u$ is then just $\nabla\widehat{u}=R\nabla u$. The invariance property gives $f(\nabla\widehat{u})=f(\nabla u)$.\\

	So, rigid-body transformation do not change the elastic energy. This is also known as ``objectivity'' (a change of frame of reference is a rigid-body transformation).
\end{itemize}
But there is a problem: There is no convex $f:\mathbb{R}^{d\times d}\longrightarrow\mathbb{R}$ which satisfies the properties (a) and (b) simultaneously. Consider the case $d=2$. Then
\[R=\begin{pmatrix}
	\cos(\vartheta)&-\sin(\vartheta)\\
	\sin(\vartheta)&\cos(\vartheta)
\end{pmatrix}\in\SO{2}.\]
If $f:\mathbb{R}^{2\times 2}\longrightarrow\mathbb{R}$ was convex, then for $A=\frac{1}{2}(R+R^\top)=\cos(\vartheta)\mathrm{Id}$ we would have
\[f(A)=f\left(\frac{1}{2}R+\frac{1}{2}R^\top\right)\leq\frac{1}{2}f(R)+\frac{1}{2}f(R^\top)=f(\mathrm{Id})\]
by property (b). But, if $\cos(\vartheta)<1$, property (a) would give us also $f(A)=f(\cos(\vartheta)\mathrm{Id})>f(\mathrm{Id})$.\\

Likewise one can show for general $d\geq2$ that no such $f$ exists.
    \section{Poly-, Quasi- and Rank-1-Convexity}

We discuss different notions of non-convexity and their relevance for weak sequential lower semicontinuity of $I(u)=\int_\Omega{f(x,\nabla u(x))\mathrm{d}x}$. We will see the following relations:
\[\text{convexity}\quad\Rightarrow\quad\text{poly-convexity}\quad\Rightarrow\quad\text{quasi-convexity}\quad\Rightarrow\quad\underset{(\text{+technical assumptions})}{\text{rank-1-convexity}}.\]
We already have seen that convexity of the density $f$ is sufficient, and it is also necessary in case $\min\{m,d\}=1$. It turns out that quasi-convexity is sufficient and necessary in order to have weak lower semicontinuity. However quasi-convexity is hard to check, so one should avoid checking this property. In comparison, poly-convexity is more easy to check, and still sufficient for weak lower semicontinuity. Rank-1-convexity is also not difficult to verify but this is more suitable for excluding functional ones from further investigation.\\

Poly-convexity is based on the minors of $A\in\mathbb{R}^{m\times d}$, i.e. the subdeterminants. In order to make our definitions for the notions of convexity, we need to introduce some notation first.\\[11pt]

\begin{notation}
Fix $A\in\mathbb{R}^{m\times d}$ and $s\in\{1,\dotsc,\min\{m,d\}\}$. For tuples $K=(k_1,\dotsc,k_s)$, $L=(l_1,\dotsc,l_s)$ where $1\leq k_1<k_2<k_3<\dotsc<k_s\leq m$ and $1\leq l_1<l_2<\dotsc<l_s\leq d$, we denote by $A_{K,L}=(A_{k_i,l_i})_{i,j=1}^s\in\mathbb{R}^{s\times s}$ the submatrix consisting of $K$-th rows and $L$-th columns, and by $a_{K,L}:=\det(A_{K,L})$ its $(K,L)$-th minor.\\

Next, we define
\[
    T_s(A)=\left\{a_{K,L}\,\middle\vert\,
    \begin{array}{c}
	K,L\in\mathbb{N}^s\text{ such that }1\leq k_1<\dotsc<k_s\leq m\\
	\text{and }1\leq l_1<\dotsc<l_s\leq d
    \end{array}\right\}
\]
as the \textit{set of minors of order $s$}. We are able to choose
\begin{itemize}
	\item $\binom{d}{s}=\frac{d!}{(d-s)!s!}$ different tuples of column indices,
	\item $\binom{m}{s}=\frac{m!}{(m-s)!s!}$ different tuples of row indices.
\end{itemize}
The number $\tau_s(m,d):=\#T_s(A)=\binom{m}{s}\binom{d}{s}$ is the 
\textit{number of minors of order $s$}. By choosing a suitable order 
of the minors, we can consider $T_s$ as a map 
$T_s:\mathbb{R}^{m\times d}\longrightarrow\mathbb{R}^{\tau_s(m,d)}$.\\

Finally, we define
\[
    T:\mathbb{R}^{m\times d}\longrightarrow\mathbb{R}^{\tau(m,d)},
    \qquad T(A):=\left(T_1(A),\dotsc,T_{\min\{m,d\}}(A)\right),
\]
where $\tau(m,d):=\sum_{s=1}^{\min\{m,d\}}{\tau_s(m,d)}$.\\[11pt]
\end{notation}

\begin{example}
Let $m=d=3$ and write
\[
    A=
    \begin{pmatrix}
	A_{11}&A_{12}&A_{13}\\
	A_{21}&A_{22}&A_{23}\\
	A_{31}&A_{32}&A_{33}
    \end{pmatrix}\in\mathbb{R}^{3\times 3}.
\]
Then
\begin{itemize}
	\item[(a)] $T_1(A)=(A_{11},A_{12},A_{13},A_{21},A_{22},A_{23},A_{31},A_{32},A_{33})\in\mathbb{R}^9$.
	\item[(b)] $T_3(A)=\det(A)\in\mathbb{R}^1$.
	\item[(c)] $T_2(A)\in\mathbb{R}^9$ and contains the nine determinants of all $2\times2$-submatrices. This is -- up to $\mathbb{R}^9\simeq\mathbb{R}^{3\times3}$ and some signs -- exactly the cofactor matrix $\Cof{A}$ which is well known from linear algebra.
\end{itemize}
So, here $T(A)$ corresponds to $(A,\Cof{A},\det(A))\in\mathbb{R}^{19}$.
\end{example}


\begin{definition}
Let $f:\mathbb{R}^{m\times d}\longrightarrow\mathbb{R}\cup\{+\infty\}$.
\begin{itemize}
	\item[(i)] $f$ is \textit{convex} if for all $A,B\in\mathbb{R}^{m\times d}$, all $\lambda\in[0,1]$ it holds
	\[f(\lambda A+(1-\lambda)B)\leq\lambda f(A)+(1-\lambda)f(B).\]
	\item[(ii)] $f$ is \textit{poly-convex} if $f$ is a convex function in its minors, that means, there exists a convex function $g:\mathbb{R}^{\tau(m,d)}\longrightarrow\mathbb{R}\cup\{+\infty\}$ such that $f(A)=g(T(A))$ for all $A\in\mathbb{R}^{m\times d}$.
	\item[(iii)] $f$ is \textit{quasi-convex} if $f$ is Borel-measurable and if for all $A\in\mathbb{R}^{m\times d}$, all $\varphi\in C_c^\infty(B_1(0);\mathbb{R}^m)$ it holds
	\[\int_{B_1(0)}{f(A+\nabla\varphi(x))\mathrm{d}x}\geq f(A)\vol{B_1(0)}.\]
	\item[(iv)] $f$ is \textit{rank-1-convex} if for all $A,B\in\mathbb{R}^{m\times d}$ with $\Rank{A-B}=1$, all $\lambda\in[0,1]$ it holds
	\[f(\lambda A+(1-\lambda)B)\leq\lambda f(A)+(1-\lambda)f(B).\]\\
\end{itemize}
\end{definition}


\begin{remark}
For $A,B\in\mathbb{R}^{2\times2}$ consider
\[w:\mathbb{R}^2\longrightarrow\mathbb{R}^2,\qquad w(x):=\left\{\begin{array}{rl}
	Ax&\text{if }x_1<0,\\
	Bx&\text{if }x_1\geq0.
\end{array}\right.\]
For the choices $A=\begin{pmatrix}1&0\\0&0\end{pmatrix}$, $B=\begin{pmatrix}0&0\\0&1\end{pmatrix}$ the map $w$ is not continuous. As an exercise, one can show that the following are equivalent:
\begin{itemize}
	\item[(a)] $w$ is continuous.
	\item[(b)] $\Rank{A-B}=1$.
	\item[(c)] There exist $a\in\mathbb{R}^d$, $b\in\mathbb{R}^m$ with $A-B=a\otimes b$.\\[11pt]
\end{itemize}
\end{remark}

\begin{remark}
\begin{itemize}
	\item[(a)] In the definition of poly-convexity, the function $g$ is in general not unique. For example consider $m=d=2$ and $f:\mathbb{R}^{2\times 2}\longrightarrow\mathbb{R}$, given by
	\[f(A)=\lvert A\rvert^2=A_{11}^2+A_{22}^2+A_{21}^2+A_{12}^2=(A_{11}-A_{22})^2+(A_{12}+A_{21})^2+2\det(A).\]
	We can choose $g_1(A,\delta)=\lvert A\rvert^2$ but also $g_2(A,\delta)=(A_{11}-A_{22})^2+(A_{12}+A_{21})^2+2\delta$, i.e., both are convex and satisfy $f(A)=g_1(T(A))=g_2(T(A))$.
	\item[(b)] In the definition of quasi-convexity, we can equivalently consider arbitrary fixed, bounded, open and nonempty sets $D\subset\mathbb{R}^d$, i.e.,
	\[\int_D{f(A+\nabla\varphi(x))\mathrm{d}x}\geq f(A)\vol{D},\]
	and we can also take test functions $\varphi$ in the far more larger set $W_0^{1,\infty}(D;\mathbb{R}^m)$ instead of $C_c^\infty(D;\mathbb{R}^m)$. We will show this in the exercises.
	\item[(c)] If $\min\{m,d\}=1$, then rank-1-convexity is the same as convexity because then $f$ is defined on vectors which have rank at most 1.\\[11pt]
\end{itemize}
\end{remark}

\begin{theorem}
Let $f:\mathbb{R}^{m\times d}\longrightarrow\mathbb{R}_\infty$ be continuous (in the sense of compactification, that means we equip $\mathbb{R}_\infty$ with the metric $d(x,y):=\lvert\arctan(x)-\arctan(y)\rvert$).
\begin{itemize}
	\item[(i)] If $f$ is convex, then it is also poly-convex.
	\item[(ii)] If $f$ is poly-convex, then it is also quasi-convex.
	\item[(iii)] If $f$ is quasi-convex and in addition finite, then it is also rank-1-convex.\\
\end{itemize}
\end{theorem}

\begin{remark}
\textit{Remark: In some of the statements the continuity assumption on $f$ can be weaken by e.g. imposing only Borel-measurability. We just need to make sure that the integral in the definition of quasi-convexity is well-defined. However, if $f$ is convex, then it is continuous anyway.}
\end{remark}

\begin{proof}
\begin{itemize}
	\item[(i)] Choose $g(T)=f(A)$ where $T=(A,\dotsc)\in\mathbb{R}^{\tau(m,d)}$ and $A$ denotes the minors of order 1.
	\item[(ii)] Since $f$ is poly-convex, there exists $g:\mathbb{R}^{\tau(m,d)}\longrightarrow\mathbb{R}_\infty$ convex such that $f(A)=g(T(A))$. Let $\varphi\in C_c^\infty(B_1(0);\mathbb{R}^m)$ be given. Use Jensen's inequality and \hyperlink{theorem_4_1_10}{Theorem 4.1.10} to obtain
	\begin{align*}
		\int_{B_1(0)}{f(A+\nabla\varphi(x))\mathrm{d}x}&=\int_{B_1(0)}{g(T(A+\nabla\varphi(x)))\mathrm{d}x}\\
		&\geq\vol{B_1(0)}g\left(\frac{1}{\vol{B_1(0)}}\int_{B_1(0)}{T(A+\nabla\varphi(x))\mathrm{d}x}\right)\\
		&=\vol{B_1(0)}g(T(A))=\vol{B_1(0)}f(A),
	\end{align*}
	so $f$ is quasi-convex.
	\item[(iii)] The third part will be discussed in the exercise lesson.\hfill$\blacksquare$\\[11pt]
\end{itemize}
\end{proof}

\begin{lemma}[Gradient minors as divergence]
\label{lem:MinorsAsDivergence}
For $u\in C^3(\Omega;\mathbb{R}^m)$ and $s\in\{1,\dotsc,\min\{m,d\}\}$ we have
\[\det\left(\frac{\partial(u_1,\dotsc,u_s)}{\partial(x_1,\dotsc,x_s)}\right)=\sum_{j=1}^s{\frac{\partial}{\partial x_j}\underbrace{\left((-1)^{j+1}u_1\det\left(\frac{\partial(u_2,\dotsc,u_s)}{\partial(x_1,\dotsc,\widehat{x}_j,\dotsc,x_s)}\right)\right)}_{=:P_j(u,\nabla u)}}=\divergence{P(u,\nabla u)},\]
where $P=(P_1,\dotsc,P_s,0,\dotsc,0)$ with $d-s$ zeros. Here, we introduced the notation
\[\frac{\partial(u_1,\dotsc,u_s)}{\partial(x_1,\dotsc,x_s)}=\left(\frac{\partial u_i}{\partial x_j}\right)_{i,j=1}^d\]
which denotes the top left $(s\times s)$-submatrix of the Jacobian matrix of $u$, and $\frac{\partial(u_2,\dotsc,u_s)}{\partial(x_1,\dotsc,\widehat{x}_j,\dotsc,x_s)}$ denotes the submatrix of $\frac{\partial(u_1,\dotsc,u_s)}{\partial(x_1,\dotsc,x_s)}$ which is obtained by deleting the first row and $j$-th column.
\end{lemma}


\begin{remark}
The main point is that the subdeterminants of the gradient have the divergence structure $\det(\nabla u_{K,L})=\divergence{P_{K,L}(u,\nabla u)}$, and $P_{K,L}$ has slightly better properties: It has one derivative less, the rest are still subdeterminants and the outer divergence can be moved to test functions. The latter is useful for e.g. weak convergence and weak formulations, because the divergence can be pushed to test functions.
\end{remark}

\begin{example}
Before we move on to the proof, we will first check the formula for the special cases $s=1$ and $s=2$. For $s=1$ it is simple; we just have
\[\frac{\partial u_1}{\partial x_1}=\frac{\partial}{\partial x_1}((-1)^2u_1\cdot1).\]
For $s=2$ we have to calculate a little bit
\begin{align*}
	\det\left(\frac{\partial(u_1,u_2)}{\partial(x_1,x_2)}\right)&=\frac{\partial u_1}{\partial x_1}\frac{\partial u_2}{\partial x_2}-\frac{\partial u_1}{\partial x_2}\frac{\partial u_2}{\partial x_1}\\
	&\overset{!}{=}\frac{\partial}{\partial x_1}\left((-1)^2u_1\frac{\partial u_2}{\partial x_2}\right)+\frac{\partial}{\partial x_2}\left((-1)^3u_1\frac{\partial u_2}{\partial x_1}\right)\\
	&=\frac{\partial u_1}{\partial x_1}\frac{\partial u_2}{\partial x_2}+u_1\frac{\partial^2u_2}{\partial x_1\partial x_2}-\frac{\partial u_1}{\partial x_2}\frac{\partial u_2}{\partial x_1}-u_1\frac{\partial^2u_2}{\partial x_2\partial x_1}.
\end{align*}
Since $u$ is in particular a $C^2$-function we can make use of Schwarz's theorem in order to see that the equation is valid.
\end{example}

\begin{proof}[Proof of Lemma \ref{lem:MinorsAsDivergence}]
First use Laplace expansion for the determinant, then the product rule for derivatives, to get
\begin{align*}
	\det\left(\frac{\partial(u_1,\dotsc,u_s)}{\partial(x_1,\dotsc,x_s)}\right)&=\sum_{j=1}^s{(-1)^{j+1}\frac{\partial u_1}{\partial x_j}\det\left(\frac{\partial(u_2,\dotsc,u_s)}{\partial(x_1,\dotsc,\widehat{x}_j,\dotsc,x_s)}\right)}\\
	&=\sum_{j=1}^s{\frac{\partial}{\partial x_j}\left((-1)^{j+1}u_1\det\left(\frac{\partial(u_2,\dotsc,u_s)}{\partial(x_1,\dotsc,\widehat{x}_j,\dotsc,x_s)}\right)\right)}\\
	&\qquad\qquad-u_1\sum_{j=1}^s{(-1)^{j+1}\frac{\partial}{\partial x_j}\det\left(\frac{\partial(u_2,\dotsc,u_s)}{\partial(x_1,\dotsc,\widehat{x}_j,\dotsc,x_s)}\right)}.
\end{align*}
The first term on the right-hand-side is exactly the one term on the right-hand-side in the claim. Hence, it remains to show that $u_1M_s=0$, where
\[M_s:=\sum_{j=1}^s{(-1)^{j+1}\frac{\partial}{\partial x_j}\det\left(\frac{\partial(u_2,\dotsc,u_s)}{\partial(x_1,\dotsc,\widehat{x}_j,\dotsc,x_s)}\right)}.\]
More precisely, we have to show $M_s\equiv0$ since $u_1$ is arbitrary, and we are going to do this via induction on $s$.\\

\textit{Base case:} $s=2$.
\begin{itemize}
	\item[] Use Schwarz's theorem and $u\in C^3(\Omega;\mathbb{R}^m)\subset C^2(\Omega;\mathbb{R}^m)$ to justify the last equality
	\[M_2=\sum_{j=1}^2{(-1)^{j+1}\frac{\partial}{\partial x_j}\det\left(\frac{\partial u_2}{\partial(x_1,\dotsc,\widehat{x}_j,\dotsc,x_2)}\right)}=\frac{\partial^2u_2}{\partial x_1\partial x_2}-\frac{\partial^2u_2}{\partial x_2\partial x_1}=0.\]
\end{itemize}

\textit{Induction step:} $s-1\rightsquigarrow s$.
\begin{itemize}
	\item[] On each determinant appearing in $M_s$ we use, as in the beginning of the proof, Laplace expansion and product rule in order to get
	\begin{align*}
		M_s&=\sum_{j=1}^s{(-1)^{j+1}\frac{\partial}{\partial x_j}\det\left(\frac{\partial(u_2,\dotsc,u_s)}{\partial(x_1,\dotsc,\widehat{x}_j,\dotsc,x_s)}\right)}\\
		&=\sum_{j=1}^s{(-1)^{j+1}\frac{\partial}{\partial x_j}\Biggl[\sum_{\ell=1}^{j-1}{\frac{\partial}{\partial x_\ell}\left((-1)^{\ell+1}u_2\det\left(\frac{\partial(u_3,\dotsc,u_s)}{\partial(x_1,\dotsc,\widehat{x}_\ell,\dotsc,\widehat{x}_j,\dotsc,x_s)}\right)\right)}}\\
		&\qquad\qquad\sum_{\ell=j+1}^s{\frac{\partial}{\partial x_\ell}\left((-1)^\ell u_2\det\left(\frac{\partial(u_3,\dotsc,u_s)}{\partial(x_1,\dotsc,\widehat{x}_j,\dotsc,\widehat{x}_\ell,\dotsc,x_s)}\right)\right)}\\
		&\qquad\qquad-u_2\sum_{\ell=1}^{j-1}{(-1)^{\ell+1}\frac{\partial}{\partial x_j}\det\left(\frac{\partial(u_3,\dotsc,u_s)}{\partial(x_1,\dotsc,\widehat{x}_\ell,\dotsc,\widehat{x}_j,\dotsc,x_s)}\right)}\\
		&\qquad\qquad-u_2\sum_{\ell=j+1}^s{(-1)^\ell\frac{\partial}{\partial x_j}\det\left(\frac{\partial(u_3,\dotsc,u_s)}{\partial(x_1,\dotsc,\widehat{x}_j,\dotsc,\widehat{x}_\ell,\dotsc,x_s)}\right)}\Biggr].
	\end{align*}
	The last two sums together are zero because they form a sum of type $M_{s-1}$, hence we can use induction hypothesis. So we are left with
	\begin{align*}
		M_s&=\sum_{j=1}^s{\Biggl[\sum_{\ell=1}^{j-1}{(-1)^{j+\ell}\frac{\partial^2}{\partial x_j\partial x_\ell}\left(u_2\det\left(\frac{\partial(u_3,\dotsc,u_s)}{\partial(x_1,\dotsc,\widehat{x}_\ell,\dotsc,\widehat{x}_j,\dotsc,x_s)}\right)\right)}}\\
		&\qquad\qquad\sum_{\ell=j+1}^s{(-1)^{j+\ell+1}\frac{\partial^2}{\partial x_j\partial x_\ell}\left(u_2\det\left(\frac{\partial(u_3,\dotsc,u_s)}{\partial(x_1,\dotsc,\widehat{x}_j,\dotsc,\widehat{x}_\ell,\dotsc,x_s)}\right)\right)}\Biggr]\\
		&=\sum_{1\leq\ell<j\leq s}{(-1)^{j+\ell}\frac{\partial^2}{\partial x_j\partial x_\ell}\left(u_2\det\left(\frac{\partial(u_3,\dotsc,u_s)}{\partial(x_1,\dotsc,\widehat{x}_\ell,\dotsc,\widehat{x}_j,\dotsc,x_s)}\right)\right)}\\
		&\qquad\qquad+\sum_{1\leq j<\ell\leq s}{(-1)^{j+\ell+1}\frac{\partial^2}{\partial x_j\partial x_\ell}\left(u_2\det\left(\frac{\partial(u_3,\dotsc,u_s)}{\partial(x_1,\dotsc,\widehat{x}_j,\dotsc,\widehat{x}_\ell,\dotsc,x_s)}\right)\right)}.
	\end{align*}
	Interchanging indices in the second term gives with Schwarz's theorem (here we really need $u\in C^3(\Omega;\mathbb{R}^m)$)
	\begin{align*}
		M_s&=\sum_{1\leq\ell<j\leq s}{\left((-1)^{j+\ell}+(-1)^{j+\ell+1}\right)\frac{\partial^2}{\partial x_j\partial x_\ell}\left(u_2\det\left(\frac{\partial(u_3,\dotsc,u_s)}{\partial(x_1,\dotsc,\widehat{x}_\ell,\dotsc,\widehat{x}_j,\dotsc,x_s)}\right)\right)}\\
		&=0.
	\end{align*}
\end{itemize}
\end{proof}

\begin{theorem}[Minors are quasi-affine]
Let $s\in\{1,\dotsc,\min\{m,d\}\}$, and $D\subset\mathbb{R}^d$ open, bounded Lipschitz domain. Then, for all $A\in\mathbb{R}^{m\times d}$ and all $\psi\in W_0^{1,s}(D;\mathbb{R}^m)$ it holds
\[\int_D{T_s(A+\nabla\psi(x))\mathrm{d}x}=\vol{D}T_s(A).\]
\end{theorem}

\begin{proof}
Set
\[J_s(\psi)=\int_D{T_s(A+\nabla\psi(x))\mathrm{d}x}\]
for $\psi\in C_c^\infty(D;\mathbb{R}^m)$. We show $J_s(\psi)=J_s(0)$, i.e., that $J$ is a Null-Lagrangian. Pick ordered row and column indices $K,L\in\mathbb{N}^s$, i.e., $K=(k_1,\dotsc,k_s)$, $L=(\ell_1,\dotsc,\ell_s)$, $1\leq k_1<\dotsc<k_s\leq s$, $1\leq\ell_1<\dotsc<\ell_s\leq s$. Set $u(x)=Ax+\psi(x)$, $w(x)=Ax$. Note that $\supp{u-w}\subset\mathrel\subset D$, and $u$ and $w$ agree close to the boundary. By \hyperlink{lemma_4_1_7}{Lemma 4.1.7} we have
\[\det((\nabla u)_{K,L})=\sum_{j=1}^s{\frac{\partial}{\partial x_{\ell_j}}\underbrace{\left((-1)^{j+1}u_{k_1}\det\left(\frac{\partial(u_{k_2},\dotsc,u_{k_s})}{\partial(x_{\ell_1},\dotsc,\widehat{x}_{\ell_j},\dotsc,x_{\ell_s})}\right)\right)}_{=:P_j(u,\nabla u).}}.\]
Writing $P_{K,L}=(P_1,\dotsc,P_s)$, integration over $D$ yields
\begin{align*}
	\int_D{\det((\nabla u(x))_{K,L})\mathrm{d}x}&=\int_D{\divergence{P_{K,L}(u(x),\nabla u(x))}\mathrm{d}x}\\
	&=\int_{\partial D}{P_{K,L}(u(x),\nabla u(x))\cdot\nu(x)\mathrm{d}a}\\
	&=\int_{\partial D}{P_{K,L}(w(x),\nabla w(x))\cdot\nu(x)\mathrm{d}a}\\
	&=\int_D{\divergence{P_{K,L}(w(x),\nabla w(x))}\mathrm{d}x}\\
	&=\int_D{\det((\nabla w(x))_{K,L})\mathrm{d}x}=\int_D{\det(A_{K,L})\mathrm{d}x},
\end{align*}
where we have used Gau{\ss} theorem in the second and fourth line, and $\supp{u-w}\subset\mathrel\subset D$ in the third line. Since $K,L\in\mathbb{N}^s$ were arbitrary, we conclude that the assertion is true for $\psi\in C_c^\infty(D;\mathbb{R}^m)$. Now we use density of $C_c^\infty(D;\mathbb{R}^m)$ in $W_0^{1,s}(D;\mathbb{R}^m)$ and continuity of $J_s$ on $W_0^{1,s}(D;\mathbb{R}^m)$. Since $J_s$ is constant on the dense subset $C_c^\infty(D;\mathbb{R}^m)$, we get $J_s(\psi)=J_s(0)$ for all $\psi\in W_0^{1,s}(D;\mathbb{R}^m)$.
\end{proof}

\textbf{Warning!} The reverse implications
\[\text{rank-1-convexity}\quad\Rightarrow\quad\text{quasi-convexity}\quad\Rightarrow\quad\text{poly-convexity}\quad\Rightarrow\quad\text{convexity}\]
do in general \underline{not} hold!\\[11pt]

\begin{example}
\begin{itemize}
	\item[(a)] (Poly-convex $\not\Rightarrow$ convex) For $m=d=2$ consider $f(A)=\det(A)$. First note that $f$ is convex by taking the convex (even linear) function $g(A,\delta)=\delta$. But $f$ is not convex, e.g. take
	\[A_0:=\begin{pmatrix}-1&-2\\2&1\end{pmatrix},\qquad A_1:=\begin{pmatrix}1&-2\\2&-1\end{pmatrix},\qquad\frac{1}{2}A_0+\frac{1}{2}A_1=\begin{pmatrix}0&-2\\2&0\end{pmatrix},\]
	and for these matrices we have $f(A_0)=3=f(A_1)$, but $f(\frac{1}{2}A_0+\frac{1}{2}A_1)=4$.
	\item[(b)] (Quasi-convex $\not\Rightarrow$ poly-convex) We consider the family of functions for $\gamma\in\mathbb{R}$
    \[
        f_\gamma:\mathbb{R}^{2\times 2}\longrightarrow\mathbb{R},\qquad f_\gamma(A):=\lvert A\rvert^2\left(\lvert A\rvert^2-2\gamma\det(A)\right).
    \]
	In \cite{AliDac1992EQFP}, it was proven that there exists some $\varepsilon>0$ such that the following four assertions hold:
	\begin{itemize}
		\item[(1)] $f_\gamma$ is convex if and only if $\lvert\gamma\rvert\leq\frac{2}{3}\sqrt{2}$.
		\item[(2)] $f_\gamma$ is poly-convex if and only if $\lvert\gamma\rvert\leq1$.
		\item[(3)] $f_\gamma$ is quasi-convex if and only if $\lvert\gamma\rvert\leq1+\varepsilon$.
		\item[(4)] $g_\gamma$ is rank-1-convex if and only if $\lvert\gamma\rvert\leq\frac{2}{\sqrt{3}}$.
	\end{itemize}
	\item[(c)] (Rank-1-convex $\not\Rightarrow$ quasi-convex, \v{S}ver\'ak's example, 1992) Consider the case $d=2$, $m\geq3$, and again a family of functions, namely
	\[f_{\alpha,\beta}(A)=g(P(A))+\alpha\left(\lvert A\rvert^2+\lvert A\rvert^4\right)+\beta\lvert A-P(A)\rvert^2\]
	for $\alpha,\beta>0$, and where
	\[P:\mathbb{R}^{3\times 2}\longrightarrow\mathbb{R}^{3\times 2},\qquad P\left((A_{ij})_{\substack{1\leq i\leq 3\\1\leq j\leq 2}}\right):=\begin{pmatrix}
		A_{11}&0\\
		0&A_{22}\\
		\frac{1}{2}(A_{31}+A_{32})&\frac{1}{2}(A_{31}+A_{32})
	\end{pmatrix}.\]
	and
	\[g:\mathbb{R}^{3\times 2}\longrightarrow\mathbb{R},\qquad g\left((P_{ij})_{\substack{1\leq i\leq 3\\1\leq j\leq 2}}\right)=-P_{11}P_{22}P_{31}.\]
	In \cite[Part I, Chapter 7, 7.3 Generalized Convexity Notions and Envelopes, Example 7.10]{Rind2018CV} one can read the following:
	\begin{itemize}
		\item[(1)] For $\alpha>0$ sufficiently small, for all $\beta>0$ the function $f_{\alpha,\beta}$ is \underline{not} quasi-convex.
		\item[(2)] For every $\alpha>0$ there exists $\beta_\alpha>0$ such that $f_{\alpha,\beta_\alpha}$ is rank-1-convex.\\
	\end{itemize}

	Until now (January 2023), the case $m=d=2$ is still a major unsolved problem.
\end{itemize}
\end{example}
    \section{Poly-Convexity}
We consider $f:\mathbb{R}^{m\times d}\longrightarrow\mathbb{R}_\infty$ and the associated functional
\[I(u)=\int_\Omega{f(\nabla u(x))\mathrm{d}x}\]
such that
\begin{itemize}
	\item[(a)] $f$ is poly-convex with $f(A)=g(T(A))$ and a suitable, convex $g:\mathbb{R}^{\tau(m,d)}\longrightarrow\mathbb{R}_\infty$,
	\item[(b)] $f(A)\geq\alpha_0+B_0:A$ for some $\alpha_0\in\mathbb{R}$, $B_0\in\mathbb{R}^{m\times d}$.\\
\end{itemize}

We can also write $I(u)=J(T(\nabla u))$, where
\[J:L^q(\Omega;\mathbb{R}^{\tau(m,d)})\longrightarrow\mathbb{R}_\infty,\qquad J(\widetilde{T}):=\int_\Omega{g(\widetilde{T}(x))\mathrm{d}x}.\]
Our goal is to establish weak lower semicontinuity for $I$ on $W^{1,q}(\Omega;\mathbb{R}^m)$. So far we know from our developed theory, since $J$ is convex, that $J$ is weakly lower semicontinuous on $L^q(\Omega;\mathbb{R}^{\tau(m,d)})$. So it remains to show weak sequential continuity of
\[\widehat{T}:W^{1,q}(\Omega;\mathbb{R}^m)\longrightarrow L^q(\Omega;\mathbb{R}^{\tau(m,d)}),\qquad\widehat{T}(u):=T(\nabla u).\]
If for example $\min\{m,d\}=1$, then $\widehat{T}$ is linear since $T(A)=A$. In the case $\min\{m,d\}>1$, $\widehat{T}$ is no longer linear. Here we will use the special structure of the gradient minors. Of course, we are talking about nothing other than the divergence structure.\\[11pt]

\begin{theorem}[Weak continuity of gradient minors]
Let $s\in\{1,\dotsc,\min\{m,d\}\}$ and $p>s$. Let $(u_n)_{n\in\mathbb{N}}\subset W^{1,p}(\Omega;\mathbb{R}^m)$, $u\in W^{1,p}(\Omega;\mathbb{R}^m)$. Then $T_s(\nabla u_n),T_s(\nabla u)\in L^{\frac{p}{s}}(\Omega;\mathbb{R}^{\tau_s(m,d)})$ for each $n\in\mathbb{N}$. If moreover $u_n\rightharpoonup u$ in $W^{1,p}(\Omega;\mathbb{R}^m)$ then $T_s(\nabla u_n)\rightharpoonup T_s(\nabla u)$ in $L^\frac{p}{s}(\Omega;\mathbb{R}^{\tau_s(m,d)})$.
\end{theorem}

\begin{remark} 
The first version was proven by Yurii Reshetnyak \textit{1968. Then, in 1977,} John Ball \textit{gave a better version of the proof.}
\end{remark}

\textit{The claim is wrong for $p=s$. The crucial point is that we will use that $C_c^\infty(\Omega;\mathbb{R}^m)$ is dense in $L^{(p/s)'}(\Omega;\mathbb{R}^m)$, but this is no longer true if $p=s$ as ``$1'=\infty$''. We will discuss counterexamples in the exercise class.}\\

\begin{proof}
Observe that each $w\in W^{1,p}(\Omega;\mathbb{R}^m)$ fulfills $T_s(\nabla w)\in L^{\frac{p}{s}}(\Omega;\mathbb{R}^{\tau_s(m,d)})$ (that's true even if $p=s$). For the weak convergence we use an induction over $s$.\\

\textit{Base case:} $s=1$.
\begin{itemize}
	\item[] In that case, $A\longmapsto T_1(A)=A$ is linear and the claim follows.\\
\end{itemize}

\textit{Induction step:} $s-1\rightsquigarrow s$.
\begin{itemize}
	\item[] By Rellich's compact theorem we obtain $u_n\to u$ in $L^p(\Omega;\mathbb{R}^m)$, and from that we also obtain $u_n\to u$ in $L^{(\frac{p}{s-1})'}(\Omega;\mathbb{R}^m)$ since
	\[\left(\frac{p}{s-1}\right)'=\frac{p}{s-1}\cdot\frac{1}{\frac{p}{s-1}-1}=\frac{p}{p-s+1}<p,\]
	since $p>s$. Consider an arbitrary minor of order $s$ with row indices $K=(k_1,\dotsc,k_s)$ and column indices $L=(\ell_1,\dotsc,\ell_s)$. For weak convergence we have to show
	\[\int_\Omega{\det\bigl((\nabla u_n(x))_{K,L}\bigr)\varphi(x)\mathrm{d}x}\to\int_\Omega{\det\bigl((\nabla u(x))_{K,L}\bigr)\varphi(x)\mathrm{d}x}\]
	as $n\to\infty$, for all $\varphi\in L^{(p/s)'}(\Omega;\mathbb{R})$. Since $u_n\rightharpoonup u$ in $W^{1,p}(\Omega;\mathbb{R}^m)$ we have that $(\nabla u_n)_{n\in\mathbb{N}}$ and $\nabla u$ are uniformly bounded in $L^p(\Omega;\mathbb{R}^{m\times d})$. Hence, $(\det((\nabla u_n)_{K,L}))_{n\in\mathbb{N}},\det((\nabla u)_{K,L})$ are uniformly bounded in $L^{\frac{p}{s}}(\Omega;\mathbb{R})$ as well. Thus, it suffices to consider $\varphi$ in the dense subset $C_c^\infty(\Omega;\mathbb{R})$. \hyperlink{lemma_4_1_7}{Lemma 4.1.7} and integration by parts yields
	\begin{align*}
		&\int_\Omega{\det\bigl((\nabla u_n(x))_{K,L}\bigr)\varphi(x)\mathrm{d}x}\\
		&\qquad\qquad=\int_\Omega{\sum_{j=1}^s{\frac{\partial}{\partial x_{\ell_j}}\Biggl((-1)^{j+1}(u_n)_{k_1}(x)\underbrace{\det\left(\frac{\partial((u_n)_{k_2},\dotsc,(u_n)_{k_s})}{\partial(x_{\ell_1},\dotsc,\widehat{x}_{\ell_j},\dotsc,x_{\ell_s})}(x)\right)}_{=:a_{n,j}(x)}\Biggr)}\varphi(x)\mathrm{d}x}\\
		&\qquad\qquad=-\int_\Omega{\sum_{j=1}^s{(-1)^{j+1}(u_n)_{k_1}(x)a_{n,j}(x)\partial_{x_{\ell_j}}\varphi(x)}\mathrm{d}x}.
	\end{align*}
	(The middle step is only allowed for $u_n\in C^3(\Omega;\mathbb{R}^m)$, but, by a density argument, the conclusion remains true for the regularity our $u_n$ have.) By induction hypothesis we have $a_{n,j}\rightharpoonup a_j$ in $L^{\frac{p}{s-1}}(\Omega;\mathbb{R})$, where $a_j=\det\Bigl(\frac{\partial((u_n)_{k_2},\dotsc,(u_n)_{k_s})}{\partial(x_{\ell_1},\dotsc,\widehat{x}_{\ell_j},\dotsc,x_{\ell_s})}\Bigr)$. Moreover, we have $(u_n)_{k_1}\partial_{x_{\ell_j}}\varphi\in L^{(\frac{p}{s-1})'}(\Omega;\mathbb{R})$ and strong convergence $(u_n)_{k_1}\partial_{x_{\ell_j}}\varphi\to u_{k_1}\partial_{x_{\ell_j}}\varphi$ in $L^{(\frac{p}{s-1})'}(\Omega;\mathbb{R})$ for $n\to\infty$. Together with the weak convergence of $(a_{n,j})_{n\in\mathbb{N}}$ we get
	\begin{align*}
		&\int_\Omega{\det\bigl((\nabla u_n(x))_{K,L}\bigr)\varphi(x)\mathrm{d}x}\\
		&\qquad\qquad\to-\int_\Omega{\sum_{j=1}^s{\frac{\partial}{\partial x_{\ell_j}}\left((-1)^{j+1}u_{k_1}(x)\det\left(\frac{\partial(u_{k_2},\dotsc,u_{k_s})}{\partial(x_{\ell_1},\dotsc,\widehat{x}_{\ell_j},\dotsc,x_{\ell_s})}(x)\right)\right)}\varphi(x)\mathrm{d}x}\\
		&\qquad\qquad=\int_\Omega{\det\bigl((\nabla u(x))_{K,L}\bigr)\varphi(x)\mathrm{d}x}.
	\end{align*}
\end{itemize}
\end{proof}


From now on, we assume that $\infty>p>\min\{m,d\}>1$. We introduce
\[Y_{p,m,d}:=\bigtimes_{s=1}^{\min\{m,d\}}{L^{\frac{p}{s}}\left(\Omega;\mathbb{R}^{\tau_s(m,d)}\right)}\]
so that $Y_{p,m,d}\longhookrightarrow L^{\frac{p}{\min\{m,d\}}}(\Omega;\mathbb{R}^{\tau(m,d)})$ is a reflexive Banach space. We have seen above that the map
\[\widehat{T}:W^{1,p}(\Omega;\mathbb{R}^m)\longrightarrow Y_{p,m,d},\qquad\widehat{T}(u):=T(\nabla u)\]
is continuous with respect to norm (since the operator only includes sums of products of $L^p$-factors) and weak topology. Define
\[\Gamma:=\widehat{T}\left(W^{1,p}(\Omega;\mathbb{R}^m)\right)\subset Y_{p,m,d}.\]
For this set observe the following:
\begin{itemize}
	\item[(a)] $\Gamma$ is weakly sequentially closed. Indeed, let $(\gamma^{(n)})_{n\in\mathbb{N}}\subset\Gamma$ be a sequence of gradient minors such that $\gamma^{(n)}\rightharpoonup\gamma$ in $Y_{p,m,d}$ as $n\to\infty$ for some $\gamma\in Y_{p,m,d}$. By definition of $\Gamma$, there exist $u^{(n)}\in W^{1,p}(\Omega;\mathbb{R}^m)$ with $\gamma^{(n)}=\widehat{T}(u^{(n)})$. Note that $u^{(n)}$ is in general not unique, but if we assume in addition $\int_\Omega{u^{(n)}(x)\mathrm{d}x}=0\in\mathbb{R}^m$, then $u^{(n)}$ is unique by Poincar\'e-Wirtinger inequality
	\[\left\lVert u-\fint_\Omega{u(x)\mathrm{d}x}\right\rVert_{L^p(\Omega;\mathbb{R}^m)}\leq C_\text{PW}\lVert\nabla u\rVert_{L^p(\Omega;\mathbb{R}^{m\times d})}\]
	and $\nabla u^{(n)}=T_1(\nabla u^{(n)})=\gamma_1^{(n)}$. So we get $\nabla u^{(n)}=T_1(\nabla u^{(n)})=\gamma_1^{(n)}\rightharpoonup\gamma_1$ in $L^p(\Omega;\mathbb{R}^{m\times d})$. In particular, $(\nabla u^{(n)})_{n\in\mathbb{N}}$ is bounded in $L^p(\Omega;\mathbb{R}^{m\times d})$ and then Poincar\'e-Wirtinger provides boundedness of $(u^{(n)})_{n\in\mathbb{N}}$ in $L^p(\Omega;\mathbb{R}^m)$. Thus, by reflexivity of $L^p(\Omega;\mathbb{R}^m)$, we have up to a subsequence which we do not relabel that $u^{(n)}\rightharpoonup u$ in $L^p(\Omega;\mathbb{R}^m)$ for some $u\in L^p(\Omega;\mathbb{R}^m)$. For all $\varphi\in C_c^\infty(\Omega;\mathbb{R}^m)$ we have
	\begin{align*}
		\int_\Omega{\gamma_{1,\bullet\ell}(x)\varphi(x)\mathrm{d}x}&=\lim_{n\to\infty}{\int_\Omega{\partial_{x_\ell}u^{(n)}(x)\varphi(x)\mathrm{d}x}}\\
		&=-\lim_{n\to\infty}{\int_\Omega{u^{(n)}(x)\partial_{x_\ell}\varphi(x)\mathrm{d}x}}\\
		&=-\int_\Omega{u(x)\partial_{x_\ell}\varphi(x)\mathrm{d}x},
	\end{align*}
	so $\nabla u=\gamma_1$ since weak derivative is unique. \hyperlink{theorem_4_2_1}{Theorem 4.2.1} yields $\gamma^{(n)}=\widehat{T}(u^{(n)})\rightharpoonup\widehat{T}(u)$ as $n\to\infty$, and, since the weak limit is unique, we infer $\gamma=\widehat{T}(u)$. Therefore $\gamma\in\Gamma$.
	\item[(b)] $\Gamma$ is not convex. Recall Mazur's lemma: If a set is strongly closed and convex then it is also weakly closed. But here $\Gamma$ is not convex. To see this we let
	\[A_1=\left(\begin{array}{c|c}
		\begin{array}{cc}
			+1&0\\
			0&+1
		\end{array}&\textbf{0}\\ \hline
		\textbf{0}&\textbf{0}
	\end{array}\right)\quad\text{and}\quad A_2=\left(\begin{array}{c|c}
		\begin{array}{cc}
			-1&0\\
			0&-1
		\end{array}&\textbf{0}\\ \hline
		\textbf{0}&\textbf{0}
	\end{array}\right)\]
	and define $u_j(x)=A_jx$ for $j=1,2$. Then $\widehat{T}(u_j)\in\Gamma$ and
	\[\frac{1}{2}T_2(A_1)+\frac{1}{2}T_2(A_2)=(1,0,\dotsc,0)\in\mathbb{R}^{\tau_2(m,d)},\]	
	where we put $\det(A_{(1,2),(1,2)})$ here in the first component of $T_2(A)$. But
	\[\frac{1}{2}\widehat{T}(u_1)+\frac{1}{2}\widehat{T}(u_2)=(\underbrace{0}_{\in\mathbb{R}^{m\times d}},\underbrace{\frac{1}{2}T_2(A_1)+\frac{1}{2}T_2(A_2)}_{\ne0\in\mathbb{R}^{\tau_2(m,d)}},\dotsc).\]
	No matrix $A$ exists such that $T(A)=\frac{1}{2}T(A_1)+\frac{1}{2}T(A_2)$.\\[11pt]
\end{itemize}

The set $\widetilde{\Gamma}=\{T(A)\mid A\in L^p(\Omega;\mathbb{R}^{m\times d})\}$ is closed with respect to norm in $Y_{p,m,d}$, but not weakly sequentially closed. The point is that general matrix-valued functions $A\in L^p(\Omega;\mathbb{R}^{m\times d})$ do not have an additional structure like a gradient structure. For example take
\[A_k:\Omega\longrightarrow\mathbb{R}^{m\times d},\qquad A_k(x_1,\dotsc,x_d):=\left(\begin{array}{c|c}
	\begin{array}{cc}
		\sin(kx_1)&0\\
		0&\sin(kx_j)
	\end{array}&\textbf{0}\\ \hline
	\textbf{0}&\textbf{0}
\end{array}\right)\]
for $j\in\{1,\dotsc,d\}$ fixed. We know that $A_k\rightharpoonup0$ in $L^p(\Omega;\mathbb{R}^{m\times d})$ for all $p\in[1,\infty)$, even $A_k\overset{*}{\rightharpoonup}0$ in $L^\infty(\Omega;\mathbb{R}^{m\times d})$. Consider $T_2(A_k(x))=(\sin(kx_1)\sin(kx_j),0,\dotsc,0)$ which converges weakly to $(\rho,0,\dotsc,0)$ as $k\to\infty$, where
\[\rho=\left\{\begin{array}{rl}
	0&\text{if }j\ne1,\\
	\frac{1}{2}&\text{if }j=1.
\end{array}\right.\]
For $j=1$, $A_k$ cannot be written as a gradient of a function $u_k$ because $\partial_{x_2}u_{k,2}=\sin(kx_1)$ and $\partial_{x_1}u_{k,2}=0$ cannot be fulfilled simultaneously. For $j=2$, we have $A_k=\nabla u_k$ with
\[u_k(x)=\begin{pmatrix}
	-\frac{1}{k}\cos(kx_1)\\
	-\frac{1}{k}\cos(kx_2)\\
	0\\
	\vdots\\
	0
\end{pmatrix}.\]\\

\begin{theorem}[Weak sequential lower semicontinuity in poly-convex case]
Let $\Omega\subset\mathbb{R}^d$ be an open, bounded Lipschitz domain, let $g:\Omega\times\mathbb{R}^{\tau(m,d)}\longrightarrow[0,\infty]$ be continuous and such that $g(x,\cdot):\mathbb{R}^{\tau(m,d)}\longrightarrow[0,\infty]$ is convex for almost all $x\in\Omega$. Then, the associated functional
\[I:W^{1,p}(\Omega;\mathbb{R}^m)\longrightarrow[0,\infty],\qquad I(u):=\int_\Omega{g(x,T(\nabla u(x)))\mathrm{d}x}\]
is weakly sequentially lower semicontinuous for all $\infty>p>\min\{m,d\}(>1)$.\\
\end{theorem}
\textit{Remark: The function $g$ can be generalized, see Chapter~\ref{ch:convexcase}}\\

\begin{proof}
Consider the auxiliary functional
\[J:Y_{p,m,d}\longrightarrow[0,\infty],\qquad J(\gamma):=\int_\Omega{g(x,\nabla\gamma(x))\mathrm{d}x}\]
so that $I=J\circ\widehat{T}$. Clearly $J$ is convex, and due to Fatou's lemma it is also lower semicontinuous with respect to norm continuity. Then $J$ is also weakly sequentially lower semicontinuous on $Y_{p,m,d}$ by \hyperlink{theorem_3_2_6}{Theorem 3.2.6}. If $u_n\rightharpoonup u$ in $W^{1,p}(\Omega;\mathbb{R}^m)$ for $u_n,u\in W^{1,p}(\Omega;\mathbb{R}^m)$, then $\widehat{T}(u_n)\rightharpoonup\widehat{T}(u)$ in $Y_{p,m,d}$ by \hyperlink{theorem_4_2_1}{Theorem 4.2.1} and hence
\[\liminf_{n\to\infty}{I(u_n)}=\liminf_{n\to\infty}{J(\widehat{T}(u_n))}\geq J(\widehat{T}(u))=I(u).\]
\end{proof}

\begin{theorem}
[Existence of minimizers in poly-convex case]
Let $\Omega\subset\mathbb{R}^d$ be an open, bounded Lipschitz domain, $g:\Omega\times\mathbb{R}^{m\times d}\longrightarrow\mathbb{R}_\infty$ continuous, and convex in the second component, $\infty>p>\min\{m,d\}$. Additionally assume that
\[g(T(A))\geq C_1\lvert A\rvert^p-C_2\]
for some $C_1,C_2\in(0,\infty)$. Then, for every $\varphi\in(W^{1,p}(\Omega;\mathbb{R}^m))'$ the functional
\[I:W_0^{1,p}(\Omega;\mathbb{R}^m)\longrightarrow\mathbb{R}_\infty,\qquad I(u):=\int_\Omega{g(T(\nabla u(x)))\mathrm{d}x}-\varphi(u)\]
has a minimizer $u_*\in W_0^{1,p}(\Omega;\mathbb{R}^m)$.\\
\end{theorem}
\begin{proof}
Combine the abstract existence result with the previous theorem. To be more precise, the space $W_0^{1,p}(\Omega;\mathbb{R}^m)$ is reflexive, $I$ is coercive because of the $p$-growth and Poincar\'e inequality, and $I$ is weakly lower semicontinuous by \hyperlink{theorem_4_2_2}{Theorem 4.2.2} (applied to $g+C_2$ which is non-negative). Hence \hyperlink{theorem_3_1_14}{Theorem 3.1.14} postulates a minimizer.
\end{proof}
    \section{Quasi-Convexity and Weak Lower Semicontinuity}
We show that quasi-convexity of $f$ is both necessary and sufficient for weak lower semicontinuity. Recall that $f:\mathbb{R}^{m\times d}\longrightarrow\mathbb{R}_\infty$ is quasi-convex if for all $A\in\mathbb{R}^{m\times d}$, all $\psi\in C_c^\infty(B_1(0);\mathbb{R}^m)$
\[\int_{B_1(0)}{f(A+\nabla\psi(x))\mathrm{d}x}\geq\vol{B_1(0)}f(A).\]
According to \hyperlink{remark_4_1_5}{Remark 4.1.5 (b)}, we can even consider $\psi\in W_0^{1,\infty}(B_1(0);\mathbb{R}^m)$ and replace the unit ball $B_1(0)$ with every arbitrary bounded, open set $D\subset\mathbb{R}^d$.\\

\begin{theorem}[Vitali covering]
Let $\Omega\subset\mathbb{R}^d$ be bounded and open. Fix $\delta>0$. Then there exists a countable family $\mathcal{G}$ of pairwise disjoint closed balls in $\Omega$ such that $\diam{B}\leq\delta$ for each $B\in\mathcal{G}$ and $\mathcal{L}^d(\Omega\setminus\bigcup_{B\in\mathcal{G}}{B})=0$.
\end{theorem}
\textit{Proof:}\\
\cite[Kapitel V, §1 Produktma{\ss}e, 5. Das $p$-dimensionale \"au{\ss}ere Hausdorff-Ma{\ss}, Satz 1.14]{Elst1996MIT}.\hfill$\blacksquare$\\[11pt]

\begin{theorem}[Quasi-convexity is necessary and sufficient]
Let $\Omega\subset\mathbb{R}^d$ be a bounded, open Lipschitz domain. Fix $p\in(1,\infty)$. Let $f:\mathbb{R}^{m\times d}\longrightarrow[0,\infty)$ be continuous such that $f(A)\leq C(1+\lvert A\rvert^p)$ for all $A\in\mathbb{R}^{m\times d}$ and some $C>0$. Consider
\[I:W^{1,p}(\Omega;\mathbb{R}^m)\longrightarrow[0,\infty),\qquad I(u):=\int_\Omega{f(\nabla u(x))\mathrm{d}x}.\]
Then, the following assertions are equivalent:
\begin{itemize}
	\item[(a)] $I$ is weakly sequentially lower semicontinuous on $W^{1,p}(\Omega;\mathbb{R}^m)$.
	\item[(b)] $f$ is quasi-convex.\\
\end{itemize}
\end{theorem}
\textit{Remark: The lower bound of $f$ by zero is a relatively strong assumption. But a lower bound of polynomial type of order $p$ would not work, if we e.g. consider the example of Tartar, $m=d=2$ and $f(A)=\det(A)$, then this is poly-convex, hence quasi-convex, and fulfills $\lvert f(A)\rvert\leq c\lvert A\rvert^2$, but its associated functional is not weakly lower semicontinuous on $H^1((0,1)^2;\mathbb{R}^2)$.}\\

\textit{However, a lower bound of the type $f(A)\geq c_1\lvert A\rvert^{p-\varepsilon}-c_2$ for $\varepsilon>0$ would be enough to prove the equivalence, see \cite[Part II, Chapter 8, Section 8.2 Weak lower semicontinuity, 8.2.4 Lower semicontinuity for general quasiconvex functions for $1\leq p<\infty$]{Daco2007DMCV}.}

\textit{Proof:}
\begin{itemize}
	\item[(a)] $\Rightarrow$ (b): Assume $I$ is weakly lower semicontinuous. There exist $x\in\mathbb{R}^d$ and $R>0$ such that $B_1(0)\subset\mathrel\subset\Omega':=y+R\Omega$, and $I$ remains weakly sequentially lower semicontinuous on $W^{1,p}(\Omega';\mathbb{R}^m)$. That means, after eventual rescaling and translation, we can assume $B_1(0)\subset\mathrel\subset\Omega$. Let $A\in\mathbb{R}^{m\times d}$, $\psi\in C_c^\infty(B_1(0);\mathbb{R}^m)$. We need to show
	\[f(A)\leq\frac{1}{\vol{B_1(0)}}\int_{B_1(0)}{f(A+\nabla\psi(x))\mathrm{d}x}.\]
	For $j\in\mathbb{N}$ consider a Vitali covering of $B_1(0)$, that means, by \hyperlink{theorem_4_3_1}{Theorem 4.3.1} we can find $r_k^{(j)}\in(0,\delta_j)$, $x_k^{(j)}\in B_1(0)$, where $\delta_j:=\frac{1}{j}$, such that
	\[B_1(0)=\dot\bigcup_{k\in\mathbb{N}}{B_{r_k^{(j)}}(x_k^{(j)})}\cup N\]
	for some $N\subset B_1(0)$ with $\mathcal{L}^d(N)=0$. Define for $j\in\mathbb{N}$ the function $u_j:\Omega\longrightarrow\mathbb{R}^m$ via
	\[u_j(x):=Ax+\left\{\begin{array}{rl}
		r_k^{(j)}\psi\left(\frac{x-x_k^{(j)}}{r_k^{(j)}}\right)&\text{if }x\in B_{r_k^{(j)}}(x_k^{(j)}),\\
		0&\text{otherwise}
	\end{array}\right.\]
	so $u_j$ just consists of scaled copies of $\psi$, where each copy is living on one of the balls $B_{r_k^{(j)}}(x_k^{(j)})$. Clearly we have $u_j\to u_*$ in $L^p(\Omega;\mathbb{R}^m)$ with $u_*(x)=Ax$, since $\psi$ is bounded and $\sup\{r_k^{(j)}\mid k\in\mathbb{N}\}\to0$ for $j\to\infty$. Furthermore, $\nabla u_j\rightharpoonup A=\nabla u_*$ in $L^p(\Omega;\mathbb{R}^{\times d})$ because $\rVert\nabla u_j\rVert_{L^\infty(\Omega)}$ is bounded uniformly in $j$, $W^{1,p}(\Omega;\mathbb{R}^{m\times d})$ is reflexive and embeds into $L^p(\Omega;\mathbb{R}^m)$. By assumption we have
	\[\vol{\Omega}f(A)=I(u_*)\leq\liminf_{j\to\infty}{\int_\Omega{f(\nabla u_j(x))\mathrm{d}x}}.\]
	Use that $\nabla u_j=A$ in $\Omega\setminus B_1(0)$. Hence we obtain
	\begin{align*}
		\vol{B_1(0)}f(A)&\leq\liminf_{j\to\infty}{\int_{B_1(0)}{f(\nabla u_j(x))\mathrm{d}x}}\\
		&=\liminf_{j\to\infty}{\sum_{k=1}^\infty{\int_{B_{r_k^{(j)}}(x_k^{(j)})}{f(A+\nabla\psi\left(\frac{x-x_k^{(j)}}{r_k^{(j)}}\right))\mathrm{d}x}}}\\
		&=\liminf_{j\to\infty}{\sum_{k=1}^\infty{(r_k^{(j)})^d\int_{B_1(0)}{f(A+\nabla\psi(y))\mathrm{d}y}}},
	\end{align*}
	where the second line follows since $f\geq0$, and the third is a change of variables. Since
	\[\vol{B_1(0)}=\sum_{k=1}^\infty{\vol{B_{r_k^{(j)}}(x_k^{(j)})}}=\sum_{k=1}^\infty{(r_k^{(j)})^d\vol{B_1(0)}}\]
	it follows that $\sum_{k=1}^\infty{(r_k^{(j)})^d}=1$. Therefore we have shown
	\[\vol{B_1(0)}f(A)\leq\int_{B_1(0)}{f(A+\nabla\psi(x))\mathrm{d}x}.\]\\

	For this implication, we haven't used the growth condition.
	\item[(b)] $\Rightarrow$ (a): Assume $f$ is quasi-convex. Since $f$ is continuous and satisfies the $p$-growth, and since $W_0^{1,\infty}(\Omega;\mathbb{R}^m)$ is dense in $W_0^{1,p}(\Omega;\mathbb{R}^m)$, we can use in the definition of quasi-convexity even test functions $\psi\in W_0^{1,p}(\Omega;\mathbb{R}^m)$.\\

	Let $(u_k)_{k\in\mathbb{N}}\subset W^{1,p}(\Omega;\mathbb{R}^m)$, $u\in W^{1,p}(\Omega;\mathbb{R}^m)$ such that $u_k\rightharpoonup u_*$ in $W^{1,p}(\Omega;\mathbb{R}^m)$ as $k\to\infty$. We are going to show $I(u_*)\leq\liminf_{k\to\infty}{I(u_k)}$ during three steps, gradually weaken the assumption on the given objects.\\

	\textit{Step 1:}
	\begin{itemize}
		\item[] Suppose first $u_*$ is affine, i.e. $\nabla u_*\equiv A$ for some constant matrix $A\in\mathbb{R}^{m\times d}$, and also that $u_k-u_*\in W_0^{1,p}(\Omega;\mathbb{R}^m)$. With these assumptions we estimate
		\[I(u_*)=\int_\Omega{f(A)\mathrm{d}x}=\vol{\Omega}f(A)\leq\int_\Omega{f(A+\nabla(u_k-u_*)(x))\mathrm{d}x}=I(u_k)\]
		for each $k\in\mathbb{N}$. Hence $I(u_*)\leq\liminf_{k\to\infty}{I(u_k)}$.\\
	\end{itemize}

	\textit{Step 2:}
	\begin{itemize}
		\item[] Now suppose that $u_*$ is affine, but the $u_k-u_*$ do no longer need to be in $W_0^{1,p}(\Omega;\mathbb{R}^m)$. Let $\Omega_0\subset\mathrel\subset\Omega$ and $R:=\frac{1}{2}\operatorname{dist}(\Omega_0,\partial\Omega)>0$. For all $N\in\mathbb{N}$, all $n=1,\dotsc,N$, set
		\[\Omega_n=\left\{x\in\Omega\,\middle\vert\,\operatorname{dist}(x,\Omega_0)<\frac{n}{N}R\right\}.\]
		Then $\Omega_0\subset\Omega_1\subset\dotsc\subset\Omega_N\subset\Omega$.

		\begin{figure}[ht]
			\centering
			\begin{tikzpicture}
				% Shape of Omega
				\node at (3.5, 2) {$\Omega$};
				\draw[thick] (-3.5, 0) arc (180:90:2);
				\draw[thick] plot[smooth, domain=-1.5:-1] (\x, {sqrt(4-(\x+1.5)^2)});
				\draw[thick] plot[smooth, domain=-1:1] (\x, {-0.14177*(\x)^3+0.11558*(\x)^2+0.39827*\x+2.07741});
				\draw[thick] plot[smooth, domain=1:1.5] (\x, {sqrt(6.25-(\x-1.5)^2)});
				\draw[thick] (4, 0) arc (0:90:2.5);
				\draw[thick] (4, 0) arc (360:270:2);
				\draw[thick] plot[smooth, domain=-1.5:1.5] (\x+0.5,{-0.03704*(\x)^3+0.25*\x-2.25});
				\draw[thick] (-3.5, 0) arc (180:270:2.5);

				% Omega_0
				\draw[thick, fill=cyan, fill opacity=0.1] (-1, -0.5) rectangle (1, 0.5);
				\node at (0, 0) {$\Omega_0$};

				% Omega_1
				\draw[dashed, fill=cyan, fill opacity=0.1] (-1, 0.9678) -- (1, 0.9678) arc (90:0:0.4678) --(1.4678, -0.5) arc (360:270:0.4678) -- (-1, -0.94678) arc (270:180:0.4678) -- (-1.4678, 0.5) arc (180:90:0.4678);
				\node at (0, -0.7339) {$\Omega_1$};

				% Omega_2
				\draw[dashed, fill=cyan, fill opacity=0.1] (-1, 1.4356) -- (1, 1.4356) arc (90:0:0.9356) --(1.9356, -0.5) arc (360:270:0.9356) -- (-1, -1.4356) arc (270:180:0.9356) -- (-1.9356, 0.5) arc (180:90:0.9356);\node at (0, -1.2017) {$\Omega_2$};

				% Omega_3
				\draw[dashed, fill=cyan, fill opacity=0.1] (-1, 1.9034) -- (1, 1.9034) arc (90:0:1.4034) --(2.4034, -0.5) arc (360:270:1.4034) -- (-1, -1.9034) arc (270:180:1.4034) -- (-2.4034, 0.5) arc (180:90:1.4034);\node at (0, -1.6695) {$\Omega_3$};

				% Radius R
				\draw[red, fill=red] (-0.73337, 0.5) circle (1.5pt) -- node[left] {$R$} (-0.73337, 1.90341) circle (1.5pt);
			\end{tikzpicture}
			\caption{Illustration of the situation with $\Omega_0\subset\dotsc\subset\Omega_3\subset\Omega$.}
		\end{figure}

		For $n=1,\dotsc,N$ consider $\varphi_n\in C^\infty(\Omega)$ with $0\leq\varphi_n\leq 1$, $\varphi_n\equiv1$ in $\Omega_{n-1}$, $\varphi_n\equiv0$ in $\Omega\setminus\Omega_n$ and $\lvert\nabla\varphi_n\rvert\leq\frac{2N}{R}$ in $\Omega$. (Such $\varphi_n$ can be constructed by mollifying certain distance functions.) Set $v_k^{(n)}:=u_*+\varphi_n\cdot(u_k-u_*)$ and compute
		\[\nabla v_k^{(n)}=\left\{\begin{array}{rl}
			\nabla u_*\equiv A&\text{in }\Omega\setminus\Omega_n,\\
			\nabla u_k&\text{in }\Omega_{n-1},\\
			A+\varphi_n\nabla(u_k-u_*)+(u_k-u_*)\nabla\varphi_n&\text{in }\Omega_n\setminus\Omega_{n-1}.
		\end{array}\right.\]
		Then $v_k^{(n)}-u_*\in W_0^{1,p}(\Omega;\mathbb{R}^m)$ and hence quasi-convexity gives
		\begin{align*}
			I(u_*)&\leq\int_\Omega{f\left(A+\nabla(v_k^{(n)}-u_*)(x)\right)\mathrm{d}x}\\
			&=\int_{\Omega_{n-1}}{f(\nabla u_k(x))\mathrm{d}x}+\int_{\Omega\setminus\Omega_n}{f(A)\mathrm{d}x}\\
			&\qquad\qquad+\int_{\Omega_n\setminus\Omega_{n-1}}{f\bigl(A+\varphi_n(x)\nabla(u_k-u_*)(x)+(u_k-u_*)(x)\nabla\varphi_n(x)\bigr)\mathrm{d}x}\\
			&\leq\int_\Omega{f(\nabla u_k(x))\mathrm{d}x}+C\int_{\Omega\setminus\Omega_0}{(1+\lvert A\rvert^p)\mathrm{d}x}\\
			&\qquad\qquad+\widetilde{C}\int_{\Omega_n\setminus\Omega_{n-1}}{\left(1+\lvert A\rvert^p+\lvert\nabla u_k(x)\rvert^p+\lvert(u_k-u_*)(x)\rvert^p\Bigl(\frac{N}{R}\Bigr)^p\right)\mathrm{d}x},
		\end{align*}
		where we have used in the last estimate that $f$ is non-negative and satisfies the $p$-growth, and that $0\leq\varphi_n\leq 1$. Now sum over $n\in\{1,\dotsc,N\}$ and divide by $N$ to get
		\begin{align*}
			I(u_*)&\leq\int_\Omega{f(\nabla u_k(x))\mathrm{d}x}+C\int_{\Omega\setminus\Omega_0}{(1+\lvert A\rvert^p)\mathrm{d}x}\\
			&\qquad\qquad+\frac{\widetilde{C}}{N}\int_{\Omega_N\setminus\Omega_0}{\left(1+\lvert A\rvert^p+\lvert\nabla u_k(x)\rvert^p+\lvert(u_k-u_*)(x)\rvert^p\Bigl(\frac{N}{R}\Bigr)^p\right)\mathrm{d}x}.
		\end{align*}
	\end{itemize}
\end{itemize}

    \printbibliography

\end{document}